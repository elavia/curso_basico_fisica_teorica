	\documentclass[10pt,oneside]{CBFT_book}
	% Algunos paquetes
	\usepackage{amssymb}
	\usepackage{amsmath}
	\usepackage{graphicx}
	\usepackage{libertine}
	\usepackage[bold-style=TeX]{unicode-math}
	\usepackage{lipsum}

	\usepackage{natbib}
	\setcitestyle{square}

	\usepackage{polyglossia}
	\setdefaultlanguage{spanish}
	



	\usepackage{CBFT.estilo} % Cargo la hoja de estilo

	% Tipografías
	% \setromanfont[Mapping=tex-text]{Linux Libertine O}
	% \setsansfont[Mapping=tex-text]{DejaVu Sans}
	% \setmonofont[Mapping=tex-text]{DejaVu Sans Mono}

	%===================================================================
	%	DOCUMENTO PROPIAMENTE DICHO
	%===================================================================

\begin{document}

% =================================================================================================
\chapter{Introducción al estudio de procesos de relajación}
% =================================================================================================


% =================================================================================================
\section{Procesos de Markov}
% =================================================================================================

Sea $Y$ una variable estocástica que puede tomar valores $y_1, y_2,...$\notamargen{Las $P$ son densidades
de probabilidad, cuando el espacio muestral sea continuo.}
\[
	P_1(y_1,t) \equiv \text{Prob. de tomar $y_1$ en $t$ (1 paso)}
\]
\[
	P_2(y_1,t_1;y_2,t_2) \equiv \text{Prob. conjunto de tomar $y_1$ en $t_1$ y $y_2$ en $t_2$}
\]
\[
	P_{1/1}(y_1,t_1 | y_2, t_2) \equiv \text{Prob. condicional de tomar $y_2$ en $t_2$ habiendo 
	tomado $y_1$ en $t_1$ (certeza de $y_1$) }
\]

Abreviaremos obviando el tiempo. Además se tiene 
\[
	P(y_1;y_2) \leq P(y_1 | y_2)
\]
donde el lhs evalúa los caminos que comunican $y_1, y_2$ del total y el rhs evalúa los cminos que comunican
$y_1, y_2$ del subconjunto de los que parten de $y_1$.

Además
\[
	P_2(y_1;y_2) = P_1(y_1) P_{1/1}(y_1|y_2)
\]
cumpliéndose lo siguiente
\begin{itemize}
 \item $ \int P_1(y_1) dy_1 = 1 \qquad  \text{normalización} $ 
 \item $ \int P_{1/1}(y_1|y_2) dy_2 = 1 \qquad \text{normalización} $ 
 \item $ \int P_2(y_1;y_2) dy_1 = \int P_1(y_1) P_{1/1}(y_1|y_2) dy_1 =  P_1(y_2) \qquad \text{reducción} $
\end{itemize}

\subsubsection{Ejemplito numérico}

\[
	P(y_1;y_2) = P(y_1)P(y_1|y_2) = \frac{4}{4}\frac{1}{2} = \frac{2}{7}
\]
\[
	P(y_2;y_1) = P(y_2)P(y_2|y_1)  = \frac{3}{7}\frac{2}{3} = \frac{2}{7}
\]
Notemos que $P(A|B) \neq P(B|A)$ aunque $P(A;B) = P(B;A)$

Las densidades de muchos pasos: $P(y_1;y_2;y_3)$ son relevantes cuando el sistema tiene ``memoria''.

Un proceso es de Markov cuando el estado del sistema depende del paso inmediato anterior únicamente.
Se define por 
\[
	P_1(y_1) , \quad P_{1/1}(y_1|y_2) \equiv \text{Probabilidad de transición} 
\]
\[
	P_{3/1}(y_1,y_2,y_3|y_4) \underbrace{\rightarrow}_{\text{Markov}} \; P_{1/1}(y_3|y_4)
\]

Se puede demostrar una ecuación de Chapman-Kolmogorov
\[
	P_{1/1}(y_1|y_3) = \int P_{1/1}(y_1|y_2) P_{1/1}(y_2|y_3) dy_2
\]

% ~~~~~~~~~~~~~~~~~~~~~~~~~~~~~~~~~~~~~~~~~~~~~~~~~~~~~~~~~~~~~~~
\subsection{Ecuación maestra}

Queremos ver la evolución de la $P_1(y_1,t)$
\[
	\dtot{P_1(y,t)}{t} = \lim_{\tau \to 0} \frac{P_1(y,t+\tau) - P_1(y,t) }{\tau}
\]
Usando que
\[
	P_1(y_2,t+\tau) = \int dy_1 P_1(y_1,t) P_{1/1}(y_1,t|y_2,t+\tau) 
\]
\[
	P_1(y_2,t) = \int dy_1 P_1(y_1,t) P_{1/1}(y_1,t|y_2,t) 
\]
\[
	\dtot{P_1(y,t)}{t} = \int dy_1 P_1(y_1,t) \left[ \lim_{\tau \to 0} 
	\frac{1}{\tau} (P_{1/1}(y_1,t|y_2,t+\tau) - P_{1/1}(y_1,t|y_2,t))   \right]
\]
que se puede escribir de modo que 
\[
	\frac{1}{\tau} \left\{ [ 1 - \tau \int dy W(y_1,y)]\delta(y_1 - y_2) + \tau W(y_1,y_2) - \delta(y_1-y_2) 
\right\}
\]
y entonces 
\[
	\dtot{P_1(y,t)}{t} = \int dy_1 P_1(y_1,t) \left[ -\int dy W(y_1,y) \delta(y_1-y_2) + W(y_1,y_2) \right]
\]
\[
	\dtot{P_1(y,t)}{t} = \int dy_1 P_1(y_1,t) W(y_1,y_2) - \int dy_1 P_1(y_1,t) \int dy W(y_1,y)\delta(y_1-y_2)
\]
\[
	\dtot{P_1(y,t)}{t} = \int dy_1 P_1(y_1,t) W(y_1,y_2) - \int dy P_1(y_2,t) W(y_2,y)
\]
\[
	\dtot{P_1(y,t)}{t} = \int dy_1 P_1(y_1,t) W(y_1,y_2) - P_1(y_2,t) \int dy  W(y_2,y)
\]
donde el primer término en el rhs se interpreta como ganancia (lo que entra) y el segundo pérdida (pues la integral es 
lo que sale).
\[
	W(y_1,y_2) \equiv \text{Transiciones $y_1\to y_2$ por la unidad de tiempo}
\]

% ~~~~~~~~~~~~~~~~~~~~~~~~~~~~~~~~~~~~~~~~~~~~~~~~~~~~~~~~~~~~~~~
\subsection{Camino aleatorio y ecuación de difusión}

Si $\ell, \Tau$ son escalas y $n_2,s$ un número entero de pasos 
\[
	P_1(n_2\ell, s\Tau ) = \sum_{n_1} P_1( n_1\ell, [s-1]\Tau )P_{1/1}( n_1\ell, [s-1]\Tau|n_2\ell, s\Tau )
\]

% \bibliographystyle{CBFT-apa-good}	% (uses file "apa-good.bst")
% \bibliography{CBFT.Referencias} % La base de datos bibliográfica

\end{document}
