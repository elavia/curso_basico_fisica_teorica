	\documentclass[10pt,oneside]{CBFT_book}
	% Algunos paquetes
	\usepackage{amssymb}
	\usepackage{amsmath}
	\usepackage{graphicx}
	\usepackage{libertine}
	\usepackage[bold-style=TeX]{unicode-math}
	\usepackage{lipsum}

	\usepackage{natbib}
	\setcitestyle{square}

	\usepackage{polyglossia}
	\setdefaultlanguage{spanish}
	



	\usepackage{CBFT.estilo} % Cargo la hoja de estilo

	% Tipografías
	% \setromanfont[Mapping=tex-text]{Linux Libertine O}
	% \setsansfont[Mapping=tex-text]{DejaVu Sans}
	% \setmonofont[Mapping=tex-text]{DejaVu Sans Mono}

	%===================================================================
	%	DOCUMENTO PROPIAMENTE DICHO
	%===================================================================

\begin{document}

% =================================================================================================
\chapter{Gas de Bose}
% =================================================================================================

Para Bose debe cumplirse $ \mu < \text{ todo } e $ 
y como $ e \geq 0$ eso dice que 
\[
	\mu < 0
\]

Pero si en un sistema tiene $ e_0 $ como mínimo y $ e_0 > 0 $ entonces, ¿puede ser $ \mu > 0 $?
Aparentemente sí (al menos recordando que la restricción sale de la serie).
\notamargen{Ya lo entendí esto: pero no para partícula libre.}

Además $ \braket{n_e} \geq 0 $, el número de partículas debe ser positivo.
\[
	\beta p V = \log (\Xi) = \sum_e - \log ( 1 - \euler^{-\beta(e-\mu)})
\]
\[
	\beta p = \sum_{e \neq 0} \frac{- \log ( 1 - \euler^{-\beta(e-\mu)}) }{V} - \frac{\log (1-z)}{V}
\]

El último término será negligible para todo $z$, incluso con $z\to 1$ pues en ese caso $V \to \infty$ mucho
más rápido
\[
	\braket{n_0} = \frac{1}{z^{-1}-1} = \frac{z}{1-z}
\]
y $ \braket{n_0} / V $ es finito incluso con $z\to 1$, entonces
\[
	\braket{n_0} - z \braket{n_0} - z = 0 \qquad z = \frac{\braket{n_0}}{1+\braket{n_0}}
\]
\[
	1-z = \frac{1}{1+\braket{n_0}}
\]
\[
	- \frac{\log (1-z)}{V} = \frac{\log (1+\braket{n_0})}{V}
\]
 
y dado que $ \log (\braket{n_0}) \ll \braket{n_0} $ despreciamos $ \log (1-z) / V $.

Como $ 0 > \mu $ entonces $ \euler^{\beta \mu} \equiv z < 1 $

En Bose la fugacidad está acotada
\[
	\frac{N}{V} = \frac{1}{\lambda^3} g_{3/2}(z) + \frac{1}{V}\Frac{z}{1-z}
\]
\[
	\frac{\lambda^3}{v} =  g_{3/2}(z) + \frac{\lambda^3}{V} n_0
\]
\[
	\underbrace{\frac{N}{V}}_{\text{ densidad total }} =
	\underbrace{\frac{1}{\lambda^3} g_{3/2}(z)}_{\text{ densidad en los excitados }} +
	\underbrace{\frac{1}{V}\Frac{z}{1-z}}_{\text{ densidad en el fundamental }}
\]

Por otro lado como $ 0 < z < 1 $ entonces $ g_{3/2}(z) $ está acotada 
\[
	g_{3/2}(1) = \sum_{j=1}^\infty \frac{1}{j^{3/2}} = 2.612
\]

Con $z\approx 1$ da
\[
	\frac{\lambda^3}{v} = g_{3/2}(1) + \lambda^3 \frac{n_0}{V} 
\]
cuando se aumenta $N$ necesariamente las partículas se apilan en el fundamental; es una
fracción macroscópica pués $ V \to \infty $ y entonces $ n_0 \to \infty $.

Se da con 
\[
	\frac{\lambda^3}{v} = \frac{\lambda^3}{V} N = \frac{h^3}{(2\pi m kT)^{3/2}} \frac{N}{V} > 2.612
\]
\notamargen{Destaco en esta expresión $T$ baja dividiendo y $n$ alta multiplicando.}

El condensado de Bose surge cuando se saturan los excitados; ello pasa con $T$ baja, $N/V$
alta y $ \mu \to 0$

GRAFIQUETE

% =================================================================================================
\section{Cuánticos IV --reubicar--}
% =================================================================================================

algunos temitas sueltos:

números de ocupación

gas de Fermi $p$ y $c_v$

gas de Fermi $p$ y $c_v$

Condensado de Bose

\notamargen{¿El condensado BE requiere población de los niveles o $V$ total de algún tipo?}

El coeficiente lineal del virial $ 1/ 2^{5/2} = 0.1767767 $ sale considerando las $ f_{\nu}(z) $ hasta orden
uno y tirando términos más allá.

\notamargen{Tenía unas consultas agarradas con clip: ¿porqué hay una cúspide en $C_v$? ¿transiciones?}

El requerimiento $ \mu < 0 $ viene de que el fundamental $ n_0 $ no puede tener población negativa
\[
	n_0 = \frac{1}{\euler^{\beta(e_0 - \mu)} -1} = \frac{1}{\euler^{-\beta\mu} -1} \geq 0
\]
\[
	\euler^{-\beta\mu} -1 > 0 \qquad \Rightarrow \quad \mu < 0
\]
Con $\mu \to 0^-$ tenemos $ n \to \infty $

En el caso del condensado establecemos desde 
\[
	\frac{\lambda^3(T)}{v} = g_{3/2}(1) 
\]
que lleva para $T_c$ (para $v$ fijo) o $v_c$ (para $T$ fija) versiones evaluadas de la anterior ecuación.

Para la población de los estados excitados
\[
	p_x = \frac{h}{V^{1/3}}n_x \Rightarrow  \vb{p} = \frac{h}{V^{1/3}} \vb{n}
\]
\[
	\frac{n_{e_i}}{V} = \frac{1}{V} \frac{1}{ z^{-1}\euler^{\beta e_i} - 1 } \leq 
	\frac{1}{V(\euler^{\beta e_i} - 1)} = \frac{1}{V(\sum_{l=1}^\infty (\beta e_i )^l/l!)}
\]
pués $z^{-1} = 1/z \leq 1$
\[
	\beta e = \frac{\beta p^2}{2m} = \frac{\beta}{2m} \frac{h^2}{V^{2/3}} ( n_x^2 + n_y^2 + n_z^2)
\]
\[
	\frac{2m}{V^{1/3} \beta h^2 (\sum_{l=1} ... )} \to 0 \quad \text{ si } \quad V \to \infty
\]
y entonces
\[
	\frac{n_e}{V} \to 0 \quad \text{ si } \quad V \to \infty
\]

Esto significa que si $V$ es muy grande, en el condensado se tenderá a que todas las partículas se hallen en
$ e = 0 $ pues 
\[
	\frac{N_e}{N} \to 0 \qquad \qquad \frac{N_0}{N} \to 1
\]

Véamoslo en la ecuación de $N$,
\[
	\frac{\lambda^3 N}{V} = g_{3/2}(1) + \frac{\lambda^3}{V} \frac{z}{1-z}
\]
y si $z \to 1$ de forma que $z/(1-z) \gg 1$ entonces $g_{3/2}(1)$ es despreciable de modo que
\[
	\frac{\lambda^3 N}{V} \approx \frac{\lambda^3}{V} \frac{z}{1-z} = \frac{\lambda^3 N_0}{V} 
\]
y se da que $ N \sim N_0 $.

En Bose se da $ 0 < z < 1$

DIBUJITOS

Con $ z \ll 1$ es $ \lambda^3 / v \approx z $ y entonces $ z \approx 1/ (v/\lambda^3) $.
Con $ z=1 $ es $ \lambda^3 / v = 2.612$n pero si $ \lambda^3 / v > 2.612 $ entonces $z$ no se mueve y
sigue en su valor 1.


\subsection{Cuánticos 5 - Cuánticos 5b --reubicar--}

presión gas de Bose

$C_V$ gas de Bose

Condensado de Bose $\to$ transición de fase de primer orden

límite clásico función de partición

cálculo de $ Tr (\euler^{-\beta A} ) = Q_N(V,T) $

diferencia con el caso clásico

potencial efectivo

\notamargen{Ver la transición de fase con el tema del calor latente. ¿Cómo era lo de Clayperon?}


% \bibliographystyle{CBFT-apa-good}	% (uses file "apa-good.bst")
% \bibliography{CBFT.Referencias} % La base de datos bibliográfica

\end{document}
