	\documentclass[10pt,oneside]{CBFT_book}
	% Algunos paquetes
	\usepackage{amssymb}
	\usepackage{amsmath}
	\usepackage{graphicx}
% 	\usepackage{libertine}
% 	\usepackage[bold-style=TeX]{unicode-math}
	\usepackage{lipsum}

	\usepackage{natbib}
	\setcitestyle{square}

	\usepackage{polyglossia}
	\setdefaultlanguage{spanish}
	



	\usepackage{CBFT.estilo} % Cargo la hoja de estilo

	% Tipografías
	% \setromanfont[Mapping=tex-text]{Linux Libertine O}
	% \setsansfont[Mapping=tex-text]{DejaVu Sans}
	% \setmonofont[Mapping=tex-text]{DejaVu Sans Mono}

	%===================================================================
	%	DOCUMENTO PROPIAMENTE DICHO
	%===================================================================

\begin{document}

% =================================================================================================
\chapter{Gas de Bose}
% =================================================================================================

Para Bose debe cumplirse $ \mu < \text{ todo } e $ 
y como $ e \geq 0$ eso dice que 
\[
	\mu < 0
\]

Pero si en un sistema tiene $ e_0 $ como mínimo y $ e_0 > 0 $ entonces, ¿puede ser $ \mu > 0 $?
Aparentemente sí (al menos recordando que la restricción sale de la serie).
\notamargen{Ya lo entendí esto: pero no para partícula libre.}

\[
	N = \sum_e \vm{n_e} = \sum_e \frac{ 1 }{ z^{-1}\euler^{\beta e} - 1 }
\]
Además $ \braket{n_e} \geq 0 $, el número de partículas debe ser positivo, lo que lleva a 
$| z\euler^{-\beta e} | < 1 $ para todo $e$ de manera que con $e \geq 0$ se tiene $0<z<1$
(esto depende de que los niveles de energía sean mayores a cero).

\[
	\beta p V = \log (\Xi) = \sum_e - \log ( 1 - \euler^{-\beta(e-\mu)})
\]
\[
	\beta p = \sum_{e \neq 0} \frac{- \log ( 1 - \euler^{-\beta(e-\mu)}) }{V} - \frac{\log (1-z)}{V}
\]

La densidad es 
\[
	\frac{N}{V} = \frac{1}{\lambda^3} g_{3/2}(z)
\]
donde la expresión general de las $g_\nu$ es
\[
	g_\nu(z) = \frac{1}{\Gamma(\nu)} \int_0^\infty \frac{ x^{\nu-1} }{z^{-1} \euler^x - 1 } dx 
\]
El paso al continuo para $N,V \to \infty$ con $N/V$ constante y la relación $e=p^2/(2m)$ resulta en
\[
	g(e) = \Frac{2 \pi V}{ h^3 } (2N)^{3/2} e^{1/2}
\]
y con ello las dos ecuaciones continuas para el gran canónico resultan en
\[
	\frac{pV}{kT} = -\frac{2\pi V}{h^3} (2m)^{3/2} \int_0^\infty e^{1/2} \log( 1 - z\euler^{-\beta e}  ) de
\]
\[
	N = \frac{2\pi}{h^3} (2m)^{3/2} \int_0^\infty \frac{ e^{1/2} }{ z^{-1}\euler^{\beta e} - 1 } \: de
\]
\notamargen{Recordemos que la longitud de onda térmica es $\lambda=\lambda(T)$.}

Al final la energía 
\[
	U = - \dpar{}{\beta} \log Q = \frac{3}{2} k T \frac{V}{\lambda^3} g_{5/2}(z)
\]
y se ve que $pV = 2/3 U$ que usó la dispersión no relativista y el factor 3 es por la dimensión.

Para la ecuación de estado hay que expresar $z = z(N)$ e introducirlo en $p/(kT)$.

El último término será negligible para todo $z$, incluso con $z\to 1$ pues en ese caso $V \to \infty$ mucho
más rápido
\[
	\braket{n_0} = \frac{1}{z^{-1}-1} = \frac{z}{1-z}
\]
y $ \braket{n_0} / V $ es finito incluso con $z\to 1$, entonces
\[
	\braket{n_0} - z \braket{n_0} - z = 0 \qquad z = \frac{\braket{n_0}}{1+\braket{n_0}}
\]
\[
	1-z = \frac{1}{1+\braket{n_0}}
\]
\[
	- \frac{\log (1-z)}{V} = \frac{\log (1+\braket{n_0})}{V}
\]
 
y dado que $ \log (\braket{n_0}) \ll \braket{n_0} $ despreciamos $ \log (1-z) / V $.

Como $ 0 > \mu $ entonces $ \euler^{\beta \mu} \equiv z < 1 $

En Bose la fugacidad está acotada
\[
	\frac{N}{V} = \frac{1}{\lambda^3} g_{3/2}(z) + \frac{1}{V}\Frac{z}{1-z}
\]
y entonces el nivel de ocupación del fundamental debo sumarlo aparte; vemos que en el
segundo término con $z\to 1$ revienta.
El primer término es la densidad de partículas en los niveles excitados
\[
	\frac{\lambda^3}{v} =  g_{3/2}(z) + \frac{\lambda^3}{V} n_0
\]
\[
	\underbrace{\frac{N}{V}}_{\text{ densidad total }} =
	\underbrace{\frac{1}{\lambda^3} g_{3/2}(z)}_{\text{ densidad en los excitados }} +
	\underbrace{\frac{1}{V}\Frac{z}{1-z}}_{\text{ densidad en el fundamental }}
\]

\begin{ejemplo}{Comentario raro}

En relación a lo del condensado anoté que: ``aparentemente habría un error es esta ecuación''
(con respecto a la ecuación de $\lambda^3/v$) pués
\[
	 \sum_e \longrightarrow \int_0^\infty \: g(e) \: de
\]
si he pesado el nivel energético  cero con el cero y la borré de la integral. Veamos que $e^{1/2}$
y $g(0)=0$.
 
\end{ejemplo}

Si $z$ está lejos de 1 el nivel fundamental no está muy poblado y las partículas se distribuyen
en los otros excitados.

Por otro lado como $ 0 < z < 1 $ entonces $ g_{3/2}(z) $ está acotada 
\[
	g_{3/2}(1) = \sum_{j=1}^\infty \frac{1}{j^{3/2}} = 2.612
\]
\notamargen{En la carpeta hablo de un $N_max$ dado por $ V (2\pi m k T)^{3/2}/h^3 g_{3/2}(1)$ y
se da $N_e \geq N_max(T)$, que imagino que implica lo del acotamiento en $g$ y causa que no
``entren más'' en los excitados (supongo).}

Con $z\approx 1$ da
\[
	\frac{\lambda^3}{v} = g_{3/2}(1) + \lambda^3 \frac{n_0}{V} 
\]
cuando se aumenta $N$ necesariamente las partículas se apilan en el fundamental; es una
fracción macroscópica pués $ V \to \infty $ y entonces $ n_0 \to \infty $.

Se da con 
\[
	\frac{\lambda^3}{v} = \frac{\lambda^3}{V} N = \frac{h^3}{(2\pi m kT)^{3/2}} \frac{N}{V} > 2.612
\]
\notamargen{Destaco en esta expresión $T$ baja dividiendo y $n$ alta multiplicando.}

El condensado de Bose surge cuando se saturan los excitados; ello pasa con $T$ baja, $N/V$
alta y $ \mu \to 0$. Se tiene en estos casos que $N_0$ es comparable a $N$.
El valor 2.612 define un punto crítico en el cual empieza a diverger la población del fundamental.
Tiene niveles macroscópicos, se puede comparar con $N$.


GRAFIQUETE


El condensado de Bose podemos pensarlo como la coexistencia de dos fluidos ($e=0$ y $e\neq 0$).
Podemos definir un $ T_c, v_c $ desde 
\[
	\frac{\lambda^3}{v} = g_{3/2}(1) = 2.612 = \frac{h^3}{(2 \pi m k T)^{3/2}} \frac{1}{v}
\]
que lleva a que para un dado $v$ tenemos una cierta $T_c$ y para una cierta $T$ tenemos un 
dado $v_c$ dados ambos por 
\[
	T_c^{3/2} = \frac{h^3}{(2 \pi m k T)^{3/2}} \frac{1}{v} \frac{1}{g_{3/2}(1)} \qquad 
	v_c = \frac{\lambda^3(T)}{g_{3/2}(1)}
\]
De esta forma si $ T<T_c$ y $v<v_c$ se tiene la condensación de Bose
\[
	\lambda^3\frac{N}{V} = g_{3/2}(1) + \lambda^3\frac{N_0}{V}
\]
que es válida a partir de la condensación ($T<T_c$)
\[
	N =  \frac{(2 \pi m k )^{3/2}}{h^3} T^{3/2} g_{3/2}(1)  V + N_0 = N \Frac{T}{T_c}^{3/2} + N_0
\]
\notamargen{$N_e = N\Frac{T}{T_c}^{3/2}$}
\[
	N_o = N \left( 1 - \Frac{T}{T_c}^{3/2} \right),
\]
que es válida por supuesto con $T<T_c$.
A partir de haber alcanzado la condensación $z=1$, añadir partículas ($N++$) o reducir el volumen 
($V--$) hace que $N_e/V \to 0 $ pues $V \to \infty$

DIBUJO con observaciones

\includegraphics[scale=0.5]{images/1606329632.jpg}

Esto de arriba corresponde a $T < T_C $ para el caso $\lambda^3/v > 2.612 $ y se ve que 
$z = N_0/(N_0+1) \to 1$ si $N_0\to\infty$.

Cuando $v/\lambda^3$ es chico se saturan los $N_e$ y entonces $ z \to 1 $.

Cuando $v/\lambda^3$ es grande no hay condensado y entonces $ \lambda^3/v \approx z $ o bien
$ 1/ (v/\lambda^3) \approx z $.

Para la presión tendremos
\[
	\beta p = \frac{1}{\lambda^3} g_{5/2}(z)
\]
Otra observación es que $P_{BE}(T=0)=0$. Las partículas $N_e$ en la fase normal hacen presión que es la
mitad de la del gas ideal mientras que las de $N_0$ no hacen presión en absoluto.

\includegraphics[scale=0.4]{images/1606329638.jpg}

con $ z = 1 ( T < T_c ) $
\[
	\frac{p}{kT} = \frac{(2\pi m k T)^{3/2}}{h^3} g_{5/2}(1) = 
	\frac{1}{ v (T_c/T)^{3/2} g_{3/2}(1) } g_{5/2}(1)
\]
\notamargen{Tenía anotado por allí $p \propto T^{5/2}$.}
\[
	p = 1.34 \frac{(2\pi m )^{3/2}}{h^3} (kT)^{5/2} \qquad \qquad 
	\frac{pV}{NkT} = 0.513 \Frac{T}{T_c}^{3/2}
\]
con $ z = 1 ( T = T_c ) $
\[
	\beta p = \frac{ g_{5/2}(1) }{ g_{3/2}(1) v } = \frac{0.513}{v}
\]
\[
	p = 0.513 \frac{NkT}{V} \qquad \text{ es aprox. $1/2 p$ gas ideal clásico }
\]
con $ z \lesssim 1 ( T > T_c ) $
\[
	\beta p = \frac{1}{v} \frac{ g_{5/2}(z) }{ g_{3/2}(z) }
\]
pero no podemos expandir en el virial porque $ \lambda^3 / v $ no es chico.

Con $ z \approx 0 \: ( T \gg T_C ) $
\[
	\beta p v = \frac{pV}{NkT} = \sum_{l=0}^\infty a_l \Frac{\lambda^3}{v}^{l-1}
\]
usando toda la serie y procediendo en modo análogo a Fermi se obtienen
\notamargen{ Los $a_\ell$ son los coeficientes del virial -que son los mismos
para Fermi-.}
\[
	\begin{cases}
	 a_1 = 1 \\
	 a_2 = -0.17678 \\
	 a_3 = -0.00330
	\end{cases}
\]
\[
	\frac{pV}{NkT} = 1 - 0.17678 \Frac{\lambda^3}{v} - 0.00330 \Frac{\lambda^3}{v}^2
\]

DIBUJO 

El virial vale en $\lambda^3/v \ll 1$ (alta $T$ y baja $N/V$ )

A bajas $T$ se comportan de modo muy diferente, $p_{\text{ Fermi }} > 0 $ y
$p_{\text{ Bose }} \approx 0$

\begin{ejemplo}{\bf Problema 2 -tip-}
Es $e_{p,n_i} = p^2/ 2m + n_i \varepsilon_1 $ donde $n_i=0,1$ y entonces es un grado de libertad interno.
 
\end{ejemplo}


% =================================================================================================
\subsection{Análisis del gas ideal de Bose}
% =================================================================================================

\begin{itemize}
 \item $ \lambda^3 / v \ll 1 $ y entonces $ z \ll 1 $ $\quad  [T \gg T_c ] $ (o sea $T$ alta y $N/V$ baja )
 tenemos un desarrollo del virial porque $z \ll 1$
 \[
	\frac{\beta p V}{N} = \sum_{l=1}^\infty a_l \Frac{\lambda^3}{v}^{l-1} = \frac{ g_{5/2}(z) }{ g_{3/2}(z) }
 \]
 \[
	\beta p V \approx 1 - \frac{\lambda^3}{v} \frac{1}{2^{5/2}} \qquad \qquad 
	U = \frac{3}{2}pV = \frac{3}{2} NkT \left( 1 - \frac{\lambda^3}{v} \frac{1}{2^{5/2}} \right)
 \]
 Como $a_2 < 0$ se tiene que la presión para Bose Einstein es menor a la presión clásica. Siguiendo
 podemos trabajar una expresión para el calor específico
 \[
	\frac{C_V}{kT} = \frac{3}{2} \left( 1 + 0.0884 \Frac{\lambda^3}{v} + 
	0.0066 \Frac{\lambda^3}{v}^2 + ... \right)
 \]
 
 \item $ \lambda^3 / v \approx 1 $ y entonces $ z < 1 $ $\quad  [T > T_c ] $
 \[
	\beta p V = \frac{ g_{5/2}(z) }{ g_{3/2}(z) }
 \]
 \item  $ \lambda^3 / v = 2.612 $ y entonces $ z = 1 $ $\quad  [T = T_c ] $
 \[
	\beta p V = \frac{ g_{5/2}(z) }{ g_{3/2}(z) } \approx \frac{1.34}{2.612} \approx 0.513
 \]
 \item $ \lambda^3 / v \gg 1 $ y entonces $ z = 1 $ $\quad  [T < T_c ] $ (baja temperatura $T$ y alta
 densidad $N/V$) y hay que considerar el  fundamental
 \[
	\beta p = \frac{1}{\lambda^3} g_{5/2}(1) \qquad \qquad \lambda^3\Frac{N-N_0}{V} = g_{3/2}(1)
 \]
 \notamargen{Con $z=1$ y $T<T_c$ expresamos todo en términos de $(T/T_c)$.}
 que lleva a 
 \[
	\left( 1 - \frac{N_0}{N} \right) = \Frac{T}{T_c}^{3/2}
 \]
 puesto que $T_c$ es tal que 
 \[
	\frac{h^3}{(2\pi m kT_c)^{3/2}} \frac{N}{V} = g_{3/2}(1) = \frac{\lambda^3}{v}\Frac{T}{T_c}^{3/2}
 \]
 \[
	\beta p V = \frac{ g_{5/2}(z) }{ g_{3/2}(z) } \Frac{T}{T_c}^{3/2} = 0.513 \Frac{T}{T_c}^{3/2}
 \]
 \[
	\frac{\lambda^3}{v} \Frac{T}{T_c}^{3/2} = g_{3/2}(1) \quad \Rightarrow \quad \frac{1}{\lambda^3} =
	\frac{1}{v}\Frac{T}{T_c}^{3/2} \frac{1}{g_{3/2}(1)}
 \]
\end{itemize}

Con el aumento de la temperatura aumentan ambos miembros en la ecuación 
\[
	\frac{\lambda^3}{v} = g_{3/2}(z)
\]
pero como $g_{3/2}$ está acotada esto lleva a la condensación de Bose.

Desde la expresión de la energía $ U = 3/2 p V $ y $C_V = \dpar{}{T}(3/2 p V)$
y entonces
\begin{itemize}
 \item $ T < T_c $ 
 \[
	C_V = \dpar{}{T}\left( \frac{3}{2} N k \Frac{T}{T_c}^{3/2} 0.513  \right) = 
	\frac{15}{4} N k \Frac{T}{T_c}^{3/2} 0.513 \qquad C_V \propto T^{3/2}
 \]
 \item $ T = T_c $ 
 \[
	C_V = N k \; 0.513 \frac{15}{4} = N k 1.92375
 \]
 \item $ T > T_c $ 
 \[
	C_V = \left( \frac{15}{4}\frac{ g_{5/2}(z) }{ g_{3/2}(z) } - 
	\frac{9}{4} \underbrace{\frac{ g_{3/2}(z) }{ g_{1/2}(z) }}_{\to \infty \text{ en } z=1} \right)
 \]
 $C_V$ es continuo.
 \item $ T \gg T_c $ 
 \[
	C_V = N k \frac{3}{2} \dpar{}{T} \left( T \sum_{l=1}^\infty a_l \Frac{\lambda^3}{v}^{l-1} \right)
 \]
 \[
	C_V = N k \frac{3}{2} \left( 1 + 0.0884 \Frac{\lambda^3}{v} + ... \right)
 \]
\end{itemize}

DIBUJO

Entonces tenemos dos fases macroscópicas, que dependen de la temperatura
\begin{itemize}
 \item Fase normal: consisten en las partículas de los niveles excitados,
 \[
	N_e = N \Frac{T}{T_C}^{3/2}
 \]
 \item Fase condensada
 \[
	\frac{N_0}{N} = 1 - \Frac{T}{T_C}^{3/2}
 \]
\end{itemize}

\includegraphics[scale=0.5]{images/1606329628.jpg}

Y vemos que $N_0 \to \infty$ en forma comparable a como $N \to \infty$. El $N_0$ macroscópicamente poblado
es la condensación.

\subsection{Condensado de Bose como transición de fase}

\[
	\frac{N_0}{N} = 1 - \Frac{T}{T_c}^{3/2}
\]
\[
	\frac{N_0}{N} = 1 - \frac{v}{v_c}
\]
que se obtiene desde las siguientes
\[
	\frac{\lambda^3(T_c)}{v} = g_{3/2}(1) \qquad \qquad  \frac{\lambda^3(T)}{v_c} = g_{3/2}(1)
\]
para llegar a la relación útil:
\[
	\Frac{T}{T_c}^{3/2} = \frac{v}{v_c}
\]
\notamargen{ $\lambda^3 = h^3/(2\pi m k T)^{3/2}$ y $\frac{\lambda^3}{v_c} = g_{3/2}(1) = \frac{\lambda^3}{v} 
\frac{v}{v_c}  $}

En $ \frac{\lambda^3}{v} \leq g_{3/2}(1) $ vale 
\[
	\frac{\lambda^3}{v} = g_{3/2}(z) \text{ no tengo en cuenta $N_0$ }
\]
\[
	\frac{v_c}{v} = \frac{ g_{3/2}(z) }{ g_{3/2}(1) } \quad \Rightarrow \quad 
	\Frac{T}{T_c}^{3/2} = \frac{ g_{3/2}(z) }{ g_{3/2}(1) } 
\]

Se vio que con $ V \to \infty $
\[
	\frac{1}{V} \log (1-z) \to 0 
\]
y entonces 
\[
	\beta p = \frac{1}{\lambda^3} g_{5/2}(z) \qquad v > v_c
\]
\[
	\beta p = \frac{1}{\lambda^3} g_{5/2}(1) \qquad v \leq v_c
\]
\[
	\beta p = \frac{ g_{5/2}(1) }{ v_c g_{3/2}(1) }
\]
es decir que la presión $p$ no depende del $v$

Con $v > v_c$ 
\[
	p = \frac{ k T g_{5/2}(z) }{\lambda^3} = \Frac{h^2}{ 2\pi m } \frac{1}{\lambda^3} g_{5/2}(z)
\]
que conlleva a 
\[
	kT = \Frac{h^2}{ 2\pi m } \frac{1}{\lambda^2} \qquad 
	p =  \Frac{h^2}{ 2\pi m } \frac{ g_{5/2}(z) }{ v^{5/3} [ g_{3/2}(z) ]^{5/3} }
\]
y con $v > v_c$ 
\[
	pv^{5/3} = \Frac{h^2}{ 2\pi m } \frac{ g_{5/2}(z) }{ [ g_{3/2}(z) ]^{5/3} }
\]
con $v \leq v_c$ 
\[
	p = \frac{ k T }{ v_c } \frac{ g_{5/2}(1) }{ g_{3/2}(1) }
\]

Vemos que en $ v = v_c $ es
\[
	pv^{5/3} = \Frac{h^2}{ 2\pi m } \frac{ g_{5/2}(1) }{ [ g_{3/2}(1) ]^{5/3} }
\]
\[
	p = \Frac{h^2}{ 2\pi m } \frac{ g_{5/2}(1) }{ v_c g_{3/2}(1) }\frac{1}{\lambda^2} =
	\frac{kT}{v_c}\frac{ g_{5/2}(1) }{ g_{3/2}(1) }
\]
y entonces se ve que es continua.

\begin{center}
\begin{tabular}{c|c}
 $\displaystyle \beta p = \frac{1}{\lambda^3} g_{5/2}(z) \quad v \geq v_c \quad $ & 
 $\displaystyle \quad \beta p = \frac{1}{\lambda^3} g_{5/2}(1) \quad v \leq v_c \quad $\\
 & \\
 $\displaystyle \frac{\lambda^3}{v} = g_{3/2}(z) \quad v > v_c \quad $ 
 & $\displaystyle \quad \frac{\lambda^3}{v} = g_{3/2}(1) \quad v = v_c \quad $
\end{tabular}
\end{center}

\begin{itemize}
 \item $ v \geq v_c $
 \[
	p = \frac{ kT }{v_c} g_{5/2}(z) = \frac{ ( 2 \pi m )^{3/2} }{ h^3 } ( kT )^{5/2} g_{5/2}(z)
 \]
 \[
	p = \Frac{h^2}{2\pi m}\frac{1}{\lambda^5}g_{5/2}(z) = 
	\Frac{h^2}{2\pi m} \frac{g_{5/2}(z)}{v_c^{5/3} [g_{3/2}(z)]^{5/3}}
 \]
 \[
	\boxed{ p v^{5/3} = \Frac{h^2}{2\pi m} \frac{g_{5/2}(z)}{ g_{3/2}(z)^{5/3}} }
 \]
 \item $ v \leq v_c$
 \[
	p = \frac{ kT }{v_c} g_{5/2}(1) = \boxed{ \Frac{kT}{v_c} \frac{g_{5/2}(1)}{g_{3/2}(1)} }
 \]
 
 Las isotermas del gas ideal de Bose serán algo como 
 
 DIBUJO
 
 Una dada $ T_1 $ determina un $v_{c_1}$ pués
 \[
	\frac{\lambda^3(T_1)}{v_{C_1}} = g_{3/2}(1) \quad \to \quad v_{C_1} = \frac{\lambda^3(T_1)}{g_{3/2}(1)} 
 \]
 y en la zona condensada $p$ no depende del $v$.
 
 \notamargen{ $ \lambda^3(T) \propto T^{-3/2} $ A medida que $T$ sube el $v_c$ es más pequeño.}
 
 Si ponemos todo en función de $T$ resulta 
 \[
	v \leq v_c \qquad p = \frac{ ( 2 \pi m )^{3/2} }{ h^3 } ( kT )^{5/2} g_{5/2}(1)
 \]
 \[
	\dtot{p}{T} =   \frac{5}{2} \frac{ ( 2 \pi m )^{3/2} }{ h^3 }  ( k )^{5/2} T^{3/2} g_{5/2}(1) =
	\frac{5}{2} \frac{k}{\lambda^3} g_{5/2}(1) = \frac{5}{2} \frac{k}{v_c} \frac{g_{5/2}(1)}{g_{3/2}(1)}
 \]
  
 DIBUJO 
 \[
	\dtot{ p }{ T } = \frac{ ( 5/2 ) k T g_{5/2}(1) }{T v_c g_{3/2}(1) } 
 \]
 pero Clapeyron era
 \[
	\dtot{ p }{ T } = \frac{ L }{ T \Delta V } \qquad \Rightarrow \qquad 
	\boxed{  \dtot{ p }{ T } = \frac{ ( 5/2 ) k T g_{5/2}(1) / g_{3/2}(1) }{T v_c} }
 \]
 
Es una transición de fase de primer orden 
 \notamargen{ $ \dtot{ p }{ T } = \frac{ L }{ T \Delta V } = \frac{ T \Delta S }{ T \Delta V } = 
 \frac{ \Delta S }{ \Delta V } $}
 \[
	S = \frac{U + pV - \mu N}{T} = \frac{ 5/2 pV - \mu N }{T}
 \]
 \[
	\frac{S}{kN} = \frac{5}{2} \frac{pV}{NkT} - \frac{\mu}{kT}
 \]
y entonces 
 \[
	T > T_c \qquad \qquad \frac{S}{kN} = \frac{5}{2} \frac{ g_{5/2}(z) }{ g_{3/2}(z) } - \log z
 \]
 \[
	T < T_c \qquad \qquad \frac{S}{kN} = \frac{5}{2} 0.513 \Frac{T}{T_c}^{3/2}
 \]
 \notamargen{$ \Frac{T_c}{T}^{3/2} = \frac{ \lambda^3 }{ g_{3/2}(1) v } $}
 \[
	\text{Con } T \to 0 \qquad \qquad \frac{S}{kN} \propto T^{3/2} 
 \]
y por lo tanto vale la tercer de la termodinámica.
Para $ T < T_c $ es
\[
	S = N k \frac{5}{2} \frac{ g_{5/2}(1) }{ g_{3/2}(1) } \Frac{v}{v_c} \quad \to \quad 
	\dpar{S}{V} = \frac{\partial S/N}{\partial V/N} = \dpar{s}{v}
\]
siendo $s$ entropía por unidad y $v$ volumen específico.
\[
	\dpar{s}{v} =  \frac{ ( 5/2 ) k g_{5/2}(1) / g_{3/2}(1) }{v_c} = \dtot{p}{T}
\]
y acá es donde vemos que es una transición de fase de primer orden.
\end{itemize}



% =================================================================================================
\section{Cuánticos IV --reubicar--}
% =================================================================================================

algunos temitas sueltos:

números de ocupación

gas de Fermi $p$ y $c_v$

gas de Fermi $p$ y $c_v$

Condensado de Bose

\notamargen{¿El condensado BE requiere población de los niveles o $V$ total de algún tipo?}

El coeficiente lineal del virial $ 1/ 2^{5/2} = 0.1767767 $ sale considerando las $ f_{\nu}(z) $ hasta orden
uno y tirando términos más allá.

\notamargen{Tenía unas consultas agarradas con clip: ¿porqué hay una cúspide en $C_v$? ¿transiciones?}

El requerimiento $ \mu < 0 $ viene de que el fundamental $ n_0 $ no puede tener población negativa
\[
	n_0 = \frac{1}{\euler^{\beta(e_0 - \mu)} -1} = \frac{1}{\euler^{-\beta\mu} -1} \geq 0
\]
\[
	\euler^{-\beta\mu} -1 > 0 \qquad \Rightarrow \quad \mu < 0
\]
Con $\mu \to 0^-$ tenemos $ n \to \infty $

En el caso del condensado establecemos desde 
\[
	\frac{\lambda^3(T)}{v} = g_{3/2}(1) 
\]
que lleva para $T_c$ (para $v$ fijo) o $v_c$ (para $T$ fija) versiones evaluadas de la anterior ecuación.

Para la población de los estados excitados
\[
	p_x = \frac{h}{V^{1/3}}n_x \Rightarrow  \vb{p} = \frac{h}{V^{1/3}} \vb{n}
\]
\[
	\frac{n_{e_i}}{V} = \frac{1}{V} \frac{1}{ z^{-1}\euler^{\beta e_i} - 1 } \leq 
	\frac{1}{V(\euler^{\beta e_i} - 1)} = \frac{1}{V(\sum_{l=1}^\infty (\beta e_i )^l/l!)}
\]
pués $z^{-1} = 1/z \leq 1$
\[
	\beta e = \frac{\beta p^2}{2m} = \frac{\beta}{2m} \frac{h^2}{V^{2/3}} ( n_x^2 + n_y^2 + n_z^2)
\]
\[
	\frac{2m}{V^{1/3} \beta h^2 (\sum_{l=1} ... )} \to 0 \quad \text{ si } \quad V \to \infty
\]
y entonces
\[
	\frac{n_e}{V} \to 0 \quad \text{ si } \quad V \to \infty
\]

Esto significa que si $V$ es muy grande, en el condensado se tenderá a que todas las partículas se hallen en
$ e = 0 $ pues 
\[
	\frac{N_e}{N} \to 0 \qquad \qquad \frac{N_0}{N} \to 1
\]

Véamoslo en la ecuación de $N$,
\[
	\frac{\lambda^3 N}{V} = g_{3/2}(1) + \frac{\lambda^3}{V} \frac{z}{1-z}
\]
y si $z \to 1$ de forma que $z/(1-z) \gg 1$ entonces $g_{3/2}(1)$ es despreciable de modo que
\[
	\frac{\lambda^3 N}{V} \approx \frac{\lambda^3}{V} \frac{z}{1-z} = \frac{\lambda^3 N_0}{V} 
\]
y se da que $ N \sim N_0 $.

En Bose se da $ 0 < z < 1$

DIBUJITOS

Con $ z \ll 1$ es $ \lambda^3 / v \approx z $ y entonces $ z \approx 1/ (v/\lambda^3) $.
Con $ z=1 $ es $ \lambda^3 / v = 2.612$n pero si $ \lambda^3 / v > 2.612 $ entonces $z$ no se mueve y
sigue en su valor 1.


\subsection{Cuánticos 5 - Cuánticos 5b --reubicar--}

presión gas de Bose

$C_V$ gas de Bose

El condensado de Bose es una transición de fase de primer orden.
Crece la población del fundamental de modo espectacular. El parámetro $ \lambda^3/V $ se encarga de
adjustar la población del fundamental.

límite clásico función de partición

cálculo de $ Tr (\euler^{-\beta A} ) = Q_N(V,T) $

diferencia con el caso clásico

potencial efectivo

\notamargen{Ver la transición de fase con el tema del calor latente. ¿Cómo era lo de Clayperon?}

Podemos comparar presión con el gas ideal para reconoder si es Fermi o Bose.

\includegraphics[scale=0.5]{images/1606329551.jpg}


El $C_V$ es continuo. Veamos que da
\[
	T<T_C \qquad \frac{C_V}{Nk} \propto T^{3/2}
\]
\[
	T=T_C \qquad \frac{C_V}{Nk} \approx 1.925 > \frac 3 2
\]
\[
	T>T_C \qquad \dpar{}{T}\left( \frac{3}{2} T \frac{g_{5/2}(z)}{g_{3/2}(z)} \right) \frac{C_V}{Nk} 
\]


\includegraphics[scale=0.5]{images/1606329555.jpg}

La flecha de abajo señala una región de coexistencia. Entonces el fundamental se empieza a poblar mucho.
Cuando tengo todos en el condensado es $ S \to 0, T \to 0$ y se ve que satisface la tercer ley.
Los boltzmanniones no cumplen esto (no están pensados para satisfacer la tercer ley).

Tiene calor latente $ \Delta H $, entonces tenemos una transición de fase de primer orden.

\subsection{Límite clásico de la función de partición}

Cuando se overlapean las funciones de onda en las partículas hay que realizar las perturbaciones 
correspondientes.
El límite clásico es la no permutación. La simetría hace surgir términos efectivos de interacción
(atractivos o repulsivos)

\includegraphics[scale=0.5]{images/1606329560.jpg}


\begin{ejemplo}{\bf Problema 5}

Se tiene lo siguiente:
\[
	\frac{m}{2}( \omega_x^2 x^2 + \omega_y^2 y^2 + \omega_z^2 z^2 )
\]
es decir un planteamiento semiclásico, de manera que considero un continuo.

\includegraphics[scale=0.3]{images/1606329642.jpg}

Primero se considera el caso 1D, entonces es
\[
	E = \frac{m}{2} \omega^2 x^2 + \frac{ p^2 }{2m}
\]
y hago la conversión al continuo según
\[
	\sum_\text{estados} \longrightarrow \frac{1}{h} \int dx dp \longrightarrow \int g(e) de
\]

Con el cambio de variables $ R^2 = X^2 + Y^2 $ se tienen
\[
	dx = \sqrt{2/(m\omega^2)} dX \qquad dp = \sqrt{2m} dY
\]
de modo que $ dX dY = 2 \pi R dR = \pi de $ y como 
\[
	\frac{1}{\hbar} dx dp = \frac{1}{\hbar\omega} de
\]
se ve que $ g(e) = 1 / (\hbar \omega )$. Hemos hallado un $g(e)$ constante, lo cual parece razonable 
porque es el espaciado entre niveles de energía para el oscilador armónico en el caso cuántico.
Entonces este enfoque semiclásico lleva al mismo resultado,
\[
	\Delta E = \hbar \omega \qquad H = \left( n + \frac{1}{2} \right) \hbar \omega \qquad 
	g(e) = \frac{\text{\# de estados}}{\text{unidad de energía}}
\]

Vayamos ahora al caso 3D
\[
	E = \frac{m \omega^2 }{ 2 }( x_1^2 + x_2^2 + x_3^2 ) + \frac{1}{2m} ( p_1^2 + p_2^2 + p_3^2 )
\]
y como $ R^2 = x_1^2 + x_2^2 + x_3^2 + y_1^2 + y_2^2 + y_3^2 $ que es el módulo al cuadrado de un
vector en $\mathbb{R}^6$. De tal suerte es
\[
	\frac{ d^3x d^3p }{ h^3 } = \frac{ 2^3 }{ h^3 \omega^3 } d^3X d^3Y = \frac{ R^5 dR }{ (\hbar \omega)^3}
\]
donde $d^3X d^3Y = \pi^3 R^5 dR $ y el volumen $\Theta_{6D} = \pi^3 R^6/6$ de tal manera que
\[
	g(e) = \frac{1}{2(\hbar\omega)^3} e^2 de
\]

Ahora, en 3D, $g(e)$ sí depende de la energía. El número de estados va de $g_{xp} d^3x d^3p$ a $g_e de$.
No interesa ver el límite termodinámico, $N\to\infty, V\to\infty$ con $N/V$ finito.
En el problema del oscilador armónico $ N/\omega^3 $ será el límite termodinámico.
Si $\hbar \omega \ll k T$ entonces lo puedo considerar un continuo. $kT_c \sim 20-200 \hbar\omega$, este
es el caso en condensación de Bose

\includegraphics[scale=0.4]{images/1606329650.jpg}

\[
	N = \sum_e \frac{ z \euler^{-\beta e} }{ 1 - z \euler^{-\beta e}} \longrightarrow
	\int_0^\infty \frac{ g(e) de }{ z^{-1}\euler^{\beta e} - 1} + \frac{z}{1-z} 
	+ \frac{ z \euler^{-\beta e_1} }{ 1 - z \euler^{-\beta e_1}} 
\]
donde separo la contribución de finitos términos lo cual no debería joder. La primer integral es,
mediante el cambio de variables $\beta e = x$
\[
	I = \frac{ 1 }{ 2 ( \hbar \omega )^3 } \int_0^\infty \frac{1}{\beta} \frac{ x^2 dx }{ z^{-1} \euler^x - 1}
\]
relacionados con la expresión de $g_\nu$ (que tendría que estar en un apéndice al final).
Importante remark: notemos que con $\nu> 1$ con $z\to$ converge pero con $\nu=1$ (lo cual tiene que
ver con la dispersión y la dimensión del problema) con $z\to 1$ diverge.
\[
	N = \frac{z}{1-z} + \Frac{kT}{\hbar\omega}^3 g_3(z) + 
	 \frac{ z \euler^{-\beta e_1} }{ 1 - z \euler^{-\beta e_1}} 
\]
\[
	N \omega^3 \Frac{ \hbar }{ kT }^3 = 
	\Frac{ \omega \hbar }{ kT }^3 \frac{ z }{ 1 - z } + g_z(z) + 
	\Frac{ \omega \hbar }{ kT }^3 \frac{ z \euler^{-\beta e_1} }{ 1 - z \euler^{-\beta e_1}} 
\]
donde $g_3(z)$ es creciente pero como $z\leq 1$ está acotada. El primer término permanece macroscópicamente
poblado en el límite termodinámico. El último término en cambio es aproximadamente nulo en dicho límite.

\includegraphics[scale=0.4]{images/1606329656.jpg}

Puedo despejar una $T_c$

\includegraphics[scale=0.4]{images/1606329660.jpg}

Para la $T_c$ sería
\[
	N \Frac{ \omega \hbar }{ kT } = g_3(z=1)
\]
de manera que 
\[
	N = N_0 + \Frac{ kT }{ \omega \hbar }g_3(z\neq 1)
\]
y para $T < T_c$ es 
\[
	N = N_0 + N \Frac{T}{T_c}^3 \qquad  \frac{N_0}{N} = 1 - \Frac{T}{T_c}^3
\]

\includegraphics[scale=0.4]{images/1606329664.jpg}

La energía será
\[
	E = \int_0^\infty \frac{ g(e) de }{ z^{-1} \euler^{\beta e} - 1} = 
	3 \frac{( k T )^4}{( \hbar \omega )^3} g_4(z)
\]
Con $N_0 \ll N $ es 
\[
	 N \sim \Frac{ kT_c }{ \omega \hbar }g_3(1)
\]
y finalmente
\[
	\frac{ E }{ N k T_c } = \frac{3 g_4(1)}{g_3(1)} \Frac{T}{T_c}^4.
\]

\end{ejemplo}



\begin{ejemplo}{\bf Problema 6}

Relacionado con excitaciones en un sólido.

\includegraphics[scale=0.4]{images/1606329667.jpg}

Entonces,
\[
	E_{\text{cin}} = \frac{1}{2} m \sum_{i=1}^{3N} \dot{x}^2_i
\]
\[
	E_{\text{pot}} = \phi_0 + 
	\frac{1}{2} \sum_{ij} \left. \dparcru{ \phi}{x_i}{x_j}\right|_{x_{i0},x_{j0}}
	(x_i-x_i^0)^2(x_j-x_j^0)^2 + ... 
\]
Pero podemos cambiar de coordenadas
\[
	H = \phi_0 + \sum_{i=1}^{3N} \frac{1}{2} ( m \dot{q}^2_i + \omega^2_i q_i^2 ),
\]
donde $\{ q_i \}$ son los modos normales. Un modo normal es un oscilador armónico.
Clásicamente tenemos los modos normales $\vb{e} \euler^{ i ( \vb{k}\cdot \vbx - \omega t)} $ donde estos 
son vectores de polarización y $\vb{k}$  es el vector de propagación y  $\vb{e}$ el vector de polarización.

Cuánticamente hablamos de fonones, que serán cuantos de excitación de cada modo normal.
El número de fonones es análogo al formalismo de los números de ocupación.
Los fonones se pueden crear y destruir sin invertir energía de manera que $\mu=0$ pero aún utilizo la
estadística de Bose.
Podemos considerar dos aproximaciones.
\begin{itemize}
 \item Aproximación de Einstein: $\omega_i = \omega$ constante para todo $i$.
 \item Aproximación de Debye: El sólido es un medio elástico, continuo y deformable, el espectro de
 $\omega$ va al continuo.
\end{itemize}

\includegraphics[scale=0.4]{images/1606337002.jpg}

Quiero usar Debye. En 3D asumo dispersión como $ \omega = c k $ 

\end{ejemplo}





% \bibliographystyle{CBFT-apa-good}	% (uses file "apa-good.bst")
% \bibliography{CBFT.Referencias} % La base de datos bibliográfica

\end{document}
