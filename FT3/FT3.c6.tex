	\documentclass[10pt,oneside]{CBFT_book}
	% Algunos paquetes
	\usepackage{amssymb}
	\usepackage{amsmath}
	\usepackage{graphicx}
	\usepackage{libertine}
	\usepackage[bold-style=TeX]{unicode-math}
	\usepackage{lipsum}

	\usepackage{natbib}
	\setcitestyle{square}

	\usepackage{polyglossia}
	\setdefaultlanguage{spanish}
	



	\usepackage{CBFT.estilo} % Cargo la hoja de estilo

	% Tipografías
	% \setromanfont[Mapping=tex-text]{Linux Libertine O}
	% \setsansfont[Mapping=tex-text]{DejaVu Sans}
	% \setmonofont[Mapping=tex-text]{DejaVu Sans Mono}

	%===================================================================
	%	DOCUMENTO PROPIAMENTE DICHO
	%===================================================================

\begin{document}

% =================================================================================================
\chapter{Gas de Bose}
% =================================================================================================


% =================================================================================================
\section{Cuánticos IV --reubicar--}
% =================================================================================================

algunos temitas sueltos:

números de ocupación

gas de Fermi $p$ y $c_v$

gas de Fermi $p$ y $c_v$

Condensado de Bose

\notamargen{¿El condensado BE requiere población de los niveles o $V$ total de algún tipo?}

El coeficiente lineal del virial $ 1/ 2^{5/2} = 0.1767767 $ sale considerando las $ f_{\nu}(z) $ hasta orden
uno y tirando términos más allá.

\notamargen{Tenía unas consultas agarradas con clip: ¿porqué hay una cúspide en $C_v$? ¿transiciones?}

El requerimiento $ \mu < 0 $ viene de que el fundamental $ n_0 $ no puede tener población negativa
\[
	n_0 = \frac{1}{\euler^{\beta(e_0 - \mu)} -1} = \frac{1}{\euler^{-\beta\mu} -1} \geq 0
\]
\[
	\euler^{-\beta\mu} -1 > 0 \qquad \Rightarrow \quad \mu < 0
\]
Con $\mu \to 0^-$ tenemos $ n \to \infty $

En el caso del condensado establecemos desde 
\[
	\frac{\lambda^3(T)}{v} = g_{3/2}(1) 
\]
que lleva para $T_c$ (para $v$ fijo) o $v_c$ (para $T$ fija) versiones evaluadas de la anterior ecuación.

Para la población de los estados excitados
\[
	p_x = \frac{h}{V^{1/3}}n_x \Rightarrow  \vb{p} = \frac{h}{V^{1/3}} \vb{n}
\]
\[
	\frac{n_{e_i}}{V} = \frac{1}{V} \frac{1}{ z^{-1}\euler^{\beta e_i} - 1 } \leq 
	\frac{1}{V(\euler^{\beta e_i} - 1)} = \frac{1}{V(\sum_{l=1}^\infty (\beta e_i )^l/l!)}
\]
pués $z^{-1} = 1/z \leq 1$
\[
	\beta e = 
\]

% \bibliographystyle{CBFT-apa-good}	% (uses file "apa-good.bst")
% \bibliography{CBFT.Referencias} % La base de datos bibliográfica

\end{document}
