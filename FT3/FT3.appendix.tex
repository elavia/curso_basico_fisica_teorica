	\documentclass[10pt,oneside]{CBFT_book}
	% Algunos paquetes
	\usepackage{amssymb}
	\usepackage{amsmath}
	\usepackage{graphicx}
% 	\usepackage{libertine}
% 	\usepackage[bold-style=TeX]{unicode-math}
	\usepackage{lipsum}

	\usepackage{natbib}
	\setcitestyle{square}

	\usepackage{polyglossia}
	\setdefaultlanguage{spanish}


	\usepackage{CBFT.estilo} % Cargo la hoja de estilo

	% Tipografías
	% \setromanfont[Mapping=tex-text]{Linux Libertine O}
	% \setsansfont[Mapping=tex-text]{DejaVu Sans}
	% \setmonofont[Mapping=tex-text]{DejaVu Sans Mono}

	%===================================================================
	%	DOCUMENTO PROPIAMENTE DICHO
	%===================================================================

\begin{document}
 
 
\appendix

% ~~~~~~~~~~~~~~~~~~~~~~~~~~~~~~~~~~~~~~~~~~~~~~~~~~~~~~~~~~~~~~~~~~~~~~~~~~~~~~~~~~~~~~~
\chapter{Volumen de esfera n-dimensional}\label{App.volumen_esfera}
% ~~~~~~~~~~~~~~~~~~~~~~~~~~~~~~~~~~~~~~~~~~~~~~~~~~~~~~~~~~~~~~~~~~~~~~~~~~~~~~~~~~~~~~~

\[
	\Omega_n (R) = \int_{x_1^2 + x_2^2 + ... + x_N^2 = R^2 } \kern-2em dx_1 dx_2 ... dx_N = c_n R^N
\]
y la integral es
\[
	\int_{-\infty}^{\infty} dx_1 ... \int_{-\infty}^{\infty} dx_N \euler^{-(x_1^2 + x_2^2 + ... + x_N^2)} =
	\left( \int_{-\infty}^{\infty} \euler^{-x^2} dx \right)^n = (\sqrt{\pi})^N
\]
pero en esféricas
\[
	\left( \int_{0}^{\infty} \dtot{\Omega_n}{R} \euler^{-R^2} dR \right)^n =
	n C_n \int_{0}^{\infty} R^{-1} \euler^{-R^2} dR = 
	n C_n \int_{0}^{\infty} t^{n/2-1} \euler^{-t} dt 
\]
\[
	= \frac{1}{2} C_n n \Gamma(n/2) = \pi^{1/2},
\]
donde es la $\Gamma$ de los matemáticos que cumple ciertas cosas
\[
	\begin{cases}
		\Gamma(n) &= \int_{0}^{\infty} t^{n-1} \euler^{-t} dt\\
		\Gamma(n+1) &= n \Gamma(n) \\
		\Gamma(n+1) &= n! \qquad n \in \mathbb{N} 
	\end{cases}
\]
Luego
\[
	C_n = \frac{ \pi^{n/2} }{ \Gamma(n/2 + 1) }
\]

%  
\end{document}
