	\documentclass[10pt,oneside]{CBFT_book}
	% Algunos paquetes
	\usepackage{amssymb}
	\usepackage{amsmath}
	\usepackage{graphicx}
	\usepackage{libertine}
	\usepackage[bold-style=TeX]{unicode-math}
	\usepackage{lipsum}

	\usepackage{natbib}
	\setcitestyle{square}

	\usepackage{polyglossia}
	\setdefaultlanguage{spanish}
	



	\usepackage{CBFT.estilo} % Cargo la hoja de estilo

	% Tipografías
	% \setromanfont[Mapping=tex-text]{Linux Libertine O}
	% \setsansfont[Mapping=tex-text]{DejaVu Sans}
	% \setmonofont[Mapping=tex-text]{DejaVu Sans Mono}

	%===================================================================
	%	DOCUMENTO PROPIAMENTE DICHO
	%===================================================================

\begin{document}

% =================================================================================================
\chapter{Gases diluidos en las proximidades del equilibrio}
% =================================================================================================


% =================================================================================================
% \section{Energía y entropía}
% =================================================================================================

Sistema clásico diluido, procesos colisionales en términos de $\sigma$, sistema grande con paredes
reflejantes
\[
	f(\vb{x}, \vb{p}, t) d^3 x d^3 p \equiv \# \text{de partículas en el cubo $d^3 p$, $d^3x$}
\]
siendo $f$ la función de distribución de un cuerpo.

La teoría cinética busca hallar $f(\vb{x}, \vb{p}, t)$ para una dada interacción molecular.
Sabemos que la interacción es a través de colisiones.
\notamargen{Clásico implica $\lambda_{\text{deB}} \ll (V/N)^{1/3}$, $h/p \ll v^{1/3}$ o bien $\frac{h}{\sqrt{2mkT}} 
\ll v^{1/3}$}

Sin colisiones las moléculas evolucionan de acuerdo a
\[
	t \to t + \delta t \qquad \vb{x} \to \vb{x} + \vb{v}\delta t \qquad 
	\vb{p} \to \vb{p} + \vb{F}\delta t 
\]
\[
	f(\vb{x}, \vb{p}, t) d^3 x d^3 p = 
	f(\vb{x} + \vb{v}\delta t , \vb{p} \to \vb{p} + \vb{F}\delta t , \vb{p}, t + \delta t) d^3 x' d^3 p'
\]

El volumencillo con sus partículas evoluciona en el espacio de fases $\mu$.
El volumen evoluciona de acuerdo al jacobiano.
\[
	d^3r' d^3p' = |J| d^3r d^3p
\]
pero 
\[
	J = \frac{\partial(x',y',z',p_x',p_y',p_z')}{\partial(x,y,z,p_x,p_y,p_z)}
\]
da 
\[
	1 + \mathcal{O}(\delta t^3)
\]
con lo cual si $ \delta t \ll 1$ será $d^3r' d^3p' = d^3r d^3p$ y entonces
\[
	f(\vb{x} + \vb{v}\delta t , \vb{p} \to \vb{p} + \vb{F}\delta t , \vb{p}, t + \delta t) = 
		f(\vb{x}, \vb{p}, t)
\]
pero si hay colisiones
\[
	f(\vb{x} + \vb{v}\delta t , \vb{p} \to \vb{p} + \vb{F}\delta t , \vb{p}, t + \delta t) = 
		f(\vb{x}, \vb{p}, t)	+ \left. \dpar{f}{t} \right|_{ \text{col}} \delta t
\]
\[
	\dpar{f}{t}  \delta t d^3r d^3p = (\bar{R}- R) \delta t d^3r d^3p
\]
donde $\bar{R} \delta t d^3r' d^3p'$ es el número de colisiones durante $\delta t$ en las que una
partícula se halla al final en $d^3r' d^3p'$ y $R \delta t d^3r d^3p$ es correspondientemente el
número de colisiones durante $\delta t$ en las que una partícula se halla al comienzo en $d^3r d^3p$.
\notamargen{$R \delta t d^3r d^3p$ será finalmente el número de partículas en el cubo $d^3r d^3p$.}

De $t$ a $t+\delta t$ algunas moléculas de A pasan a B y otras van hacia otros lados. Hacia B
llegan moléculas de A y desde fuera.

Dada la dilución consideramos colisiones binarias.
\notamargen{Queremos ver cómo varía f en $\mu$.}

$R$ es el número de colisiones en las cuales la partícula se halla en A y consecuentemente no llega 
a B (pérdida) (en el cubo $d^3V_2$) y $\bar{R}$ es el número de colisiones en las cuales la partícula
se halla fuera de A y consecuentemente por colisión llega a B (ganancia) (en el cubo $d^3V_2$).
\[
	\underbrace{ f( \vb{v}_2,t ) d^3V_2 }_{\text{d. blancos}}  
	\underbrace{| \vb{V}_2 - \vb{V}_1 |}_{ \text{condición de colisión} } 
	\underbrace{ f( \vb{v}_1,t ) d^3V_1 }_{\text{d. incidentes}}  
	\underbrace{\sigma}_{ V_1V_2 \to V_1'V_2'} d^3V_1' d^3V_2'
\]

Si quiero conocer $R$ debo integrar: si la partícula con $\vb{V}_2$ se halla en A integrao en todas
las $\vb{V}_1$ y en todos los destinos $\vb{V}_1'$ y $\vb{V}_2'$.
\[
	\underbrace{ f( \vb{v}_2',t ) d^3V_2' }_{\text{d. blancos}}  
	\underbrace{| \vb{V}_2' - \vb{V}_1' |}_{ \text{condición de colisión} } 
	\underbrace{ f( \vb{v}_1',t ) d^3V_1' }_{\text{d. incidentes}}  
	\underbrace{\sigma}_{ V_1V_2 \to V_1'V_2'} d^3V_1 d^3V_2
\]

Si quiero conocer $\bar{R}$ debo integrar: si la partícula con $\vb{V}_2$ se halla en B integrao en todas
las $\vb{V}_1'$ $\vb{V}_2'$ (orígenes) y en todos los destinos $\vb{V}_1'$.

\[
	d^3V_2 R = \int_{V_1} \int_{V_1'} \int_{V_2'}  f(\vb{V}_2,t) d^3V_2 | \vb{V}_2 - \vb{V}_1 |
		f(\vb{V}_1,t) d^3V_1 \underbrace{\sigma}_{12 \to 1'2'}  d^3V_1' d^3V_2'
\]
\[
	d^3V_2 \bar{R} = \int_{V_1} \int_{V_1'} \int_{V_2'}  f(\vb{V}_2',t) d^3V_2' | \vb{V}_2' - \vb{V}_1' |
		f(\vb{V}_1',t) d^3V_1' \underbrace{\sigma}_{1'2' \to 12}  d^3V_1 d^3V_2
\]

\[
	d^3V_2 R = \int_{V_1} \int_{V_1'} \int_{V_2'}  f_2 f_1  | \vb{V}_2 - \vb{V}_1 |
		\underbrace{\sigma}_{12 \to 1'2'}  d^3V_1' d^3V_2' d^3V_2 d^3V_1
\]
\[
	d^3V_2 \bar{R} = \int_{V_1} \int_{V_1'} \int_{V_2'}  f_2' f_1' | \vb{V}_2' - \vb{V}_1' |
		 \underbrace{\sigma}_{1'2' \to 12}  d^3V_1 d^3V_2 d^3V_2' d^3V_1'
\]
y si usamos que $| \vb{V}_2 - \vb{V}_1 |=| \vb{V}_2' - \vb{V}_1' |$ y $  \underbrace{\sigma}_{12 \to 1'2'} =  
\underbrace{\sigma}_{1'2' \to 12} $ entonces 
\[
	\left. \dpar{f_2}{t}\right|_{\text{col}} =(\bar{R}-R) d^3V_2 =
	\int_{V_1} \int_{V_1'} \int_{V_2'}  ( f_1' f_2' -f_1 f_2 ) | \vb{V}_2 - \vb{V}_1 |
		\underbrace{\sigma}_{12 \to 1'2'}  d^3V_1' d^3V_2' d^3V_2 d^3V_1
\]

Bajo estas líneas pueden verse los esquemas de integración,


\subsection{Construcción de una cuenta}

\[
	\overbrace{\frac{|\vb{V}_2-\vb{V}_1|\delta t \delta A}{\delta t \delta A}}^{\text{Volumen dentro
	del cual una partícula con $\vb{V}_1$ chocaría a una de $\vb{V}_2$.}}
	\underbrace{f(\vb{V}_1,t) d^3V_1}_{\text{densidad de incidentes}}
\]
es el \# de partículas incidentes con $\vb{V}_1$ que podría colisionar con una de $\vb{V}_2$ en la unidad
de tiempo y por unidad de área.
\[
	\sigma (\vb{V}_1\vb{V}_2 \to \vb{V}_1' \vb{V}_2') d^3V_1' d^3V_2'
\]
es la sección eficaz de dispersión del proceso $V_1V_2 \to V_1'V_2'$ teniendo como destinos $\vb{V}_1'$ y
$\vb{V}_2'$.

\[
	\left[ |\vb{V}_2-\vb{V}_1| f(\vb{V}_1,t) d^3V_1 \right] \sigma_{ 12 \to 1'2'} d^3V_1' d^3V_2'
\]
es el \# de partículas incidentes con $\vb{V}_1$ dispersadas en $\vb{V}_1'$ y con el blanco yendo a $\vb{V}_2'$
por unidad de tiempo y volumen.

\[
	[ f(\vb{V}_2,t) d^3V_2 ] |\vb{V}_2-\vb{V}_1| f(\vb{V}_1,t) d^3V_1 \sigma d^3V_1' d^3V_2'
\]
es el \# de partículas dispersadas hacia $\vb{V}_1'$ y $\vb{V}_2'$ proviniendo de $\vb{V}_1$ y $\vb{V}_2$ por 
unidad de tiempo y de volumen.

Quisiera conocer $R dt d^3r d^3v$ (\# de colisiones durante $dt$ en las cuales una partícula incial --blanco-- se
halla en $d^3r$ con $d^3v_2$)\notamargen{pérdida; si golpeo un blanco en $\vb{V}_2$ lo saco del volumen}
\[
	R dt d^3r d^3v = \int_{V_1}\int_{V_1'}\int_{V_2'} dt d^3r 
	f(\vb{V}_2,t) d^3V_2 |\vb{V}_2-\vb{V}_1| f(\vb{V}_1,t) d^3V_1 \sigma d^3V_1' d^3V_2'
\]
\notamargen{Se integra en las incidentes $V_1$ y en las destinos $V_1', V_2'$.}
y también $\bar{R} dt d^3r d^3v$ (\# de colisiones durante $dt$ en las cuales una partícula final se halla en 
$d^3r$ con $d^3v_2$)
\notamargen{ganancia si golpeo}
\[
	\bar{R} dt d^3r d^3v = \int_{V_1}\int_{V_1'}\int_{V_2'} dt d^3r 
	f(\vb{V}_2',t) d^3V_2' |\vb{V}_2'-\vb{V}_1'| f(\vb{V}_1',t) d^3V_1' \sigma d^3V_1 d^3V_2
\]
\[
	\left. \dpar{f}{t}\right|_{col} \delta t = (\bar{R} - R)\delta t
\]
Usando
\[
	|\vb{V}_2 - \vb{V}_1| = |\vb{V}_2' - \vb{V}_1'| \quad \sigma(12 \to 1'2') = \sigma(1'2' \to '2)
\]
\[
	\left. \dpar{f}{t} \right|_{\text{col}} =
	\int_{V_1} \int_{V_1'} \int_{V_2'} d^3v_1 d^3v_1' d^3v_2' |\vb{V}_2 - \vb{V}_1| \sigma 
	( f(\vb{V}_1',t)f(\vb{V}_2',t) - f(\vb{V}_1,t)f(\vb{V}_2,t) ) 
\]

Por otro lado
\[
	f(\vb{r} + \vb{v} \delta t, \vb{p} + \vb{F} \delta t, t + \delta t) - f(\vb{r}, \vb{p}, t) = 
	f(\vb{r},\vb{v} + \frac{\vb{F}}{m}\delta t, t + \delta t) - f(\vb{r},\vb{v}, t)
\] 
\[
	\dpar{f}{\vb{r}} \vb{v} \delta t + \dpar{f}{\vb{v}} \frac{\vb{F}}{m} \delta t + \dpar{f}{t} \delta t = 
	\vb{v}\cdot\nabla_{\vb{r}} + \frac{\vb{F}}{m}\cdot\nabla_{\vb{v}} + \dpar{f}{t} \delta t 
\]
y entonces con $\delta t \to 0$ es
\[
	\left( \vb{v}\cdot\nabla_{\vb{r}} + \frac{\vb{F}}{m}\cdot\nabla_{\vb{p}} + \dpar{}{t} \right)f =
	\left. \dpar{f}{t} \right|_{\text{col}} 
\]
y somos conducidos a
\[
	\boxed{( \vb{v}\cdot\nabla_{\vb{r}} + \frac{\vb{F}}{m}\cdot\nabla_{\vb{v}} + \dpar{}{t} )f_2 =
	\int_{V_1} \int_{V_1'} \int_{V_2'} d^3v_1 d^3v_1' d^3v_2' V \sigma 
	(f_1'f_2' - f_1f_2) }
\]
la ecuación de transporte de Boltmann.
\notamargen{La solución de equilibrio será aquella independiente del tiempo.
Es decir $\dpar{f}{t} = 0$, $\int\int\int dV...V\sigma(f_1'f_2' - f_1f_2) = 0$}

Se ha supuesto CAOS MOLECULAR, de modo que la correlación de dos cuerpos
(función de distribución de dos cuerpos en el mismo punto espacial)
\[
	f(\vb{r},\vb{v}_1,\vb{v}_2, t) = f(\vb{r},\vb{v}_1, t) f(\vb{r},\vb{v}_2, t)
\]
y esto nos lleva a que las velocidades de dos partículas en el elemento $d^3r$
no están correlacionadas. La probabilidad de encontrarlas simultáneamente es el
producto de hallarlas a cada una por separado.

Una condición suficiente es
\[
	f_1'f_2' - f_1f_2 = 0 \Rightarrow \left. \dpar{f}{t} \right|_{\text{col}} = 0
\]
y veremos que es también necesaria.

\subsection{otra}

Supusimos un sistema diluido, con colisiones binarias y llegamos a
\be
	\left( \vb{v}\cdot\nabla_{\vec{r}} + \frac{1}{m}\vb{F}\cdot\nabla_{\vec{v}} + \dpar{}{t} \right)f_2 =
	\dpar{f_2}{t} = \int\int\int d^3v_1 d^3v_1' d^3v_2' V \sigma (f_{1'}f_{2'} - f_1f_2)
	\label{equilibrio_fdist}
\ee

Pensamos que en el equilibrio será $\partial f_2 /\partial t = 0$ y sabemos que 
\[
	\text{si} \;  f_{1'}f_{2'} - f_1f_2 = 0  \Rightarrow  \dpar{f}{t} = 0
\]
\notamargen{La función del equilibrio es MB, $f_0(\vb{v}) \to \dpar{f_0}{t}=0$}

Definiendo $H(t) = \int d^3V f(\vb{v},t) \log( f(\vb{v},t) )$ vemos que 
\[
	\text{si} \; \dpar{f(\vb{v},t)}{t} = 0 \Rightarrow \dtot{H}{t} = 0
\]

Ahora, considerando que $f$ satisface \eqref{equilibrio_fdist} probamos que 
\[
	\text{si $f$ verifica \eqref{equilibrio_fdist}}  \Rightarrow \dtot{H}{t} \leq 0
\]
pero como el integrando en $dH/dt$ no cambia de signo nunca debe anularse para obtener el cero
con lo cual 
\[
	\dtot{H}{t} = 0 \Rightarrow f_{1'}f_{2'} - f_1f_2 = 0 \Rightarrow \dpar{f}{t} = 0
\]
y en definitiva
\[
	\boxed{ \dtot{H}{t} = 0 \Leftrightarrow  \dpar{f}{t} = 0 }
\]
y prueba que con 
\[
	f(\vb{v},t)_{t\to \infty} \to f_0(\vb{v}) \quad \text{con} \quad \dpar{f_0}{t} = 0
\]

La ecuación \eqref{equilibrio_fdist} asume la hipótesis de CAOS MOLECULAR para su validez.

$f(\vb{p},t)$ en principiao sólo satisface la ecuación de transporte de Boltzmann cuando vale
CAOS MOLECULAR. Una tal $f$ es tal que 
\[
	\dtot{H}{t} \leq 0 \qquad \text{$H$ es decreciente siempre (un instante luego del CAOS MOLECULAR)}
\]
\[
	\dtot{H}{t} = 0 \qquad \text{si $f(\vb{p},t) = f_{MB}$ con $\dpar{f}{t} = 0$ }
\]

CAOS MOLECULAR entonces significa que $H$ es máximo local, luego decrece rápidamente y además se sale
de $f_{MB}$

\section{Teorema H y consecuencias}

\[
	H(t) = \int d^3p f( \vb{p}, t ) \log ( f( \vb{p}, t ) ) =
	< \log f( \vb{p}, t ) >_{\text{no normalizado}}
\]
\[
	\dpar{H(t)}{t} = \int d^3p \left( \dpar{f}{t} \log f + f \frac{1}{f} \dpar{f}{t} \right)
\]
\[
	\dpar{H(t)}{t} = \int d^3p \dpar{f}{t}  \left( 1 + \log f \right)
\]
\[
	\text{Si} \; \dpar{f}{t} = 0 \Rightarrow \dpar{H}{t} = 0
\]
Entonces la anulación de la derivada de $H$ es condición necesaria pero no suficiente para
que la derivada de $f$ se anule.

Por otro lado, también vale que si $f$ satisface la ecuación de Boltzmann, entonces
\[
	 \dtot{H}{t} = \dtot{}{t}< \log f( \vb{p}, t ) >_{\text{no normalizado}} \leq 0
\]

\[
	\dpar{H(t)}{t} = \int d^3p \dpar{f}{t}(\vb{p},t)  \left( 1 + \log f \right)
\]
y si consideramos función de $\vb{v}_2$,
\[
	\dtot{H}{t} = \int d^3V_2 
	\int_{V_1} \int_{V_1'} \int_{V_2'} d^3v_1 d^3v_1' d^3v_2' V \sigma 
	(f_1'f_2' - f_1f_2) [1 + \log f_2]
\]
pero el intercambio de $V_1$ con $V_2$ no afecta la integral y podemos sumar dos medios,
\begin{multline*}
 \dtot{H}{t} = \frac{1}{2} \left[ \int d^3V_2 
	\int_{V_1} \int_{V_1'} \int_{V_2'} d^3v_1 d^3v_1' d^3v_2' V \sigma 
	(f_2'f_1' - f_2f_1) [1 + \log f_1] + \right.\\
	\left. \int d^3V_2 \int_{V_1} \int_{V_1'} \int_{V_2'} d^3v_1 d^3v_1' d^3v_2' V \sigma 
	(f_1'f_2' - f_1f_2) [1 + \log f_2] \right]
\end{multline*}
\[
 \dtot{H}{t} = \frac{1}{2} \left[ \int d^3V_2 
	\int_{V_1} \int_{V_1'} \int_{V_2'} d^3v_1 d^3v_1' d^3v_2' V \sigma 
	(f_2'f_1' - f_2f_1) [2 + \log (f_1 f_2)] \right] 
\]
pero intercambio de $V_1',V_2'$ con $V_1,V_2$ tampoco afecta, entonces
\begin{multline*}
  \dtot{H}{t} = \frac{1}{4} \left[ \int d^3V_2 
	\int_{V_1} \int_{V_1'} \int_{V_2'} d^3v_1 d^3v_1' d^3v_2' V \sigma 
	(f_2f_1 - f_2'f_1') [2 + \log (f_1' f_2')] + \right. \\
	\left. int d^3V_2 
	\int_{V_1} \int_{V_1'} \int_{V_2'} d^3v_1 d^3v_1' d^3v_2' V \sigma 
	(f_2'f_1' - f_2f_1) [2 + \log (f_1 f_2)] \right] 
\end{multline*}
\[
	\dtot{H}{t} = \frac{1}{4} \int d^3V_2 
	\int_{V_1} \int_{V_1'} \int_{V_2'} d^3v_1 d^3v_1' d^3v_2' V \sigma 
	(f_2f_1 - f_2'f_1') [\log \left( \frac{f_1' f_2'}{f_1f_2} \right)]
\]
y como siempre es
\[
	(X-Y)\log\left( \frac{Y}{X} \right) \leq 0
\]
luego
\[
	\dtot{H}{t} \leq 0 
\]
y si 
\[
	\dpar{f}{t} = 0 \Rightarrow \dtot{H}{t} = 0 
\]
pero de la prueba que acabamos de finalizar vemos que si 
\[
	\dtot{H}{t} = 0 \Rightarrow f_1f_2 - f_1'f_2' = 0 \Rightarrow \dpar{f}{t} = 0
\]
luego 
\[
	\dtot{H}{t} = 0 \qquad \Leftrightarrow \qquad \dpar{f}{t}(\vb{v},t) = 0
\]
con $f$ de Boltzmann.

Entonces $dH/dt = 0$ si y sólo si $f_1f_2 = f_1'f_2'$ para todas las colisiones. Esta 
condición se conoce como {\it balance detallado} y es la condición de equilibrio para
el gas.

\[
	E = \int d^3V f(\vb{v},t)|\vb{v}|^2 < \infty
\]
\[
	H = \int d^3V f(\vb{v},t) \log f(\vb{v},t) 
\]





\notamargen{$H$ es el promedio en la distribución de $\log f(\vb{p},t)$ no normalizado.}


% \bibliographystyle{CBFT-apa-good}	% (uses file "apa-good.bst")
% \bibliography{CBFT.Referencias} % La base de datos bibliográfica

\end{document}
