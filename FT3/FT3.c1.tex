	\documentclass[10pt,oneside]{CBFT_book}
	% Algunos paquetes
	\usepackage{amssymb}
	\usepackage{amsmath}
	\usepackage{graphicx}
% 	\usepackage{libertine}
% 	\usepackage[bold-style=TeX]{unicode-math}
	\usepackage{lipsum}

	\usepackage{natbib}
	\setcitestyle{square}

	\usepackage{polyglossia}
	\setdefaultlanguage{spanish}
	



	\usepackage{CBFT.estilo} % Cargo la hoja de estilo

	% Tipografías
	% \setromanfont[Mapping=tex-text]{Linux Libertine O}
	% \setsansfont[Mapping=tex-text]{DejaVu Sans}
	% \setmonofont[Mapping=tex-text]{DejaVu Sans Mono}

	%===================================================================
	%	DOCUMENTO PROPIAMENTE DICHO
	%===================================================================

\begin{document}

% =================================================================================================
\chapter{Básicos de termodinámica}
% =================================================================================================

El que un cuerpo sea, a efectos termodinámicos, macroscópico o no dependerá de la relación entre las 
distancias características entre los componentes y el tamaño total teniendo en cuenta el potencial que 
hace interactuar a las partículas.
Así, por ejemplo, una galaxia no es un cuerpo macroscópico mientras que un barril de cerveza sí lo es.

El equilibrio termodinámica se da cuando macroscópicamente nada pasa en el sistema, aunque microscópicamente
siempre está sucediendo algo.

Una pared rígida y adiabática implica que no puedo interactuar con él; en cambio una pared rígida implica
solamente que no puedo interactuar por medios mecánicos [explicar].

El concepto de estado conlleva el desinterés por la manera en que llego a ese estado. Una variable 
termodinámica no puede depender de cómo el sistema llega a ese estados; por ello esa variable $F$ es un
diferencial exacto, es decir que
\[
	\oint dF = 0 \qquad \text{ y } \qquad F(b)-F(a) = \int_a^b dF
\]
implica que $F$ es variable de estado.

El gas ideal es tal que las partículas no interactúan entre sí. Es una aproximación de gases diluídos que
funciona bien a baja densidad.
Podemos expandir una expresión en términos de la densidad, lo cual se conoce como ecuación del virial.

Se busca hallar los coeficientes de dicha expansión a cada orden
\[
	p = \frac{nRT}{V}\left( 1 + \frac{n}{V} B(T) + \Frac{n}{V}^2 C(T) + ... \right)
\]
donde $B,C,...$ son los coeficientes del virial.
Van der Waals, según veremos después, es una primera corrección al gas ideal.

Se tiene el potencial de Lenard-Jones, que es de la forma $V \sim (x^{-12} - x^{-6} ) $


En la ecuación de estado del gas ideal se tiene que todo el volumen es accesible todo el tiempo (el gas
puede ocupar todo el volumen).

Partículas no interactuantes (no tienen volumen)

En la expresión de Van der Waals,
\[
	p = \frac{nRT}{V - nb} - a \Frac{n}{V}^2
\]
el denominador del primer término tiene en cuenta el carozo, mientras que el segundo término tiene en
cuenta la disminución de presión (que es un efecto de las colas).

% =================================================================================================
\section{Energía y entropía}
% =================================================================================================

Una de las formulaciones de la 2da ley es definir la entropía. Surge de:
\[
	\frac{Q_1}{Q_2} = -\frac{T_1}{T_2} \qquad \Rightarrow \frac{Q_1}{Q_2} + \frac{T_1}{T_2} = 0 \;
	\text{reversible}
\]
\[
	\int \frac{dQ}{T} \leq 0 \qquad \text{desigualdad de Clausius}
\] 
Proceso reversible en un sistema aislado
\[
	S_{A\to B} = \int_A^B dS = 0
\]
pues 
\[
	dS =\frac{dU}{T} - \frac{p}{V}dV + \frac{\mu}{T}dN = 0
\]
pero en procesos irreversibles la variación de $S$ es cota superior:
\notamargen{La existencia de $S$ es independiente de su cálculo}
\[
	\int_A^B \frac{dQ}{T} < \int_A^B dS = S_{A\to B}.
\]

Luego, para un sistema aislado, en un proceso irreversible 
\[
	dS_I = 0 \qquad \Rightarrow \qquad \frac{dQ_I}{T} = 0
\]
y entonces
\[
	0 < \int_A^B  dS =  S_{A\to B}
\]

La entropía solo aumenta. Podría calcular $S_{A\to B}$ con un proceso reversible de $A\to B$ pero ahí 
ya tengo que intervenir sobre el sistema (no hay procesos espontáneos --en un sistema aislado-- reversibles).

En reversibles
\[
	dU = TdS - pdV + \mu dN
\]
mientras que en irreversibles
\[
	dU = ddQ_I - pdV +\mu dN, \quad \text{pero} \quad dQ_I < TdS 
\]
y entonces
\[
	dU < TdS - pdV + \mu dN
\]
% \begin{figure}[htb]
% 	\begin{center}
% 	\includegraphics[width=0.8\textwidth]{images/teo2_1.pdf}	 
% 	\end{center}
% 	\caption{}
% \end{figure} 
Si $S$ es homogénea, se tiene
\[
	S = S(\lambda U, \lambda X, \{\lambda N_i\}) = \lambda S( U, X, \{ N_i\})
\]
y además si \notamargen{En un sistema $PVT$ $Y=-p$.}
\[
	TdS = dU - YdX - \mu_i dN_i
\]
\[
	\dtot{S}{\lambda} = S = \dpar{S}{\lambda U}\dtot{\lambda U}{\lambda} +
	\dpar{S}{\lambda X}\dtot{\lambda X}{\lambda} +
	\dpar{S}{\lambda N_i}\dtot{\lambda N_i}{\lambda}
\]
\[
	S = \dpar{S}{\lambda U} U + \dpar{S}{\lambda X} X + \dpar{S}{\lambda N_i} N_i
\]
\[
	\dpar{}{\lambda U}\left[ S(\lambda U)\right] = 
	\dpar{}{\lambda U}\left[ \lambda S( U)\right] = \dpar{S}{U} = \frac{1}{T}
\]
y procediendo del mismo modo con $Y,\mu$
\[
	S = \frac{1}{T} U + \frac{-Y}{T} X + \frac{-\mu_i}{T} N_i
\]
y arribamos a la ecuación fundamental
\[
	TS = U - YX - \mu_i N_i 
\]
o bien
\[
	U = TS + YX + \sum_i \mu_i N_i
\]

La primera ley (en sistemas reversibles) era 
\[
	dU = TdS + YdX + \sum_i \mu_i dN_i
\]
y a $S,V,N$ constantes 
\[
	dU^R = 0 \qquad dU^I \leq 0
\]
la mínima $U$ es equilibrio.
Si existe trabajo que no es de volumen resulta 
\[
	dU < -dW_\text{libre}
\]
\[
	\frac{dQ}{dT} = \frac{dU}{T} + \frac{p}{T}dV - \frac{\mu}{T}dN = \frac{dQ}{dT} \leq dS
\]

Si el sistema está aislado será
\[
	0 \leq dS \quad \text{condición de equilibrio}
\]
alcanzando el máximo ya no puede disminuir la entropía.


% =================================================================================================
\section{Transformadas de Legendre de las funciones termodinámicas}
% =================================================================================================

\[
	f(x,y,z) \qquad \text{con pendientes} \quad \dpar{f}{x},\dpar{f}{y},\dpar{f}{z}
\]
entonces 
\[
	\varphi(f_x,y,z) = f(x,y,z) - \left. x \dpar{f(x,y,z)}{x}\right|_{y,z}
\]
es la transformada de Legendre respecto de $x$, mientras que 
\[
	\varphi(f_x,f_y,z) = f(x,y,z) - x \dpar{f}{x} - y \dpar{f}{y}
\]
es la transformada de Legendre respecto de $y$.

La transformada de Legendre transforma una función homogénea en otra función homogénea, mantiene el
carácter de función de estado.
\[
	d\varphi(f_x,y,z) = df - dx \dpar{f}{x} - x d\left( \dpar{f}{y} \right)
\]

Para el caso de la energía
\[
	U=U(S,V,N) \qquad \qquad dU = TdS - pdV + \mu dN
\]
y entonces
\[
	A = U - \left. S\dpar{U}{S}\right|_{V,N} = U - ST \qquad \Rightarrow \qquad  A=A(T,V,N)
\]
\[
	H = U - \left. V\dpar{U}{V}\right|_{S,N} = U + pV \qquad \Rightarrow \qquad  H=H(S,p,N)
\]
\[
	G = U - \left. S\dpar{U}{S}\right|_{V,N} - \left. V\dpar{U}{V}\right|_{S,N} = 
	U - ST + pV \qquad \Rightarrow \qquad  G=G(T,p,N)
\]
\[
	dA = dU - SdT - TdS = -SdT - pdV + \mu dN
\]
\[
	dA \leq -SdT - pdV + \mu dN 
\]
entonces $A$ mínimo es equilibrio a $T,V,N$ constantes.

La idea de las transformadasd de Legendre es pasar la dependencia de cierto juego de variables a otro
que podría ser más apropiado par el sistema en cuestión.

Sistema aislado en equilibrio, entonces se tendrá $S$ máxima y como $S(U,V,N)$ y considero fluctuación
energética
\[
	\left. \dpar{S}{U} \right|_{\text{eq}} = 0 \qquad \left. \dpar2{S}{U} \right|_{\text{eq}} < 0
\]
\[
	\delta S_{\mathrm{orden 2}} = \frac{1}{2} \left. \dpar2{S}{U} \right|_{\text{eq}} \delta U^2
\]

% =================================================================================================
\section{Gas de Van der Waals}
% =================================================================================================

Van der Waals incorpora la interacción molecular. \notamargen{Esta subsección tiene cinco gráficos}
\[
	\left( p +\frac{an^2}{V^2} \right)(V- nb) = nRT
\]
donde $a,b(T)$ caracterizan al gas en cuestión.

La función $p=p(V)$ tiene tres extremos para $T<T_c$,
\[
	\dpar{p}{V} = 0
\]
En $T=T_c$ es
\[
	\left.\dpar{p}{V} \right|_{T_c} = 0 \qquad \left.\dpar[2]{p}{V} \right|_{T_c} = 0
\]
punto de inflexión
\[
	v_c = 3b \qquad p_c = \frac{a}{27b^2} \qquad T_c = \frac{8a}{27Rb}
\]
y eso lleva a la ley de estados correspondientes
\[
	\left( \bar{p} + \frac{3}{\bar{v}^2} \right)(3 \bar{v} - 1)= 8\bar{T}
\]

De Van der Waals al virial
\[
	p = \frac{nRT}{(V-nb)} - a \left(\frac{n}{V} \right)^2 = 
	\frac{nRT}{V(1-b/v)}- \frac{a}{v^2}
\]
\[
	p = \frac{RT}{v}\left[ 1 + \frac{b}{v} - \frac{a}{vRT} \right] =
	p = \frac{RT}{v}\left[ 1 + \frac{1}{v}\left( b - \frac{a}{RT} \right) \right]
\]
y el último paréntesis es el primer coeficiente del virial.

Un potencial intermolecular está compuesto de una zona repulsiva (carozo duro) y una atractiva (cola)
\[
	V_{eff} = V-b \qquad (\mathrm{menor volumen por el carozo})
\]
\[
	p = \frac{RT}{V-b} - \left(\frac{a}{V}\right)^2 \qquad (\mathrm{menor presión por la atractividad})
\]
y entonces, por mol de sustancia,
\[
	\left( p +\frac{a^2}{V^2} \right)(V- b)= RT
\]

$b$ corrige el volumen que es ahora menor porque las partículas ocupan espacio. $a$ corrige la presión dado que la 
atracción tiende a formar pares bajando la presión sobre las paredes.

Las funciones respuesta tienen signo errado dentro de la zona del rulo\notamargen{Recordemos que
\[ -\frac{1}{v}\dpar{v}{p}=\kappa_T > 0\]}
\[
	\dpar{p}{V}>0 \rightarrow  \dpar{v}{p} >0 \Rightarrow \kappa_T < 0 \qquad (\mathrm{MAL})
\]
\[
	dT = -SdT + VdP + \mu dN
\]
dada la isoterma y que $N$ es constante 
\[
	dG = Vdp \rightarrow dg = v dP \quad (\mathrm{molar})
\]
$G$ es cóncava en $p$ entonces 
\[
	v = \left.\dpar{g}{p} \right|_{T,N}, \qquad  
	\dpar{v}{p} =\left.\dpar[2]{g}{p} \right|_{T,N} < 0
\]
y luego 
\[
	\Delta g = \int_{p_c}^{p_G} v dp = 0
\]
entonces 
\[
	\int_C^D + \int_D^E + \int_E^F + \int_F^G = 0
\]
y si se invierten puntos para tener un recorrido según las flechas se llega a 
\[
	\int_C^D - \int_E^D = \int_F^E - \int_F^G 
\]

Áreas inguales determinan entonces los puntos C y G de forma que se corrige Van Der Waals para dar curvaturas
correctas. En la región de coexistencia hemos trocado
\[
	\dpar{p}{V} > 0 \quad \mathrm{por} \quad \dpar{p}{V}=0
\]
lo cual da $\kappa_T \to \infty$ en lugar del $\kappa_T < 0$ (que es incorrecto).









% \bibliographystyle{CBFT-apa-good}	% (uses file "apa-good.bst")
% \bibliography{CBFT.Referencias} % La base de datos bibliográfica

\end{document}
