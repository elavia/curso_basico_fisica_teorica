	\documentclass[10pt,oneside]{CBFT_article}
	% Algunos paquetes
	\usepackage{amssymb}
	\usepackage{amsmath}
	\usepackage{graphicx}
	\usepackage{libertine}
	\usepackage[bold-style=TeX]{unicode-math}
	\usepackage{lipsum}

	\usepackage{natbib}
	\setcitestyle{square}

	\usepackage{polyglossia}
	\setdefaultlanguage{spanish}


	\usepackage{CBFT.estilo} % Cargo la hoja de estilo

	% Tipografías
	% \setromanfont[Mapping=tex-text]{Linux Libertine O}
	% \setsansfont[Mapping=tex-text]{DejaVu Sans}
	% \setmonofont[Mapping=tex-text]{DejaVu Sans Mono}

	%===================================================================
	%	DOCUMENTO PROPIAMENTE DICHO
	%===================================================================

\title{CBFT Mecánica clásica}
\author{Cuerpos rígidos}
\date{\today}

\begin{document}
\maketitle
\tableofcontents


% =================================================================================================
\section{Cuerpos rígidos}
% =================================================================================================

Los vínculos constituyen la condición de rigidez,
\be
	|\vb{r}_i \vb{r}_j | = d_{ij}	\qquad i \neq j
\label{vinculos}
\ee

Del discreto al continuo
\[
	\vb{R} = \frac{\sum_i m_i\vb{r}_i}{\sum_i m_i} \longrightarrow 
	\vb{R} = \frac{\int \rho \vb{r}_i dv }{\int \rho dv} 
\]

\subsection{Grados de libertad de un cuerpo rígido}

Cada punto tiene como vínculos las ecuaciones \eqref{vinculos}

El cuerpo rígido tiene seis grados de libertad.
Si las condiciones de rigidez son lineales resultan cinco grados de libertad.

\subsection{Velocidad de un cuerpo rígido}

Lo único que pueden hacer los puntos de un cuerpo rígido es rotar.

\[
	\delta r_{p_0} = r_{p_0} \sin(\beta) \delta \alpha
\]
\[
	\frac{\delta r_{p_0}}{\delta t} = r_{p_0} \sin(\beta) \frac{\delta\alpha}{\delta t}
\]
\[
	v_{p_0} = \dot{\alpha} r_{p_0} \sin(\beta)
\]
pero $v_{p_0} \perp \hat{n}$ y $v_{p_0} \perp r_{p_0}$ de manera que 
\[
	\vb{V}_{p_0} = \vb{\Omega} \times \vb{r}_{p_0}.
\]

Luego, para ir a un sistema inercial le sumo la V de algún punto del rígido (el origen O)
medido desde un sistema inercial. Entonces, el campo de velocidad del cuerpo rígido es
\[
	\vb{V}_{p} = \vb{V}_0 + \vb{\Omega} \times \vb{r}_{p_0}.
\]

\subsection{Unicidad de la velocidad de rotación}

\[
	\vb{V}_{p} = \vb{V}_0' + \vb{\Omega}' \times \vb{r}_{p_0'}
\]
siendo \vb{\Omega}' la \vb{\Omega} como se ve desde el sistema O'
\[
	\vb{V}_{p} = \vb{V}_0 + \vb{\Omega} \times \vb{r}_{p_0}
\]
y donde \vb{\Omega} es la vista desde el sistema O.
\[
	\vb{V}_0' + \vb{\Omega}' \times \vb{r}_{p_0'} = \vb{V}_0 + \vb{\Omega} \times \vb{r}_{p_0} 
\]
y descomponiendo de acuerdo con el dibujo resulta 
\[
	\vb{\Omega} \times \vb{r}_{OO'} + \vb{\Omega}' \times \vb{r}_{0'p} = \vb{\Omega} \times \vb{r}_{p_0} 
\]
\[
	\vb{\Omega} \times ( \vb{r}_{00'} - \vb{r}_{0p} ) + \vb{\Omega}' \times \vb{r}_{0'p}  = 0
\]
\[
	( \vb{\Omega}' - \vb{\Omega}  ) \times \vb{r}_{0'p} = 0 ,
\]
de la cual se deduce que $\vb{\Omega}'=\vb{\Omega}$. Entonces, \vb{\Omega} es la misma para cualquier
punto del cuerpo rígido.

\[
	\vb{\Omega} \cdot \vb{V}_p = \vb{\Omega} \cdot \vb{V}_0  + \vb{\Omega}\cdot(\vb{\Omega}\times \vb{r}_{0p} )
\]
\[
	\vb{\Omega} \cdot \vb{V}_p = \vb{\Omega} \cdot \vb{V}_0
\]
lo cual se cumple para todo punto $p$ perteneciente al cuerpo rigido. Si es $\vb{\Omega} \cdot \vb{V}_0 = 0$
entonces serán $\vb{\Omega} \perp \vb{V}_0$ y $\vb{\Omega} \perp \vb{V}_p$.

Si en un instante dado \vb{\Omega} es perpendicular a $\vb{V}_p$ entonces \vb{\Omega} es perpendicular a 
$\vb{V}_{p'}$ para todo punto del cuerpo rígido.

\subsection{Eje instantáneo de rotación}

Si $p$ es tal que $\vb{V}_p = 0$ entonces
\[
	\vb{V}_0 = - \vb{\Omega} \times \vb{r}_{p0}
\]
donde $\vb{V}_0$ es una velocidad desde un sistema inercial.
Desde el sistema inercial el cuerpo rígido realiza una rotación pura, puesto que veo al
punto O rotar en torno a algún eje.
\[
	\vb{V}_0 = - \vb{\Omega} \times ( r_{\perp} + r_{\parallel} ) = -\vb{\Omega} \times  r_{\perp} 
\]
y esto define un eje instantáneo de rotación.

% =================================================================================================
\section{Ángulos de Euler}
% =================================================================================================

Se toma un sistema 123 inicialmente coincidente con uno XYZ paralelo al inercial, 123 tiene origen
en el centro de masa del cuerpo.

\[
	A_1(\phi) = 
	\begin{pmatrix}
		\cos(\phi) & \sin(\phi) & 0 \\
		-\sin(\phi) & \cos(\phi) & 0 \\ 
		0 & 0 & 1  \\
	\end{pmatrix}
\]
\[
	A_2(\theta) = 
	\begin{pmatrix}
		1 & 0 & 0 \\
		0 & \cos(\theta) & \sin(\theta) \\ 
		0 & -\sin(\theta) & \cos(\theta)  \\
	\end{pmatrix}
\]
\[
	A_3(\psi) = 
	\begin{pmatrix}
		\cos(\psi) & \sin(\psi) & 0 \\
		-\sin(\psi) & \cos(\psi) & 0 \\ 
		0 & 0 & 1  \\
	\end{pmatrix}
\]
\[
	\vb{\Omega} = \dot{\phi}\hat{z} + \dot{\theta}\hat{n} + \dot{\psi}\hat{3}
\]
y expresando $\hat{z},\hat{n}$ en $\hat{1},\hat{2}, \hat{3}$ resulta
\[
	\vb{\Omega} = [\dot{\phi}\sin(\theta)\sin(\psi) + \dot{\theta}\cos(\psi) ]\hat{1} +
			[\dot{\phi}\sin(\theta)\cos(\psi) - \dot{\theta}\sin(\psi) ] \hat{2} +
			[\dot{\phi}\cos(\theta) + \dot{\psi} ]\hat{3}
\]

Ahora estamos interesados en el momento angular.
\[
	\vb{L}_0^{sist} = \vb{L}^{cm} + \vb{L}_{cm}^{sist} 
\]
\[
	\vb{L}_{spin} = \sum_i^N m_i ( \vb{r}_i' \times \vb{v}_i' )
\]
que están en el sistema 123.



\bibliographystyle{CBFT-apa-good}	% (uses file "apa-good.bst")
\bibliography{CBFT.Referencias} % La base de datos bibliográfica

\end{document}
