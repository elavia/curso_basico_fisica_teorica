	\documentclass[10pt,oneside]{CBFT_article}
	% Algunos paquetes
	\usepackage{amssymb}
	\usepackage{amsmath}
	\usepackage{graphicx}
	\usepackage{libertine}
	\usepackage[bold-style=TeX]{unicode-math}
	\usepackage{lipsum}

	\usepackage{natbib}
	\setcitestyle{square}

	\usepackage{polyglossia}
	\setdefaultlanguage{spanish}


	\usepackage{CBFT.estilo} % Cargo la hoja de estilo

	% Tipografías
	% \setromanfont[Mapping=tex-text]{Linux Libertine O}
	% \setsansfont[Mapping=tex-text]{DejaVu Sans}
	% \setmonofont[Mapping=tex-text]{DejaVu Sans Mono}

	%===================================================================
	%	DOCUMENTO PROPIAMENTE DICHO
	%===================================================================

\title{CBFT Mecánica clásica}
\author{Fuerzas centrales}
\date{\today}

\begin{document}
\maketitle
\tableofcontents

% =================================================================================================
\section{Fuerzas centrales}
% =================================================================================================

Una fuerza central es aquella que cumple
\[
	\vb{F}(r) = f(r)\hat{r} = - \dpar{V}{r}
\]
de tal suerte que la parte cinética del lagrangiano es 
\[
	\Lag = \frac{1}{2}m \left( \dot{r}^2 + r^2 \dot{\theta}^2 + r^2 \sin(\theta)^2\dot{\phi}^2 \right)
\]

El momento angular $\vb{L}$ se conserva puesto que $\vb{\tau} = \vb{r} \times \vb{F}=0$. Como es 
$\vb{L} = \vb{r} \times \vb{p} = \vb{r} \times m\dot{\vb{r}} = cte$ entonces se sigue que $\vb{r},\vb{p}$
se hallan contenidos en el mismo plano.

Puedo pedir, sin pérdida de generalidad, que $\theta=\pi/2$ y entonces 
\[
	\Lag = \frac{1}{2}m \left( \dot{r}^2 + r^2 \dot{\theta}^2 \right) - V(r).
\]

Como $\phi$ es cíclica se tiene
\[
	\dpar{\Lag}{\dot{\phi}} = L = mr^2\dot{\phi}
\]
que no es otra cosa que la conservación del momento angular, información que puede ser llevada al
lagrangiano,
\[
	\Lag = \frac{1}{2}m \dot{r}^2 + \left[ \frac{L^2}{2 m r^2} - V(r) \right]
\]
donde el último corchete será lo que llamaremos un potencial efectivo $V_{eff}$,
\[
	\Lag = \frac{1}{2}m \dot{r}^2 + V_{eff}(r)
\]

La ecuación de Euler-Lagrange resulta en
\[
	m\ddot{r} - \frac{L^2}{mr^3} + \dpar{V}{r} = 0
\]
pero es más sencillo utilizar la conservación de la energía que explícitamente tiene la expresión
\[
	E = \frac{1}{2}m \dot{r}^2 + \frac{L^2}{2 m r^2} + V(r)
\]
desde la cual se puede integrar directamente la trayectoria $r=r(t)$ según
\[
	\dtot{r}{t} = \sqrt{ \frac{2}{m}\left( E - \frac{L^2}{2 m r^2} - V(r) \right)},
\]
aunque suele ser más útil la trayectoria en el espacio físico $r=r(\phi)$ o bien $\phi=\phi(r)$.
\[
	m r^2 \dtot{\phi}{t} = L \quad \longrightarrow m r^2 \dtot{\phi}{r} \dot{r} = L
\]
luego
\[
	\dot{r} d\phi = \frac{L}{m r^2} dr
\]
\[
	\int d\phi = \int \frac{L/mr^2}{\sqrt{ \frac{2}{m}\left( E - \frac{L^2}{2 m r^2} - V(r) \right)}}dr
\]

% =================================================================================================

\bibliographystyle{CBFT-apa-good}	% (uses file "apa-good.bst")
\bibliography{CBFT.Referencias} % La base de datos bibliográfica

\end{document}
