	\documentclass[10pt,oneside]{CBFT_book}
	% Algunos paquetes
	\usepackage{amssymb}
	\usepackage{amsmath}
	\usepackage{graphicx}
	\usepackage{libertine}
% 	\usepackage[bold-style=TeX]{unicode-math}
	\usepackage{lipsum}

	\usepackage{natbib}
	\setcitestyle{square}

	\usepackage{polyglossia}
	\setdefaultlanguage{spanish}


	\usepackage{CBFT.estilo} % Cargo la hoja de estilo

	% Tipografías
	% \setromanfont[Mapping=tex-text]{Linux Libertine O}
	% \setsansfont[Mapping=tex-text]{DejaVu Sans}
	% \setmonofont[Mapping=tex-text]{DejaVu Sans Mono}

	%===================================================================
	%	DOCUMENTO PROPIAMENTE DICHO
	%===================================================================

\begin{document}

\chapter{Pequeñas oscilaciones}

Es un formalismo para analizar el movimiento que realiza un sistema cuando está sometido a
ligeras perturbaciones en la posición de equilibrio.
Esto desarrollará un método sistemático para tratar todo tipo de problemas con muchos grados
de libertad pero en forma aproximada.

\subsection{Idea para un grado de libertad}

Para un grado de liberada la idea es que 

\includegraphics[scale=0.5]{images/fig_mc_oscil_1.jpg}

en un potencial $V(x)$ con un mínimo, es decir que cumple 
\[
	\dtot{V(x)}{x} = 0 ,\dtot[2]{V(x)}{x} > 0
\]
para algún $x_{eq}$, en la expresión de la energía
\be
	E = \frac 1 2 m \dot{x}^2 + V(x),
	\label{energia_1d}
\ee
se aproxima el potencial según\footnote{Nótese que esta es la expansión de Taylor en la cual el término lineal 
está justamente ausente porque la derivada primera en el punto es nula.}
\be
	V(x) \approx V_0 + \frac{1}{2} \left.\dtot[2]{V(x)}{x}\right|_{x_{eq}} (x-x_{eq})^2,
	\label{potencial_aproximado}
\ee
y si definimos $ k \equiv d^2V/dx^2|_{x_{eq}} $ se llega a 
\[
	E = \frac 1 2 m \dot{x}^2 + V_0 + \frac{1}{2} k (x-x_{eq})^2, 
\]
que derivada con respecto al tiempo resulta en 
\[
	m\ddot{x} + k (x-x_{eq}) = 0,
\]
la cual no es otra cosa que una ecuación de oscilador armónico, cuya solución general es
\[
	x(t) = A \cos (\omega t + \varphi ),
\]
donde $ \omega =  \sqrt{ k / m } $ y $ \varphi $ está asociada a la energía $E$. Ver Apéndice X para la resolución
de oscilador armónico.
\notamargen{Un apéndice más: oscilador armónico con término no homogéneo (usar 76R carpeta). Acá habría que llegar a despejar quién es
$\varphi$.}

El problema físico tiene dos constantes aunque la resolución presenta cuatro (dos complejos, con parte real e imaginaria).

Nótese que el desarrollo del potencial a orden dos equivale a una fuerza linealizada, merced a que $ m \ddot{x} = - dV/dx$.

\subsection{Varias variables}

En el caso de un potencial $V(\vb{x}_1, ...,\vb{x}_n)$ hay que hallar las raíces del mismo y luego desarrollar en torno a los puntos
de equilibrio. Se empieza desde 
\[
	\dpare{V}{\vb{x}}{x_{eq}} = 0,
\]
y habría que desarrollar 
\[
	V( \vb{x}_1, ...,\vb{x}_n ) = V( \vb{x}_1, ...,\vb{x}_n ) + 
	\frac 1 2 \sum_{i,j} \dparcru{V}{\vb{x}_i}{\vb{x}_j}(\vb{x}-\vb{x}_i)(\vb{x}-\vb{x}_j)
\]

No obstante, el problema se puede enfocar mejor en términos de las coordenadas generalizadas. Entonces, el potencial es
\[
	V(q_1,...,q_n) \approx V(q_1^0,...,q_n^0) + \sum_{i=1}^n \left. \dpar{V}{q_i} \right|_{q_i^0} (q_i - q_i^0)
		+ \frac{1}{2} \sum_{i,j=1}^n \left. \dparcru{V}{q_j}{q_i}\right|_{q_i^0}(q_i -q_i^0)(q_j -q_i^0)
\]
y la energía cinética,
\[
	T(q_1,...,q_n,\dot{q}_1,...,\dot{q}_n) \approx \frac{1}{2} \left( m(q_1^0,...,q_n^0) + \sum_{i=1}^n 
				\left. \dpar{m}{q_i} \right|_{q_i^0} (q_i - q_i^0) + ... \right) \sum_{i,j}^n \dot{q}_i\dot{q}_j
\]
[Esta expresión hay que revisarla y reubicarla!]

La energía cinética es 
\[
	T = \frac 1 2 \sum_{i,j} m_{ij}(q_1,...,q_n) \dot{q}_i \dot{q}_j
\]
donde $m_{ij}$ son los coeficientes de las coordenadas generalizadas y se desarrollarán en serie en torno al equilibrio (caracterizado
por un supraíndice $0$), es decir,
\[
	m_{ij} \approx m_{ij}( q_i^0, ..., q_n^0 ) + \sum_k \dpare{ m_{ij} }{ q_k }{q^0}( q_k - q_k^0 ).
\]

Estamos considerando que la energía cinética es $ T = T_2 $, pero cabría pensar que existe un $ T_0( q_1,...,q_n)$ y se lo 
sumaríamos en ese caso al potencial $V$. En el lagrangiano que consideraremos no está presente $T_1$; queremos un potencial
que no depende de las velocidades.

\begin{ejemplo}{Sobre el término $T_1$}

Para el caso de una masa fija, enhebrada en varilla que gira con velocidad angular $\omega$, el lagrangiano es
\[
	\Lag = \frac 1 2 m (\dot{r}^2 + r^2 \dot{\vp}^2),
\]
con energía
\[
	T = T_0 + T_2 = \frac{2}{2}mr^2\dot{\vp}^2 + \frac{2}{2}m\dot{r}^2
\]
donde $\dot{\vp} = \omega$ el último término no depende de la velocidad pero sí de la posición.
Es {\it como} un potencial que genera la fuerza ficticia.
\end{ejemplo}

\notamargen{Esta aproximación y formalismo sirve para un mínimo y un sistema que hace pequeños apartamientos
respecto de ese mínimo.}

Haciendo la aproximación consistente resulta 
\[
	\Lag = T - V = - \frac{1}{2} \sum_{i,j}^n \left. \dparcru{V}{q_j}{q_i}\right|_{q_i^0}(\eta_i)(\eta_j) +
		\frac{1}{2} \sum_{i,j}^n \left. m_{ij}\right|_{q_i^0} \dot{\eta}_i \dot{\eta}_j
\]
con $V_{ij} \equiv \partial^2 V / ( \partial q_i \partial q_j ) |_{q_i^0}, m_{ij} = m_{ij}|_{q_i^0}$, ambos simétricos, 
y donde se ha definido $\eta_i = q_i - q_i^0$, que es un apartamiento típico de la posición de equilibrio. 
Notemos que $\dot{q}_i = \dot{\eta}_i $. Nótese también que el término lineal en
la aproximación de $m_{ij}$ al verse multiplicado por el producto $\dot{q}_i\dot{q}_j$ es ya de orden cúbico por lo cual debe
descartarse para ser consistentes con las aproximaciones hechas en el potencial.

Con esta nomenclatura puede escribirse el {\it lagrangiano de pequeñas oscilaciones}
\[
	\Lag = \frac{1}{2} \sum_{i,j=1}^n m_{ij} \dot{\eta}_i \dot{\eta}_j - \frac{1}{2} \sum_{i,j=1}^n V_{ij} \eta_i \eta_j
\]
siendo ambas sumatorias formas bilineales cuadráticas reales y definidas positivas. Matricialmente,
\[
	\Lag = \frac{1}{2} \dot{\vb{\eta}}^t \mathbb{T} \dot{\vb{\eta}} - \frac{1}{2} \dot{\vb{\eta}}^t \mathbb{V} \dot{\vb{\eta}}
\]
y si ahora evaluamos las ecuaciones de Euler-Lagrange para este formalismo resulta que 
\[
	\frac{d}{dt}\left( \dpar{\Lag}{\dot{\eta}_k} \right) - \dpar{\Lag}{\eta_k} = 
		\frac{d}{dt} \left( \frac{1}{2} \sum_{i,j=1}^n m_{ij} \frac{d}{d\dot{\eta}_k}(\dot{\eta}_i \dot{\eta}_j) \right) - 
		\frac{1}{2} \sum_{i,j=1}^n V_{ij} \frac{d}{d\eta_k} (\eta_i \eta_j) = 0
\]
son $n$ ecuaciones diferenciales de Euler, 
\[
	\sum_{j=1}^n m_{kj} \ddot{\eta}_j + V_{kj} \eta_j = 0 \qquad k=(1,...,n).
\]

Esto es un oscilador armónico para cada partícula. Se puede pensar en todas las partículas unidas por resortes acoplados.

Se propone como solución 
\[
	\eta_j(t)  = A_j e^{i\omega t}
\]
de frecuencia $\omega$, idéntica para todas las partículas, tomando al final del proceso $\Re\{A_j e^{i\omega t}\}$ como 
solución física. Esta elección lleva a
\[
	\sum_{j=1}^n ( - \omega^2 m_{kj} + V_{kj} ) A_j = 0
\]
que equivale a
\[
	(\mathbb{V} -\omega^2\mathbb{T})\vb{A} = 0
\]
que no es otra cosa que un problema de autovalores y autovectores generalizado. Necesito
\[
	\left| \mathbb{V} -\omega^2\mathbb{T} \right| = 0
\]
lo cual me hará buscar un polinomio característico $P^n[\omega^2]$ de orden $n$ en $\omega^2$.
Así se trendrán $n$ valores para $\omega^2$ con $\omega^2_s \in \mathbb{R}$ y $\omega^2_s \geq 0$, que serán las
autofrecuencias o frecuencias propias $\omega^2_1, ...,\omega^2_n$. 

Para cada $\omega$ se tiene una solución
\[
	\eta_j^s = A_j^s e^{i\omega_s t}	 \qquad s=1,...,N
\]
pero el movimiento general será una combinación de todas las frecuencias,
\[
	\eta_j(t) = \sum_{s=1}^N c_s A_j^s e^{i\omega_s t}.
\]

En general, dado un $V=V(q_i)$ puede ser más fácil obtener explícitamente la serie de Taylor con 
$\partial^2 V/ \partial q_i \partial q_j |_{q_i^0}$ o bien cambiar variable $\eta = q_i - q_i^0$ y quedarse
con los términos cuadráticos en $\eta_i \eta_j$. Para la energía cinética $T=T(q,\dot{q})$ puede ser más
fácil evaluar $m_{ij}(q_i)|_{q_i^0}$ y quedarnos con los términos cuadráticos en $\dot{\eta}_i \dot{\eta}_j$.

Veamos la solución para una frecuencia dada,
\[
	\sum_j ( V_{kj} - \omega_s^2 m_{kj} ) A_j^s = 0
\]
y como usamos una raíz $\omega_s$ se tendrá una ecuación linealmente dependiente que tiraremos. Serán
ahora $N-1$ ecuaciones,
\[
	\sum_j ( V_{kj} - \omega_s^2 m_{kj} ) \frac{A_j^s}{A_1^s} = 0
\]
y definimos el cociente $a_j^s \equiv {A_j^s}/{A_1^s}$ al pasar dividiendo la amplitud del modo cuya frecuencia estamos
considerando. Entonces
\[
	\sum_j ( V_{kj} - \omega_s^2 m_{kj} ) a_j^s = - V_{k1} - \omega_s^2 m_{k1} \qquad k=1,...,N-1
\]
\notamargen{Acá sería bueno poner explícitamente hasta donde llega la sumatoria y explicitar qué $\omega$ se usa.}

Entonces como $N-1$ ecuaciones no homogéneas tienen solución real, entonces $a_j$ es un cociente real y todo los
$A_s^j$ tienen que tener la misma fase. [mmm?]
La fase viene determinada por las condiciones iniciales.

Veamos ahora que las frecuencias son reales. Para ello se multiplica por el complejo conjugado y se suma
\[
	\sum_k A_k^{s*} \sum_j V_{kj} A_j^s = \omega_s^2 \sum_k A_k^{s*} \sum_j m_{kj} A_j^s
\]
\[
	\sum_k A_k^{s} \sum_j V_{kj} A_j^{s*} = \omega_s^{2*} \sum_k A_k^{s} \sum_j m_{kj} A_j^{s*}
\]
y usando la simetría de $m_{kj}, V_{kj}$ se restan estas ecuaciones y se obtiene
\[
	0 = ( \omega^2_s - \omega^{2*}_s ) \sum_k \sum_j  A_k^{s*} m_{kj} A_j^{s}
\]
y como la doble sumatoria es no nula se sigue que las frecuencias son reales.
Incluso se puede despejar
\[
	\omega_s^2 = \frac{ \sum_k \sum_j  A_k^{s*} V_{kj} A_j^{s} }{ \sum_k \sum_j  A_k^{s*} m_{kj} A_j^{s} }
\]
Ambos, numerador y denominador son definidos positivos.
Si el numerador fuese negativo para alguna dirección, eso significa que en esa dirección será un máximo (sería una
especie de punto silla); pequeñas oscilaciones no valdrá en esa dirección.

Por otra parte, si se consideran dos frecuencias diferentes
\[
	0 = ( \omega^2_s - \omega^{2*}_p ) \sum_k \sum_j  A_k^{s*} m_{kj} A_j^{p}
\]
entonces lo que debe ser nulo es la doble sumatoria. Entonces, en la {\it métrica} dada por $m_{jk}$ $A_j$ y $A_k$
son perpendiculares.
Para determinar el $A_1$ (que era el parámetro que permanecía indeterminado) impongo
\[
	{A^t}^{p*} M A^p = 1
\]
y los $A_j$ se {\it consideran} reales pués todos tienen la misma fase y son los modos normales.
Se está pidiendo que de uno la norma en la métrica dada por $M$.

Si la raíz del polinomio $P^n$ tiene multiplicidad $k$, se tienen $k$ ecuaciones linealmente dependientes y hay que
arrojar al cesto de la basura $k$ ecuaciones.

Si construyo la matriz
\[
	A = \begin{pmatrix}
	A_1^1 & A_1^2 & ... \\
	A_2^1 &       &      \\
	...
	\end{pmatrix},
\]
donde cada columna de esta matriz es un autovector. Entonces se ve que esta matriz {\it diagonaliza} a $M$, i.e.
\[
	A^t M A = \begin{pmatrix}
	1 & 0 & ... \\
	0 &  1  &      \\
	...
	\end{pmatrix} = \mathbb{1}.
\]

Asimismo, como 
\[
	V A = \omega^2 M A,
\]
eso conduce a que 
\[
	A^t V A = \begin{pmatrix}
	\omega_1^2 & 0 & ... \\
	0 &  \omega_2^2  &      \\
	...
	\end{pmatrix}.
\]
\notamargen{Hay que repasar esto.}

\begin{ejemplo}{\bf Lo de las matrices}

Si $ A^{p*} M A^s =0 $ con $ p \neq s $ entonces $M$ está definiendo una métrica pués si $ M = \mathbb{1} $ entonces
\[
	A^{p*} \mathbb{1} A^s = A^{p*} A^s = 0,
\]
lo cual significa que $A^{p*}$ y $ A^s$ son perpendiculares.
 
\end{ejemplo}

\subsection{Expresión vectorial}

Vectorialmente es 
\[
	\vb{\eta}^s = \vb{A}_j^s e^{i\omega_s t} = \begin{pmatrix}
	                A_1 e^{i\omega_s t} \\
	                A_2 e^{i\omega_s t} \\
	                ... \\
	                A_N e^{i\omega_s t} \\
	               \end{pmatrix}
\]
para la frecuencia $\omega_s$, siendo cada uno un grado de libertad moviéndose con frecuencia $\omega_s$.

Luego, es
\[
	\vb{\eta}_{tot} = c_1 \vb{\eta}^1 + c_2 \vb{\eta}^2 + ... + c_N \vb{\eta}^N
\]
\[
	\vb{\eta}_{tot} = \begin{pmatrix}
				\eta_1 \\
				\eta_2 \\
				... \\
				\eta_n 
	                  \end{pmatrix}
	                 = \begin{pmatrix}
				c_1 A_1^1 e^{i\omega t} + c_2 A_1^2 e^{i\omega t} + ... + c_n A_1^n e^{i\omega t} \\
				c_1 A_2^1 e^{i\omega t} + c_2 A_2^2 e^{i\omega t} + ... + c_n A_2^n e^{i\omega t} \\
				... \\
				c_1 A_n^1 e^{i\omega t} + c_2 A_n^2 e^{i\omega t} + ... + c_n A_n^n e^{i\omega t}
	                  \end{pmatrix}
\]
entonces $\vb{A}^s$ es un modo normal de frecuencia $s$.
\[
	\vb{A}^s = \begin{pmatrix}
	            A_1^s \\
	            A_2^s \\
	            ... \\
	            A_n^s
	           \end{pmatrix}
	           e^{i\theta_0}
\]

La solución total ($j$ es el grado de libertad) se puede escribir 
\[
	\eta_j(t) = \sum_{s=1}^N c_s A_j^s e^{i\omega_s t}
\]
\[
	\vb{\eta}(t) = \sum_{s=1}^N c_s \vb{A}^s e^{i\omega_s t}
\]
y finalmente 
\[
	\vb{\eta}(t) = \Re \left\{ \sum_{s=1}^N c_s \vb{A}^s e^{i\omega_s t} \right\}
\]

Matricialmente,
\[
	\vb{A}^\dagger \mathbb{T} \vb{A} = 1
\]
siendo el $\dagger$ el traspuesto conjugado. Se pide que la norma (en la métrica dada por $\mathbb{T}$ de la unidad)
\[
	A^t \mathbb{T} A = \mathbb{1}
\]
lo cual significa que $A$ diagonaliza a $\mathbb{T}$, siendo 
\[
	A = \begin{pmatrix}
	     A_1^1 & A_1^2 & ... & A_1^n \\
	     A_2^1 & ... \\
	     A_n^1 & A_n^2 & ... & A_n^n 
	    \end{pmatrix}
\]
la matriz modal donde sus columnas son autovectores.

\[
	(\mathbb{V} - \omega^2 \mathbb{T})\vb{A} = 0
\]
interpolando a la matriz 
\[
	A^t \mathbb{V} A = \omega^2 A^t \mathbb{T} A= \omega^2 \mathbb{1}
\]

\subsection{Un cambio de coordenadas}

Se puede incluso realizar un cambio de coordenadas
\[
	\vb{\eta} = A \vb{\xi}
\]
tal que 
\[
	A^{n\times n} \xi^{n\times 1} \qquad \qquad  (A \vb{\xi} )^t = {\xi^t}^{1 \times n} {A^t}^{n \times n}
\]
y que se llaman coordenadas normales. Se resuelve el problema en estas coordenadas $\xi$ y luego se regresa a las
originales $\eta$
\[
	\Lag = \frac{1}{2} \dot{\vb{\eta}}^t \mathbb{T} \dot{\vb{\eta}} - \frac{1}{2} \dot{\vb{\eta}}^t \mathbb{V} \dot{\vb{\eta}}
\]
\[
	\Lag = \frac{1}{2} A^t\dot{\vb{\xi}}^t \mathbb{T} A \dot{\vb{\xi}} - \frac{1}{2} A^t\dot{\vb{\xi}}^t \mathbb{V} 
\dot{\vb{\xi}}
\]
\[
	\Lag = \frac{1}{2} \dot{\vb{\xi}}^t \mathbb{1} \dot{\vb{\xi}} - \frac{1}{2} \dot{\vb{\xi}}^t \omega^2 \mathbb{1} \dot{\vb{\xi}}
\]

\[
	\Lag = \frac{1}{2} \sum_i \dot{\vb{\xi}}_i^2 - \frac{1}{2} \sum_i \vb{\xi}_i^2 \omega^2_i 
\]

Los autovectores son los modos normales. Son $N$ osciladores armónicos independientes. Se pasa de un problema de muchas partículas
interactuantes a uno de $N$ partículas que no interactúan.

\[
	\frac{d}{dt}\left( \dpar{\Lag}{\dot{\xi}_i} \right)- \dpar{\Lag}{\xi_i} = \sum_i \ddot{\xi}_i + \omega^2_i \xi_i = 0 
\]
y son $N$ ecuaciones de Euler-Lagrange.
\[
	\sum_i ( -\omega^2 + \omega^2_i ) A_i = 0
\]
de modo que si $\omega^2 = \omega^2_i$ entonces
\[
	\xi_i = C_i e^{i\omega_i t}.
\]

\notamargen{Hay que consolidar este material disperso y confuso!}

\[
	\xi_\ell(t) = C_\ell \cos( \omega_\ell t + \vp_\ell )
\]
y entonces
\begin{align*}
	\eta_1(t) &= \sum_{\ell} A_1^\ell C_\ell \cos( \omega_\ell t + \vp_\ell ) \\
	\eta_2(t) &= \sum_{\ell} A_2^\ell C_\ell \cos( \omega_\ell t + \vp_\ell ) \\
	... & \\
	\eta_N(t) &= \sum_{\ell} A_N^\ell C_\ell \cos( \omega_\ell t + \vp_\ell )
\end{align*}
que son soluciones con $\omega_i \neq 0$. Son $\eta$ coordenadas normales y $\xi$ coordenadas colectivas [no es al revés?].

Digamos que en coordenadas normales
\[
	\xi_j = C_j e^{i \omega_j t}
\]
grados de libertad en $\xi$ (un grado de libertad es una $\omega$) y se desacoplan los grados de libertad
en lo que hace a $\omega_s$.
Por otro lado,
\[
	\eta_j = \sum_{s=1}^N c_s A_j^s e^{i \omega_j t}
\]
grados de libertad en $\eta$, un grado de libertad entonces es combinación lineal de todas las $\omega$.

Si $\omega=0$ es 
\[
	\xi_j = At + B 
\]
\[
	\eta_j = \sum_{s=1}^{N-1} c_s A_j^s e^{i \omega_j t} + A_j(Gt + D)
\]
siendo el último término asociado a la $\omega=0$.
Para volver atrás es 
\[
	A^{\dagger} \mathbb{T} A = \mathbb{1}
\]
y entonces 
\[
	A^{\dagger} \mathbb{T} \vb{\eta} = A^{\dagger} \mathbb{T} A \vb{\xi}  
\]
\[
	A^{\dagger} \mathbb{T} \vb{\eta} = \mathbb{1} \xi
\]
coordenadas normales en función de las de desplazamiento.

En conclusión podemos decir varias cosas,
\begin{itemize}
 \item Las frecuencias nulas están asociadas a momentos conservados.
 \item En coordenadas normales cada grado de libertad oscial con una frecuencia única (son $N$
	osciladores independientes)
 \item Las amplitudes cumplen
 \[ \vb{A}^s =
 \begin{pmatrix}
  a_1^s e^{i \phi_s} \\
  a_2^s e^{i \phi_s} \\
  ... \\
  a_n^s e^{i \phi_s}
 \end{pmatrix}
 \]
 donde tienen la misma fase los $A_j^s$ para toda frecuencia $\omega_s$
 \item Los modos normales pueden excitarse por separado (son ortogonales).
 \item Frecuencias iguales generarán modos normales que son físicamente los
 mismos. Son generados por la simetría del problema.
 \[
	\vb{A} = a_1(v_1) + a_2(v_2)
 \]
 si por ejemplo generan dos autovectores de esta forma.
\end{itemize}

\subsection{Coordenadas colectivas y normales}

\includegraphics[scale=0.5]{images/fig_mc_coord_normales_colectivas.jpg}

Las ecuaciones de Newton del sistema son
\[
	m_1 \ddot{x}_1 = k(x_2-x_1-\ell_0) \qquad \qquad m_2 \ddot{x}_2 = -k(x_2-x_1-\ell_0)
\]
que verifican
\[
	m_1 \ddot{x}_1 + m_2 \ddot{x}_2 = 0,
\]
y entonces
\[
	\ddot{x}_2 - \ddot{x}_1 = -k \left( \frac{1}{m_1} + \frac{1}{m_2} \right) ( x_2-x_1-\ell_0 )
\]

Definiendo $x_2-x_1 = x_{rel}$ se pasa de un problema de dos partículas acopaladas $(x_1,x_2)$ a otro de dos partículas
desacopladas; una oscila y la otra se traslada,
\[
	\mu \ddot{x}_{rel} + k (x_{rel} - \ell_0 ) = 0 \qquad \ddot{x}_{cm} = 0
\]
y $x_{rel}, x_{cm}$ son coordenas colectivas, pero no corresponden a un movimiento real de un sistema.
Tendré dos problemas separados que pueden, dado el caso, excitarse por separado.

En el caso de $N$ osciladores, si hay algún $\omega_i=0$ se tendrá
\[
	\Lag = \sum_{i=1}^N \frac{ 1 }{ 2} \dot{\xi}_i
\]
y como $\ddot{\xi}_i = 0 $ entonces $\xi(t) = At + B$ es solución y 
\[
	\eta_N(t) = \sum_\ell A_n C_\ell \cos (\omega_\ell t + \vp_\ell) + \sum_k A_n^k (Bt + D)
\]
donde el primer término es por $\omega_\ell\neq 0$ y el segundo por $\omega_k=0$.

% ~~~~~~~~~~~~~~~~~~~~~~~~~~~~~~~~~~~~~~~~~~~~~~~~~~~~~~~~~~~~~~~~~~~~~~~~~~~~
\begin{ejemplo}{\bf Aro fijo con bolas engarzadas}
 
\includegraphics[scale=0.5]{images/fig_mc_problema_aro_modos_normales_1.jpg}  

El lagrangiano correspondiente a este {\it setup} es 
\[
	\Lag = \frac 1 2 \left[ \dot{\theta}^2_1 + \dot{\theta}^2_2 + \dot{\theta}^2_3 + \dot{\theta}^2_4 \right] m \ell^2 
	- \frac 1 2 k \ell^2 \left[ ( \theta_2 - \theta_1 )^2 + ( \theta_3 - \theta_2 )^2 + ( \theta_4 - \theta_3 )^2 +
	( \theta_1 - \theta_4 )^2 \right]
\]
Este lagrangiano, como está, ya {\it es} de pequeñas oscilaciones. En efecto, $\theta_2^0 = \theta_1^0 $ de modo que 
$\theta_2 - \theta_1 = \eta_2 - \eta_1 = \theta_2 - \theta_2^0 + \theta_1^0 - \theta_1$. 
Luego,
\[
	V = \begin{pmatrix}
		2k\ell^2 & - k\ell^2 & 0 & -k\ell^2 \\
		\\
		-k\ell^2 & 2 k\ell^2 & -k\ell^2 & 0 \\
		\\
		0 & - k\ell^2 & 2k\ell^2 & -k\ell^2 \\
		\\
		-k\ell^2 & 0 & -k\ell^2 & 2k\ell^2 
	    \end{pmatrix}
\]
donde los ceros reflejan la inexistencia de resorte entre dichas partículas. Entonces
\[
	V - \omega^2 M = \begin{pmatrix}
		2k\ell^2 - m \omega^2 \ell^2 & - k\ell^2 & 0 & -k\ell^2 \\
		\\
		-k\ell^2 & 2 k\ell^2 - m \omega^2 \ell^2 & -k\ell^2 & 0 \\
		\\
		0 & - k\ell^2 & 2k\ell^2 - m \omega^2 \ell^2& -k\ell^2 \\
		\\
		-k\ell^2 & 0 & -k\ell^2 & 2k\ell^2 - m \omega^2 \ell^2
	    \end{pmatrix}
\]
 
Ahora hay que calcular el determinante de esta matriz $V - \omega^2 M$, que luego de desarrollar y usar el método que más le
gusta al dilegencioso lector permite arribar a
\[
	P(\omega) = \mathrm{det}(V - \omega^2 M) = 
	m\ell^2\omega(2k\ell^2 - m\ell \omega^2)^2( m\ell\omega^2-4k\ell^2),
\]
cuyas raíces son:
\[
	\omega_1^2 = 2\frac{k}{m} \qquad 
	\omega_2^2 = 2\frac{k}{m} \qquad 
	\omega_3^2 = 0 \qquad 
	\omega_4^2 = 4\frac{k}{m}
\]

En este ejemplo se conserva el momento angular, de manera que hubiese sido razonable obtener una frecuencia nula 
asociada como de hecho apareció en $\omega_3$. Resolvamos ahora ese modo. Será
\[
	2 k \ell^2 A_1^3 - k \ell^2 A_2^3 - k \ell^2 A_4^3 = 0,
\]
\[
	- k \ell^2 A_1^3 + 2 k \ell^2 A_2^3 - k \ell^2 A_3^3 = 0,
\]
\[
,	- k \ell^2 A_2^3 + 2 k \ell^2 A_3^3 - k \ell^2 A_4^3 = 0
\]
y resulta $A_1^3 = A_2^3 = A_3^3 = A_4^3 $.

Entonces
\[
	\vb{A}^3 = a \begin{pmatrix}
	              1 \\
	              1 \\
	              1 \\
	              1 
	             \end{pmatrix} 
\]
donde $ a = 1 / ( 2 \ell \sqrt{m} ) $ y se da 
\[
	\vb{A}^{3\dagger} M \vb{A}^3 = \mathbb{1}
\]
Este es el modo normal de $\omega^2_3=0$, que se ve dibujado bajo estas líneas
 
\includegraphics[scale=0.5]{images/fig_mc_problema_aro_modos_normales_2.jpg} 

Para la frecuencia $\omega_4$ es 
\[
	- 2 k \ell^2 A_1^4 - k \ell^2 A_2^4 - k \ell^2 A_4^4 = 0,
\]
\[
	k \ell^2 A_1^4 - 2 k \ell^2 A_2^4 - k \ell^2 A_3^4 = 0,
\]
\[
,	- k \ell^2 A_2^4 - 2 k \ell^2 A_3^4 - k \ell^2 A_4^4 = 0
\]
de manera que 
\[
	\vb{A}^4 = a \begin{pmatrix}
	              1 \\
	              -1 \\
	              1 \\
	              -1 
	             \end{pmatrix} 
\]
donde $a = 1/( 2 \ell \sqrt{m})$. El dibujo asociado será

\includegraphics[scale=0.5]{images/fig_mc_problema_aro_modos_normales_3.jpg} 

Para las frecuencias {\it mellizas} $\omega_1,\omega_2 $ es 
\[
	- k \ell^2 A_1^1 - k \ell^2 A_3^1 = 0,
\]
\[
	- k \ell^2 A_2^1 - k \ell^2 A_4^1 = 0,
\]
o bien 
\[
	A_1 = A_3 = 0 \qquad A_2 = A_4,
\]
y
\[
	A_2 = A_4 \qquad A_1 = -A_3
\]
y consecuentemente 
\[
	\vb{A}^1 = a \begin{pmatrix}
	              0 \\
	              1 \\
	              0 \\
	              -1 
	             \end{pmatrix} 
	             \qquad \qquad 
	\vb{A}^2 = a \begin{pmatrix}
	              1 \\
	              0 \\
	              -1 \\
	              0 
	             \end{pmatrix} 
\]
con $ a = 1 /(\ell\sqrt{2m})$. Los dibujos siguientes ilustran los movimientos esperados

\includegraphics[scale=0.5]{images/fig_mc_problema_aro_modos_normales_4.jpg} 

Con respecto a lo de aquí arriba son situaciones físicas iguales (si cambio masas no será la misma situación).
La introducción de $M$ y $m$ (ver figurillas siguientes) rompe la degeneración y serán modos normales pero de diferente 
frecuencia.
Pero podría haberse elegido, ver bajo estas líneas,

\includegraphics[scale=0.5]{images/fig_mc_problema_aro_modos_normales_5.jpg} 

Para el caso siguiente 

\includegraphics[scale=0.5]{images/fig_mc_problema_aro_modos_normales_6.jpg} 

esta configuración no conserva el momento angular.

Si tomamos $M=2m$ entonces podemos considerar momento angular nulo y obtengo un modo normal

\includegraphics[scale=0.5]{images/fig_mc_problema_aro_modos_normales_7.jpg} 
 
\[
	\begin{pmatrix}
	 1 & 1 & 1 & 1 
	\end{pmatrix}
	\begin{pmatrix}
	 1 & 0 & 0 & 0 \\
	 0 & 2 & 0 & 0 \\
	 0 & 0 & 1 & 0 \\
	 0 & 0 & 0 & 2
	\end{pmatrix}
	\begin{pmatrix}
	 1 \\
	 -1/2 \\
	 1 \\
	 -1/2
	\end{pmatrix}
\] 

\notamargen{Este ejemplo se continuó en dos clases consecutivas de modo que puede haber algún pise.}

\includegraphics[scale=0.35]{images/fig_mc_problema_aro_modos_normales_8.jpg} 

En resumen 
\[
	\omega_3^2 = 0 \qquad \qquad A^3 = \frac{1}{2\sqrt{m}\ell}
						\begin{pmatrix} 1 \\ 1 \\ 1 \\ 1 \end{pmatrix}
\]
\[
	\omega_4^2 = \frac{4k}{m} \qquad \qquad A^4 = \frac{1}{2\sqrt{m}\ell}
						\begin{pmatrix} 1 \\ -1 \\ 1 \\ -1 \end{pmatrix}
\]
\[
	\omega_1^2 = \omega_2^2 = \frac{2k}{m}
\]
\[
	A^1 = \frac{1}{\sqrt{2m}\ell}
		\begin{pmatrix} 0 \\ 1 \\ 0 \\ -1 \end{pmatrix}
	A^2 = \frac{1}{2\sqrt{2m}\ell}
		\begin{pmatrix} 1 \\ 0 \\ -1 \\ 0 \end{pmatrix}
\]
\[
	A^t M A = 1
\]

El problema dependerá de cómo excitemos al sistema. Si $\eta = A \xi $ entonces con $\omega_i \neq 0$ es
\[
	\xi_i = C_i \cos (\omega_i t + \vp_i),
\]
pero con $\omega_i = 0$ se tendrá
\[
	\xi_i = B_i t + D_i.
\]

Luego, como $\dot{\eta} = A \dot{\xi} $ de tal modo que 
\[
	A^t M \eta = A^t M A \xi \qquad \qquad  A^t M \dot{\eta} = A^t M A \dot{\xi} = \mathbb{1} \dot{xi}
\]

Las condiciones iniciales serán 
\[
	\eta_1(t=0) = \theta_0 \qquad \eta_i(t=0) = 0, i \neq 1 
\]
y los $\dot{\eta}_i(t=0)$ están en función de las $\eta$ y quiero pasarlas a $\xi$.

\[
	A^t M = \begin{pmatrix}
	        0 & \frac{1}{\sqrt{2m}\ell} & 0 & -\frac{1}{\sqrt{2m}\ell} \\
	        \frac{1}{\sqrt{2m}\ell} & 0 & -\frac{1}{\sqrt{2m}\ell} & 0 \\
	        \frac{1}{2\sqrt{m}\ell} & \frac{1}{2\sqrt{m}\ell} & \frac{1}{2\sqrt{m}\ell} & \frac{1}{2\sqrt{m}\ell} \\
	        \frac{1}{2\sqrt{m}\ell} & -\frac{1}{2\sqrt{m}\ell} & \frac{1}{2\sqrt{m}\ell} & -\frac{1}{2\sqrt{m}\ell} 
	        \end{pmatrix}
	        \begin{pmatrix}
	         m & 0 & 0 & 0 \\
	         0 & m & 0 & 0 \\
	         0 & 0 & m & 0 \\
	         0 & 0 & 0 & m 
	        \end{pmatrix}
\]
mientras que las posiciones iniciales serán 
\[
	\begin{pmatrix}
	0 \\
	\\
	\frac{\sqrt{m}}{\sqrt{2}\ell} \theta_0 \\
	\\
	\frac{\sqrt{m}}{\sqrt{2}\ell} \theta_0 \\
	\\
	\frac{\sqrt{m}}{\sqrt{2}\ell} \theta_0
	\end{pmatrix} =
	\begin{pmatrix}
	\\
	C_1 \cos\vp_1 \\
	\\
	C_2 \cos\vp_2 \\
	\\
	Bt + D \\
	\\
	C_4 \cos\psi_4 \\
	\\
	\end{pmatrix}
\]
y las velocidades iniciales por su parte,
\begin{align*}
	0 & = - C_1 \omega_1 \sin \vp_1 \\
	0 & = - C_2 \omega_2 \sin \vp_2 \\
	0 & = B \\
	0 & = - C_3 \omega_4 \sin \vp_4 
\end{align*}
[tal vez un typo acá arriba]

Al separar de las posiciones de equilibrio se excitan modos que no son el cero, y tendrán momento angular nulo; el
único modo que dará momento angular no nula y por ende generará rotación es el asociado a $\omega_3^2$.

Según se ve el modo uno tampoco se está excitando,
\[
	\vp_2 = 0 \qquad C_2 = \frac{\sqrt{m}}{2\ell} \theta_0
\]
\[
	\eta = A \xi = \begin{pmatrix}
	        0 & \frac{1}{\sqrt{2m}\ell} & \frac{1}{2\sqrt{m}\ell} & \frac{1}{2\sqrt{m}\ell} \\
	        \\
	        \frac{1}{\sqrt{2m}\ell} & 0 & \frac{1}{2\sqrt{m}\ell} & -\frac{1}{2\sqrt{m}\ell} \\
	        \\
	        0 & -\frac{1}{2\sqrt{m}\ell} & \frac{1}{2\sqrt{m}\ell} & \frac{1}{2\sqrt{m}\ell} \\
	        \\
	        -\frac{1}{\sqrt{2m}\ell} & 0 & \frac{1}{2\sqrt{m}\ell} & -\frac{1}{2\sqrt{m}\ell} 
	        \end{pmatrix}
	        \begin{pmatrix}
		0 \\
		\\
		\frac{\sqrt{m}}{\sqrt{2}\ell} \theta_0 \cos(\omega_2 t)\\
		\\
		\frac{\sqrt{m}}{\sqrt{2}\ell} \theta_0 ( Bt + D ) \\
		\\
		\frac{\sqrt{m}}{\sqrt{2}\ell} \theta_2 \cos(\omega_4 t)
	        \end{pmatrix}
\]
\notamargen{Chequear estas ecuaciones vectoriales!}

Esto nos da los $\eta$ solución.
Notemos que para el dibujo de los modos normales no hace falta hallar el valor de las constantes $1/(\sqrt{2m}\ell)$, etc. de
normalización.

\notamargen{La normalización está en $A^t M A$.}

Si hay simetrías en el problema ($V,M$ son invariantes frente a cierta transformación)
\[
	VA^\ell = \omega^2 M A^\ell
\]

Una matriz,
\[
	B = \begin{pmatrix}
	0 & 1 & 0 & 0 \\
	0 & 0 & 1 & 0 \\
	0 & 0 & 0 & 1 \\
	1 & 0 & 0 & 0 \\
	\end{pmatrix}
\]
y su inversa $B^{-1}$ transforman un versor en otro

\includegraphics[scale=0.5]{images/fig_mc_problema_aro_modos_normales_9.jpg} 

Pasa según las flechas rojas el sistema; pero ante este cambio el lagrangiano es invariante.
Las matrices $B$ y $B^{-1}$ son rotaciones (la inversa es en el otro sentido) y que dejan invariante, como se dijo, al lagrangiano.
Se ve que pasan $1000 \to 0001$ y $0100 \to 1000$,
\[
	BVB^t = V \qquad BVB^tB = VB \qquad BV - VB = 0
\]
Los autovectores de $V$ se pueden elegir dentro de los de $B$.

Si $VA^\ell = \omega^2 M A^\ell$ entonces
\[
	BVB^t BA^\ell = \omega^2 BMB^t BA^\ell
\]
\[
	VBA^\ell = \omega^2 MBA^\ell,
\]
aunque para esto necesito que no sean degenerados los autovectores.
Entonces, podemos escribir así:
\[
	- \omega^2 M + V = (2 k \ell^2 - \omega^2 m \ell^2 ) \mathbb{1} - B k \ell^2 - B^t K \ell^2
\]

Sea la misma $B$, que cumple la propiedad
\[
	B\hat{n} = \hat{n}+1 \qquad \qquad B^\dagger\hat{n} = \hat{n}-1
\]
con 
\[
	\hat{n} = \begin{pmatrix}
	           0 \\
	           ...\\
	           1\\
	           ...\\
	           0
	          \end{pmatrix}
\]
y $\hat{n}=\hat{N}$ y $\hat{N}+1=\hat{1}$ y $\hat{0}=\hat{N}$.
Ahora estoy considerando el problema con $N$ bolas

\includegraphics[scale=0.5]{images/fig_mc_problema_aro_modos_normales_10.jpg} 

\notamargen{La idea acá es tomar el lagrangiano de pequeñas oscilaciones y ver qué transformaciones dejan invariantes a las matrices del mismo.}

Si tomo un vector $u$ y le aplico $B$,
\[
	\vb{u} = \sum \euler^{i\theta n} \hat{n},
\]
entonces
\[
	B \vb{u} = \sum \euler^{i\theta n} (\hat{n} + 1) = \euler^{-i \theta } \sum \euler^{i\theta (n+1)} (\hat{n} + 1)
\]
y $\theta$ es un parámetro que puede valer cualquier cosa; para ajustar [los bordes?] si veo que 
\[
	\euler^{i\theta (N+1)} (\hat{N} + 1) \qquad \text{en} n=N
\]
y
\[
	\euler^{i\theta 1} (\hat{1}) 
\]
que es lo mismo que 
\[
	\sum \euler^{i\theta n} \hat{n} \qquad \text{en} n=1
\]
entonces puedo vincular
\[
	\euler^{i\theta (N+1)} = \euler^{i\theta },
\]
de lo cual deducimos que
\[
	\theta = \frac{2k\pi}{N}.
\]

Ahora operamos
\[
	( 2 k \ell^2 \mathbb{1} - B k \ell^2 - B^\dagger k \ell^2 ) u = \omega^2 m \ell^2 \mathbb{1} u,
\]
\[
	2 k \ell^2 u - k \ell \euler^{-i\theta} u - k \ell^2 \euler^{i\theta} u
\]
lo que lleva a
\[
	\left[ 2 k \ell^2 - k \ell 2 ( \euler^{i\theta} + \euler^{-i\theta} ) \right] u = \omega^2 m \ell^2 u
\]
o bien 
\[
	\omega^2 = \frac{2k}{m}\left( 1 - \cos \frac{2k'\pi}{N} \right)
\]
con $k'= 0, \pm 1, \pm 2, ..., \pm \frac{N-1}{2}$.

Con $N=4$ es 
\[
	k'=0 \qquad \to \quad \omega^2=0
\]
que corresponde a la no degenerada y representa la rotación de todo el {\it bodoque}.
Acá están las cuatro autofrecuencias
\[
	k' = \pm 1 \qquad \omega^2_1 = \omega_{-1}^2 = \frac{2k}{m}
\]
\[
	k' = \pm 2 \qquad \omega^2_2 = \omega_{-2}^2 = \frac{4k}{m}
\]
donde que sea el $+$ o el $-$ en la frecuencia depende de si $N$ es impar la mitad son positivos y la mitad negativos.
Para $N$ par el que sobra lo tomo positivo o negativo (cualquiera).

Con los autovectores no degenerados tengo los autovalores del modo real. Para los siguientes (como son degenerados) tendré 
que tomar una combinación lineal de ellos.

Los autovectores acompañantes serán:
\[
	k' = 0 \qquad \begin{pmatrix}
	               1 \\
	               1 \\
	               1 \\
	               1
	              \end{pmatrix}
\]
\[
	k' = 1 \qquad \begin{pmatrix}
	               \euler^{i \pi/2 } \\
	               \euler^{i \pi } \\
	               \euler^{i 3\pi/2 } \\
	               \euler^{i 2\pi }
	               \end{pmatrix} =
		       \begin{pmatrix}
	               i \\
	               -1 \\
	               -i \\
	               0
	               \end{pmatrix}
\] 
\[
	k' = 1 \qquad \begin{pmatrix}
	               \euler^{-i \pi/2 } \\
	               \euler^{-i \pi } \\
	               \euler^{-i 3\pi/2 } \\
	               \euler^{-i 2\pi }
	               \end{pmatrix} =
		       \begin{pmatrix}
	               -i \\
	               -1 \\
	               i \\
	               0
	               \end{pmatrix}
\]

Con estos dos debo generar independientes en combinación lineal
\[
	\oplus \to  \begin{pmatrix}
	 0 \\
	 -2 \\
	 0 \\
	 0
	\end{pmatrix}
\]
\[
	\ominus \to \begin{pmatrix}
	 2 \\
	 0 \\
	 -2 \\
	 0
	\end{pmatrix}
\]
donde el último se multiplica por $i$.

\notamargen{Esto es muy sketchi, habría que completarlo.}
\end{ejemplo}




% =================================================================================================
\section{Oscilaciones viscosas}
% =================================================================================================

El lagrangiano usual de pequeñas oscilaciones,
\[
	\Lag = \frac 1 2 \sum_{i,j} \dot{\eta}_i\dot{\eta}_j m_{ij} -  \frac 1 2 \sum_{i,j} {\eta}_i {\eta}_j V_{ij},
\]
lleva naturalmente a 
\[
	\sum_j m_{ij} \ddot{\eta}_j + V_{ij} \eta_j = 0
\]
cuya resolución es la diagonalización simultánea de $M$ y $V$.
Si hay fuerzas viscosas presentes se obtiene
\be
	\sum_j m_{ij} \ddot{\eta}_j + V_{ij}\eta_j + B_{ij}\dot{\eta}_j = 0
	\label{ec_viscosas}
\ee

Se necesitaría diagonalizar tres formas cuadráticas simultáneas, las tres matrices, lo cual
en general no es posible.
Este problema \eqref{ec_viscosas} no tiene en general modos normales. Entonces \eqref{ec_viscosas}
no se podrá convertir en osciladores independientes.
La resolución utiliza la misma idea; proponer $A\euler{\lambda t}$ donde $\lambda \in \mathbb{C}$
entonces
\[
	\mbox{det}\left\{ \mathbb{V} + \lambda^2 \mathbb{M} + \lambda \mathbb{B}\right\} = 0
\]
debe solicitarse para hallar la resolución.


\begin{ejemplo}{\bf Problema 14 Método de Lagrange}

\includegraphics[scale=0.5]{images/fig_mc_lagrangebola_1.jpg}

El lagrangiano es
\[
	\Lag = \frac 1 2 m ( (b-a)^2 \dot{\beta}^2 + \dot{z}^2 ) + 
	\frac 1 2 I ( \dot{\vp}^2 + \dot{\theta}^2 + \dot{\psi}^2 + 2 \dot{\psi} \dot{\vp} \cos\theta ) 
	- m g z
\]
y como la velocidad $ \vb{v}_p $ es nula, se tiene 
\[
	\vb{v}_p = 0 = \vb{V}_{cm} + \vb{\Omega} \times a \hat{\vp}
\]
que lleva a
\[
	0 = (b-a)\dot{\beta}\hat{\beta} + \dot{z}\hat{z} +
	[\omega_\rho\hat{\rho} + \omega_\beta\hat{\beta} + \omega_z\hat{z} ]\times a\hat{\rho}
\]
\[
	0 = (b-a)\dot{\beta}\hat{\beta} + \dot{z}\hat{z} + a\omega_z\hat{\beta} - a\omega_\beta\hat{z} 
\]

La condición de rodadura es
\[
	\begin{cases}
	 (b-a)\dot{\beta} + a \omega_z = 0 \\
	 \dot{z} - a \omega_\beta = 0
	\end{cases}
\]
y como la velocidad en cartesianas es $ \vb{\Omega} = \Omega_x \hat{x} + \Omega_y \hat{y} + \Omega_z \hat{z} $,
la conversión a los ejes del problema es
\[
	\hat{\rho} = \cos \beta \hat{x} + \sin \beta \hat{y} \qquad 
	\hat{\beta} = -\sin \beta \hat{x} + \cos \beta \hat{y}
\]
o bien
\[
	\hat{x} = \cos \beta \hat{\rho} - \sin \beta \hat{\beta} \qquad 
	\hat{y} = \sin \beta \hat{\rho} + \cos \beta \hat{\beta}
\]
entonces
\[
	\vb{\Omega} = \omega_\rho \hat{\rho} + \omega_\beta \hat{\beta} + \omega_z \hat{z}
\]
donde 
\[
	\omega_\rho = \Omega_x \cos\beta + \Omega_y \sin\beta \qquad 
	\omega_\beta = \Omega_y \cos\beta + \Omega_x \sin\beta 
\]

Luego de algún álgebra
\[
	\omega_\rho = \dot{\psi} \sin\theta \sin(\vp-\beta) + \dot{\theta} \cos(\vp-\beta)
\]
\[
	\omega_\beta = -\dot{\psi} \sin\theta \cos(\vp-\beta) + \dot{\theta} \sin(\vp-\beta)
\]
\[
	\omega_z = \dot{\psi} \cos\theta + \dot{\vp}
\]

Pero como
\[
	(b-a) \beta + a \dot{\vp} + a \dot{\psi} \cos\theta = 0 ,
\]
\[
	\dot{z} -a \dot{\theta} \sin(\vp-\beta) + a\dot{\psi}\sin\theta\cos(\vp -\beta) = 0,
\]
conviene utilizar multiplicadores de Lagrange,
\[
	\dtot{}{t}\left( \dpar{\Lag}{\dot{q}_k} \right) - \dpar{\Lag}{q_k} = \sum_{\ell=1}^2 \lambda_\ell a_{\ell k}
\]
lo cual lleva a 
\[
	m(b-a)^2 \ddot{\beta} = \lambda_1 (b-a) \qquad \qquad 
	m \ddot{z} + m g = \lambda_2
\]
\[
	I \ddot{\vp} + I \dtot{}{t}( 2 \dot{\psi} \cos\theta ) = a \lambda_1
\]

Haciendo gradiente en los vínculos,
\[
	\lambda_1 ( [b-a] \delta\beta + a\delta\vp + a\cos\theta\delta\psi) = 0
\]
\[
	\lambda_2 (\delta z - a\sin(\vp-\theta)\delta\theta + a\sin\theta \cos(\vp-\beta)\delta\psi) = 0
\]
y entonces
\[
	I \ddot{\theta} +  I \dot{\psi} \dot{\vp} \sin\theta ) =
	- a \lambda_2 \vp \sin (\vp - \beta)
\]
\[
	I \ddot{\psi} +  I \dtot{}{t}( \dot{\psi} \cos\theta ) =
	\lambda_1 a \cos\theta + \lambda_2 \vp \sin \theta \cos(\vp - \theta)
\]

Tenemos siete ecuaciones con siete incógnitas. Una sugerencia para resolverlo alternativamente
es a través de las ecuaciones de Newton,
\[
	m\dot{\vb{V}_{cm}} = \vb{f} + \vb{N} \qquad \qquad 
	I \dot{\vb{\omega}} = \vb{t}
\]

\includegraphics[scale=0.5]{images/fig_mc_lagrangebola_2.jpg}

\[
	\dtot{\omega}{t} = \left.\dtot{\omega}{t}\right| + \dot{\beta}\hat{\beta} \times \vb{\omega}
\]
lo cual nos debería conducir a algo de la forma
\[
	(I + ma^2)\ddot{\omega}\vp + \dot{\vp} I \omega_\vp = 0
\]
y sale que $\omega_\rho = \dot{\vp} \omega_\vp$ siendo $\dot{\vp}$ y $\dot{z}$ constantes.

\end{ejemplo}

% ~~~~~~~~~~~~~~~~~~~~~~~~~~~~~~~~~~~~~~~~~~~~~~~~~~~~~~~~~~~~~~~~~~~~~~~~~~~
\begin{ejemplo}{\bf Problema de la molécula diatómica}

\includegraphics[scale=0.5]{images/fig_mc_molecula_1.jpg}

Acá hay que escribir el potencial con cuidado,
\[
	V = V_\alpha(\alpha) + V_{Cso}(r) + V_{OH}(r'),
\]
donde 
\[
	V_\alpha(\alpha) = \frac{k\ell^2}{2}(\pi - \alpha)^2
\]
y $\ell$ es un $r, r'$ de equilibrio.
\[
	V_{Cso}(r) = 4 \epsilon 
	\left[ \Frac{\sigma}{r}^{12} - \Frac{\sigma}{r}^6\right]
\]
\[
	V_{OH}(r') = \frac{V_{Cso}(r') }{15}
\]
y según se ve ya está separado el mismo.
Calculamos las derivadas del potencial,
\[
	V_{\alpha\alpha} = \dpar[2]{V}{\alpha} = k\ell^2 
\]
y de 
\[
	\dpar{V_{Cso}}{r}(r) = 0
\]
sale un $r_{eq}$ que cumple $r_{eq} = \sigma 2^{1/12}\equiv \ell$ y luego
\[
	\left. V_{Cso}{''}\right|_{eq} = 24 \epsilon 
	\left[ \frac{26}{r^2} \Frac{\sigma}{r}^{12} - \frac{7}{r^2} \Frac{\sigma}{r}^6\right] = k_r
\]
donde los términos con $\sigma$ equivalen a $4\ell^2$ y $2\ell^2$. Además,
\[
	V_{rr} = k_r \qquad V_{r'r'} = \frac{k_r}{15}
\]

Esto define
\[
	V = \begin{pmatrix}
	 V_{\alpha\alpha} & 0 & 0 & 0 \\
	 0 & V_{rr} & 0 & 0 \\
	 0 & 0 & V_{r'r'} & 0  \\
	 0 & 0 & 0 & ...
	 \end{pmatrix}
\]

En general tenemos más grados de libertad que tres. Ubicamos el centro de masa en
el Cesio por ser muy masivo. Entonces pierdo tres grados de libertad y me quedan
seis. Ignoro rotación, y otra cosa más [¿?]

\includegraphics[scale=0.5]{images/fig_mc_molecula_3.jpg}

\includegraphics[scale=0.5]{images/fig_mc_molecula_2.jpg}

Restan cuatro grados de libertad $r, \vp, r', \beta$.
\[
	\dot{X}_0^2 = \dot{r}^2 + r^2 \dot{\vp}^2
\]
\[
	\vb{X}_0 = r \cos (\vp) \hat{x} + r \sin (\vp) \hat{y}
\]
\[
	\vb{X}_H = \vb{X}_0 + r' \cos (\vp + \beta) \hat{x} + r' \sin (\vp + \beta) \hat{y}
\]
y el cuadrado es
\[
	\dot{X}_H^2 = \dot{X}_0^2 + \dot{r'}^2 + {r'}^2 ( \dot{\vp} + \dot{\beta} )^2
\]
\begin{multline}
	2 \dot{r} \dot{r'} [ \cos \vp \cos (\beta + \vp) + \sin\vp \sin(\beta + \vp) ] + \\
	2 {r} {r'} [ \sin(\beta + \vp) \sin\vp (\dot{\beta} + \dot{\vp}) \dot{\vp} + \cos(\beta + \vp) \cos\vp \dot{\vp} (\dot{\beta} + \dot{\vp}) ] + \\
	2 \dot{r} {r'} [ \cos (\beta + \vp) \sin \vp - \cos \vp \sin (\beta + \vp)] (\dot{\beta} + \dot{\vp}) + \\
	2 {r} \dot{r'} [ \dot{\vp} \cos\vp \sin(\beta + \vp) - \dot{\vp} \sin\vp \cos (\beta + \vp)] 
\end{multline}
donde los últimos dos se {\it mueren} al aproximar.
Finalmente el lagrangiano de pequeñas oscilaciones resulta en
\[
	\Lag = \frac {17} 2 m ( \dot{r}^2 + r^2 \dot{\vp}^2 ) + 
	\frac m 2 ( \dot{r'}^2 + {r'}^2( \dot{\vp} + \dot{\beta} ) ) +
	\frac m 2 ( 2 \dot{r} \dot{r'} + 2 \ell^2 ( \dot{\beta} + \dot{\vp} ) \dot{\vp} ) - 
	\frac{k_r r^2}{2} - \frac{k_{r'} {r'}^2}{2} - \frac{k_\beta \beta^2}{2}
\]
en donde los $r^2$ y ${r'}^2$ son ambos $\ell^2$.

Definimos
\[
	\eta = \begin{pmatrix}
	       r - \ell \\
	       r' - \ell \\
	       \beta - 0 \\
	       \vp - 0
	      \end{pmatrix} \qquad     
	\qquad  \qquad 
	\dot{\eta} = \begin{pmatrix}
	       \dot{r} \\
	       \dot{r'} \\
	       \dot{\beta} \\
	       \dot{\vp}
	      \end{pmatrix} \qquad 
\]
siendo la posición de equilibrio 

\includegraphics[scale=0.5]{images/fig_mc_molecula_4.jpg}

La matriz del potencial $V$ es 
\[
	V =  \begin{pmatrix}
		k_r	 &	0	&	0	&	0	\\
		\\
		0 	&	k_{r'} 	&	0	&	0	\\
		\\
		0	&	0	&	k_{\alpha}  &	0  \\
		\\
		0	&	0	&	0  	&   0
	    \end{pmatrix}
\]
donde $k_r = 15 k_{r'}$ y $M_0 = 16 M_H$ (check!).

Y ahora hay que armar la energía cinétic $T$ que resulta
\[
	T = \begin{pmatrix}
		17 m	 &	m 	&	0	&	0	\\
		\\
		m 	&	m 	&	0	&	0	\\
		\\
		0	&	0	&	20 m \ell^2  &	2 m \ell^2  \\
		\\
		0	&	0	&	2 m \ell^2  	&   m \ell^2
	    \end{pmatrix}
\]

Luego, el lagrangiano
\[
	\Lag = \frac{ 1 }{ 2 } \dot{\vb{\eta}}^t \mathbb{T} \dot{\vb{\eta}} -
	\frac{1}{2} \dot{\vb{\eta}}^t \mathbb{V} \dot{\vb{\eta}}
\]
puede transformarse a 
\[
	\Lag = \frac{ 1 }{ 2 } \dot{\vb{\xi}}\dot{\vb{\xi}} - \frac{1}{2} \dot{\vb{\xi}} \omega^2 \dot{\vb{\xi}}
\]
y en estas nuevas coordenadas,
\[
	\ddot{\xi}_i + \omega^2 \xi_i = 0
\]
y la solución son osciladores armónicos.

Para ello debería hallar $A$, donde 
\[
	\bar{\eta} = A \bar{\xi} 
\]
que verifica $A\mathbb{T}A^t = \mathbb{1}$ y $A\mathbb{V}A^t = \omega$ de modo que necesito $|\omega^2\bar{T} - 
\bar{V}|=|M|=0$ donde es
\[
	M = \begin{pmatrix}
		17 m \omega^2 - k_r	&	m \omega^2	&	0	&	0	\\
		\\
		m \omega^2	&	m \omega^2 - k_{r'}	&	0	&	0	\\
		\\
		0	&	0	&	20 m \ell^2 \omega^2 - k_\alpha	 &	2 m \ell^2 \omega^2 \\
		\\
		0	&	0	&	2 m \ell^2 \omega^2 	&   m \ell^2 \omega^2
	    \end{pmatrix}
\]

Gracias a los bloques se hace menos trabajo, pués los autovalores de la matriz serán los de cada bloque.
El primer bloque es
\[
	\begin{pmatrix}
	 \displaystyle 17 \frac{m\omega^2}{k_{r'}} -\frac{k_r}{k_{r'}} &  \displaystyle \frac{m\omega^2}{k_{r'}}  \\
	 \\
	 \displaystyle \frac{m\omega^2}{k_{r'}} &  \displaystyle \frac{m\omega^2}{k_{r'}} - 1
	 \end{pmatrix} =
	\begin{pmatrix}
	\\
	 \displaystyle 17 \lambda - 15 & \lambda  \\
	 \\
	  \displaystyle \lambda & \lambda - 1 \\
	  \\
	 \end{pmatrix}
\]
y entonces
\[
	(17\lambda -15)(\lambda -1) - \lambda^2 =  \lambda^2 - 2\lambda + \frac{15}{16} = 0
\]
resulta en $\lambda_{1,2} = 5/4, 3/4$ de modo que 
\[
	\omega^2_1 = \frac 5 4 \frac {k_{r'}}{m} \qquad \qquad 
	\omega^2_2 = \frac 3 4 \frac {k_{r'}}{m}
\]

El otro bloque es
\[
	\begin{pmatrix}
	 20 \lambda - 1 & 2 \lambda  \\
	 2 \lambda & \lambda
	 \end{pmatrix} = 0 
\]
de manera que 
\[
	(20 \lambda - 1) \lambda  - 4\lambda^2 = 0 
\]
lo cual conduce a $\lambda_{3,4} = 1/16, 0 $. 

Los coeficientes de normalización serán
\[
	a_4 = c_4 \begin{pmatrix}
	       0 \\
	       0 \\
	       0 \\
	       1
	      \end{pmatrix} \qquad 
	a_3 = c_3 \begin{pmatrix}
	       0 \\
	       0 \\
	       1 \\
	       -2
	      \end{pmatrix} \qquad 
	a_1 = c_1 \begin{pmatrix}
	       1 \\
	       -5 \\
	       0 \\
	       0
	      \end{pmatrix} \qquad 
	a_2 = c_2 \begin{pmatrix}
	       1 \\
	       3 \\
	       0 \\
	       0
	      \end{pmatrix}
\]

Aquí la matriz se separó en bloques y entonces los autovalores serán independientes en cada
bloque; no se mezclan entre sí.

\includegraphics[scale=0.5]{images/fig_mc_molecula_5.jpg}

Habría que calcular ahora la matriz modal
\[
	\bar{a}_1^t \mathbb{T} \bar{a}_1 = 1,
\]
que conduce a 
\[
	\begin{pmatrix}
	17 & 1 \\
	1  & 1
	\end{pmatrix}
	\begin{pmatrix}
	1 \\
	-3
	\end{pmatrix}
	c_1^2 m = 1
\]
lo que arroja $c_1 = 1 /(4\sqrt{2m})$
\[
	\begin{pmatrix}
	1 & 3 
	\end{pmatrix}
	\begin{pmatrix}
	17 & 1 \\
	1  & 1
	\end{pmatrix}
	\begin{pmatrix}
	1 \\
	3
	\end{pmatrix}
	c_2^2 m = 1
\]
que da $ c_2 = c_1 $. Luego $c_3 = 1 / (4 \sqrt{m} \ell^2 )$ y $c_4$ quedó vacante.
La matriz modal resulta
\[
	A = \begin{pmatrix}
		\frac{1}{4\sqrt{2m}}	&	\frac{1}{4\sqrt{2m}}	&	0	&	0	\\
		\\
		\frac{-5}{4\sqrt{2m}}	&	\frac{3}{4\sqrt{2m}}	&	0	&	0	\\
		\\
		0	&	0	&	\frac{1}{4\sqrt{m}\ell}	&	0	\\
		\\
		0	&	0	&	\frac{-2}{4\sqrt{m}\ell}	&	\frac{1}{\sqrt{m}\ell}
	    \end{pmatrix}
\]

Supongamos ahora el sistema moviéndose de acuerdo con
\[
	\bar{X} = 1 \bar{\xi}_2 + 3 \bar{\xi}_3
\]
y entonces podemos pasar a las coordenadas originales,
\[
	A \bar{\xi}_2 + 3 A \bar{\xi}_3
\]

\notamargen{Hay que revisar la notación aquí.}

\end{ejemplo}


% \bibliographystyle{CBFT-apa-good}	% (uses file "apa-good.bst")
% \bibliography{CBFT.Referencias} % La base de datos bibliográfica

\end{document}

