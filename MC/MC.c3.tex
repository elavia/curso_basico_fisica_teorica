	\documentclass[10pt,oneside]{CBFT_book}
	
	% Algunos paquetes
	
	\usepackage{amssymb}
	\usepackage{amsmath}
	\usepackage{graphicx}
	\usepackage{libertine}
	\usepackage{lipsum}
	\usepackage[numbers]{natbib}
% 	\usepackage{natbib}
	\setcitestyle{square}

	\usepackage{polyglossia}
	\setdefaultlanguage{spanish}


	\usepackage{CBFT.estilo} % Cargo la hoja de estilo

	% Tipografías
	% \setromanfont[Mapping=tex-text]{Linux Libertine O}
	% \setsansfont[Mapping=tex-text]{DejaVu Sans}
	% \setmonofont[Mapping=tex-text]{DejaVu Sans Mono}

	%===================================================================
	%	DOCUMENTO PROPIAMENTE DICHO
	%===================================================================

% \title{CBFT Mecánica clásica}
% \author{Simetrías}
% \date{\today}

\begin{document}
% \maketitle
% \tableofcontents
\chapter{Simetrías}

% =================================================================================================
\section{Constantes de movimiento y simetrías}
% =================================================================================================

Si en las ecuaciones de Euler-Lagrange
\[
	\frac{d}{dt}\left( \dpar{\Lag}{\dot{q}_j} \right) - \dpar{\Lag}{q_j}  = 0, 
\]
se daba el caso de que $ \Lag $ no dependía de $ q_j $ entonces
$ \partial {\Lag} /\partial {q_j}  = 0  $ y
\[
	\frac{d}{dt}\left( \dpar{\Lag}{\dot{q}_j} \right) = 0
\]
significa que 
\[
	\dpar{\Lag}{\dot{q}_j} \equiv p_j
\]
es una constante ($\dot{p}_j=0$).

Por otra parte, si $\delta q_i$ es traslación rígida en una dirección $\hat{n}$ entonces 
\[
	p_i = \vb{P}\cdot\hat{n} \qquad \text{ y } \qquad Q_j= \vb{F}\cdot\hat{n}. 
\]	
En cambio, si $\delta q_i$ es una rotación rígida en torno a un eje $\hat{n}$ se tiene 
\[
p_i = \vb{L}\cdot\hat{n} \qquad  \text{ y } \qquad Q_j= \vb{\Tau}\cdot\hat{n}.
\]

En estos dos casos
\[
	\dpar{T}{q_i} = 0
\]
puesto que:
\begin{itemize}
 \item Como $T$ depende de las velocidades (y no de las coordenadas) no depende del origen y por lo tanto no 
varía ante una traslación rígida (que es un cambio de origen).
 \item Como $T$ es un escalar no cambia ante una rotación.
\end{itemize}

% \[
% 	\dpar{\Lag}{q_i} = 0 = \dpar{T}{q_i} - \dpar{V}{q_i} = 0
% \]
Luego, si $V \neq V(\dot{q})$ (el potencial $V$ no depende explícitamente de las velocidades) entonces las ecuaciones 
de Euler-Lagrange adoptan la forma
% \[
% 	\frac{d}{dt}\left( \dpar{T}{\dot{q}_j} \right) + \dpar{V}{q_j}  = 0 
% \]
\[
	\frac{d}{dt}\left( \dpar{T}{\dot{q}_j} \right) = - \dpar{V}{q_j}   
\]
\[
	\frac{d}{dt}\left( p_j \right) = - \dpar{V}{q_j}   
\]
y entonces 
\[
	\dot{p}_j = -\dpar{V}{q_j}   
\]
es la fuerza total proyectada en la dirección $\hat{n}$.

\notamargen{Acá parecen estar separadas los movimientos rígidos del hecho de que V sea de las coordenadas solamente. 
En un caso tenemos $\dot{p}=0$ y en otro $\dot{p} = -\partial V / \partial q$.
Creo que lo del potencial sería para las otras coordenadas no afectadas por la simetría?.}

Para examinar constantes de movimiento podemos ver primreo las variables cíclicas. Sin embargo, si elegimos otras 
coordenadas tal vez no aparezca la constante de movimiento como coordenada cíclica (aunque por supuesto sigue 
existiendo dicha constante).

% =================================================================================================
\section{El teorema de Noether}\index{Noether, teorema de}
% =================================================================================================

Si existe una transformación continua $q_i \longrightarrow q_i + \delta q_i$ que deje invariante el
$\Lag$ entonces hay una constante de movimiento asociada a dicha transformación.

La transformación es 
\[
	q_i \longrightarrow q_i' = q_i + \delta q_i
\]
y cumple 
\[
	\Lag(q_i, \dot{q}_i , t) = \Lag(q_i', \dot{q}_i' , t) =
	\Lag(q_i[q_i',t], \dot{q}_i[\dot{q}_i',t] , t)
\]
y así si consideramos una variación a $t$ fijo,
\[
	\delta \Lag = \sum_i \dpar{\Lag}{q_i}\delta q_i + \dpar{\Lag}{\dot{q}_i}\delta \dot{q}_i =
	\sum_i \dpar{\Lag}{q_i}\delta q_i + \frac{d}{dt}\left( \dpar{\Lag}{\dot{q}_i}\delta q_i \right)
	- \frac{d}{dt}\left( \dpar{\Lag}{\dot{q}_i} \right) \delta q_i = 0
\]
\[
	\delta \Lag = \sum_i \left[ \dpar{\Lag}{q_i} - \frac{d}{dt}\left( \dpar{\Lag}{\dot{q}_i} \right) \right]
	\delta q_i + \frac{d}{dt}\left( \dpar{\Lag}{\dot{q}_i}\delta q_i \right) = 0
\]
pero como el primer término del RHS es nulo por las ecuaciones de Euler-Lagrange tenemos que 
\[
	\delta \Lag = \frac{d}{dt}\left( \sum_i \dpar{\Lag}{\dot{q}_i}\delta q_i \right)  = 0,
\]
lo que está dentro del paréntesis es la cantidad conservada. 
Recordemos que 
\[
	\delta q_i = q'_i - q_i 
\]
y una traslación infinitesimal es 
\[
	\vb{r}_i' - \vb{r}_i = \delta \vb{r}. 
\]

La variable cíclica es un caso particular de teorema de Noether, pero hay constantes de movimiento que 
no provienen de ninguna simetría.
\[
	\frac{d}{dt}\left( \sum_i \dpar{\Lag}{\dot{q}_i} ( \delta \alpha \hat{n}\times \vb{r}_i ) \right) 
\]
\[
	\frac{d}{dt}\left( \delta\alpha \sum_i \vb{p}_i \times \vb{r}_i  \right) =
	\delta\alpha \frac{d}{dt}\left(  \sum_i \vb{p}_i \times \vb{r}_i  \right) = 0
\]
siendo $\delta \alpha \equiv \epsilon$ un parámetro infinitesimal.
Para $k$ grados de libertad
\begin{align*}
	q'_i &= q_i + \underbrace{\epsilon_i g_i(q_1,...,q_n,t)}_{\delta q} \\
	... \\
	q'_k &= ...
\end{align*}
\[
	\vb{r}_i' = \vb{r}_i + \delta\vb{r} \quad \textrm{traslación rígida}
\]
\[
	\vb{r}_i' = \vb{r}_i + \delta\alpha \; \hat{n}\times\vb{r}_i \quad \textrm{rotación rígida}
\]
o también 
\[
	\delta \vb{r} \times \vb{r}
\]

$T$ es invariante siempre frente a (por ser un escalar)
\[
	T = T' 
\]
entonces habrá que examinarlo.
Constatemos que 
\[
	V = V(|\vb{r}_i - \vb{r}_j|)
\]
es invariancia ante una traslación rígida, y
\[
	V = V(x_1,x_2)
\]
es una invariancia de traslación en $x_3$.

$\Lag$ tendrá como constante un momento lineal si $V$ es invariante frente a traslación.
$\Lag$ tendrá como constante un momento angular si $V$ es invariante frente a rotación.
$\Lag$ tendrá como constante una combinación si $V$ es invariante frente a una roto-traslación.

Otra construcción posible es 
\[
	\delta \Lag = 0
\]
\[
	\Lag( q_i , \dot{q}_i, t ) - \Lag(q_i' , \dot{q}_i' , t) = 0 
\]
pidiendo que $d\Lag = 0$ llego a 
\[
	\sum \left\{ \frac{d}{dt}\left( \dpar{\Lag}{\dot{q}_i} \delta q \right) - 
	\frac{d}{dt}\left( \dpar{\Lag}{\dot{q'}_i} \delta q' \right)  \right\} = 0
\]
\notamargen{Las primas están mal. Hay que pensar una construcción adecuada.
Queda odd.}
\[
	\sum \left\{ \frac{d}{dt}\left( \dpar{\Lag}{\dot{q}_i} \delta q \right) - 
	\frac{d}{dt}\left( \dpar{\Lag}{\dot{q'}_i} \delta q \right) -
	\frac{d}{dt}\left( \dpar{\Lag}{\dot{q'}_i} \sum_\ell^s \epsilon_\ell g_i^\ell \right) \right\} = 0
\]
y podemos usar que 
\[
	\dpar{\Lag}{\dot{q'}_i} = \dpar{\Lag}{\dot{q}_i}
\]
pues $g\neq g(t)$ y es todo a tiempo fijo. Se tiene 
\[
	q' = q + \delta q
\]
\[
	q_i' = q_i + \sum_\ell^s \epsilon_\ell g_i^\ell
\]
siendo esta la transformación general
\[
	\delta q_i' = \delta q_i + \sum_\ell^s \epsilon_\ell g_i^\ell
\]

Extraemos también que 
\[
	\dpar{\Lag}{\dot{q'}_i} \sum_\ell^s \epsilon_\ell g_i^\ell = C
\]

Se puede pensar también como que $\Lag$ es invariante ante la transformación infinitesimal
$\delta q$
\[
	\delta\Lag = 0 = \sum_i^N \dpar{\Lag}{q_i}\delta q_i + \dpar{\Lag}{\dot{q}_i}\delta \dot{q}_i
\]
\[
	\delta\Lag = 0 = \sum_i^N \left[ \dpar{\Lag}{q_i} - \frac{d}{dt}\left( \dpar{\Lag}{\dot{q}_i} \right)
	\right]\delta q_i  + \sum_i^N \frac{d}{dt}\left( \dpar{\Lag}{\dot{q}_i} \delta q_i \right)  = 0
\]
siendo el primer término nulo, y siendo lo que se conserva lo que aparece en el segundo término,
donde 
\[
	\delta q_i =  \sum_\ell^s \epsilon_\ell g_i^\ell(q_1,q_2,...,q_n)
\]
Finalmente 
\[
	\delta \Lag = 0 = \frac{d}{dt}\left( \sum_i^N \dpar{\Lag}{\dot{q}_i} \delta q_i \right) 
\]


% =================================================================================================


% \bibliographystyle{CBFT-apa-good}	% (uses file "apa-good.bst")
% \bibliography{CBFT.Referencias} % La base de datos bibliográfica

\end{document}
