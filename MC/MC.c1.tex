	\documentclass[10pt,oneside]{CBFT_book}
	
	% Algunos paquetes
	\usepackage{amsmath}
	\usepackage{amssymb}
	\usepackage{graphicx}
	\usepackage{libertine}
	\usepackage{lipsum}
	\usepackage[numbers]{natbib}
	\setcitestyle{square}


	\usepackage{polyglossia}
	\setdefaultlanguage{spanish}

	\usepackage{CBFT.estilo} % Cargo la hoja de estilo
	

	% Tipografías
	% \setromanfont[Mapping=tex-text]{Linux Libertine O}
	% \setsansfont[Mapping=tex-text]{DejaVu Sans}
	% \setmonofont[Mapping=tex-text]{DejaVu Sans Mono}

	%===================================================================
	%	DOCUMENTO PROPIAMENTE DICHO
	%===================================================================

% \title{CBFT Mecánica clásica}
% \author{Mecánica lagrangiana}
% \date{\today}

\begin{document}
% \maketitle
% \tableofcontents
\chapter{Conceptos de mecánica newtoniana}

Tal vez sea una simplificación, pero no una muy terrible, decir que el curso de mecánica clásica
busca reemplazar la mecánica basada en las ecuaciones de Newton,
\[
	\vb{F} = m \vb{a} 
\]
por un \emph{formalismo} más poderoso y que se podrá aplicar luego a otros campos.
Este formalismo constituye el corazón de la mecánica clásica.

El contenido de este capítulo forma un núcleo básico de los resultados de le mecánica newtoniana que necesitaremos 
tener a mano para lo subsiguiente (leyes de conservación del momento lineal, momento angular y energía) así como 
ciertos rudimentos mínimos de la matemática usual en la resolución de los problemas.

% =================================================================================================
\section{Leyes de conservación}
% =================================================================================================

Repasaremos a continuación las leyes de conservación fundamentales de la mecánica para sistemas de partículas.

\subsection{Momento lineal}

La segunda ley de Newton se podía escribir en función del momento lineal de una partícula de masa $ m $ como
\[
	\dtot{\vb{p}}{t} = \vb{F}
\]
siendo $ \vb{p} = m\vb{v} $ el momento de la partícula y $ \vb{F} $ la fuerza total que actuaba sobre la misma.
Si el resultado de las fuerzas sobre la partícula era nulo entonces se tiene que $ \vb{p} = cte. $ (el momento lineal 
es una constante de movimiento).

En el caso de un sistema de $N$ partículas como el mostrado en la Figura \ref{fig_mc_leyes_cons} el momento total del 
sistema es la suma de los momentos individuales, es decir
\[
	\vb{P} = \Sum{i=1}{N} \vb{p}_i = \Sum{i=1}{N} m_i \vb{v}_i = \Sum{i=1}{N} m_i \dtot{\vb{x}_i}{t}
\]
luego la segunda ley para el sistema serán las $ N $ ecuaciones
\[
	\dtot{\vb{P}}{t} = \Sum{i=1}{N} m_i \dtot[2]{\vb{x}_i}{t} = \Sum{i=1}{N} \vb{F}_i
\]
donde $ \vb{F}_i$ es la fuerza total sobre la partícula $i$-ésima que puede descomponerse según
\be
	\vb{F}_i = \vb{F}_i^\text{ext} + \Sum{ j \neq i }{N} \vb{F}_{ij}
	\label{descomp_fuerzas}
\ee
siendo $ \vb{F}_i^\text{ext} $ las fuerzas debidas a agentes externos y $\vb{F}_{ij}$ la fuerza sobre la partícula $i$ 
debido a la partícula $j$.

\begin{figure}[hbt]
	\begin{center}
	\includegraphics[width=0.6\textwidth]{images/fig_mc_leyes_cons.pdf}
	\end{center}
	\caption{Sistema de partículas de masas $m_i$ con sus correspondientes vectores de
	posición $\vb{x}_i$. La partícula $m_1$ tiene además indicado su vector velocidad $\vb{v}_1$.}
	\label{fig_mc_leyes_cons}
\end{figure} 

Entonces 
\[
	\dtot{\vb{P}}{t} = \Sum{i=1}{N} m_i \dtot[2]{\vb{x}_i}{t} = \Sum{i=1}{N} \vb{F}_i^\text{ext} + 
	\Sum{i=1}{N} \Sum{ j \neq i }{N} \vb{F}_{ij}
\]
pero el último término del RHS es nulo puesto que por cada sumando $ \vb{F}_{ij} $ también aparece el sumando 
$ \vb{F}_{ji} $ y por acción y reacción estas fuerzas tienen la misma dirección y sentido opuesto, i.e.
\[
	\vb{F}_{ij} = - \vb{F}_{ji}.
\]

De esta forma la ley de conservación para el sistema es 
\[
	\dtot{\vb{P}}{t} = \Sum{i=1}{N} \vb{F}_i^\text{ext} = \vb{F}^\text{ext}_\text{total}
\]
y el momento $ \vb{P} $ del sistema se conserva si la resultante de todas las fuerzas externas es nula. 

Definiendo el vector de posición del centro de masa como 
\[
	\vb{x}_\text{cm} = \frac{\sum_i m_i \vb{x}_i }{\sum_i m_i} = \frac{\sum_i m_i \vb{x}_i }{M}
\]
donde $ M $ es la masa del sistema, se tiene el resultado clásico de que
\[
	\dtot{}{t}( M \vb{x}_\text{cm} ) = \Sum{i=1}{N} m_i \vb{v}_i = M \vb{v}_\text{cm} = \vb{P},
\]
el sistema como un todo tiene un momento total que puede asociársele al de una única partícula {\it centro de masa} de 
masa $M$ y que se mueve con velocidad $ \vb{v}_\text{cm} $.

Si $\vb{P}$ se conserva, entonces $ \vb{v}_\text{cm} $ es una constante, el sistema posee un punto (el centro de masas) 
que se mueve con velocidad constante sin importar qué tan complejo sea el movimiento del conjunto total.

% ~~~~~~~~~~~~~~~~~~~~~~~~~~~~~~~~~~~~~~~~~~~~~~~~~~~~~~~~~~~~~~~~~~~~~~~~~~~~~~~~~~~~~~~~~~~~~~~~~~~~~~~~~~~~~~~~~~~~~
\subsection{Momento angular}

El momento angular de una partícula con momento lineal \vb{p} es 
\[
	\vb{l} = \vb{x} \times \vb{p} = m \: \vb{x} \times \vb{v}.
\]
% El hecho de entre el vector de posición de la partícula en su definición implica que el momento angular
% dependerá del origen del sistema de coordenadas elegido y por ende también su conservación. 
En la Figura 
\ref{fig_mc_mom_ang_particula} se ilustra sobre la trayectoria de una partícula el vector momento angular. 
La variación temporal del momento angular,
\[
	\dtot{\vb{l}}{t} = \dtot{\vb{x}}{t} \times \vb{p} + \vb{x} \times \dtot{\vb{p}}{t} 
\]
se reduce al segundo término, puesto que $ d\vb{x}/dt = \vb{v} $ es paralela a $ \vb{p} $, y
se tiene finalmente el resultado conocido
\be
	\dtot{\vb{l}}{t} = \vb{x} \times \dtot{\vb{p}}{t} = \vb{x} \times \vb{F} = \vb{\tau}
	\label{conserv_mom_ang}
\ee
de que la variación del momento angular es el torque $\vb{\tau}$ causado por la fuerza $ \vb{F} $ 
que actúa sobre la partícula.

\notamargen{Cambiar en el dibujo f por F. Igualmente habría que ser consistente con qué quiero decir
para las mayúsculas y qué para las minúsculas.}
\begin{figure}[hbt]
	\begin{center}
	\includegraphics[width=0.6\textwidth]{images/fig_mc_mom_ang_particula.pdf}	
	\end{center}
	\caption{Una partícula de masa $m$ se desplaza en una trayectoria. En un punto \vb{x} de la misma se
	indican su velocidad \vb{v}, su momento angular \vb{l} y la fuerza \vb{F} a la que está sometida y el
	torque resultante \vb{\tau} por esa fuerza.
	El momento angular es perpendicular al plano (en marrón) definido por los vectores \vb{x} y \vb{v} mientras que 
	el torque lo es al plano (en gris) definido por \vb{x} y \vb{F}.}
	\label{fig_mc_mom_ang_particula}
\end{figure} 

Dado que la definición de $ \vb{l} $ y de $ \vb{\tau} $ implica el vector de posición $ \vb{x} $ se sigue que ambas 
magnitudes dependen de la elección del origen del sistema de coordenadas. 
Es decir que una determinación de $ \vb{l} $ y $ \vb{\tau} $ tiene sentido únicamente con respecto a un cierto origen 
de coordenadas.

De la ecuación \eqref{conserv_mom_ang} se deduce que si la fuerza es siempre paralela al vector de posición de una 
partícula ($\vb{F} \parallel \vb{x}$) entonces el momento angular \vb{l} se conserva puesto que el torque es 
$\vb{\tau}=0$ en ese caso. Es lo que se llama una fuerza central.
\notamargen{Habría que destacar lo de fuerza central con un dibujo. Es importante.}

\subsubsection{Momento angular para un sistema de partículas}

Si ahora tenemos un sistema de $N$ partículas el momento angular correspondiente (con respecto a un dado origen de
coordenadas) será
\[
	\vb{L} = \Sum{i=1}{N} \: \vb{x}_i \times \vb{p}_i
\]

De manera equivalente, la variación temporal es 
\[
	\dtot{\vb{L}}{t} = \Sum{i=1}{N} \vb{x}_i \times \vb{F}_i
\]
y si utilizamos la descomposición \eqref{descomp_fuerzas} para la fuerza $\vb{F}_i$ resulta
\[
	\dtot{\vb{L}}{t} = \Sum{i=1}{N} \vb{x}_i \times \vb{F}_i^\text{ext}  +
	\Sum{i=1}{N} \Sum{j\neq i}{N}  \vb{x}_i \times \vb{F}_{ij}
\]

Es claro\footnote{Nota \ref{nota_suma_ineqj}} que el segundo término puede expresarse de manera equivalente como 
\[
	\Sum{i=1}{N} \Sum{j\neq i}{N}  \vb{x}_i \times \vb{F}_{ij} =
	\frac{1}{2} 
	\Sum{i=1}{N} \Sum{j \neq i}{N}  \left[ \vb{x}_i \times \vb{F}_{ij} + \vb{x}_j \times \vb{F}_{ji} \right]
\]
y aceptando que las fuerzas internas son pares acción-reacción se tiene 
\[
	\Sum{i=1}{N} \Sum{j\neq i}{N}  \vb{x}_i \times \vb{F}_{ij} =
	\frac{1}{2} 
	\Sum{i=1}{N} \Sum{j \neq i}{N}  \left[ \vb{x}_i - \vb{x}_j \right] \times \vb{F}_{ij},
\]
de manera que la derivada del momento angular total es 
\be
	\dtot{\vb{L}}{t} = \Sum{i=1}{N} \vb{x}_i \times \vb{F}_i^\text{ext}  +
	\frac{1}{2} \Sum{i=1}{N} \Sum{j \neq i}{N}  \left[ \vb{x}_i - \vb{x}_j \right] \times \vb{F}_{ij} 
	\label{dL_sistema}
\ee

La conservación de \vb{L},
\[
	\dtot{\vb{L}}{t} = 0	
\]
requiere entonces que las fuerzas externas sean centrales, lo cual anula el primer término en \eqref{dL_sistema},
y que se verifique 
\[
	\vb{F}_{ij}  \parallel ( \vb{x}_i - \vb{x}_j ),
\]
es decir que la fuerza sobre $i$ ejercida por la partícula $j$ tenga la dirección del vector que une las dos 
partículas, para anular el segundo término de \eqref{dL_sistema}.

\begin{figure}[htb]
	\begin{center}
	\includegraphics[width=0.4\textwidth]{images/fig_mc_parstrong.pdf}	
	\end{center}
	\caption{Principio de acción y reacción fuerte para dos partículas de masas $m_i$ y $m_j$.}
	\label{fig_mc_parstrong}
\end{figure} 

Esto establece lo que se llama un ``principio de acción y reacción {\it fuerte}''; las fuerzas son iguales y opuestas 
(de esto se trata el principio de acción y reacción), pero además colineales.
Dadas dos partículas del sistema cualesquiera con posiciones $ \vb{x}_i, \vb{x}_j $ y de masas $ m_i, m_j $, como se 
muestra en la Figura \ref{fig_mc_parstrong}, la fuerza $\vb{F}_{ij}$ sobre $i$ debido a $j$ debe estar contenida en la 
dirección del vector $ \vb{x}_i - \vb{x}_j $ lo cual le otorga las dos posibilidades indicadas por las flechas rojas 
gruesas. Para la fuerza $\vb{F}_{ji}$ el razonamiento es, por supuesto, idéntico.

\notamargen{Acá hay más para extraer: poner un gráfico con lo que no puede pasar. Poner un código de colores para
las flechas, puesto que si son iguales y opuestas las fuerzas están hermanadas las externas por un lado y las internas
por el otro.}

La existencia de un principio de acción y reacción fuerte sobreviene [es una consecuencia?] de la naturaleza puntual de 
los cuerpos. De no ser puntuales se tendrá principio de acción y reacción a secas.

Existe otra descomposición interesante para el momento angular \vb{L} de un sistema de $N$ partículas en términos de 
sus distancias al centro de masas.

Para cada partícula $i$-ésima con posición $ \vb{x}_i $ y velocidad $ \vb{v}_i $ definimos una coordenada $ \vb{x}_i' $ 
y una velocidad $\vb{v}_i' $ en términos de la posición \vb{X} y velocidad \vb{V} del centro de masa, ver Figura 
\ref{fig_mc_angularmom}, de acuerdo a
\[
	\vb{x}_i = \vb{X} + \vb{x}_i' \qquad \vb{v}_i = \vb{V} + \vb{v}_i',
\]
es decir que consideramos coordenadas respecto al centro de masa.

\begin{figure}[hbt]
	\begin{center}
	\includegraphics[width=0.6\textwidth]{images/fig_mc_angularmom.pdf}	
	\end{center}
	\caption{}
	\label{fig_mc_angularmom}
\end{figure} 

\notamargen{Actualizar el $X_{cm}$ en el gráfico y poner el origen O.}

En términos de estas nuevas variables primadas el momento angular es
\[
	\vb{L}_O = \Sum{i=1}{N} \vb{x}_i \times \vb{p}_i = 
	\Sum{i=1}{N} (\vb{X} + \vb{x}_i') \times m_i (\vb{V} + \vb{v}_i')
\]
\[
	\vb{L}_O = \Sum{i=1}{N} ( \vb{X} \times m_i \vb{V}  + \vb{X} \times m_i \vb{v}_i'
	+ \vb{x}_i' \times m_i \vb{V} 	+ \vb{x}_i' \times m_i \vb{v}_i' )
\]

Como la posición del centro de masa es
\be
	\vb{X} = \frac{1}{M} \Sum{i=1}{N} m_i \vb{x}_i  
	\label{R_cm}
\ee
se tendrá 
\[
	M \vb{X} = \Sum{i=1}{N} m_i \vb{x}_i = \Sum{i=1}{N} m_i ( \vb{X} + \vb{x}_i' ) =
	\vb{X} \Sum{i=1}{N} m_i + \Sum{i=1}{N} m_i \vb{x}_i'
\]
pero el primer término del RHS es $M\vb{X}$ de manera que 
\be
	\Sum{i=1}{N} m_i \vb{x}_i' = 0 .
	\label{Condicion_cm}
\ee
La velocidad del centro de masa es la derivada temporal de \eqref{R_cm}, i.e.
\be
	\vb{V} = \frac{1}{M} \Sum{i=1}{N} m_i \dtot{\vb{x}_i}{t} = 
	\frac{1}{M} \Sum{i=1}{N} m_i \vb{v}_i
	\label{V_cm}
\ee

Con estos resultados volvemos a la expresión del momento que resulta 
\[
	\vb{L}_O = \vb{X} \times M \vb{V}  + \vb{X} \times \left( \Sum{i=1}{N} m_i \vb{v}_i' \right) +
	\left( \Sum{i=1}{N} m_i \vb{x}_i' \right) \times \vb{V} + \Sum{i=1}{N} \vb{x}_i' \times m_i \vb{v}_i',
\]
pero debido a \eqref{Condicion_cm} y a su derivada temporal (que resulta nula) el segundo y tercer sumando de la 
expresión anterior son nulos y entonces 
\[
	\vb{L}_O = \left( \vb{X} \times M \vb{V} \right) + \Sum{i=1}{N} ( \vb{x}_i' \times m_i \vb{v}_i' )
\]
% \[
% 	\vb{L}^T_O = \vb{L}^{cm} + \vb{L}^{sist}_{cm}
% \]
siendo el primer término del RHS el momento angular orbital y el segundo el momento angular de spin.

% Con respecto a la conservacion del momento angular, se tendrá
% \[
% 	\dtot{\vb{L}_O}{t} = \sum \vb{\tau}_O
% \]
% que se puede ver como suma del torque de fuerzas externas y de fuerzas internas. En el primer caso,
% los torques externos sumarán cero si las fuerzas externas son nulas o centrales.
% En el segundo caso los torques internos son nulos si vale el principio de acción y reacción fuerte;
% es decir si
% \[
% 	\vb{r}_i - \vb{r}_j \parallel F_{ij}.
% \]

% ~~~~~~~~~~~~~~~~~~~~~~~~~~~~~~~~~~~~~~~~~~~~~~~~~~~~~~~~~~~~~~~~~~~~~~~~~~~~~~~~~~~~~~~~~~~~~~~~~~~~~~~~~~~~~~~~~~~~~
\subsection{Trabajo y energía}

Consideremos una partícula de masa $ m $ que se mueve sobre una cierta trayectoria suave $\vb{x}(t)$, ver {Figura} 
\ref{fig_mc_workenergy}, debido a la acción de una fuerza \vb{F}.
Su velocidad \vb{v} es en todo momento tangente a la trayectoria y define de esta forma un versor $ \hat{t} $
colineal con la misma. Esto define un plano, mostrado en la parte derecha de la figura, para el cual todo vector
perteneciente al mismo es normal a la trayectoria. Elegimos un versor $ \hat{n} $ que está en la dirección de
la proyección de \vb{F} sobre dicho plano.

\begin{figure}[!h]
	\begin{center}
	\includegraphics[width=0.9\textwidth]{images/fig_mc_workandenergy.pdf}	
	\end{center}
	\caption{Partícula de masa $m$ que se mueve sobre una trayectoria $\vb{x}(t)$ bajo la acción de una fuerza 
\vb{F} (izquierda). En el detalle de la derecha se muestra la descomposición del movimiento en direcciones
tangencial $\hat{t}$ y normal $\hat{n}$.}
	\label{fig_mc_workenergy}
\end{figure} 

Descomponiendo la fuerza y la velocidad en estas dos direcciones, se tiene 
\[
	\vb{F} = F^t \: \hat{t}  + F^n \: \hat{n} \qquad \qquad \vb{v} = v \: \hat{t}
\]
de manera que la segunda ley de Newton, 
\[
	m \: \dtot{\vb{v}}{t} = \vb{F},
\]
para la componente $\hat{t}$ resulta
\[
	m \dtot{v}{t} = F^t
\]
\notamargen{Notemos que el versor desplazamiento $d\vb{s}$ {\it camina} por la trayectoria.}
Involucrando al diferencial de arco $ ds = | d\vb{x} | $ a lo largo de la trayectoria, la ecuación anterior se
puede escribir como
\be
	m \: dv \:\dtot{s}{t} = m \: v \: dv = F^t \: ds = \vb{F} \cdot d\vb{x},
	\label{ec_trabajo}
\ee
donde la última igualdad es posible en virtud de que $ F^n \perp d\vb{x} $ por construcción.

Podemos integrar ambos miembros de \eqref{ec_trabajo} entre $\vb{x}(t_0) \equiv \vb{x}_0$ y su 
correspondiente velocidad $v(t_0) \equiv v_0$ hasta $\vb{x}_1, \vb{v}_1$, 
\[
	m \int_{v_0}^{v_1} \: v \: dv = \int_{\vb{x}_0}^{\vb{x}_1}  \vb{F} \cdot d\vb{x}
\]
obteniendo
\[
	\left. \frac{1}{2} m v^2 \right|_{v_0}^{v_1} = W_{\vb{x}_0 \to \vb{x}_1} 
\]
que es el llamado \emph{teorema de las fuerzas vivas} para una partícula de masa $m$ y nos dice que la
variación de energía cinética en la trayectoria es igual al trabajo de todas las fuerzas que actúan
sobre la misma, i.e.
\be
	T_1 - T_0 = \Delta T_{\vb{x}_0 \to \vb{x}_1}  = W_{\vb{x}_0 \to \vb{x}_1} .
	\label{conser_energia}
\ee

En el caso particular en que la fuerza sea normal a la trayectoria en todo el intervalo $[t_0,t_1]$ se 
tendrá $\Delta T = 0 $, es decir que se conserva la energía cinética a lo largo de toda la trayectoria.
Sólo las componentes tangenciales de la fuerza producen trabajo y esto es solamente debido a que este proviene
de un producto escalar (una proyección); las componentes normales no hacen trabajo.

\notamargen{ Falta meter lo de \[ 	 m \frac{v^2}{\rho} = F_n \] }

Si la fuerza proviene de un potencial\footnote{El menos delante del gradiente es una convención, como se verá a
continuación.}, se tiene 
\be
	\vb{F} = - \nabla V
	\label{fuerza_prov_potencial}
\ee
y podemos expresar en coordenadas cartesianas esta equivalencia \eqref{fuerza_prov_potencial}
\[
	\vb{F} = -\left( \dpar{V}{x_1}, \dpar{V}{x_2}, \dpar{V}{x_3} \right)
\]
y evaluar la integral del trabajo para obtener
\[
	W = \int_{\vb{x}_0}^{\vb{x}_1}  \vb{F} \cdot d\vb{x} =
	\int_{t_0}^{t_1}  \vb{F}(\vb{x}[t]) \cdot \dotvb{ x } \: dt =
	- \int_{t_0}^{t_1}  \sum_{i=1}^3 \left[ \dpar{V}{x_i} \dtot{x_i}{t} \right] \: dt = V_0 - V_1
\]
donde la última igualdad se obtiene por integración de un gradiente. Esto 
significa que la integral es independiente de la trayectoria $\vb{x}_0 \to \vb{x}_1$.

Entonces, volviendo a \eqref{conser_energia}
\[
 	\rlap{ $\overbrace{\phantom{T_1 - T_0 = W_{0 \to 1}}}^{\text{Vale siempre}} $}  T_1 - T_0 =
	\underbrace{ W_{0 \to 1} = V_0 - V_1 }_{\text{Si $\vb{F}$ proviene de potencial} }
\]
y pasando de miembros se tiene 
\[
	(T_1 + V_1) = (T_0 + V_0 ) 
\]
que viene a significar que la cantidad $ E = T + V $ (la energía mecánica) se conserva si la fuerza $\vb{F}$ 
proviene de un potencial $V$. 
Por dicha razón, las fuerzas para las cuales se verifica \eqref{fuerza_prov_potencial} se llaman {\it fuerzas
conservativas}. En una dimensión, cualquier $ F(x) $ se puede hacer provenir de un potencial si verifica ser
integrable, es decir si podemos definir
\be
	V(x) = \int F(x) \: dx.
	\label{potencial_1d}
\ee
Para tres dimensiones no cualquier $ F(\vb{x}) $ es conservativa.

El signo negativo en \eqref{fuerza_prov_potencial} hace que la cantidad conservada sea $T+V$ en lugar de $T-V$.
Tiene más sentido físico que se conserve una suma de energías antes que una resta de las mismas.

\subsubsection{Trabajo y energía para un sistema de partículas}

Para un sistema de $ N $ partículas la energía cinética simplemente es la suma de las energías cinéticas de cada 
partícula,
\[
	T = \sum_{i=1}^N \: \frac{1}{2} \: m_i v_i^2.
\]

Utilizando la expresión en función del centro de masa, $\vb{v}_i = \vb{V} + \vb{v}_i'$ en la energía se llega a
\[
	T = \frac 1 2 M V^2 + \frac{1}{2} \: \sum_{i=1}^N \: m_i {v_i'}^2,
\]
donde el primer término es la energía cinética de traslación del centro de masa y el segundo término (la sumatoria) es 
la energía cinética interna.
En el caso de dos cuerpos la anterior expresión se reduce a
\[
	T = \frac 1 2 M V^2 + \frac 1 2 \mu v_r^2
\]
donde $\mu$ es la masa reducida y $v_r$ es la velocidad relativa.
\notamargen{Las conservaciones de las cosas permiten reducir la cantidad de integraciones necesarias.}

La definición del trabajo, en cambio, es un poco más complicada. Entre dos instantes de tiempo $ t $ y $ t + \Delta t $ 
el sistema está caracterizado por las $ N $ posiciones $ \{\vb{x}_i\} $ de todos sus integrantes y cada partícula 
experimenta un desplazamiento $ \Delta \vb{x}_i $ asociado con la fuerza que actúa sobre ella.

En principio la fuerza sobre cada partícula puede dividirse en interna (debida a las otras partículas del sistema) y 
externa (debida a agentes exteriores al sistema), lo cual permite escribir
\[
	\vb{F} = \vb{F}^{\text{int}} + \vb{F}^{\text{ext}}
\]
y consecuentemente
\[
	W = W^{\text{int}} + W^{\text{ext}}
\]

El $ W $ entre dos instantes de tiempo $t_0$ y $t_1$ corresponde ahora a la integral entre la configuración del sistema 
a $t_0$ dada por $ \{\vb{x}_i(t_0)\} $ hasta la configuración $ \{\vb{x}_i(t_1)\} $, las cuales etiquetaremos como 0 y 
1 respectivamente. 
\notamargen{El rozamiento depende de la velocidad, entonces no es conservativo.}

Entonces el trabajo externo es
\[
	W^{\text{ext}} = \sum_{i=1}^N \int_0^1 \vb{F}^{\text{ext}}_i \cdot \: d\vb{x}_i
\]
siendo $ \vb{F}^{\text{ext}}_i $ la fuerza externa sobre la partícula $i$. Para que valga la conservatividad es 
necesario que 
\begin{itemize}
 \item La fuerza sobre $i$ dependa solamente de las coordenadas $\vb{x}_i$ de esa partícula. Es decir:
 \[
	\vb{F}_i = \vb{F}_i(\vb{x}_i)
 \]
 \item Se verifique para cada $\vb{F}_i$ 
 \[
	\Nabla \times \vb{F}_i = 0,
 \]
 donde el operador $\nabla$ se toma con respecto a las coordenadas de la partícula $i$ en cuestión.
\end{itemize}

\begin{figure}[!hb]
	\begin{center}
	\includegraphics[width=0.75\textwidth]{images/fig_mc_work_system.pdf}	
	\end{center}
	\caption{Elementos implicados en la evaluación del trabajo interno $ W^{\text{int}} $ para un sistema
	de partículas.}
	\label{fig_mc_work_system}
\end{figure} 

\notamargen{Arreglar flechas en este gráfico.}

Estas condiciones permiten escribir la fuerza como el gradiente de un potencial y entonces el trabajo externo es la 
suma de las diferencias entre las energías potenciales de las partículas entre las configuraciones 0 y 1, o bien
\[
	W^{\text{ext}} = - \sum_{i=1}^N \left. \Delta V_i(\vb{x}_i) \right|_0^1
\]

El trabajo interno corresponde a la suma sobre cada partícula $i$ de la fuerza ejercida por todas las otras partículas 
$j \neq i$ del sistema, es decir
\be
	W^{\text{int}} = \sum_{i=1}^N \sum_{j\neq i}^N \int_0^1 \vb{F}_{ij} \cdot d\vb{x}_i
	\label{internal_work}
\ee
donde $\vb{F}_{ij} $ es la fuerza sobre $i$ ejercida por $j$. La restricción en la sumatoria sobre $j$ descarta la suma 
de autofuerzas. Es claro que la expresión \eqref{internal_work} se puede escribir equivalentemente como
\[
	\frac{1}{2} \sum_{i=1}^N \sum_{j\neq i}^N \int_0^1 
	\left( \vb{F}_{ij} \cdot d\vb{x}_i + \vb{F}_{ji} \cdot d\vb{x}_j \right) 
\]
\notamargen{¿nota final con la justificación de que se puede escribir así?}
% justificación :
% Esta sumatoria tiene N(N-1) términos que resultan de los N por N posibilidades excluyendo los N tales que i=j
% Por cada término del tipo F_ij dx_i hay un correspondiente F_ji dx_j; por ejemplo está F_13 dx_1 (i=1 u j=3) y F_31 
% dx_3 que viene de i=3 y j=1. Luego podemos sumar todo dos veces considerando un sumando general F_ijdx_i + F_jidx_j 
% preo dividiendo por dos para compensar. 
%
y si ahora aceptamos que vale el principio de acción y reacción
\[
	\frac{1}{2} \sum_{i=1}^N \sum_{j\neq i}^N \int_0^1 
	\vb{F}_{ij} \cdot \left(d\vb{x}_i - d\vb{x}_j \right) .
\]
Definiendo luego un vector de separación relativa $ \vb{r}_{ij} = \vb{x}_i - \vb{x}_j $ se tiene que las integrales son 
de la forma 
\[
	\int \vb{F}_{ij} \cdot d\vb{r}_{ij}
\]
y sabemos, por analogía con lo anterior, que si $\vb{F}_{ij}$ depende del vector de separación $ \vb{r}_{ij} $ y es de 
rotor nulo entonces las fuerzas internas son conservativas.
Entonces,
\[
	\vb{F}_{ij} = -\Nabla_i V(\vb{r}_{ij}) \qquad \vb{F}_{ji} = -\Nabla_j V(\vb{r}_{ij})
\]
y como vale acción y reacción $\vb{F}_{ij} = - \vb{F}_{ji}$ esto lleva a que $\nabla_i = \nabla_j$.
\notamargen{Un potencial que depende solo de la distancia entre dos partículas $|r_{ij}|$ cumple PAR fuerte.}

Un ejemplo numérico aclarará esta relación. Sea un potencial que depende de la distancia entre dos partículas,
$r = |\vb{r}_{ij}|$, es decir que si $i=2$ y $j=1$ se tendrá
\[
	r = \sqrt{ (x_2-x_1)^2 + (z_2-z_1)^2 + (z_2-z_1)^2 },
\]
luego,
\begin{multline*}
	\: \vb{F}_{21} = -\Nabla_2 V = -\dpar{V}{r} \left( \dpar{r}{x_2}\hat{x} + \dpar{r}{y_2}\hat{y} + 
	\dpar{r}{z_2}\hat{z} \right) = \\ 
	-\dpar{V}{r} \frac 1 r \left( (x_2-x_1)\hat{x} + (y_2-y_1)\hat{y} + (z_2-z_1)\hat{z} \right)
\end{multline*}
y, en cambio,
\begin{multline*}
	\: \vb{F}_{12} = -\Nabla_1 V = -\dpar{V}{r} \left( \dpar{r}{x_1}\hat{x} + \dpar{r}{y_1}\hat{y} + 
	\dpar{r}{z_1}\hat{z} \right) = \\ 
	-\dpar{V}{r} \frac 1 r \left( -(x_2-x_1)\hat{x} - (y_2-y_1)\hat{y} - (z_2-z_1)\hat{z} \right)
\end{multline*}
de manera que $ \vb{F}_{21} = -\vb{F}_{12} $.

En estos casos, en presencia de fuerzas conservativas
\[
	E = T + V^e(\vb{x}_1,...,\vb{x}_N) + \frac 1 2 \sum_{i=1}^N \sum_{j=1}^N V_{ij}
\]
donde $V^e$ es el trabajo externo. Luego, la variación de energía $\Delta E$ será
\[
	\Delta E = \sum_{i=1}^N ( \Delta T_i - \Delta V_i ) + \frac{1}{2} \sum_{i=1}^N \sum_{j=1}^N \Delta V_{ij}.
\]
\notamargen{Revisar la escritura de la energía y de la variación, qué pienso con respecto a T, V?}


% ~~~~~~~~~~~~~~~~~~~~~~~~~~~~~~~~~~~~~~~~~~~~~~~~~~~~~~~~~~~~~~~~~~~~~~~~~~~~~~~~~~~~~~~~~~~~~~~~~~~~~~~~~~~~~~~~~~~~~
\begin{ejemplo}{\bfseries Análisis energético de un potencial }

\label{ejemplo_analisis_potencial}
Dada una fuerza 1D 
\[
	F(x) = -k x + \frac{a}{x^3} ,	\qquad \qquad a > 0
\]
se realiza un análisis del potencial resultante y de la energía.

\cajacostado{12cm}{12.4cm}{
La fuerza $-kx$ es una fuerza restitutiva mientras que $a/x^3$ es una fuerza
repulsiva pués $a > 0$. 
% También podríamos decir que se tiene:
% \[
% 	\partial \Psi = 0
% \]
}

\vspace*{1mm}
A partir de esta fuerza, que es la del oscilador armónico 1D sumada a una perturbación controlada por el parámetro 
$a$, procedemos a calcular el potencial, a través de la relación \eqref{potencial_1d} de modo que (a menos de una 
constante aditiva que no interesa aquí) se obtiene
\[
	V(x) = \frac{k}{2} x^2 + \frac{a}{2 x^2}
\]
siendo la energía total 
\[
	E = T + V( x ), 
\]
la cual es una cantidad conservada.
% Para analizar el movimiento bajo este potencial consideraremos tres relaciones diferentes entre los parámetros $k,a$ 
% para poder graficar $ V(x) $. Consideraremos tres casos numéricos $ k = 100 a, 20 a, 5 a $ (dado que $k$ y $a$ son 
% magnitudes dimensionalmente diferentes es evidente que estos números 100, 20 y 5 tienen unidades, aunque no interesan 
% para este análisis).

Para analizar el movimiento bajo este potencial dividimos ambos miembros sobre $ a $ y se define $ v \equiv V/a$ para 
considerar tres casos representativos $ k / a = 5, 20, 100 $. Este potencial $ v $ es una especie de potencial por 
unidad de $ a $. Consecuentemente, tendremos una energía reescalada $ e = t + v $.

Para el caso $ k / a = 100 $ la Figura \ref{fig_mc_problema2_1} muestra la gráfica de $ v $ junto con la de cada 
uno de los términos que componen este potencial; el término $k/(2a) x^2$ (cuadrático) y $1/(2x^2)$ (una ley de potencias 
de exponente -2). En la zona de $ x $ pequeña domina la ley de potencias mientras que para $ x $ grande domina la 
cuadrática. 

\begin{figure}[!ht]
	\begin{center}
	\includegraphics[width=0.85\textwidth]{images/fig_mc_problema2_1.pdf}	
	\end{center}
	\vspace*{-5mm}
	\caption{Gráfico del potencial $v=V(x)/a$ y energía $e=E/a$ escalados para el ejemplo del oscilador perturbado 
	($ k/a $= 100).}
	\label{fig_mc_problema2_1}
\end{figure} 

También está indicada una línea de $ e $ constante que define la energía total $t + v$. Dada la restricción $ e = t + 
v$, la energía cinética $t$ está representada por la distancia vertical entre $e$ y $v$ para todo $x$ comprendido entre 
los puntos indicados por cuadrados azules. Estos son aquellos puntos para los cuales $t=0$ (la velocidad es nula) y 
definen por ende un punto de cambio de movimiento; son los llamados {\it turning points} (puntos de retroceso). 
En la figura se indica con una doble flecha vertical la magnitud de $t$ para $x =$ 0.4.

Las regiones por fuera de los puntos de retroceso están prohibidas puesto que $ t < 0 $ allí.
El movimiento posible para este potencial es entonces acotado y se halla dentro del intervalo definido por dichos 
puntos.

La línea vertical punteada $x=0.31622$ indica el mínimo del potencial (que sale desde $ dV(x)/dx = 0 $ y es 
equivalente por ello a la condición $ F(x)= 0 $). Ese punto es, debido a la forma particular de $F$, donde son iguales 
los aportes del oscilador (término cuadrático) y de su perturbación (ley de potencias).

A medida que $ a $ es más importante ($k/a$ disminuye) la parte $ 1 / x^2 $ del potencial actúa hasta valores de $ x $ 
mayores, como puede verse en la Figura XXX donde aparece graficado $v$ para los casos $ k/a = 100, 20, 5 $.

\notamargen{Poner la cuenta genérica en las notas. Why not?. FALTA un gráfico.}

Para una partícula que se mueva bajo este potencial la energía cinética
\[
	T = \frac{1}{2} m \dot{x}^2 = E - \frac{k}{2} x^2 - \frac{a}{2 x^2},
\]
permite llegar a la integral de la trayectoria
\[
% 	\sqrt{m} \: \int \left[ 2E  - kx^2 - \frac{a}{x^2} \right]^{-1/2} dx = \int \; dt.
	\sqrt{m} \: \int \frac{1}{ \left[ 2E - kx^2 - a/x^2 \right]^{1/2} } \: dx = \int \; dt.
\]

La existencia de solución cerrada para esta integral dependerá, por supuesto, de la forma del potencial $V(x)$.
En este caso particular el reemplazo $ u = x^2 $ permite escribir el argumento de la raíz como una diferencia de
cuadrados merced a un nuevo reemplazo $ y = u - E/k $. 
Si se integra entre $ x_0 = x(t=0) $ y $ x = x(t) $ se obtiene 
\[
	\frac{1}{2}\sqrt{\frac{m}{k}} \int_{x_0^2 - E/k}^{{x^2 - E/k}} \frac{1}{\sqrt{C^2 - y^2}} \; dy = t - t_0
\]
donde la constante es $ C = \sqrt{E^2 - ka}/k $.

La solución de esta integral es del tipo $ \arcsin(y/|C|) $ de manera que obtenemos
\[
	x^2 = \frac{\sqrt{E^2 - ak}}{k} \: \sin 
	\left[ \sqrt{4k/m}(t-t_0) + \arcsin ( (kx_0^2- E)/\sqrt{E^2 -ak }) \right] + \frac{E}{k}.
\]

Si se supone ahora que $ \dot{x}_0 = 0 $ (la cinética es nula en el instante $t=0$) resulta $ \arcsin(1) = \pi/2 $ 
y entonces 
\[
	x^2 = \frac{\sqrt{E^2 - ak}}{k} \: \cos \left[ \sqrt{4k/m}(t-t_0) \right] + \frac{E}{k},
\]
o bien 
\[
	x = \sqrt{ \frac{E}{k} } \left( 1 + \sqrt{1 - ak/E^2}  \: \cos \left[ 2\sqrt{k/m}(t-t_0) \right] \right)^{1/2}
\]
donde hemos tomado el valor positivo de la raíz porque en este problema es $ x > 0 $ .

El caso límite $ a = 0 $ recupera el oscilador armónico usual, como era de esperarse, pues en este caso se tiene
\[
	\underset{(a = 0)}{x} = \sqrt{ \frac{2E}{k} } \left( \cos \left[ \sqrt{\frac{k}{m}}(t-t_0) \right] \right),
\]
donde se ha utilizado la fórmula trigonométrica para el semiángulo.

\notamargen{Completar esta solución. Ver en práctica?.
Lo de $E=E(a)$ no lo entendí.}
\end{ejemplo}


\begin{notasfinales}

\label{nota_suma_ineqj}
\item{ \bf Sumatoria de torques}
Una manera de convencerse de que esta escritura es posible es hacer un diagrama de los diferentes términos que
aparecen en esta doble sumatoria. Es fácil de ver que con el añadido del término $\vb{x}_j \times \vb{F}_{ji} $ se está 
haciendo un doble conteo que justifica el $1/2$ que aparece luego.

Una demostración más matemática puede lograrse escribiendo la sumatoria $ j\neq i $ sin esta restricción, lo cual se 
puede hacer así:
\[
	\Sum{i=1}{N} \Sum{j\neq i}{N}  \vb{x}_i \times \vb{F}_{ij} = 
	\Sum{i=1}{N} \Sum{j=1}{N}  \vb{x}_i \times \vb{F}_{ij} ( 1 - \delta_{ij} )
\]
siendo $ \delta_{ij} $ la delta de Kronecker. Es claro que podemos hacer un cambio de etiquetas en las sumatorias 
puesto que los índices sumados son {\it mudos}, i.e.
\[
	\Sum{i=1}{N} \Sum{j=1}{N}  \vb{x}_i \times \vb{F}_{ij} ( 1 - \delta_{ij} ) = 
	\Sum{j=1}{N} \Sum{i=1}{N}  \vb{x}_j \times \vb{F}_{ji} ( 1 - \delta_{ij} )
\]
y dado que el orden de las sumatorias es irrelevante llegamos a
\[
	\Sum{i=1}{N} \Sum{j\neq i}{N}  \vb{x}_i \times \vb{F}_{ij} = \frac{1}{2}
	\Sum{i=1}{N} \Sum{j=1}{N} \left[ \vb{x}_i \times \vb{F}_{ij} ( 1 - \delta_{ij} ) +
	\vb{x}_j \times \vb{F}_{ji} ( 1 - \delta_{ij} )
	\right] 
\]

Regresando ahora a las sumatoria restringida obtenemos 
\[
	\Sum{i=1}{N} \Sum{j\neq i}{N}  \vb{x}_i \times \vb{F}_{ij} = \frac{1}{2}
	\Sum{i=1}{N} \Sum{j \neq i }{N} \left[ \vb{x}_i \times \vb{F}_{ij} + \vb{x}_j \times \vb{F}_{ji} \right] 
\]
que es el resultado buscado.
\end{notasfinales}

% ============================================================================


% ~~~~~~~~~~~~~~~~~~~~~~~~~~~~~~~~~~~~~~~~~~~~~~~~~~~~~~~~~~~~~~~~~~~~~~~~~~~~~~~~~~~~~~~~~~~~~~~~~~~~~~~~~~~~~~~~~~~~~
\begin{ejercicios}

\label{ej1}
\item{ \bf }
Dos masas $m_1$ y $m_2$ están unidas por una barra rı́gida. Se coloca la barra sobre una superficie horizontal sin rozamiento tal que la masa $m_1$ la toque pero no la $m_2$. Si se la deja en libertad, ¿dónde golpea $m_2$ a la superficie?

\label{ej2}
\item{ \bf }
Una partı́cula está sometida a una fuerza $F (x) = −kx + xa3$.
\begin{enumerate}[label=(\alph*)]
 \item Hallar el potencial $U(x)$. Discutir los tipos de movimiento posibles. Hallar las
posiciones de equilibrio estable y encontrar la solución general $x(t)$.
 \item Interpretar el movimiento en el lı́mite $E_2 \gg ka$.
 \item ¿Cuánto vale el perı́odo de las oscilaciones?
\end{enumerate}

\label{ej3}
\item{ \bf }
Hallar el vector velocidad y el vector aceleración en coordenadas polares y esféricas. (es
muy conveniente visualizar a partir de gráficos, además de efectuar el cálculo analı́tico).

\label{ej4}
\item{ \bf }
Un disco homogéneo de masa M y radio R está girando con velocidad angular ω.
Una mosca de masa m que inicialmente se encuentra en el centro del disco camina
radialmente hacia afuera con velocidad relativa constante.
\begin{enumerate}[label=(\alph*)]
 \item Si el disco es obligado a girar con velocidad relativa constante por un motor, qué
torque debe hacer éste para compensar el movimiento de la mosca? Cuál es la
fuerza de Coriolis que siente la mosca?
 \item Si el disco gira libremente, cuál será la velocidad angular del disco cuando la
mosca está a una distancia d del centro?
\end{enumerate}

\label{ej5}
\item{ \bf }
Un disco homogéneo de masa m y radio r rueda sin deslizar sobre un plano, inclinado
un ángulo α respecto de la horizontal.
\begin{enumerate}[label=(\alph*)]
\item Halle su aceleración angular y la aceleración lineal de su centro.
\item Si en t = 0 el disco estaba en reposo a una altura h del suelo, cuál es su velocidad
angular y lineal al llegar a éste?
\item Qué magnitudes se conservan en el movimiento del disco?
\end{enumerate}

\label{ej6}
\item{ \bf }
Dos partı́culas de masas ma y mb están sobre una mesa horizontal sin fricción. Se
encuentran unidas por una cuerda tensa que pasa por un anillo pequeño, sin fricción,
fijo a la mesa. Inicialmente las partı́culas están quietas a distancias Ra y Rb del anillo
y en t = 0 se le da un impulso a la masa mb , perpendicular a la cuerda, de modo que
ésta adquiere una velocidad v0 .
\begin{enumerate}[label=(\alph*)]
\item Qué magnitudes se conservan?
\item Dar la velocidad de las partı́culas en función de su distancia al anillo.
\item Hallar la tensión de la cuerda en función de la distancia de una masa al anillo.
\end{enumerate}

\label{ej7}
\item{ \bf }
Se lanza una partı́cula por una vı́a horizontal sin rozamiento con velocidad v0 . En un
determinado lugar la vı́a tiene forma circular, de radio a, como se indica en la figura.
\begin{enumerate}[label=(\alph*)]
\item Calcular la fuerza de vı́nculo en función de la posición y la energı́a inicial de la
partı́cula.
\item Encontrar en qué punto se despega del aro en función de la velocidad inicial.
\item Describir las posibles trayectorias.
\end{enumerate}

\label{ej8}
\item{ \bf }
Se tiene una pelotita de masa m enhebrada en una barra, como se indica en la figura.
\begin{enumerate}[label=(\alph*)]
\item Cuántos grados de libertad tiene el sistema?
\item. Existen ecuaciones de vı́nculo?
\item. Analice y compare los casos en que φ varı́a libremente y en que la barra gira en
el plano con velocidad constante.
\item Qué pasa si se agrega una segunda bolita en la barra. (Considere nula la masa
de la barra y plantee en todos los casos las ecuaciones de Newton).
\end{enumerate}

\label{ej9}
\item{ \bf }
Para los sistemas (en equilibrio) de las figuras, dibuje todas las fuerzas aplicadas.
Indique qué interacciones representan y cuales forman pares de acción y reacción.

\label{ej10}
\item{ \bf }
Tiene sentido decir que la fuerza de un resorte es conservativa ?. Si se verifica ∇×F = 0,
es F conservativa ?. Analice los siguientes ejemplos, indicando claramente cuál es el
sistema mecánico cuya energı́a se considera (ver figura).

\label{ej11}
\item{ \bf }
Cuánto marca el dinamómetro ? (suponga su masa nula) (ver figura)
\begin{enumerate}[label=(\alph*)]
\item Si m = M .
\item Si m 6= M .
\end{enumerate}


\label{ej12}
\item{ \bf }
Dadas dos masas puntuales, expresar matemáticamente el hecho que las fuerzas de
interacción entre ambas están sobre la recta que las une.

\label{ej13}
\item{ \bf }
Es posible que se conserve el impulso lineal y no se conserve la energı́a ?. Y viceversa?.
Dé ejemplos.

\label{ej14}
\item{ \bf }
Pueden conservarse dos componentes del impulso angular de una partı́cula y no con-
servarse la tercera ?. Justifique su respuesta.

\label{ej15}
\item{ \bf }
Fr = 0 implica pr = cte.? Justifique, dé ejemplos. pr = cte. implica Fr = 0?
Justifique, dé ejemplos.

\label{ej16}
\item{ \bf }
Suponga que un sistema de masas puntuales es descripto desde un sistema inercial S
(la masa ma tiene posición ra y momento pa ) y desde el sistema centro de masa S 0 (la
masa tiene posición r0 a y momento p0 a ).
\begin{enumerate}[label=(\alph*)]
\item Compare las siguientes definiciones del impulso angular referido al centro de
masa:
	\begin{enumerate}
	\item $\vb{L}_1 =$
	\item $\vb{L}_2 =$
	\item $\vb{L}_3 =$
	\end{enumerate}
\item Encuentre la relación entre el impulso angular referido a S y aquel referido a S 0
(centro de masa).
\end{enumerate}

\label{ej17}
\item{ \bf }
Dos partı́culas aisladas interactúan tal que LCM es constante.
\begin{enumerate}[label=(\alph*)]
\item Es válido afirmar que LS = cte. donde S es un sistema arbitrario distinto del
centro de masa? De un ejemplo fı́sico.
\item Visto desde el sistema CM, bajo qué condiciones el movimiento de las partı́culas
es unidimensional, bidimensional o tridimensional?
\item Si LS = cte. con S = CM, entonces el movimiento de las partı́culas será plano
(en S)? Porqué?
\end{enumerate}

\label{ej18}
\item{ \bf }
Si el centro de masa de un sistema está acelerado, sigue siendo válida la relación:
dLCM /dt = NCM ? (LCM es el momento angular y NCM el momento de las fuerzas
con respecto al centro de masa).


\label{ej19}
\item{ \bf }
Las esferas del dibujo se mueven sin rozamiento por un carril horizontal. Si se aplica
una fuerza F (colineal con el carril) sobre la primera encuentre la fuerza neta sobre
cada una de ellas y los valores de las fuerzas de contacto sobre la tercera y la quinta.

\label{ej20}
\item{ \bf }
Para las condiciones de las figuras, indique cuánto vale la fuerza de rozamiento (μe 6= 0)

\label{ej21}
\item{ \bf }
Para cada uno de los ejemplos que se muestran, indique detalladamente que magnitudes
se conservan y porqué ?. Hágalo para cada partı́cula y para todo el sistema.

\label{ej22}
\item{ \bf }
Considere un protón en reposo en un sistema fijo. Desde el infinito incide un electrón
con velocidad v0 , cuyo movimiento (cuando está muy lejos del protón) es aproximada-
mente rectilı́neo uniforme. Al acercarse al protón la trayectoria del electrón se curva
debido a la interacción electrostática entre ambos (V = −e2 /r).
Debido a que la masa mp del protón es mucho mayor que la del electrón me, puede
suponerse fijo al primero (como $mp \gg me$ , debe ser $vp \ll ve$ ). Si no interactuaran, la
trayectoria del electrón serı́a rectilı́nea y la distancia de máximo acercamiento electrón
protón serı́a b.
\begin{enumerate}[label=(\alph*)]
\item Qué magnitudes se conservan?
\item Usando las leyes de conservación halladas en a., calcule la distancia de máximo
acercamiento.
\end{enumerate}

\label{ej23}
\item{ \bf }
Utilizando coordenadas cilı́ndricas y esféricas, obtenga las ecuaciones de movimiento
para un péndulo plano y para uno esférico, respectivamente.

\label{ej24}
\item{ \bf }
Una masa m sólo puede moverse en el interior de un tubo cilı́ndrico, sin fricción, como
indica la figura. El tubo rota con velocidad angular constante.
\begin{enumerate}[label=(\alph*)]
\item Analizar qué magnitudes se conservan.
\item Hallar las ecuaciones de movimiento para la partı́cula en coordenadas cartesianas
y en polares.
\item Hallar la fuerza de vı́nculo en función del tiempo si en el instante inicial la masa
está quieta con respecto al tubo a una distancia a.
\end{enumerate}

\end{ejercicios}


% \bibliographystyle{CBFT-apa-good} % (uses file "apa-good.bst")
% \bibliography{CBFT.Referencias} % La base de datos bibliográfica


\end{document}
