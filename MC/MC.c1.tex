	\documentclass[10pt,oneside]{CBFT_book}
	
	% Algunos paquetes
	\usepackage{amsmath}
	\usepackage{amssymb}
	\usepackage{graphicx}
	\usepackage{libertine}
%   	\usepackage[bold-style=TeX]{unicode-math}
  	\usepackage[ math-style=TeX, bold-style=ISO ]{unicode-math}
% 	Este package es el que configura cómo salen las cosas en negrita, y en general
% 	en math mode.
  	
	\usepackage{lipsum}

	\usepackage[numbers]{natbib}
	\setcitestyle{square}


	\usepackage{polyglossia}
	\setdefaultlanguage{spanish}

	\usepackage{CBFT.estilo} % Cargo la hoja de estilo
	

	% Tipografías
	% \setromanfont[Mapping=tex-text]{Linux Libertine O}
	% \setsansfont[Mapping=tex-text]{DejaVu Sans}
	% \setmonofont[Mapping=tex-text]{DejaVu Sans Mono}

	%===================================================================
	%	DOCUMENTO PROPIAMENTE DICHO
	%===================================================================

% \title{CBFT Mecánica clásica}
% \author{Mecánica lagrangiana}
% \date{\today}

\begin{document}
% \maketitle
% \tableofcontents
\chapter{Mecánica newtoniana}

Tal vez sea una simplificación, pero no una muy terrible, decir que el curso de mecánica clásica
busca reemplazar la mecánica basada en las ecuaciones de Newton,
\[
	\vb{F} = m \vb{a} 
\]
por un \emph{formalismo} más poderoso y que se podrá aplicar luego a otros campos.
Este formalismo es el corazón de la mecánica clásica.

\begin{figure}[hbt]
	\begin{center}
	\includegraphics[width=0.6\textwidth]{images/fig_mc_leyes_cons.pdf}
	\end{center}
	\caption{Sistema de partículas de masas $m_i$ con sus correspondientes vectores de
	posición $\vb{x}_i$. La partícula $m_1$ tiene además indicado su vector velocidad $\vb{v}_1$.}
	\label{fig_mc_leyes_cons}
\end{figure} 


% =================================================================================================
\section{Momento angular}
% =================================================================================================



\[
	\vb{L} = \vb{r} \times \vb{p}
\]
\begin{figure}[hbt]
	\begin{center}
	\includegraphics[width=0.6\textwidth]{images/fig_mc_angularmom.pdf}	
	\end{center}
	\caption{}
\end{figure} 
\[
	\vb{r}_i = \vb{R} + \vb{r}_i' \qquad \vb{v}_i = \vb{V} + \vb{v}_i'
\]
\[
	\vb{L}^T_O = \sum_i \vb{r}_i \times \vb{p}_i = \sum_i (\vb{R} + \vb{r}_i') \times m_i (\vb{V} + \vb{v}_i')
\]
\[
	\vb{L}^T_O = \sum_i ( \vb{R} \times m_i \vb{V}  + \vb{R} \times m_i \vb{v}_i'
	+ \vb{r}_i' \times m_i \vb{V} 	+ \vb{r}_i' \times m_i \vb{v}_i' )
\]
pero si recordamos que se cumplen 
\[
	\vb{R} = \sum_i \frac{m_i \vb{r}_i}{M}
\]
\[
	M\vb{R} = \sum_i m_i (\vb{R} + \vb{r}_i') = \sum_i m_i \vb{R} + \sum_i m_i \vb{r}_i'
\]
\[
	M\vb{R} = M \vb{R} + \sum_i m_i \vb{r}_i' \quad \Longrightarrow 0 = \sum_i m_i \vb{r}_i'
\]
podemos volver a las ecuaciones anteriores para poner
\[
	\vb{L}^T_O = \vb{R} \times M \vb{V}  + \vb{R} \times \frac{d}{dt}\left( m_i \vb{r}_i' \right)
	+ \left( \sum_i m_i \vb{r}_i' \right) \times \vb{V} + \sum_i \vb{r}_i' \times m_i \vb{v}_i'
\]
\[
	\vb{L}^T_O = \vb{R} \times M \vb{V}  + \sum_i \vb{r}_i' \times m_i \vb{v}_i'
\]
\[
	\vb{L}^T_O = \vb{L}^{cm} + \vb{L}^{sist}_{cm}
\]
siendo el primer término del lado derecho el momento angular orbital y el segundo el momento angular
de spin.

Con respecto a la conservacion del momento angular, se tendrá
\[
	\dtot{\vb{L}_O}{t} = \sum \vb{\tau}_O
\]
que se puede ver como suma del torque de fuerzas externas y de fuerzas internas. En el primer caso,
los torques externos sumarán cero si las fuerzas externas son nulas o centrales.
En el segundo caso los torques internos son nulos si vale el principio de acción y reacción fuerte;
es decir si
\[
	\vb{r}_i - \vb{r}_j \parallel F_{ij}.
\]

% =================================================================================================
\section{Trabajo y energía}
% =================================================================================================

Consideremos una partícula de masa $ m $ que se mueve sobre una cierta trayectoria suave $\vb{x}(t)$, ver {Figura} 
\ref{fig_mc_workenergy}, debido a la acción de una fuerza \vb{F}.
Su velocidad \vb{v} es en todo momento tangente a la trayectoria y define de esta forma un versor $ \hat{t} $
colineal con la misma. Esto define un plano, mostrado en la parte derecha de la figura, para el cual todo vector
perteneciente al mismo es normal a la trayectoria. Elegimos un versor $ \hat{n} $ que está en la dirección de
la proyección de \vb{F} sobre dicho plano.

\begin{figure}[!h]
	\begin{center}
	\includegraphics[width=0.9\textwidth]{images/fig_mc_workandenergy.pdf}	
	\end{center}
	\caption{Partícula de masa $m$ que se mueve sobre una trayectoria $\vb{x}(t)$ bajo la acción de una fuerza 
\vb{F} (izquierda). En el detalle de la derecha se muestra la descomposición del movimiento en direcciones
tangencial $\hat{t}$ y normal $\hat{n}$.}
	\label{fig_mc_workenergy}
\end{figure} 

Descomponiendo la fuerza y la velocidad en estas dos direcciones, se tiene 
\[
	\vb{F} = F^t \: \hat{t}  + F^n \: \hat{n} \qquad \qquad \vb{v} = v \: \hat{t}
\]
de manera que la segunda ley de Newton, 
\[
	m \: \dtot{\vb{v}}{t} = \vb{F},
\]
para la componente $\hat{t}$ resulta
\[
	m \dtot{v}{t} = F^t
\]
\notamargen{Notemos que el versor desplazamiento $d\vb{s}$ {\it camina} por la trayectoria.}
Involucrando al diferencial de arco $ ds = | d\vb{x} | $ a lo largo de la trayectoria, la ecuación anterior se
puede escribir como
\be
	m \: dv \:\dtot{s}{t} = m \: v \: dv = F^t \: ds = \vb{F} \cdot d\vb{x},
	\label{ec_trabajo}
\ee
donde la última igualdad es posible en virtud de que $ F^n \perp d\vb{x} $ por construcción.

Podemos integrar ambos miembros de \eqref{ec_trabajo} entre $\vb{x}(t_0) \equiv \vb{x}_0$ y su 
correspondiente velocidad $v(t_0) \equiv v_0$ hasta $\vb{x}_1, \vb{v}_1$, 
\[
	m \int_{v_0}^{v_1} \: v \: dv = \int_{\vb{x}_0}^{\vb{x}_1}  \vb{F} \cdot d\vb{x}
\]
obteniendo
\[
	\left. \frac{1}{2} m v^2 \right|_{v_0}^{v_1} = W_{\vb{x}_0 \to \vb{x}_1} 
\]
que es el llamado \emph{teorema de las fuerzas vivas} para una partícula de masa $m$ y nos dice que la
variación de energía cinética en la trayectoria es igual al trabajo de todas las fuerzas que actúan
sobre la misma, i.e.
\be
	T_1 - T_0 = \Delta T_{\vb{x}_0 \to \vb{x}_1}  = W_{\vb{x}_0 \to \vb{x}_1} .
	\label{conser_energia}
\ee

En el caso particular en que la fuerza sea normal a la trayectoria en todo el intervalo $[t_0,t_1]$ se 
tendrá $\Delta T = 0 $, es decir que se conserva la energía cinética a lo largo de toda la trayectoria.
Sólo las componentes tangenciales de la fuerza producen trabajo y esto es solamente debido a que este proviene
de un producto escalar (una proyección); las componentes normales no hacen trabajo.

\notamargen{ Falta meter lo de \[ 	 m \frac{v^2}{\rho} = F_n \] }

Si la fuerza proviene de un potencial\footnote{El menos delante del gradiente es una convención, como se verá a
continuación}, se tiene 
\be
	\vb{F} = - \nabla V
	\label{fuerza_prov_potencial}
\ee
y podemos expresar en coordenadas cartesianas esta equivalencia \eqref{fuerza_prov_potencial}
\[
	\vb{F} = -\left( \dpar{V}{x_1}, \dpar{V}{x_2}, \dpar{V}{x_3} \right)
\]
y evaluar la integral del trabajo para obtener
\[
	W = \int_{\vb{x}_0}^{\vb{x}_1}  \vb{F} \cdot d\vb{x} =
	\int_{t_0}^{t_1}  \vb{F}(\vb{x}[t]) \cdot \dotvb{ x } \: dt =
	- \int_{t_0}^{t_1}  \sum_{i=1}^3 \left[ \dpar{V}{x_i} \dtot{x_i}{t} \right] \: dt = V_0 - V_1
\]
donde la última igualdad se obtiene por integración de un gradiente. Esto 
significa que la integral es independiente de la trayectoria $\vb{x}_0 \to \vb{x}_1$.

Entonces, volviendo a \eqref{conser_energia}
\[
 	\rlap{ $\overbrace{\phantom{T_1 - T_0 = W_{0 \to 1}}}^{\text{Vale siempre}} $}  T_1 - T_0 =
	\underbrace{ W_{0 \to 1} = V_0 - V_1 }_{\text{Si $\vb{F}$ proviene de potencial} }
\]
y pasando de miembros se tiene 
\[
	(T_1 + V_1) = (T_0 + V_0 ) 
\]
que viene a significar que la cantidad $ E = T + V $ (la energía mecánica) se conserva si la fuerza $\vb{F}$ 
proviene de un potencial $V$. 
Por dicha razón, las fuerzas para las cuales se verifica \eqref{fuerza_prov_potencial} se llaman {\it fuerzas
conservativas}. En una dimensión, cualquier $ F(x) $ se puede hacer provenir de un potencial si verifica ser integrable,
es decir si podemos definir
\[
	V(x) = \int F(x) \: dx.
\]
Para tres dimensiones no cualquier $ F(\vb{x}) $ es conservativa.

El signo negativo en \eqref{fuerza_prov_potencial} hace que la cantidad conservada sea $T+V$ en lugar de $T-V$.
Tiene más sentido físico que se conserve una suma de energías antes que una resta de las mismas.

\subsection{Trabajo y energía para un sistema de partículas}

Para un sistema de $ N $ partículas la energía cinética simplemente es la suma de las energías cinéticas de cada 
partícula,
\[
	T = \sum_{i=1}^N \: \frac{1}{2} \: m_i v_i^2.
\]

La definición del trabajo, en cambio, es un poco más complicada. Entre dos instantes de tiempo $ t $ y $ t + \Delta t $ 
el sistema está caracterizado por las $ N $ posiciones $ \{\vb{x}_i\} $ de todos sus integrantes y cada partícula 
experimenta un desplazamiento $ \Delta \vb{x}_i $ asociado con la fuerza que actúa sobre ella.

En principio la fuerza sobre cada partícula puede dividirse en interna (debida a las otras partículas del sistema) y 
externas (debida a agentes exteriores al sistema), lo cual permite escribir
\[
	\vb{F} = \vb{F}^{\text{int}} + \vb{F}^{\text{ext}}
\]
y consecuentemente
\[
	W = W^{\text{int}} + W^{\text{ext}}
\]

El $ W $ entre dos instantes de tiempo $t_0$ y $t_1$ corresponde ahora a la integral entre la configuración del sistema 
a $t_0$ dada por $ \{\vb{x}_i(t_0)\} $ hasta la configuración $ \{\vb{x}_i(t_1)\} $, las cuales etiquetaremos como 0 y 
1 respectivamente. 

Entonces el trabajo externo es
\[
	W^{\text{ext}} = \sum_{i=1}^N \int_0^1 \vb{F}^{\text{ext}}_i \cdot \: d\vb{x}_i
\]
siendo $ \vb{F}^{\text{ext}}_i $ la fuerza externa sobre la partícula $i$. Para que valga la conservatividad es 
necesario que 
\begin{itemize}
 \item La fuerza sobre $i$ dependa solamente de las coordenadas $\vb{x}_i$ de esa partícula. Es decir:
 \[
	\vb{F}_i = \vb{F}_i(\vb{x}_i)
 \]
 \item Se verifique para cada $\vb{F}_i$ 
 \[
	\Nabla \times \vb{F}_i = 0,
 \]
 donde el operador $\nabla$ se toma con respecto a las coordenadas de la partícula $i$ en cuestión.
\end{itemize}

Estas condiciones permiten escribir la fuerza como el gradiente de un potencial y entonces el trabajo externo es la 
suma de las diferencias entre las energías potenciales de las partículas entre las configuraciones 0 y 1, o bien
\[
	W^{\text{ext}} = - \sum_{i=1}^N \left. \Delta V_i(\vb{x}_i) \right|_0^1
\]

El trabajo interno corresponde a la suma sobre cada partícula $i$ de la fuerza ejercida por todas las otras partículas 
$j \neq i$ del sistema, es decir
\be
	W^{\text{int}} = \sum_{i=1}^N \sum_{j\neq i}^N \int_0^1 \vb{F}_{ij} \cdot d\vb{x}_i
	\label{internal_work}
\ee
donde $\vb{F}_{ij} $ es la fuerza sobre $i$ ejercida por $j$. La restricción en la sumatoria sobre $j$ descarta la suma 
de autofuerzas. Es claro que la expresión \eqref{internal_work} se puede escribir equivalentemente como
\[
	\frac{1}{2} \sum_{i=1}^N \sum_{j\neq i}^N \int_0^1 
	\left( \vb{F}_{ij} \cdot d\vb{x}_i + \vb{F}_{ji} \cdot d\vb{x}_j \right) 
\]
\notamargen{¿nota final con la justificación de que se puede escribir así?}
% justificación :
% Esta sumatoria tiene N(N-1) términos que resultan de los N por N posibilidades excluyendo los N tales que i=j
% Por cada término del tipo F_ij dx_i hay un correspondiente F_ji dx_j; por ejemplo está F_13 dx_1 (i=1 u j=3) y F_31 
% dx_3 que viene de i=3 y j=1. Luego podemos sumar todo dos veces considerando un sumando general F_ijdx_i + F_jidx_j 
% preo dividiendo por dos para compensar. 
%
y si ahora aceptamos que vale el principio de acción y reacción
\[
	\frac{1}{2} \sum_{i=1}^N \sum_{j\neq i}^N \int_0^1 
	\vb{F}_{ij} \cdot \left(d\vb{x}_i - d\vb{x}_j \right) .
\]
Definiendo luego un vector de separación relativa $ \vb{r}_{ij} = \vb{x}_i - \vb{x}_j $ se tiene que las integrales son 
de la forma 
\[
	\int \vb{F}_{ij} \cdot d\vb{r}_{ij}
\]
y sabemos, por analogía con lo anterior, que si $\vb{F}_{ij}$ depende del vector de separación $ \vb{r}_{ij} $ y es de 
rotor nulo entonces las fuerzas internas son conservativas.
En estos casos
\[
	\Delta E = \sum_{i=1}^N ( \Delta T_i - \Delta V_i ) + \frac{1}{2} \sum_{i=1}^N \sum_{j=1}^N \Delta V_{ij}
\]


% =================================================================================================
\section{Definiciones}
% =================================================================================================

El número de grados de libertad es el número de coordenadas independientes para resolver el problema.
Las fuerzas de vínculo $F^v$ se {\it acomodan} en todo momento para satisfacer las ligaduras.
Entonces las $\vb{F}^v$ son perpendiculares a los desplazamientos compatibles con los vínculos de
manera que 
\[
	W_{F^v} = 0
\]
\begin{figure}[hbt]
	\begin{center}
	\includegraphics[width=0.3\textwidth]{images/fig_mc_vinculos1.pdf}	 
	\includegraphics[width=0.3\textwidth]{images/fig_mc_vinculos2.pdf}
	\end{center}
	\caption{}
\end{figure} 

Los vínculos se clasifican en
\[
\textrm{holónomos} 
\begin{Bmatrix}
 f(r_i,t) = 0 \qquad \textrm{reónomos} \\
\; f(r_i) = 0 \qquad \textrm{esclerónomos} \;\\
\end{Bmatrix} 
\]
los cuales cumplen que  $W_{virtual}^{F^v}=0$, y
\[
\textrm{no holónomos} 
\begin{Bmatrix}
 f(r_i,t) \geq 0  \\
 f(r_i) \geq cte. \quad f(\dot{r}_i) = 0  \; \\
\end{Bmatrix}
\]
los cuales no cumplen, en general, que $\vb{F}^v$ perpendicular al desplazamiento posible.
\begin{figure}[hbt]
	\begin{center}
	\includegraphics[width=0.3\textwidth]{images/fig_mc_vinculos3.pdf}	
	\end{center}
	\caption{}
\end{figure} 
donde un desplazamiento virtual es un desplazamiento a $t_0$ fijo compatible con los vínculos,
mientras que un desplazamiento real es un desplazamiento en $\delta t$ durante el cual varían
fuerzas y ligaduras.

A tiempo fijo el desplazamiento es en $\hat{r} \perp \vb{F}^v$.

\[
	f(x_i,t) = cte. \Longrightarrow \sum_i^N \dpar{f}{x_i} \delta x_i + \dpar{f}{t} \delta t = 0 
\]
o bien
\[
	\nabla f \cdot \vb{\delta r} = 0
\]
% Como ejemplo citamos \cite{einstein}.
% O bien \cite{example}.
% Tal vez \citep{Aspnes:1973}

% ============================================================================

% \bibliographystyle{CBFT-apa-good} % (uses file "apa-good.bst")
% \bibliography{CBFT.Referencias} % La base de datos bibliográfica


\end{document}
