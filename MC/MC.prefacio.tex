\thispagestyle{empty}
\chapter*{Prefacio}

Al momento de escribir este volumen tomo conciencia de la lejanía que me separa de aquel que fui yo al tomar 
las notas originales del curso de mecánica clásica que dictase el profesor Alejandro Fendrik en 2005.

En cierto sentido creo que esa distancia fue beneficiosa porque me situó casi en la perspectiva de un
extranjero que por primera vez tuviera que recorrer esas tierras.

La mecánica clásica forma la estructura basal sobre la cual se construye todo el resto de la física teórica. 
En ella uno empieza a manipular ecuaciones más complicadas y aprende formalismos nuevos que le permitirán atacar
viejos y nuevos problemas con otra mirada. Mucho de lo que aquí se ve se utiliza después en física menos
intuitiva y más abstracta, como por ejemplo la mecánica cuántica y la relatividad general, donde es más difícil 
lograr una intuición física y pensar las cosas en términos de modelos mecánicos. 
Si esta intuición puede ser ganada aquí, en mecánica clásica, ello redundará en un mejor soporte mental para los
próximos pasos.


