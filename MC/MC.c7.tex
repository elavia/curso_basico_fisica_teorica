	\documentclass[10pt,oneside]{CBFT_book}
	% Algunos paquetes
	\usepackage{amssymb}
	\usepackage{amsmath}
	\usepackage{graphicx}
	\usepackage{libertine}
% 	\usepackage[bold-style=TeX]{unicode-math}
	\usepackage{lipsum}

	\usepackage{natbib}
	\setcitestyle{square}

	\usepackage{polyglossia}
	\setdefaultlanguage{spanish}


	\usepackage{CBFT.estilo} % Cargo la hoja de estilo

	% Tipografías
	% \setromanfont[Mapping=tex-text]{Linux Libertine O}
	% \setsansfont[Mapping=tex-text]{DejaVu Sans}
	% \setmonofont[Mapping=tex-text]{DejaVu Sans Mono}

	%===================================================================
	%	DOCUMENTO PROPIAMENTE DICHO
	%===================================================================

\begin{document}

\chapter{Ecuaciones de Hamilton}

Se pasa de las variables $(q, \dot{q})$ hacia el par $(q,p)$ con 
\[
	p = \dpar{\Lag}{\dot{q}}
\]
Se parte del 
\be
	H(q_i, p_i, t) = \sum_{i}^{3N-k} p_i \dot{q}_i - \Lag(q_i, \dot{q}_i, t)
	\label{hamiltoniano_def}
\ee
y consideramos el diferencial
\[
	dH = \sum_i p_i d\dot{q}_i + \dot{q}_i dp_i - \dpar{\Lag}{q_i} dq_i - \dpar{\Lag}{\dot{q}_i}d\dot{q}_i - \dpar{\Lag}{t}dt
\]
\[
	dH = \sum_i \dot{q}_i dp_i - \frac{d}{dt}\left( \dpar{\Lag}{\dot{q}_i} \right) dq_i - \dpar{\Lag}{t}dt
\]
\[
	dH = \sum_i \dot{q}_i dp_i - \dot{p}_i dq_i - \dpar{\Lag}{t}dt
\]
se deducen entonces,
\be
	\dpar{H}{p_i} = \dot{q}_i \qquad \qquad \dpar{H}{q_i} = -\dot{p}_i 
	\label{variables_canonicas}
\ee
y
\[
	\dpar{H}{t} = -\dpar{\Lag}{t}
\]
que son las ecuaciones de Hamilton. Donde $(p,q)$ son $2N$ grados de libertad del sistema llamados las variables canónicas.
Si $V\neq V(\dot{q})$ y los vínculos no dependen del tiempo entonces $T=T_2$ (la energía cinética es cuadrática en las 
velocidades) y $H = T + V = E$.

\notamargen{En el caso $H=E$ se puede escribir fácil el hamiltoniano.}

No presenta gran economía respecto a la formulación lagrangiana.
Aquí las coordenadas y los momentos adquieren un carácter simétrico.
Se define el espacio de fases del sistema, donde cada punto describe un estado dinámico en el tiempo.

Una transformación canónica es aquella transformación en variables canónicas.
Las variables canónicas son un conjunto de $2N$ ($N$ llamadas $p$ y $N$ llamadas $q$) variables que cumplen con las
ecuaciones \eqref{variables_canonicas}.

\[
	T = \frac 1 2 \sum_i \sum_j m_{ij} \dot{q}_i \dot{q}_j
\]
Aquí despejo según $ \dot{q}_i =  \dot{q}_i( q_i, p_i ) $ y se introduce en $T$ entonces se puede obtener un
$H = H(q_i,p_i) $.

\[
	T = \frac 1 2 \sum_i \sum_j m_i ( \dot{x}_i^2 + \dot{y}_i^2 + \dot{z}_i^2 )
\]
de donde se deduce 
\[
	p_{x_i} = m_i \dot{x}_i \qquad \to \qquad  \dot{x}_i = \frac{p_{x_i}}{m_i}, 
\]
y $p_{y_i}, p_{z_i}$ se obtienen análogamente.
Si $H \neq E$ para obtener $H=H(q_i,p_i)$ no hay más remedio que utilizar la definición \eqref{hamiltoniano_def}.

\begin{ejemplo}{\bf Sencillo}

Sea un hamiltoniano
\[
	H = \sum_i \frac{ p_{x_i}^2 + p_{y_i}^2 + p_{z_i}^2 }{2m} + V( \vb{x}_1,...,\vb{x}_n )
\]
\[
	T = \frac 1 2 m ( \dot{r}^2 + r^2 \dot{\theta}^2 + r^2 \sin^2 \theta \dot{\vp}^2 )
\]
y los momentos son
\[
	p_r = m \dot{r} \to \dot{r} = \frac{p_r}{m}
\]
\[
	p_\vp = m r^2 \sin^2 \theta \dot{\vp} \to \dot{\vp} = \frac{p_\vp}{m r^2 \sin^2 \theta}
\]
\[
	p_\theta = m r^2 \dot{\theta} \to \dot{\theta} = \frac{p_\theta}{m r^2}
\]

En este caso es fácil porque no hay términos cruzados. En general pasa a ser como invertir una matriz
\[
	\begin{pmatrix}
	p_r \\
	p_\vp \\
	p_\theta
	\end{pmatrix} =
	\begin{pmatrix}
	 a & b & c \\
	 d & e & f \\
	 g & h & i 
	\end{pmatrix}
	\begin{pmatrix}
	\dot{q}_r \\
	\dot{q}_\vp \\
	\dot{q}_\theta
	\end{pmatrix}
\]
Cuando no hay términos cruzados es equivalente a una matriz diagonal.
\end{ejemplo}

\subsection{Otra nomenclatura, elegante y compacta}

Si definimos
\[
	\xi = \begin{cases}
		q_i \qquad 1 \leq i \leq n \\
		p_i \qquad n + 1 \leq i \leq 2n
	      \end{cases}
\]
entonces
\[
	\dot{\xi} = J \nabla_\xi H
\]
donde el símbolo de la nabla es el gradiente de $H$ según el $\xi$ previamente definido.
Se ve que $\dot{\xi}$ y $\nabla H$ resultan ortogonales.
Donde la matriz $J$ es
\[
	J = \begin{pmatrix}
	0 & 0 & 0 & 0 & 1 & 0 & 0 & 0 \\
	0 & 0 & 0 & 0 & 0 & 1 & 0 & 0 \\	
	0 & 0 & 0 & 0 & 0 & 0 & 1 & 0 \\	
	0 & 0 & 0 & 0 & 0 & 0 & 0 & 1 \\
	-1 & 0 & 0 & 0 & 0 & 0 & 0 & 0 \\	
	0 & -1 & 0 & 0 & 0 & 0 & 0 & 0 \\
	0 & 0 & -1 & 0 & 0 & 0 & 0 & 0 \\
	0 & 0 & 0 & -1 & 0 & 0 & 0 & 0 
	\end{pmatrix}
\]
que es la matriz simpléctica.

% =================================================================================================
\section{Transformación canónica del hamiltoniano}
% =================================================================================================

Es una transformación que verifica
\[
	H \longrightarrow K
\]
donde $K=K(Q_i,P_i,t)$ es un nuevo hamiltoniano proveniente de
\[
	\dpar{H}{p_i} = \dot{q}_i \longrightarrow \dot{Q}_i = \dpar{K}{P_i}
\]
\[
	-\dpar{H}{q_i} = \dot{p}_i \longrightarrow \dot{P}_i = -\dpar{K}{Q_i}
\]

Es decir, que proponemos un nuevo hamiltoniano $K$ que es el $H$ pero en otras coordenadas.
La idea es que en esas nuevas coordenadas, el nuevo hamiltoniano sea trivial.
Las transformaciones canónicas preservan el carácter simpléctico del espacio.

Ahora usamos el Principio Variacional de Hamilton,
\[
	S = \int_{t_i}^{t_f} \Lag dt = \int_{t_i}^{t_f}  \left\{ \sum_i p_i \dot{q}_i - H(p_i,q_i,t) \right\} dt
\]
\[
	\delta S = \sum_i p_i \delta \dot{q}_i +  \dot{q}_i \delta p_i  - \dpar{H}{p_i}\delta p_i 
	-\dpar{H}{q_i}\delta q_i  - \dpar{H}{t}\delta t
\]
pero el último término es nulo porque la variación es a tiempo fijo.
Usando integración por partes en el primer término resulta que 
\[
	\delta S = \int_{t_i}^{t_f}  \left\{ \sum_i \left(-\dot{p}_i -\dpar{H}{q_i} \right) \delta q_i +
	\left( \dot{q}_i - \dpar{H}{p_i} \right) \delta p_i + \frac{d}{dt}\left( p_i \delta q_i \right) \right\} dt
\]
y llego pidiendo que sea extremo $S$ a las ecuaciones de Hamilton (dos primeros paréntesis) mientras que el 
último término resulta 
\[
	\int_{t_i}^{t_f}  \left\{ \frac{d}{dt}\left( p_i \delta q_i \right) \right\} dt =
	\left. p_i \delta q_i \right|_{t_i}^{t_f},
\]
que es nulo porque la variación se hace a extremos fijos.

Entonces, usando la misma idea que el $\Lag$ se tiene 
\[
	\Lag' = \Lag + \frac{dF}{dt}
\]
siendo $F$ una función generatriz. Luego,
\be
	\sum_i p_i \dot{q}_i - H(p_i,q_i,t) = \sum_i P_i\dot{Q}_i - K(P_i,Q_i,t) + \dtot{F}{t}
	\label{equivalencia_hamiltonianos}
\ee

Si la anterior ecuación vale me aseguro de que los hamiltonianos, en función de $(p,q)$ o de $(P,Q)$ sean equivalentes.
Obviamente esta es la condición sobre los lagrangianos.

Un mismo lagrangiano puede llevar a diferentes formas de $H$, de acuerdo a las coordenadas generalizadas.
La transformación la expresamos como
\[
	q_i = q_i(Q_i,P_i) \qquad \qquad Q_i = Q_i(p_i,q_i)
\]
\[
	p_i = p_i(Q_i,P_i) \qquad \qquad P_i = P_i(p_i,q_i)
\]
y se tienen $4n$ variables pero siendo $2n$ independientes en [¿?]
\[
	\dtot{F}{t}(2n,t)
\]

Se elegirán como variables combinaciones de variables mayúsculas y minúsculas (o imprentas y cursivas). $F$ se denomina función 
generatriz.
\[
	F_1 = F_1(q_i,Q_i,t) \qquad F_2 = F_2(q_i,P_i,t)
\]
\[
	F_3 = F_3(Q_i,p_i,t) \qquad F_4 = F_4(p_i,P_i,t)
\]

Con este sistema se introduce en \eqref{equivalencia_hamiltonianos} y se resuelve aquello que se necesita para que valga:
\begin{multline*}
	\sum_i p_i \dot{q}_i - H(p_i,q_i,t) = \sum_i P_i\dot{Q}_i - K(P_i,Q_i,t) + \dpar{F_i}{q_i} \dot{q}_i + 
\dpar{F_i}{Q_i} \dot{Q}_i + \dpar{F_i}{t} 
\end{multline*}
que conduce a 
\[
	\sum_i \underbrace{\left( p_i - \dpar{F_i}{q_i} \right)}_{=0} \dot{q}_i  - 
	\sum_i \underbrace{\left( \dpar{F_i}{Q_i} + P_i \right)}_{=0} \dot{Q}_i + 
	\underbrace{K( Q_i, P_i, t ) - H( q_i, p_i, t ) - \dpar{F_i}{t}}_{=0}  = 0
\]
que debe valer para todo tiempo $t$ y deben ser cero dentro de cada llave.
Entonces
\[
	\sum_i \left( p_i - \dpar{F_i}{q_i} \right) dq_i  - \sum_i \left( \dpar{F_i}{Q_i} + P_i \right) dQ_i + \left( K - H - 
	\dpar{F}{t} \right) dt = 0
\]
de lo cual se deducen
\begin{align*}
	\dpar{F_1}{q_i} & = p_i(q_1,...,q_n,Q_1,...,Q_n,t) \\
	\dpar{F_1}{Q_i} & = -P_i(q_1,...,q_n,Q_1,...,Q_n,t) \\
	K( Q_i, P_i, t ) & = H( q_i, p_i, t ) + \dpar{F_i}{t},
\end{align*}
que son las ecuaciones que definen la transformación canónica.

Para 
\[
	F_2(q_i,P_i) = F_1 + \sum_i P_i Q_i
\]
se tiene un diferencial
\[
	dF_2 = \dpar{F_1}{q_i} dq_i + \sum Q_i dP_i + \dpar{F_1}{t}
\]
del cual se identifican
\[
	\dpar{F_2}{q_i} = p_i \quad \dpar{F_2}{P_i} = Q_i \quad  \dpar{F_1}{t} = \dpar{F_2}{t}
\]
y así se obtiene la transformación canónica a partir de una generatriz $F_2$.
Asimismo, usando transformadas de Legendre,
\[
	F_3 = F_1 - \sum_i q_i p_i
\]
\[
	F_4 = F_1 - \sum_i q_i p_i + \sum_i Q_i P_i
\]

Entonces, basta que exista una $F$ para definir la transformación canónica.

% =================================================================================================
\section{Transformaciones canónicas. Una picture}
% =================================================================================================

\[
	(q_i,...,q_n,p_1,...,p_n) \qquad \qquad (Q_i,...,Q_n,P_1,...,P_n)
\]
\[
	\dot{p} = -\dpar{H}{q_i} \qquad \longrightarrow \qquad \dot{P} = \dpar{K}{Q_i}
\]
\[
	\dot{q} = \dpar{H}{p_i} \qquad \longrightarrow \qquad \dot{Q} = \dpar{K}{P_i}
\]

La existencia de una función generatriz $F_1,F_2,F_3,F_4$ garantiza la existencia de la parte derecha.
\[
	K(P_i,Q_i) = H(p,q,t) + \dpar{F_2}{t}
\]
con el corchete de Poisson de A con B siendo 
\[
	[A,B] = \sum_i \dpar{A}{q_i} \dpar{B}{p_i} - \dpar{A}{} \dpar{B}{q_i}
\]
\[
	\dtot{A}{t}(q_i,p_i,t) = [A,H] + \dpar{A}{t}
\]
\[
	\dot{p}_i = [p_i, H] \qquad \qquad \dot{q}_i = [q_i, H]
\]
donde requiero las condiciones
\[
	[p_i, p_k] = [q_i, q_k] = 0 \qquad [q_i, p_k]= \delta_{ik}
\]
para utilizar $p_i,q_i$ como momentos.

Recoredemos que $\partial p_i / \partial q_k = 0$ puesto que $p$ no depende de $q$, son variables independientes.

% \bibliographystyle{CBFT-apa-good}	% (uses file "apa-good.bst")
% \bibliography{CBFT.Referencias} % La base de datos bibliográfica

\end{document}
