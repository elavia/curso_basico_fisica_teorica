	\documentclass[10pt,oneside]{CBFT_book}
	% Algunos paquetes
	\usepackage{amssymb}
	\usepackage{amsmath}
	\usepackage{graphicx}
	\usepackage{libertine}
% 	\usepackage[bold-style=TeX]{unicode-math}
	\usepackage{lipsum}

	\usepackage{natbib}
	\setcitestyle{square}

	\usepackage{polyglossia}
	\setdefaultlanguage{spanish}


	\usepackage{CBFT.estilo} % Cargo la hoja de estilo

	% Tipografías
	% \setromanfont[Mapping=tex-text]{Linux Libertine O}
	% \setsansfont[Mapping=tex-text]{DejaVu Sans}
	% \setmonofont[Mapping=tex-text]{DejaVu Sans Mono}

	%===================================================================
	%	DOCUMENTO PROPIAMENTE DICHO
	%===================================================================

\begin{document}

\chapter{Ecuaciones de Hamilton}

Se pasa de las variables $(q, \dot{q})$ hacia el par $(q,p)$ con 
\[
	p = \dpar{\Lag}{\dot{q}}
\]
Se parte del 
\be
	H(q_i, p_i, t) = \sum_{i}^{3N-k} p_i \dot{q}_i - \Lag(q_i, \dot{q}_i, t)
	\label{hamiltoniano_def}
\ee
y consideramos el diferencial
\[
	dH = \sum_i p_i d\dot{q}_i + \dot{q}_i dp_i - \dpar{\Lag}{q_i} dq_i - \dpar{\Lag}{\dot{q}_i}d\dot{q}_i - \dpar{\Lag}{t}dt
\]
\[
	dH = \sum_i \dot{q}_i dp_i - \frac{d}{dt}\left( \dpar{\Lag}{\dot{q}_i} \right) dq_i - \dpar{\Lag}{t}dt
\]
\[
	dH = \sum_i \dot{q}_i dp_i - \dot{p}_i dq_i - \dpar{\Lag}{t}dt
\]
se deducen entonces,
\be
	\dpar{H}{p_i} = \dot{q}_i \qquad \qquad \dpar{H}{q_i} = -\dot{p}_i 
	\label{variables_canonicas}
\ee
y
\[
	\dpar{H}{t} = -\dpar{\Lag}{t}
\]
que son las ecuaciones de Hamilton. Donde $(p,q)$ son $2N$ grados de libertad del sistema llamados las variables canónicas.
Si $V\neq V(\dot{q})$ y los vínculos no dependen del tiempo entonces $T=T_2$ (la energía cinética es cuadrática en las 
velocidades) y $H = T + V = E$.

\notamargen{En el caso $H=E$ se puede escribir fácil el hamiltoniano.}

No presenta gran economía respecto a la formulación lagrangiana.
Aquí las coordenadas y los momentos adquieren un carácter simétrico.
Se define el espacio de fases del sistema, donde cada punto describe un estado dinámico en el tiempo.

Una transformación canónica es aquella transformación en variables canónicas.
Las variables canónicas son un conjunto de $2N$ ($N$ llamadas $p$ y $N$ llamadas $q$) variables que cumplen con las
ecuaciones \eqref{variables_canonicas}.

\[
	T = \frac 1 2 \sum_i \sum_j m_{ij} \dot{q}_i \dot{q}_j
\]
Aquí despejo según $ \dot{q}_i =  \dot{q}_i( q_i, p_i ) $ y se introduce en $T$ entonces se puede obtener un
$H = H(q_i,p_i) $.

\[
	T = \frac 1 2 \sum_i \sum_j m_i ( \dot{x}_i^2 + \dot{y}_i^2 + \dot{z}_i^2 )
\]
de donde se deduce 
\[
	p_{x_i} = m_i \dot{x}_i \qquad \to \qquad  \dot{x}_i = \frac{p_{x_i}}{m_i}, 
\]
y $p_{y_i}, p_{z_i}$ se obtienen análogamente.
Si $H \neq E$ para obtener $H=H(q_i,p_i)$ no hay más remedio que utilizar la definición \eqref{hamiltoniano_def}.

\begin{ejemplo}{\bf Sencillo}

Sea un hamiltoniano
\[
	H = \sum_i \frac{ p_{x_i}^2 + p_{y_i}^2 + p_{z_i}^2 }{2m} + V( \vb{x}_1,...,\vb{x}_n )
\]
\[
	T = \frac 1 2 m ( \dot{r}^2 + r^2 \dot{\theta}^2 + r^2 \sin^2 \theta \dot{\vp}^2 )
\]
y los momentos son
\[
	p_r = m \dot{r} \to \dot{r} = \frac{p_r}{m}
\]
\[
	p_\vp = m r^2 \sin^2 \theta \dot{\vp} \to \dot{\vp} = \frac{p_\vp}{m r^2 \sin^2 \theta}
\]
\[
	p_\theta = m r^2 \dot{\theta} \to \dot{\theta} = \frac{p_\theta}{m r^2}
\]

En este caso es fácil porque no hay términos cruzados. En general pasa a ser como invertir una matriz
\[
	\begin{pmatrix}
	p_r \\
	p_\vp \\
	p_\theta
	\end{pmatrix} =
	\begin{pmatrix}
	 a & b & c \\
	 d & e & f \\
	 g & h & i 
	\end{pmatrix}
	\begin{pmatrix}
	\dot{q}_r \\
	\dot{q}_\vp \\
	\dot{q}_\theta
	\end{pmatrix}
\]
Cuando no hay términos cruzados es equivalente a una matriz diagonal.
\end{ejemplo}


% =================================================================================================
\section{Transformación canónica del hamiltoniano}
% =================================================================================================

Es una transformación que verifica
\[
	H \longrightarrow K
\]
donde $K=K(Q_i,P_i,t)$ es un nuevo hamiltoniano proveniente de
\[
	\dpar{H}{p_i} = \dot{q}_i \longrightarrow \dot{Q}_i = \dpar{K}{P_i}
\]
\[
	-\dpar{H}{q_i} = \dot{p}_i \longrightarrow \dot{P}_i = -\dpar{K}{Q_i}
\]
y ahora usamos el Principio Variacional de Hamilton,
\[
	S = \int_{t_i}^{t_f} \Lag dt = \int_{t_i}^{t_f}  \left\{ \sum_i p_i \dot{q}_i - H(p_i,q_i,t) \right\} dt
\]
\[
	\delta S = \sum p_i \delta \dot{q}_i +  \dot{q}_i \delta p_i  - \dpar{H}{p_i}\delta p_i 
	-\dpar{H}{q_i}\delta q_i  - \dpar{H}{t}\delta t
\]
pero el último término es nulo porque la variación es a tiempo fijo.
Usando las ecuaciones de Euler-Lagrange en el primer término resulta que 
\[
	\delta S = \int_{t_i}^{t_f}  \left\{ \sum_i \left(-\dot{p}_i -\dpar{H}{q_i} \right) \delta q_i +
	\left( \dot{q}_i - \dpar{H}{p_i} \right) \delta p_i + \frac{d}{dt}\left( p_i \delta q_i \right) \right\} dt
\]
y llego pidiendo que sea extremo $S$ a las ecuaciones de Hamilton (dos primeros paréntesis) mientras que el 
último término resulta 
\[
	\int_{t_i}^{t_f}  \left\{ \frac{d}{dt}\left( p_i \delta q_i \right) \right\} dt =
	\left. p_i \delta q_i \right|_{t_i}^{t_f}.
\]

Entonces, usando la misma idea que el $\Lag$ se tiene 
\[
	\Lag' = \Lag + \frac{dF}{dt}
\]
siendo $F$ una función generatriz. Luego,
\[
	\sum p_i \dot{q}_i - H(p_i,q_i,t) = \sum P_i\dot{Q}_i - K(P_i,Q_i,t) + \dtot{F}{t}
\]









% \bibliographystyle{CBFT-apa-good}	% (uses file "apa-good.bst")
% \bibliography{CBFT.Referencias} % La base de datos bibliográfica

\end{document}
