	\documentclass[10pt,oneside]{CBFT_book}
	% Algunos paquetes
	\usepackage{amssymb}
	\usepackage{amsmath}
	\usepackage{graphicx}
	\usepackage{libertine}
	\usepackage[bold-style=TeX]{unicode-math}
	\usepackage{lipsum}

	\usepackage{natbib}
	\setcitestyle{square}

	\usepackage{polyglossia}
	\setdefaultlanguage{spanish}


	\usepackage{CBFT.estilo} % Cargo la hoja de estilo

	% Tipografías
	% \setromanfont[Mapping=tex-text]{Linux Libertine O}
	% \setsansfont[Mapping=tex-text]{DejaVu Sans}
	% \setmonofont[Mapping=tex-text]{DejaVu Sans Mono}

	%===================================================================
	%	DOCUMENTO PROPIAMENTE DICHO
	%===================================================================

\begin{document}

\chapter{Ecuaciones de Hamilton-Jacobi}

\[
	q_i \longrightarrow Q_i \equiv \beta_i \qquad p_i \longrightarrow P_i \equiv \alpha_i
\]
Pasamos a unas nuevas coordenadas y momentos $(\beta_i,\alpha_i)$ que son constantes. Entonces
la acción es del tipo $F_2$, i.e.
\[
	S = S(q_i, \alpha_i, t).
\]
Entonces
\be
	\dpar{S}{q_i} = p_i \qquad \dpar{S}{\alpha_i} = \beta_i \qquad \dpar{S}{t} = H - K  
\label{ecshamjac}	
\ee
donde 
\[
	H(q_i,p_i,t) - \dpar{S}{t} = K = 0
\]
y esto lleva a la ecuación de Hamilton-Jacobi,
\[
	H(q_i,p_i,t) - \dpar{S}{t} = 0
\]
que no es otra cosa que una ecuación en derivadas parciales (PDE). Notemos que 
\[
	\dpar{S}{q_i} = p_i(q_i,\alpha_i,t) \qquad \dpar{S}{\alpha_i} = \beta_i(q_i,\alpha_i,t)
\]
y además que Hamilton-Jacobi tiene solución si el problema es totalmente separable.
Si $H=H(q_i,\alpha_i)$ entonces $dH/dt = \partial H/\partial t=0$ y en ese caso es $H=cte.$ y
podemos poner $H=\alpha_1$.
Entonces
\[
	\dpar{S}{t} = -\alpha_1 \quad \longrightarrow \quad S=W(q_i,\dpar{S}{q_i}) -\alpha_1 t .
\]

Se procede en la misma forma con cada coordenada hasta obtener $S$.

Podemos ver que si $\alpha_1 = \alpha_1(\alpha_i)$, y me quedo con $H=\alpha_1 \equiv K$ entonces
\[
	\dpar{K}{\alpha_i} = a = \dot{Q}_i \longrightarrow Q_i = \beta = a t + \beta_0 
\]
\[
	\dpar{K}{\beta_i} = 0 = -\dot{P}_i \longrightarrow P_i = \alpha_i (ctes.).
\]

La $\alpha_1$ no puede depender de $q_i$ pues si se tuviera $\partial \alpha_1 /\partial q_i \neq 0$ 
no sería constante $\alpha_1$ pues $\dot{q}\neq 0$.

Luego, invirtiendo las ecuaciones \eqref{ecshamjac} determinamos las trayectorias
\[
	q_i = q_i(\alpha_i, \beta_i, t).
\]

Además, si el problema es totalmente separable, entonces
\[
	S = \sum_i^N \; W(q_i, \alpha_1,...,\alpha_n) - \alpha_1 t
\]
y tendré tantas constantes de movimiento como grados de libertad. La solución se compone de problemas
independientes en una variable.

\subsection{Preservación del volumen en una transformación canónica}

Definamos un hipervolumen $\mathcal{V}$ en el espacio de fases de acuerdo a
\[
	\int dq_1 dq_2 ... dq_n dp_1 dp_2 ... dp_n = \mathcal{V}_{p,q}
\]
\[
	\int dQ_1 dQ_2 ... dQ_n dP_1 dP_2 ... dP_n = \mathcal{V}_{P,Q}
\]
El jacobiano de la transformación es 
\[
	\frac{\partial (Q_1,...,Q_n,P_1,...,P_n)}{\partial (q_1,...,q_n,p_1,...,p_n)} =
	\frac{\partial (Q_1,...,Q_n,P_1,...,P_n)/\partial (q_1,...,q_n,P_1,...,P_n)|_{P_i=cte}}
	{\partial (q_1,...,q_n,p_1,...,p_n)/\partial (q_1,...,q_n,P_1,...,P_n)|_{q_i=cte}}
\]
que en notación de matriz es 
\[
	\begin{pmatrix}
	\dpar{Q_1}{q_1} & \dpar{Q_1}{q_2} & ... & \dpar{Q_1}{p_n} \\
	.. & .. & .. & .. \\
	\dpar{P_n}{q_1} & .. & .. & \dpar{P_n}{p_n}
	\end{pmatrix}
\]
Entonces 
\[
	J_{ij}^{num} = \dpar{Q_i}{q_j} = \frac{\partial}{\partial q_j} \left(\dpar{F_2}{P_i} \right)
\]
y
\[
	J_{ij}^{den} = \dpar{p_i}{P_j} = \frac{\partial}{\partial P_j} \left(\dpar{F_2}{q_i} \right)
\]
pero como estas dos expresiones son iguales se tiene que $J=1$ y entonces se conserva
el volumen, aunque cambiando de forma.

En sistemas de un grado de libertad
\[
	A_{p,q} = \int dp dq \qquad A_{P,Q} = \int dP dQ
\]
y el jacobiano
\[
	J = \begin{vmatrix}
	     \dpar{Q}{q} & \dpar{Q}{p} \\
	     \dpar{P}{q} & \dpar{P}{p}
	    \end{vmatrix} =
	    \dpar{Q}{q} \dpar{P}{p} - \dpar{Q}{p} \dpar{P}{q} = [Q, P] = 1
\]
Notamos que le corchete de Poisson para una transformación canónica en un grado de
libertad es el corchete que ya sabíamos da uno. El área se conserva.

Comentemos que un sistema disipativo achica el área de la transformación.

\subsection{Variables ángulo-acción}

Consideremos una transformación canónica 
\[
	p,q \longrightarrow J,\theta
\]
la cual requiere
\begin{itemize}
 \item Conservativos $S = W - Et $
 \item Totalmente separables $W = \sum_i^N \; W_i(q_1,\alpha_1,...,\alpha_n)$
 \item Problemas periódicos
\end{itemize}

El movimiento periódico es de rotación o libración,

La periodicidad de cada coordenada no implica periodicidad de todo el movimiento real.
\[
	S = \sum_i^N \; W_i(q_i,J_i) - Et
\]

Libración y rotación son dos movimientos de naturaleza diferente. No se puede pasar de
uno a otro mediante pequeñas perturbaciones.

La integral de acción es
\[
	J_i = \frac{1}{2\pi}\int_{ciclo} p_i(q_1,\alpha_1,...,\alpha_n) dq_i
\]
donde 
\[
	J_i = J_i(\alpha_1,...,\alpha_n)
\]
son constantes y a su vez los $\alpha_i$ son constantes de separación.
Asimismo $\alpha_i=\alpha_i(J_1,...,J_n)$. 
La transformación $S$ es 
\[
	\dpar{S}{q_i} = p_i = \dpar{W}{q_i} \qquad \dpar{S}{J_i} = \theta_i = \dpar{W}{J_i}
\]
siendo $p_i = p_i(q_1,J_1,...,J_n)$.
El nuevo hamiltoniano es $E=E(J_1,...,J_n)$
\[
	\dpar{E}{J_i} = \dot{\theta}_i \equiv \omega \qquad \dpar{E}{\theta_i} = -\dot{J}_i
\]






























% \bibliographystyle{CBFT-apa-good}	% (uses file "apa-good.bst")
% \bibliography{CBFT.Referencias} % La base de datos bibliográfica

\end{document}
