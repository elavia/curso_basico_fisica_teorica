	\documentclass[10pt,oneside]{CBFT_book}
	% Algunos paquetes
	\usepackage{amssymb}
	\usepackage{amsmath}
	\usepackage{graphicx}
	\usepackage{libertine}
% 	\usepackage[bold-style=TeX]{unicode-math}
	\usepackage{lipsum}

	\usepackage{natbib}
	\setcitestyle{square}

	\usepackage{polyglossia}
	\setdefaultlanguage{spanish}


	\usepackage{CBFT.estilo} % Cargo la hoja de estilo

	% Tipografías
	% \setromanfont[Mapping=tex-text]{Linux Libertine O}
	% \setsansfont[Mapping=tex-text]{DejaVu Sans}
	% \setmonofont[Mapping=tex-text]{DejaVu Sans Mono}

	%===================================================================
	%	DOCUMENTO PROPIAMENTE DICHO
	%===================================================================

\begin{document}

\chapter{Ecuaciones de Hamilton-Jacobi}


\section{Introducción a la formulación de Hamilton}

Un sistema mecánico está caracterizado por $\{ q_i, \dot{q}_i \}$ las cuales dan un estado posible del sistema, y además
dan toda la información dinámica del mismo.

Ahora se describirá al sistema en términos de $ q_i, p_i \equiv \partial\Lag/\partial\dot{q}_i$  que tiene la característica
de producir una simetría en la mecánica así definida (la mecánica hamiltoniana).
La simetría es tal que son intercambiables $q_i$ y $p_i$.

Para las ecuaciones de movimiento se parte del hamiltoniano
\[
	\Ham = \sum_i p_i \dot{q}_i - \Lag( q_i, \dot{q}_i, t ),
\]
donde $\Ham =\Ham(q_i,p_i,t)$ tiene la misma información que $\Lag=\Lag(q_i,\dot{q}_i,t)$.

Se puede hacer una analogía con la termodinámica, pues la primer ley se escribe
\[
	dE = dQ - dW = \left. \dpar{E}{S} \right|_V dS - \left. \dpar{E}{V} \right|_S dV,
\]
lo cual implica que usando $S,V$ tengo como ``potencial'' a la energía.
Un estado termodinámico se define por dos variables; $(S,V), (T,P), (S,P), (T,V)$ que son cada par variables conjugadas.

Para definir estos potenciales se usan transformadas de Legendre. Así,
\[
	d(E-TS) = TdS -PdV - TdS -SdT = -PdV - SdT \equiv dA
\]
siendo $A$ la energía libre de Helmholtz.
\[
	d(E+PV) = TdS -PdV + PdV + VdP = TdS + VdP \equiv dH
\]
siendo $H$ la entalpía.

En el caso del Hamiltoniano se tiene 
\[
	d\Ham = \sum_i \dot{q}_i dp_i  + \sum_i p_i d\dot{q}_i - \sum_i \dpar{\Lag}{q_i} dq_i -
	\sum_i d \dpar{\Lag}{\dot{q}_i}\dot{q}_i - \dpar{\Lag}{t}dt 
\]
la cual usando las ecuaciones de Euler-Lagrange y el hecho de que $\partial{\Lag}/\partial{\dot{q}_i} $ es el momento conjugado
$p_i$ se tiene 
\[
	d\Ham = \sum_i \dot{q}_i dp_i  - \sum_i \dot{p}_i dq_i - \dpar{\Lag}{t}dt 
\]
y como esta ecuación es el diferencial total del hamiltoniano se tiene que
\[
	\dpar{\Ham}{t} = -\dpar{\Lag}{t} = \dtot{\Ham}{t}
\]
siendo la última igualdad una derivación vista oportunamente. Asimismo,
\[
	\dpar{\Ham}{p_i} = \sum_i \dot{q}_i \qquad \qquad 
	\dpar{\Ham}{q_i} = -\sum_i \dot{p}_i
\]
que son una mayor cantidad de ecuaciones pero de orden uno (comparando con las ecuaciones del sistema en el formalismo 
lagrangiano).

Con esto definimos un espacio de fases $(p_i,q_i)$ de $2N$ dimensiones para estudiar el movimiento de un sistema de partículas.
En el caso particular de una única partícula tendremos dos variables, $(p,q)$.

\subsection{La idea de Hamilton-Jacobi}

La idea es que se busca una transformación canónica que me transporte a un hamiltoniano nuevo donde toda la solución son 
constantes. Es decir 
\[
	H(q_i,p_i) \longrightarrow K(Q_i,P_i),
\]
donde
\be
	q_i \longrightarrow Q_i \equiv \beta_i \qquad p_i \longrightarrow P_i \equiv \alpha_i
	\label{coord_hamjac}
\ee
% para $\alpha,\beta$ constantes.
Pasamos a unas nuevas coordenadas y momentos $(\beta_i,\alpha_i)$ que son constantes. 
Aunque esto requiere conocer el problema (su solución). Esta transformación existe porque es ir atrás en el tiempo;
la antievolución.

Supongamos una generatriz del tipo $F_2 = S$, llamada {\it función principal de Hamilton}
\[
	S = S(q_i, \alpha_i, t).
\]
Entonces
\be
	\dpar{S}{q_i} = p_i \qquad \dpar{S}{\alpha_i} = \beta_i \qquad \dpar{S}{t} = H - K  
	\label{ecshamjac}
\ee
donde 
\[
	H(q_i,p_i,t) - \dpar{S}{t} = K = 0
\]
que es la condición necesaria para garantizar las condiciones \eqref{coord_hamjac}.
Esto lleva a la ecuación de Hamilton-Jacobi,
\be
	H(q_i,p_i,t) - \dpar{S}{t} = 0
	\label{ecshamjac2}
\ee
que no es otra cosa que una ecuación en derivadas parciales (PDE) al especializar $H$ en las derivadas parciales
\[
	H \left( q_i, \dpar{S}{q_i}, t \right) - \dpar{S}{t} = 0,
\]
que son $n+1$ variables y $n+1$ constantes (una es trivial porque la ecuación \eqref{ecshamjac2} no depende de $S$ sino de sus 
derivadas).

\begin{ejemplo}{\bf ejemplito}
\[
	H = \frac{p^2}{2m} + V(q)
\]
\[
	\frac{1}{2m} \Dpar{S}{q}^2 + V(q) + \dpar{S}{t} = 0
\]
hay que resolverlo utilizando condiciones iniciales
\[
	H(q_i,p_i,t) \qquad p_i(t=0) \qquad q_i(t=0).
\]
\end{ejemplo}

Notemos que 
\[
	\dpar{S}{q_i} = p_i(q_i,\alpha_i,t) \qquad \dpar{S}{\alpha_i} = \beta_i(q_i,\alpha_i,t)
\]

Cuando la ecuación es separable se puede garantizar la solución de Hamilton-Jacobi.
Si $H=H(q_i,\alpha_i)$, el hamiltoniano no depende del tiempo, entonces $dH/dt = \partial H/\partial t=0$ y en ese caso es 
$H=cte.$ (la energía). Se tiene
\[
	H \left( \dpar{S}{q_1},...,\dpar{S}{q_N},q_1,...,q_N \right) + \dpar{S}{t} = 0, 
\]
o bien
\[
	\dpar{S}{t} = -E,
\]
por lo tanto es separable en el tiempo. Entonces
\[
	S = W(q_1,...,q_N,\alpha_1,...,\alpha_N) - E t,
\]
donde $W$ no depende del tiempo, que sólo aparece explícito en el segundo término. Luego
\[
	\dpar{S}{q_i} = \dpar{W}{q_i}
\]
con lo cual 
\[
	H \left( \dpar{W}{q_1},...,\dpar{W}{q_N},q_1,...,q_N \right) = E
\]
y tengo un nuevo hamiltoniano que no vale cero sino que vale $E$.
Pase a {\it un lugar} donde los momentos son constantes y las coordenadas son cíclicas.
\[
	E = E(\alpha_1, ..., \alpha_N )
\]
de manera que $\partial E/\partial \alpha_N = a$ y luego,
\[
	Q = at + Q_0 \qquad \text{ las $Q$ son lineales}
\]
Entonces,
\[
	\dpar{E}{\alpha} = \dot{Q} = a \qquad \text{ una constante}
\]
y
\[
	\dpar{E}{p} = \dpar{E}{\alpha} = \dot{Q}.
\]

Remarquemos que si fuera $E=E(q)$ entonces $\partial E \ \partial q = \dot{\alpha}$ y no sería constante $\alpha$, pero $E$ no 
depende de $Q_i$.

\begin{ejemplo}{\bf Ejemplito de un grado de liberad}

\[
	H = \frac{p^2}{2m} + V(q) \qquad \qquad S = W - Et
\]
\[
	E = \frac{1}{2m} \Dtot{W}{q}^2 + V(q), 
\]
de modo que 
\[
	\dtot{W}{q} = \pm \sqrt{2m[E-V(q)]},
\]
\[
	W(q,E) = \pm \int \sqrt{2m[E-V(q)]} dq - Et
\]

Para un grado de libertad siempre tendrá esta forma.
\end{ejemplo}

Para más grados de libertad deberíamos poder separar alguna coordenada en igual forma. Si $q_1$ no aparece, entonces la derivada 
del hamiltoniano $H$ respecto a $q_1$ dice que será constante el $\dot{q}_1$, es decir
\[
	\dpar{W}{q_1} = \alpha_2 \quad \to \quad 
	H\left( \alpha_2, \dpar{W}{q_2},...,\dpar{W}{q_N},q_2,...,q_N \right) = E.
\]

Es más, si la $S$ es totalmente separable de la forma 
\[
	S(q_i,\alpha_0) = \sum_{i=1}^N \; W_i(q_i,\alpha_1,...,\alpha_N) - Et
\]
lo cual, dicho sea de paso, requerirá $N$ constantes de movimiento, entonces se puede resolver completamente.

[Lo que sigue es un refrito de lo anterior o viceversa; habría que consolidarlo]
[...] y podemos poner $H=\alpha_1$.
Entonces
\[
	\dpar{S}{t} = -\alpha_1 \quad \longrightarrow \quad S=W(q_i,\dpar{S}{q_i}) -\alpha_1 t .
\]

Se procede en la misma forma con cada coordenada hasta obtener $S$.

Podemos ver que si $\alpha_1 = \alpha_1(\alpha_i)$, y me quedo con $H=\alpha_1 \equiv K$ entonces
\[
	\dpar{K}{\alpha_i} = a = \dot{Q}_i \longrightarrow Q_i = \beta = a t + \beta_0 
\]
\[
	\dpar{K}{\beta_i} = 0 = -\dot{P}_i \longrightarrow P_i = \alpha_i (ctes.).
\]

La $\alpha_1$ no puede depender de $q_i$ pues si se tuviera $\partial \alpha_1 /\partial q_i \neq 0$ 
no sería constante $\alpha_1$ pues $\dot{q}\neq 0$.

Luego, invirtiendo las ecuaciones \eqref{ecshamjac} determinamos las trayectorias
\[
	q_i = q_i(\alpha_i, \beta_i, t).
\]

Además, si el problema es totalmente separable, entonces
\[
	S = \sum_i^N \; W(q_i, \alpha_1,...,\alpha_n) - \alpha_1 t
\]
y tendré tantas constantes de movimiento como grados de libertad. La solución se compone de problemas
independientes en una variable.

% =================================================================================================
\section{Preservación del volumen en una transformación canónica}
% =================================================================================================

Definamos un hipervolumen $\mathcal{V}$ en el espacio de fases de acuerdo a
\[
	\int dq_1 dq_2 ... dq_n dp_1 dp_2 ... dp_n = \mathcal{V}_{p,q}
\]
que en otras coordenadas es
\[
	\int dQ_1 dQ_2 ... dQ_n dP_1 dP_2 ... dP_n = \mathcal{V}_{P,Q}.
\]

\begin{figure}
	\begin{center}
	\includegraphics[width=0.4\textwidth]{images/fig_mc_hamjac2.pdf}	 
	\end{center}
	\caption{}
\end{figure} 

El jacobiano de la transformación, que permite convertir una integral en la otra, es 
\[
	\frac{\partial (Q_1,...,Q_n,P_1,...,P_n)}{\partial (q_1,...,q_n,p_1,...,p_n)} 
\]
y puede verse que vale 1. En efecto, como vale una especie de {\it chain rule}
\[
	\frac{\partial (Q_1,...,Q_n,P_1,...,P_n)}{\partial (q_1,...,q_n,p_1,...,p_n)}  =
	\frac{\partial (Q_1,...,Q_n,P_1,...,P_n)/\partial (q_1,...,q_n,P_1,...,P_n)|_{P_i=cte}}
	{\partial (q_1,...,q_n,p_1,...,p_n)/\partial (q_1,...,q_n,P_1,...,P_n)|_{q_i=cte}}
\]

El jacobiano en notación matricial es 
\[
	\begin{pmatrix}
	\dpar{Q_1}{q_1} & \dpar{Q_1}{q_2} & ... & \dpar{Q_1}{p_n} \\
	.. & .. & .. & .. \\
	\dpar{P_n}{q_1} & .. & .. & \dpar{P_n}{p_n}
	\end{pmatrix}
\]
Entonces se puede ver que para el numerador es
\[
	J_{ij}^{num} = \dpar{Q_i}{q_j} = \frac{\partial}{\partial q_j} \left(\dpar{F_2}{P_i} \right)
\]
mientras que para el denominador,
\[
	J_{ij}^{den} = \dpar{p_i}{P_j} = \frac{\partial}{\partial P_j} \left(\dpar{F_2}{q_i} \right)
\]
pero como estas dos expresiones son iguales se tiene que $J=1$ y entonces se conserva
el volumen, aunque cambiando de forma (se deforma la cáscara pero el volumen se conserva).

Usando $|M| = |M^t|$ entonces se ve que vale uno el cociente de los jacobianos. Siempre y cuando sea
la transformación canónica.

Son invariantes canónicos
\[
	\int\int \sum_{i=1}^N \; dq_i dp_i \qquad \qquad  \int\int \sum_{j=1}^N \sum_{i=1}^N \; dq_i dp_i dq_j dp_j
\]

En sistemas de un grado de libertad
\[
	A_{p,q} = \int dp dq \qquad A_{P,Q} = \int dP dQ
\]
y el jacobiano
\[
	J = \begin{vmatrix}
	     \dpar{Q}{q} & \dpar{Q}{p} \\
	     \dpar{P}{q} & \dpar{P}{p}
	    \end{vmatrix} =
	    \dpar{Q}{q} \dpar{P}{p} - \dpar{Q}{p} \dpar{P}{q} = [Q, P] = 1.
\]
\begin{figure}[!h]
	\begin{center}
	\includegraphics[width=0.6\textwidth]{images/fig_mc_hamjac1gl.pdf}	 
	\end{center}
	\caption{}
\end{figure} 

Notamos que el jacobiano en una transformación canónica para un sistema de un grado de libertad es el corchete de Poisson 
de $Q,P$, y además da uno. El área se conserva.

Comentemos que un sistema disipativo achica el área de la transformación.

En la transformación, el número de puntos en $p,q$  es el mismo en $PQ$ pero la forma que adopta varía. Es como un líquido
incompresible.

La transformación canónica cumple que, se parte de una punto a otro por una trayectoria que no corta con ninguna otra.

\includegraphics[width=0.6\textwidth]{images/fig_mc_espacio_fases.jpg}	

No confundir espacio de fases con espacio de configuración.

% =================================================================================================
\section{Variables ángulo-acción}
% =================================================================================================

La idea es introducir un nuevo juego de variables canónicas. Para ello consideremos una transformación 
canónica 
\[
	(p, q) \longrightarrow (J, \theta)
\]
con generatriz $W$ que cumple $W-Et=S$. 
Estas variables se pueden utilizar solamente con
\begin{itemize}
	\item Problemas conservativos, donde es $S = W - Et $
	\item Problemas completamente separables, con
	\[
		W = \sum_i^N \; W_i(q_1,\alpha_1,...,\alpha_n)
	\]
	\item Problemas de movimiento periódico (condicionalmente)
\end{itemize}

Solamente si se verifican estos tres supuestos, se pueden utilizar las variables $(J,\theta)$.

El movimiento periódico es de dos tipos, de rotación o libración,

% \begin{figure}[htb]
% 	\begin{center}
	\includegraphics[width=0.7\textwidth]{images/fig_mc_rot_lib.jpg}	 
% 	\end{center}
% 	\caption{}
% \end{figure} 

Si tengo más de un grado de libertad tendré un gráfico $pq$ para cada par y puedo tener una
libración o una rotación (si el problema es separable). De lo contrario no podré expresar
cada $p$ en función de su $q$ correspondiente.

\subsection{Algunos claims sueltos}

Dado un problema puede ser conveniente pasar a un hamiltoniano $H'$ más sencillo, donde la relación
con el anterior es
\[
	H' = H + \dpar{S}{t},
\]
y si $S$ se elige del tipo $F_2$ se tiene $S = W - Et$. La $S$ depende de variables viejas y nuevas
(en mitades). El hamiltoniano más conveniente es $H'=0$
\[
	S = W_1(q_1, \alpha_1, ..., \alpha_n) + W_2(q_2, \alpha_1, ..., \alpha_n) - Et,
\]
y es tal que 
\[
	p = \dpar{S}{q} \qquad Q = \dpar{S}{\alpha} = B = cte
\]

Las integrales que son $W$ conviene esperar para primero derivarlo [?].
Ángulo-acción es un caso particular.
Necesito periodicidad de los grados de libertad (no de todos igual período a la vez).
La acción es $J$ y la tomamos como el nuevo momento,
\[
	J = \frac{1}{2\pi } \oint p dq,
\]
donde la integral se hace en un ciclo y el $2\pi$ está para que sea angular. Sin este factor será
la frecuencia natural. Entonces el hamiltoniano será la energía, $H=H'=E$ e ignoramos la parte
temporal.
\[
	\omega = \dpar{E}{J}
\]

Con esto ya obtenemos la frecuencia. Si queremos la solución completa habría que invertir.
La geometría da $p,q \to J,\theta$ es $W = \int pdq$ generatriz.
\[
	\theta = \dpar{W}{J} = \omega t + \theta_0.
\]

En el espacio de fases el área de una trayectoria está asociada con la generatriz

\includegraphics[width=0.5\textwidth]{images/fig_mc_angacc.jpg}

\begin{ejemplo}{\bf Partícula en un pozo de potencial}

Tenemos una partícula en un pozo de potencial

\includegraphics[width=0.5\textwidth]{images/fig_mc_angacc2.jpg}
 
Se tienen 
\[
	H = \frac{p^2}{2m} \qquad E = \frac{m v^2}{2} \qquad p = m v
\]
y entonces
\[
	J = \frac{1}{\pi} \int \sqrt{2 m E} dq = \sqrt{2 m E} d,
\]
y como 
\[
	\omega = \frac{\pi p}{m d} = \frac{2\pi v}{2d}
\]
siendo el período $\tau = 2d/v$. Podemos conectar $E$ con $p$ a través de estas fórmulas.
Pasamos a un problema donde todas las coordenadas son cíclicas.
\[
	W = \int p dq = \begin{cases}
			\sqrt{ 2 m E } \; q = \frac{J \pi q}{d} \text{ ida } \\
			\\
			\sqrt{ 2 m E } \; ( d + d - q) = \frac{J \pi (2d-q)}{d} \text{ vuelta }
	                \end{cases}
\]
y
\[
	p = \dpar{W}{q} = \begin{cases}
			\sqrt{ 2 m E } \\
			\\
			-\sqrt{ 2 m E }
	                 \end{cases}
\]
\[
	\theta = \dpar{W}{J} = \begin{cases}
	                        \frac{\pi q}{d} \\
	                        \\
	                        \frac{\pi}{d}(2d-q)
	                       \end{cases}
\]
que conducen a
\[
	q = \begin{cases}
		\frac{d}{\pi} ( \omega t + \theta_0 ) \\
		\\
		2d - \frac{d}{\pi} ( \omega t + \theta_0 )
	    \end{cases}
\]
 
\end{ejemplo}


\subsection{Péndulo}

La energía en el caso del péndulo es
\[
	E = \frac{p^2}{2m} + m g \ell ( 1 - \cos q )
\]
y un grado de libertad es separable siempre si se conserva la energía.
El péndulo realiza dos movimientos,
\[
	E > 2mg\ell \quad \text{ rotación} \qquad \qquad E < 2mg\ell \quad \text{ libración}
\]

\includegraphics[width=0.7\textwidth]{images/fig_mc_pendulo_angacc1.jpg}

\includegraphics[width=0.7\textwidth]{images/fig_mc_pendulo_angacc2.jpg}

En un sistema con varios grados de libertad, cada grado de libertad deberá tener un grafo de
esta forma.
Se ve que para pequeñas apartaciones $q$ es
\[
	E \approx \frac{p^2}{2m} + \frac{ m g \ell q^2 }{2},
\]
lo cual no es otra cosa que el gráfico de una elipse. Luego, será una elipse {\it toda la vida} mediante
suma de pequeños desplazamientos (perturbaciones). La idea es, entonces, que no podemos pasar de una
libración a una rotación mediante perturbaciones pequeñas. 
Libración y rotación son dos movimientos de naturaleza diferente.
No se puede atravesar la separatriz perturbando una solución de un movimiento.

\includegraphics[width=0.3\textwidth]{images/fig_mc_pendulo_angacc3.jpg}

La integral de acción es
\[
	J_i = \frac{1}{2\pi}\int_{ciclo} p_i(q_1,\alpha_1,...,\alpha_n) dq_i
\]
donde 
\[
	J_i = J_i(\alpha_1,...,\alpha_n)
\]
son los nuevos momentos, constantes, y a su vez los $\alpha_i$ son constantes de separación.
Asimismo $\alpha_i=\alpha_i(J_1,...,J_n)$. Una integral de acción para cada grado de libertad.
Esta integral las puedo hacer siempre dadas las condiciones que se supusieron.

\includegraphics[width=0.8\textwidth]{images/fig_mc_pendulo_angacc4.jpg}

\begin{ejemplo}{\bf Comentario central forces}

En un problema de fuerzas centrales, por ejemplo, se tienen
\[
	p_r = p_r(r,\ell^2,\ell_z,E) \qquad p_\vp = p_\vp(\vp) \quad p_\theta =cte
\]
que son periódicas en cada coordenada pero no el movimiento total porque en general
no coinciden los períodos de todos los movimientos en $r,\theta,\phi$.
Es decir, que la a periodicidad de cada coordenada no implica periodicidad de todo el movimiento real.

\includegraphics[width=0.7\textwidth]{images/fig_mc_pendulo_angacc5.jpg}

\includegraphics[width=0.7\textwidth]{images/fig_mc_pendulo_angacc6.jpg}

\end{ejemplo}

\subsection{Oscilador armónico}

La energía es
\[
	E = \frac{p^2}{2m} + \frac{1}{2} m \omega^2 q^2
\]
 
\includegraphics[width=0.5\textwidth]{images/fig_mc_osciladorarm_angacc.jpg}

de manera que
\[
	1 = \frac{p^2}{2mE} + \frac{m\omega^2 q^2}{2E}
\]
y
\[
	J = \frac{1}{2\pi}(\pi a b )= \frac{E}{\omega}.
\]

Para el oscilador armónico es entonces
\[
	\dpar{E}{J} = \dot{\theta}
\]
y luego
\[
	\theta = \omega t + \theta_0
\]
de modo que 
\[
	E = E(J_1, J_2, ..., J_n)
\]
siempre puedo despejar $E$ en función de los nuevos momentos.
\[
	\dpar{E}{J_i} = \dot{\theta}_i = \omega_i(J_1,...,J_n)
\]
\be
	\theta_i = \omega_i(J_1,...,J_n) t + \theta_0
	\label{sol_theta}
\ee
y habrá órbitas cerradas dependiendo de las condiciones iniciales de los $J_i$.
Ahora faltaría obtener los $q(t),p(t)$ a partir de la  \eqref{sol_theta}. Sabemos que 
\[
	W = \sum_i^N W_i(q_i,\alpha_1,...,\alpha_n),
\]
la generatriz total es una suma y los $\alpha_i$ son las constantes de separación. Necesito
\[
	W = W( q_i, J_1, ..., J_n ).
\]

Utilizo $\alpha_i( J_1, J_2, ..., J_n )$ (que se obtuvieron de las $n$ integrales de acción) y $\partial W /\partial J_i = \omega_t t + \theta_0 $.
Luego tendré $n$ relaciones del tipo
\be
	\theta_i( q_i, J_1, ..., J_n ) = \dpar{\theta}{J_i} = \omega_i t + \theta_0
	\label{sistema_theta}
\ee
y $n$ ecuaciones 
\[
	\dpar{W_i}{q_i} = p_i = p_i( q_i, J_1, ..., J_n ) 
\]
usando condiciones iniciales $(q_1, ..., q_n), (p_1, ..., p_n)$ en $t=0$ especializando a $t=0$ obtengo $J_1, ..., J_n$ constantes.
\[
	W = \sum_i^N \; W_i(q_i,J_1,...,J_n). 
\]

En el sistema \eqref{sistema_theta} con la $E$ hago 
\[
	\dpar{E}{J_i}(J_1,...,J_n) = \dot{\theta}_i(J_1,...,J_n)
\]
entonces $\theta_i = \omega_i (J_1,...,J_n) t + \theta_0$ que es la solución del problema en el espacio mecánico transformado.
Los $J_1,...,J_n$ son constantes porque dependen de constantes $\alpha_i$ .

Luego, consideramos $t=0$ y $\theta_i = \theta_i ( q_1,...,q_n,J_1,...,J_n)$ provee los $q_i = q_i(t)$. Se tiene 
\[
	p_\ell = \dpar{W_\ell}{q_\ell} = p_\ell( q_\ell, J_1, ..., J_n ).
\]

Para el oscilador armónico
\[
	E = \omega J \qquad \qquad \theta = \omega t + \theta_0
\]
\[
	W(q,E) = \int \frac{dq}{ [ 2 m \left( E - \frac{1}{2} m \omega^2 q^2 \right) ]^{-1/2} } = 
	W(q,J) = \int \frac{dq}{ [ 2 m \left( \omega J - \frac{1}{2} m \omega^2 q^2 \right) ]^{-1/2} },
\]
donde el denominador contiene el $\acos$ de algo.
\[
	\dpar{W}{J}(q,J) = \theta (t) = \omega t + \theta_0 = \int \frac{ m \omega dq }{ [ 2 m \left( \omega J - \frac{ m \omega^2 q^2 }{2} \right) ]^{1/2} }
\]
y
\[
	\dpar{W}{q} = \sqrt{ \omega J - \frac{ m \omega^2 q^2 }{2} \left( \right) } = p.
\]

\subsubsection{Quedó remanente}

\begin{figure}
	\begin{center}
	\includegraphics[width=0.7\textwidth]{images/fig_mc_hamjac.pdf}	 
	\end{center}
	\caption{}
\end{figure} 


La transformación $S$ es 
\[
	\dpar{S}{q_i} = p_i = \dpar{W}{q_i} \qquad \dpar{S}{J_i} = \theta_i = \dpar{W}{J_i}
\]
siendo $p_i = p_i(q_1,J_1,...,J_n)$.
El nuevo hamiltoniano es $E=E(J_1,...,J_n)$
\[
	\dpar{E}{J_i} = \dot{\theta}_i \equiv \omega \qquad \dpar{E}{\theta_i} = -\dot{J}_i
\]
de manera que tenemos
\[
	\theta_i = \omega t + \theta_{0_i} \qquad  \dpar{W}{J_i} = \theta_i = \theta_i(q_i, J_i)
\]
y entonces despejamos las $q_i$ desde
\[
	\theta_i(q_i, J_i) = \omega t + \theta_{0_i}.
\]

Las condiciones iniciales $(q_i, J_i)$ se introducen en
\[
	\dpar{W}{q_i} = p_i(q_1,J_1,...,J_n)
\]
y obtengo las $J_1, ..., J_n$ constantes.

% =================================================================================================
\section{Transformación canónica infinitesimal}
% =================================================================================================

Difieren de la identidad en un infinitésimo
\[
	F_2 = 	F_2(q_i,P_i) = \sum_i^N q_iP_i
\]
es la identidad
\[
	\dpar{F_2}{q_i} =  p_i \equiv P_i \qquad \dpar{F_2}{P_i} =  Q_i \equiv q_i
\]
y donde considero
\[
	F_2(q_i,P_i) = \sum q_i P_i + \epsilon G(q_1,...,q_n,P_1,...,P_n) \qquad \textrm{con} \; \epsilon \ll 1
\]
\[
	\dpar{F_2}{q_i} = p_i = P_i + \epsilon\dpar{G}{q_i} \longrightarrow P_i = p_i - \epsilon\dpar{G}{q_i} 
\]
\[
	\dpar{F_2}{P_i} = Q_i = q_i + \epsilon\dpar{G}{P_i} \longrightarrow q_i = Q_i - \epsilon\dpar{G}{P_i} 	
\]
donde $\partial G/\partial P_i \approx \partial G/\partial p_i$ diferirán en un orden $\epsilon^2$ el cual
descarto. Entonces
\[
	\delta p_\ell = -\epsilon \dpar{G}{q_\ell} \qquad \delta q_\ell = \epsilon \dpar{G}{P_\ell}.
\]

Si considero $H$ en lugar de $G$ y $\epsilon = \delta t$ entonces
\[
	\frac{\delta p_\ell}{\delta t} = -\dpar{H}{q_\ell} \qquad \frac{\delta q_\ell}{\delta t} = \dpar{H}{p_\ell}
\]
de tal manera que 
\[
	\dot{p}_\ell = -\dpar{H}{q_\ell} \qquad \dot{q}_\ell = -\dpar{H}{p_\ell}
\]
y donde se ve que el $H$ genera la transformación evolución temporal.
Por otra parte, se puede ver cómo varía una cierta cantidad $A$ ante la transformación canónica.
\[
	\delta A = A(q_i + \delta q_i, p_i + \delta p_i) - A(q_i,p_i)
\]
y
\[
	\delta A = \sum_i \left( \dpar{A}{q_i} \delta q_i + \dpar{A}{p_i} \delta p_i \right)
\]
\[
	\delta A = \epsilon \sum_i \left( \dpar{A}{q_i} \dpar{H}{p_i} - \dpar{A}{p_i} \dpar{H}{q_i} \right) =
	\epsilon[A,H] \longrightarrow \frac{\delta A}{\delta t} = [A,H]
\]
entonces las constantes de movimiento generan transformaciones canónicas infinitesimales que dejan invariante
al hamiltoniano $H$. Si
\[
	\dtot{A}{t} = 0 \Longrightarrow [A,H] = 0
\]

Consideremos una rotación infinitesimal. Una rotación en torno al eje $z$
\[
	\begin{cases}
		x_i' = x_i - \delta \alpha y_i \\
		y_i' = y_i + \delta \alpha x_i \\
		z_i' = z_i
	\end{cases}
\]
que implica 
\[
	\delta x_i = -  \delta \alpha y_i \qquad \qquad \delta y_i = \delta \alpha x_i \qquad \qquad 
	\delta z_i = 0
\]
\notamargen{Las constantes de movimiento están generadas por simetrías (Noether).}

Luego,
\[
	G = \sum_i ( x_i p_{y_i} - y_i p_{x_i} ) = \ell_z
\]
y una rotación en torno a $\hat{n}$ es
\[
	\delta\alpha [ A, \vb{L}\cdot\vb{n} ] = \delta A,
\]
si $A$ es un vector $\vb{V}$ entonces
\[
	\delta\alpha [ \vb{V}, \vb{L}\cdot\vb{n} ] = \delta \vb{V} = \delta\alpha \vb{n} \times \vb{V},
\]
de modo que 
\[
	[ \vb{V}, \vb{L}\cdot\vb{n} ] = \vb{n} \times \vb{V}
\]
es una relación vectorial; es decir que valen
\[
	[ V_x, \vb{L}\cdot\vb{n} ] = (\vb{n} \times \vb{V})_x
\]
y lo mismo para los componentes $y,z$. Además
\[
	[L_x, L_z] = ( \hat{z} \times \vb{L} )_z = - L_y
\]
\[
	[L_y, L_z] = ( \hat{z} \times \vb{L} )_y = L_x
\]
\[
	[L_x, L_y] = ( \hat{y} \times \vb{L} )_x = L_z
\]
o bien 
\[
	[L_i, L_j] = \varepsilon_{ijk} L_k
\]
donde 
\[
	\varepsilon_{ijk} = \begin{cases}
	 0 \quad \text{ si se repite índice } \\
	 1 \quad \text{ si es permutación cíclica } \\
	 -1 \quad \text{ si es permutación anticíclica }
	\end{cases}
\]

Esto nos dice que no podemos elegir como momentos estas constantes de movimiento puesto que el
corchete de Poisson entre ellas es nulo.
No va a existir transformación canónica donde $p_1 = L_x, p_2 = L_y, p_3 = Lz$ pues su corchete de
Poisson entre ellas no se anula.

% =================================================================================================
\section{Volviendo a Hamilton-Jacobi}
% =================================================================================================

\be
	H\left( \dpar{S}{q_1}, ..., \dpar{S}{q_n}, q_1, ..., q_n, t \right) - \dpar{S}{t} = 0
	\label{ham-jac}
\ee
y se pasaba de coordenadas $q,p$ a constantes $\alpha,\beta$.

Lo único que se puede asegurar son condiciones para hallar solución a \eqref{ham-jac} pero no resolverla.
La condición es que \eqref{ham-jac} sea separable, que existan tantas constantes de movmiento como grados
de libertad. Si el hamiltoniano tiene alguna coordenada cíclica o no depende del tiempo entonces se podría
separar, pero en general no es el caso.

Podría suceder que
\[
	W = \sum_{i=1}^N W_i(q_i,\alpha_1,...,\alpha_N)
\]
y entonces no podré llegar a una solución que se compone de problemas independientes en una variable.

En fuerzas centrales tenemos un ejemplo. Escribamos
\[
	H = \frac{p_r^2}{2m} + \frac{p_\vp^2}{2mr^2\sin^2\theta} + \frac{p_\theta^2}{2mr^2} + V(r) = E
\]
y
\[
	W = W_r(r,\alpha_1,\alpha_2,\alpha_3) + W_\vp(\vp,\alpha_1,\alpha_2,\alpha_3) + 
		W_\theta(\theta,\alpha_1,\alpha_2,\alpha_3)
\]
donde
\[
	\frac{1}{2m} \Dtot{W_r}{r}^2 + \frac{1}{2mr^2} \Dtot{W_\theta}{\theta}^2 + \frac{1}{2mr^2\sin^2\theta} \Dtot{W_\vp}{\vp}^2
	+ V(r)  = E \equiv \alpha_1
\]
y siendo que $E$ es una constante, la denominamos $\alpha_1$.

Luego, como en 
\[
	\left[\frac{1}{2m} \Dtot{W_r}{r}^2 + \frac{1}{2mr^2} \Dtot{W_\theta}{\theta}^2 + V(r) - \alpha_1 \right] 
	2 m r^2 \sin^2\theta = - \Dtot{W_\vp}{\vp}^2
\]
el miembro izquierdo sólo depende de $r,\theta$ y el derecho de $\vp$ tienen que ser una constante ambos
miembros.

Asimismo,
\[
	\dtot{W_\vp}{\vp} = \alpha_2,
\]
y $W_\vp = \alpha_2 \vp$ porque $\vp$ es cíclica en fuerzas centrales. Es más, $\alpha_s$ es el momento angular
en $z$ ($L_z$, que es constante).
Entonces
\[
	\frac{1}{2m} \Dtot{W_r}{r}^2 + \frac{1}{2mr^2} \:
	\left[ \frac{\alpha_2^2}{\sin^2\theta} + \Dtot{W_\theta}{\theta}^2 \right]  + V(r) = \alpha_1,
\]
donde el corchete será $\alpha_3^2$. Ahora puedo separar otra vez y surge $\alpha_3^2$ que será el $|\vb{L}|^2$ total.
Las $\alpha_i$ son constantes de separación.
Si hubiese escrito con $\theta = \pi / 2$ entonces resultaba más detectable.
\[
	W(\theta) = \int \sqrt{ \left( \alpha_3^2 - \frac{\alpha_2^2}{\sin^2\theta} \right) } \; d\theta
\]
y luego
\[
	\frac{1}{2m} \Dtot{W_r}{r}^2 + \frac{1}{2mr^2} \: \alpha_3^2 + V(r) = \alpha_1
\]
al solucionar lo anterior
\[
	W_r = \int \sqrt{ 2 m ( \alpha_1 - V(r) - \alpha_2^2 / (2mr^2) ) } \; dr
\]
y el nuevo hamiltoniano es $ K = \alpha_1 $. Entonces
\[
	\dpar{K}{\alpha_2} = 0 \qquad \text{ que lleva a } \qquad \beta_2 = cte.
\]
\[
	\dpar{K}{\alpha_3} = 0 \qquad \text{ que lleva a } \qquad \beta_3 = cte.
\]
\[
	\dpar{K}{\alpha_1} = 1 \qquad \text{ que lleva a } \qquad \beta_2 = t + \beta_{10}
\]
Como
\[
	\dpar{W}{\alpha_1} = \dpar{W_r}{\alpha_1}
\]
se tiene 
\[
	\beta_1 = \int \frac{m}{\sqrt{ 2 m ( \alpha_1 - V(r) - \alpha_2^2 / (2mr^2) ) }} dr,
\]
que es igual a la ecuación ya calculada en el caso de fuerzas centrales.

La forma de separar también funciona si
\[
	V = V(r) + \frac{a(\vp)}{r^2 \sin ^2 \theta} + \frac{b(\theta)}{r^2},
\]
es decir, si el $V$ tiene una forma como la de arriba en coordenadas esféricas.

Consideremos unas coordenadas $\xi,\eta,\vp$ que se relacionan por
\[
	\rho = \xi \eta \qquad z = \frac{1}{2}( \xi - \eta ) \qquad \vp = \vp
\]
donde $\rho,z,\vp$ son las polares cilíndricas usuales.
Un potencial de la forma 
\[
	\frac{ a(\xi) + b( \eta ) }{ \xi + \eta }
\]
en coordenadas parabólicas puede separarse.

En coordenadas elípticas
\[
	\rho = \sigma \sqrt{ ( \xi^2 - 1 )( 1 - \eta^2 ) } \qquad z = \sigma \xi \eta \qquad \vp = \vp
\]
un potencial de la forma 
\[
	V = \frac{ a(\xi) + b( \eta ) }{ \xi^2 - \eta^2 }
\]
puede separarse.

En cartesianas es
\[
	V(\vb{x}) = A(x) + B(y) + C(z)
\]
condición suficiente de verificación para Hamilton-Jacobi. Si el potencial es separable entonces tengo tantas coordenadas como 
constantes de movimiento, entonces si tengo la solución [?].

\subsection{Comentario Schrödinger}

La ecuación de Schrödinger es
\[
	- \frac{\hbar^2}{2m} \nabla^2 \Psi(\vb{x},t) + V(r) \Psi(\vb{x},t) = i \hbar  \dpar{\Psi}{t} (\vb{x},t).
\]
Si ensayamos como solución 
\[
	\Psi(\vb{x},t) = b(\vb{x},t)  \euler^{i A(\vb{x},t) / \hbar},
\]
se tendrá 
\[
	b V(r) + \frac{b}{2m} \left[ \Dpar{A}{x}^2 + \Dpar{A}{y}^2 + \Dpar{A}{z}^2 \right] = - b \dpar{A}{t}
\]
o bien 
\[
	\frac{1}{2m} \left[ \Dpar{A}{x}^2 + \Dpar{A}{y}^2 + \Dpar{A}{z}^2 \right] + V(r) + \dpar{A}{t} = 0
\]

Ecuación de Hamilton-Jacobi (con igualar el orden cero)
\[
	\frac{1}{m} \Dpar{b}{x} \dpar{A}{x} + \frac{b}{2m} \dpar[2]{A}{x} = - \dpar{B}{t}
\]
\[
	\dpar{}{x}\left( b^2 \frac{1}{m} \dpar{S}{x} \right) + \dpar{b^2}{t} = 0
\]

Esto lleva a 
\[
	\nabla \left( \rho \frac{\vb{p}}{m} \right) + \dpar{\rho}{t} = 0,
\]
que es una ecuación de continuidad para la densidad de probabilidad.
De algún lado sacamos, por la situación estacionaria,
\[
	- \dpar{b}{x} \dpar{A}{x} = \frac{1}{2} \Dpar[2]{A}{x} b,
\]
que equivale a 
\[
	\frac{1}{b} \dpar{b}{x} = - \frac 1 2 \Dpar[2]{A}{x} \Dpar{A}{x}^{-1}
\]
o bien a 
\[
	\dtot{\log b}{x} = \dtot{}{x} \log\left( \Dpar{A}{x}^{-1/2} \right)
\]
de modo que $b = 1 / \sqrt{p}$ siendo el $p$ clásico.


\subsection{Hamilton-Jacobi particular}

Esto apareció en la práctica. Consideramos
\[
	S = \int \Lag dt,
\]
donde $S = S(q_0,t)$. Pictóricamente

\includegraphics[scale=0.5]{images/fig_mc_ham-jac_1.jpg}

El diferencial de la integral resulta, como hemos visto en incontables ocasiones para Euler-Lagrange, en
\[
	\delta S = \left. \dpar{ \Lag }{ q_i } \delta q_i \right|_{t_1}^{t_2} + 
	\int_{t_1}^{t_2} \left[ \dpar{ \Lag }{ q_i } - \dtot{}{t}\Dpar{\Lag}{\dot{q}_i} \right] \delta q_i \: dt. 
\]

Luego, $ \delta S = p_i \delta q_i  $ y entonces $ p_i = dS / dq_i $. Como $\Lag = dS / dt $
\[
	\dtot{S}{t} = \dpar{S}{t} + \dpar{S}{q_i} \dot{q}_i \qquad \Lag = \dpar{S}{t} + p_i \dot{q}_i
\]
A partir de esta última, el diferencial $d\Lag$ es
\[
	d\Lag = \dot{p}_i dq_i + p_i d\dot{q}_i + \dpar{\Lag}{t}
\]
\[
	d\Lag = \dot{p}_i dq_i + p_i d\dot{q}_i + d( p_i \dot{q}_i ) + \dpar{\Lag}{t},
\]
de manera que 
\[
	d( \Lag - p_i \dot{q}_i ) = p_i dq_i - q_i dp_i
\]
lo que nos lleva a las ecuaciones de Hamilton
\[
	\dot{p}_i = - \dpar{H}{q_i} \qquad \dot{q}_i = \dpar{H}{p_i} \qquad \dpar{H}{t} = - \dpar{\Lag}{t} 
\]

Entonces
\[
	S = \int ( p_i \dot{q}_i - H ) \: dt,
\]
y pidiendo $\delta S=0$ se tiene 
\[
	\delta S = \int_{t_1}^{t_2}  \: \left[ \dtot{}{t}( p_i \delta q_i ) - \dot{p}_i \delta q_i + \dot{q}_i \delta p_i 
	- \dtot{H}{p_i} \delta q_i - \dpar{H}{p_i} \delta p_i \right] \: dt
\]
o bien 
\[
	\left. p_i \delta q_i \right|_{t_1}^{t_2} - \int dt 
	\left[ \left( \dot{p}_i + \dpar{H}{q_i} \right) \delta q_i + ( \dtot{H}{p_i} - \dot{q}_i ) \delta p_i \right]
\]

Entonces las ecuaciones de Hamilton las podemos obtener con 
\[
	\delta \int ( p_i \dot{q}_i - H ) dt = 0 \qquad \delta \int ( P_i \dot{Q}_i - H' ) dt = 0
\]
y
\[
	dF_1 = p_i \dot{q}_i dt - P_i \dot{Q}_i dt + (H'-H)dt
\]
\[
	dF( qq_i, Q_i, t ) = P_i dq_i - P_i dQ_i + (H'- H)dt
\]
y de la {\it lectura} de 
\[
	dF_1 =
\]
se identifican 
\[
	\dtot{F_1}{q_i} = p_i \qquad \qquad \dtot{F_1}{P_i} = -P_i.
\]

De modo ídem se tiene 
\[
	d(F_1 + P_iQ_i) = P_i dq_i + Q_i dP_i + (H'-H)dt
\]
($F_2(q_i,P_i,t)$) lo que lleva a 
\[
	\dtot{F_2}{q_i} = p_i \qquad \qquad \dtot{F_2}{P_i} = Q_i \qquad \qquad \dtot{F_2}{t} = H'- H.
\]

Entonces
\[
	\dpar{S}{t} + H(q_i,\dpar{S}{q_i},t) = 0
\]
pero la derivada parcial en el argumento es $p_i$. La transformación $(q_i,p_i) \to (Q_i,P_i) = (\beta_i,\alpha_i)$ 
permite la escritura
\[
	\dpar{S}{t} + H(q_i,\alpha_i,t) = 0,
\]
y $S = S'(\alpha_i) + A$ donde los $\alpha_i$ son $n$ variables (dado que $S$ aparece sólo derivada).
\[
	\dpar{S}{\alpha_i} = \beta_i \qquad H'= H + \dpar{S}{t} = 0
\]
y entonces $ \dot{\alpha}_i = 0, \dot{\beta}_i=0 $. Ahora como los momentos y las coordenadas son constantes el problema
es trivial pero la transformación es muy jodida.
\[
	\dpar{S}{t} + H(q_i,\dpar{S}{q_i}) = 0 \qquad \qquad S = S(q_i,t)
\]
en este caso $ \dtot{S}{t} = -E $ y se tiene 
\[
	S = W\left( q_i,\dpar{S}{q_i} \right) - E t.
\]



% =================================================================================================
\section{Potencial electromagnético}
% =================================================================================================

Arranquemos por los momentos canónicamente conjugados
\[
	\dpar{\Lag}{\dot{q}_i} = p_i \quad \textrm{pero} \; si V \neq V(q) \longrightarrow \dpar{T}{\dot{q}_i} = p_i
\]
entonces
\[
	U(q,\dot{q}) =  e \phi - e/c \vb{A}\cdot\vb{V} \longrightarrow \Lag = T - e \phi + e/c  \vb{A}\cdot\vb{V}
\]
\[
	p_x = \dpar{T}{\dot{x}} - \dpar{U}{\dot{x}} = m V_x - (e/c) A_x.
\]

Hacemos un cambio de gauge, en un potencial generalizado
\[
	U =  e \Phi(\vb{x},t) - (q/c) \vb{A}(\vb{x},t)\cdot\vb{V}(t)
\]
y el cambio de gauge es
\[
	\vb{A}' = \vb{A} + \nabla f,
\]
que no altera las ecuaciones de movimiento.


\begin{ejemplo}{\bf Problema de parcial}

El problema cuya geometría se ilustra a continuación. Se consideran $m_1 = m_2 = m$, $m_3 = M$ y  $k$ {\it slinkies}.

\includegraphics[scale=0.5]{images/fig_mc_problema_parcial_osc_0.jpg}
 
\includegraphics[scale=0.5]{images/fig_mc_problema_parcial_osc_1.jpg}

\includegraphics[scale=0.5]{images/fig_mc_problema_parcial_osc_2.jpg}

Es claramente un problema de seis grados de libertad, $\theta_M, \theta_1, \theta_2, Z_M, Z_1, Z_2$ 

Podemos escribir la energía cinética y el potencial como 
\[
	T = \frac{1}{2} M a^2 \dot{\theta}_M^2 + \frac{1}{2} M a^2 (\dot{\theta}_1^2 + \dot{\theta}_2^2 ) +
	\frac{1}{2} M \dot{Z}_M^2 + \frac{1}{2} m ( \dot{Z}_1^2 + \dot{Z}_2^2 )
\]
\[
	V = \frac{1}{2} \left[ a^2 ( \theta_2 - \theta_M ) + ( Z_2 - Z_M )^2 \right] +
	\frac{1}{2} \left[  a^2 ( \theta_1 - \theta_2 ) + ( Z_1 - Z_2 )^2 \right] +
	\frac{1}{2} \left[ a^2 ( \theta_1 - \theta_M ) + ( Z_1- Z_M )^2 \right] 
\]
\[
	\Lag = T - V
\]
Como ya está en forma cuadrática no es necesario aproximar. Si tenemos expresiones lineales habría que desarrollar
a orden dos, por ejemplo $ \cos \theta \sim 1 - \theta^2/2 $ y me quedo con los términos cuadráticos.

Definimos coordenadas referidas al equilibrio. Nos paramos en el equilibrio y oscilamos en torno a él.
\[
 	\eta_1 = ( \theta_M - \theta_{\mbox{eq}} ) 
 	\qquad 
 	\eta_2 = ( \theta_M - \theta_{ 1 \mbox{eq} } )a 
 	\qquad 
 	\eta_3 = ( \theta_2 - \theta_{ 2 \mbox{eq} } )a
\]
\[
	\eta_4 = Z_M - Z_{M\mbox{eq}} \qquad \eta_5 = Z_1 - Z_{1\mbox{eq}} \qquad \eta_6 = Z_2 - Z_{2\mbox{eq}}
\]
con sus correspondientes velocidades
\[
	\dot{\eta}_1 = a \dot{\theta}_M \qquad \dot{\eta}_2 = a \dot{\theta}_1 \qquad \dot{\eta}_3 = a \dot{\theta}_2
\]
\[
	\dot{\eta}_4 = \dot{Z}_M \qquad \dot{\eta}_5 = \dot{Z}_1 \qquad \dot{\eta}_6 = \dot{Z}_2
\]

\includegraphics[scale=0.5]{images/fig_mc_problema_parcial_osc_3.jpg}

El lagrangiano será 
\begin{multline*}
	\Lag  =  \frac{1}{2} M \dot{\eta}_1^2 + \frac{1}{2}\dot{\eta}_4^2 + \frac{1}{2} m (\dot{\eta}_2^2 + \dot{\eta}_3^2 + \dot{\eta}_5^2 + \dot{\eta}_6^2 ) \\
	- \frac{k}{2}\left[ (\eta_1 - \eta_2 )^2 + (\eta_5 - \eta_4 )^2 + (\eta_1 - \eta_3 )^2 + (\eta_6 - \eta_4 )^2 + (\eta_2 - \eta_3 )^2 + (\eta_5 - \eta_6 )^2 \right]
\end{multline*}


Habria que identificar los coeficientes para armar $T,V$ en 
\[
	\Lag = \frac{1}{2} \dot{\bar{eta}}^\dagger \mathbb{T} \dot{\bar{eta}} 
	- \frac{1}{2} \bar{eta}^\dagger \mathbb{V} \bar{eta}
\]

Como ejemplo, desarrollemos algún término
\[
	k(\eta_1 - \eta_2)^2 = k \eta_1 \eta_2 - 2k \eta_1 \eta_2 + k \eta_2 \eta_2 =
	k \eta_1 \eta_2 - k \eta_1 \eta_2 -  k \eta_1 \eta_2 + k \eta_2 \eta_2
\]

Las matrices resultan 
\[
	V = \begin{pmatrix}
	     2 k & -k & -k & 0 & 0 & 0 \\
	     -k & 2k & -k & 0 & 0 & 0 \\
	     -k & -k & 2k & 0 & 0 & 0 \\
	     0 & 0 & 0 & 2k & -k & -k \\
	     0 & 0 & 0 & -k & 2k & -k \\
	     0 & 0 & 0 & -k & -k & 2k
	    \end{pmatrix}
\]
\[
	T = \begin{pmatrix}
	     M & 0 & 0 & 0 & 0 & 0 \\
	     0 & m & 0 & 0 & 0 & 0 \\
	     0 & 0 & m & 0 & 0 & 0 \\
	     0 & 0 & 0 & M & 0 & 0 \\
	     0 & 0 & 0 & 0 & m & 0 \\
	     0 & 0 & 0 & 0 & 0 & m 
	    \end{pmatrix}
\]
y como se ve ambas resultan en bloques de Jordan y son cada bloque igual. El a incluido en la coordenada hace que se
obtenga esa forma simétrica.
En
\[
	\mbox{det}( \mathbb{V} - \omega^2 \mathbb{T} )
\]
se reduce a calcular el determinante de la submatriz de 3 $\times$ 3,
\[
	\begin{bmatrix}
	2 k - \omega^2 M & -k & -k \\
	-k & 2 k - \omega^2 m & -k \\
	-k & -k & 2 k - \omega^2 m 
	\end{bmatrix}
\]
que resulta en la ecuación 
\[
	\omega^2 \left( \omega^4 - \omega^2 \left[  \frac{2k}{M} + \frac{2k}{M} \right] +
	\left[ \frac{6k^2}{mM} + \frac{3k^2}{m^2} \right] \right) = 0,
\]
que da 
\[
	\omega_1^2 = 0 \qquad \qquad \omega_2^2 = \frac{2k}{M} + \frac{k}{m} \qquad \qquad \omega_3^2 = \frac{3k}{m} 
\]

Para $\omega^2 = 0$ la ecuación $ (V - \omega^2 T )A^1 = 0 $ se verifica para 
\[
	A^1 = \alpha \begin{pmatrix} 1 \\ 1 \\ 1 \end{pmatrix}
\]
cuya normalización se ajusta con $ {A^\dagger}^1 T A^1 = 1$ o bien 
\[
	\alpha^2 \begin{pmatrix} 1 & 1 & 1 \end{pmatrix}
	\begin{pmatrix}
	 M & 0 & 0 \\
	 0 & m & 0 \\
	 0 & 0 & m
	\end{pmatrix}
	\begin{pmatrix} 1 \\ 1 \\ 1 \end{pmatrix} = 1,
\]
siendo el valor de $\alpha$ dado por 
\[
	\alpha^2 = \frac{1}{2m+M} \qquad \alpha = \frac{1}{ \sqrt{2m+M} }.
\]

Los autovectores son 
\[
	A^1 = \frac{1}{\sqrt{M + 2m}} \begin{pmatrix} 1 \\ 1 \\ 1 \end{pmatrix} \qquad 
	A^2 = \frac{1}{\sqrt{4m^2/M + 2m}} \begin{pmatrix} -2m/M \\ 1 \\ 1 \end{pmatrix} \qquad 
	A^3 = \frac{1}{\sqrt{ 2m }} \begin{pmatrix} 0 \\ 1 \\ -1 \end{pmatrix} \qquad 
\]
 
Los movimientos están dados por

\includegraphics[scale=0.5]{images/fig_mc_problema_parcial_osc_4.jpg}
 
Ahora hay que completar hasta la sexta dimensión
\[
	\bar{\eta}_1 = \frac{1}{\sqrt{2m+M}} \begin{pmatrix} 1 \\ 1 \\ 1 \\ 0 \\ 0 \\ 0
	                                     \end{pmatrix} \qquad \qquad
	\bar{\eta}_3 = \frac{1}{\sqrt{2m}} \begin{pmatrix} 0 \\ 1 \\ -1 \\ 0 \\ 0 \\ 0
	                                     \end{pmatrix} \qquad \qquad
	\bar{\eta}_5 = \frac{1}{\sqrt{2m + 4m^2/M}} \begin{pmatrix} 0 \\ 0 \\ 0 \\ -2m/M0 \\ 1 \\ 1
	                                     \end{pmatrix}
\]
\[
	\bar{\eta}_2 = \frac{1}{\sqrt{2m+ 4m^2/M}} \begin{pmatrix} -2m/M \\ 1 \\ 1 \\ 0 \\ 0 \\ 0 
						\end{pmatrix} \qquad \qquad
	\bar{\eta}_4 = \frac{1}{\sqrt{M + 2m}} \begin{pmatrix} 0 \\ 0 \\ 0 \\ 1 \\ 1 \\ 1
	                                     \end{pmatrix} \qquad \qquad
	\bar{\eta}_6 = \frac{1}{\sqrt{2m}} \begin{pmatrix} 0 \\ 0 \\ 0 \\ 0 \\ 1 \\ -1
	                                     \end{pmatrix}
\] 
 
Luego, para desacoplar la solución habría que plantear la matriz
\[
	\mathbb{B} = [ \eta^\dagger_1 ... \eta_i^\dagger ]
\]
 
\end{ejemplo}

\begin{ejemplo}{\bf Problema 12}

El setup se ilustra en la figura siguiente.

\includegraphics[scale=0.5]{images/fig_mc_problema_12.jpg}

El torque 
\[
	\tau = - k \theta
\]
lo suponemos un potencial $ V = 1/2 k \theta^2 $, donde $k$ tiene unidades de energía.
Las barras solo rotan de manera que 
\[
	T_1 = \frac{1}{2} ( m_1 \ell_1^2 \dot{\theta}_1^2  ) + \frac{1}{2}( m_1 \ell_1^2 \dot{\theta}^2_1 )
\]
donde estamos pensando como dos partículas. En cambio, pensándolo como una barra con momento de inercia es
\[
	T = \frac{1}{2} I \Omega^2
\]

Entonces,
\[
	T = T_1 + T_2 = \frac{1}{2} ( 2 m_1 \ell_1^2 \dot{\theta}_1^2  ) + \frac{1}{2}( 2 m_2 \ell_2^2 \dot{\theta}^2_2 )
\]
\[
	V_1 = \frac{1}{2} k_1 \theta^2_1 \qquad 
	V_{12} = \frac{1}{2} k_2 ( \theta_2 - \theta_1 )^2  \qquad 
	V_2 = \frac{1}{2} k_1 \theta^2_2 
\]

Definiendo $\eta_i = \theta_i - \theta_{\mbox{eq}}$ que implican $\dot{\eta}_i = \dot{\theta}_i $ $(i=1,2)$ se puede escribir el lagrangiano como 
\[
	\Lag = \ell_1^2 m_1 \dot{\eta}_1^2 + \ell_2^2 m_2 \dot{\eta}_2^2 - \frac{1}{2} k_1 \eta_1^2 - \frac{1}{2} k_2 \eta_2^2 - \frac{1}{2} k_2 (\eta_2 - \eta_1)^2
\]
de manera que 
\[
	\mathbb{T} = \begin{pmatrix}
	 2 \ell^2_1 m_1 & 0 \\
	 0 & 2 \ell^2_2 m_2 
	\end{pmatrix}
	\qquad 
	\mathbb{V} = \begin{pmatrix}
	 k_1 + k_2 & - k_2 \\
	 -k_2  & k_1 + k_ 2 
	\end{pmatrix}
\]

Faltaría entonces
\[
	\mbox{det} \{ \mathbb{V} - \omega^2 \mathbb{T} \} = 0
\]
y los autovectores $A^1, A^2$.


\end{ejemplo}

\begin{ejemplo}{\bf Problema 8}

Un problema de pequeñas oscilaciones.

\includegraphics[scale=0.5]{images/fig_mc_problema_8.jpg} 

En este ejemplo hay que suponer que el lagrangiano es ya de entrada de pequeñas oscilaciones.

\end{ejemplo}

\begin{ejemplo}{\bf Problema 1 P96}

El lagrangiano para el {\it setup} es 
\[
	\Lag = \frac 1 2 m ( \dot{x}^2 + \dot{y}^2 + \dot{z}^2 ) - \frac{k_x}{2} x^2 - \frac{k_y}{2} y^2 - \frac{k_z}{2} z^2,
\]
donde los momentos son 
\[
	p_i = \dpar{\Lag}{\dot{x}_i} = m\dot{x}_i,
\]
para cada una de las coordenadas. El hamiltoniano es $h = \sum_i p_i \dot{q}_i - \Lag$, que explícitamente
\[
	h = \frac{p_x^2}{2m} + \frac{p_y^2}{2m} + \frac{p_z^2}{2m} + \frac{k_x}{2} x^2 + \frac{k_y}{2} y^2 + \frac{k_z}{2} z^2
\]
de donde leemos
\[
	\dpar{H}{x} = k_x x = - \dot{p}_x \qquad \dpar{H}{p_x} = \frac{p_x}{m} = \dot{x},
\]
que conduce, derivando una vez más, a $ \ddot{x} = \dot{p}_x / m $ y $ \ddot{x} = - k_x / m $, que es la ecuación del oscilador
armónico en $x$.
 
El diagrama de fases es algo como lo que muestra la figura siguiente
 
\includegraphics[scale=0.35]{images/fig_mc_problema_1_p96.jpg} 

donde bajo el primer gráfico aparecen las curvas de nivel del potencial.

Para fuerza central resulta $U(r)$ en esféricas y el lagrangiano es
\[
	\Lag =
\]
Calculando el hamiltoniano según la definición resulta, después de algo de álgebra, en
\[
	\Ham = \frac{p_r^2}{2m} + \frac{1}{2mr^2} \left( p_\theta^2 + \frac{p_\vp^2}{\sin^2\theta}\right) + U(r) .
\]
El hamiltoniano es constantes puesto que el lagrangiano no depende del tiempo y $p_\vp$, por la ciclicidad de $\vp$
es constante.

Si espedificamos como potencial el de Kepler, vemos que es separable en $\theta, r$ puesto que se tiene 
\[
	f(\theta,\vp) = g(r)
\]
de modo que cada una de estas funciones es una constante.

La conservación del momento angular en términos del hamiltoniano resulta en
\[
	\dtot{}{t} \left( p_\theta^2 + \frac{p_\vp^2}{\sin^2\theta}\right) = 2 p_\theta \dot{p}_\theta +
	\frac{ 2 p_\vp^2 \dot{p}_\vp^2}{\sin^2\theta} - \frac{2\cos\theta\dot{\theta}p_\vp^2}{\sin^3\theta}
\]
y entonces
\[
	\dot{p}_\theta = -\dpar{\Ham}{\theta} \left( \frac{-1}{2mr^2} \frac{-4p_\vp^2 \sin^2\theta \cos\theta}{\sin^2\theta} \right),
\]
y eso lleva a 
\[
	 \left( p_\theta^2 + \frac{p_\vp^2}{\sin^2\theta}\right) = cte
\]

EL hamiltoniano tiene un potencial efectivo dado por los dos últimos términos de la derecha,
\[
	\Ham = \frac{p_r^2}{2m} + \frac{1}{2mr^2} L^2 + U(r)
\]

El diagrama de fases aparece aquí abajo.

\includegraphics[scale=0.35]{images/fig_mc_problema_1_p96_2.jpg} 
 
\end{ejemplo}


\begin{ejemplo}{\bf Problema 8 P97}

Consideramos un potencial generalizado
\[
	U = q \vp - \frac q c \vb{v}\cdot\vb{A},
\]
donde $\vb{A} = 1/2 \pv{B}{x}$ y asimismo $\pv{\nabla}{A}=\vb{B}$ siendo el lagrangiano,
\[
	\Lag = \frac 1 2 m ( \dot{x}^2 + \dot{y}^2 + \dot{z}^2 ) + \frac{q}{c} \vb{v}\cdot\vb{A}
\]

Si el campo está en el eje $z$, es decir $\vb{B}=(0,0,B)$ se tiene $\vb{A} = 1 /2 ( -B y \hat{x} + B x \hat{y} )$, lo
cual conduce a 
\[
	\Lag = \frac 1 2 m ( \dot{x}^2 + \dot{y}^2 + \dot{z}^2 ) + \frac{q}{c} \frac{B}{2} ( - \dot{x} y + \dot{y} x )
\]
y los momentos son
\[
	P_x = m \dot{x} - \frac{qB}{2c}y \qquad
	P_y = m \dot{y} + \frac{qB}{2c}x \qquad 
	P_z = m \dot{z}  
\]
de manera que $z$ es cíclica. Luego, calculando el hamiltoniano a través de la definición es 
\[
	\Ham = \frac{1}{2} m ( \dot{x}^2 + \dot{y}^2 + \dot{z}^2 ),
\]
o bien (usando las equivalencias anteriores)
\[
	\Ham = \frac{1}{2m} \left[ p_x^2 + p_y^2 + p_z^2 + \frac{qB}{c}\left( p_x y - p_y x \right) +
	\Frac{qB}{2c}^2( x^2 + y^2 ) \right], 
\]
y el paréntesis dentro del corchete es el $L_z$.
Paso a usar un hamiltoniano $\Ham' = \Ham + cte.$
\[
	\dot{x} = \dpar{\Ham'}{p_x} = \frac{p_x}{m} \qquad \dot{p_x} = -\dpar{\Ham'}{x} = \frac{1}{2m}\Frac{qB}{2c}^2 2x
\]

Obtenemos para la ecuación de Newton,
\[
	\ddot{x} - \frac{q^2 B^2}{4 c^2 m^2} x = 0
\]
mientras que para la coordenada $y$ obtenemos una ecuación similar. Entonces
\[
	\ddot{x} + \omega^2 x = 0 \qquad \ddot{y} + \omega^2 y = 0
\]
con $\omega = q B / ( 2 c m )$. Luego
\[
	x = \frac{1}{\sqrt{ }}( \sqrt{ 2p } + ) \qquad y = \frac{1}{\sqrt{ }}( \sqrt{ 2p } + )
\]
y sus correspondienes momentos,
\[
	p_x = \frac{\sqrt{ }}{2} () \qquad p_y = \frac{\sqrt{ }}{2} ()
\]
 
Deberíamos probar que es una transformación canónica chequeando que se verifican
\[
	[ x, p_x ], [ y, p_y ] = cte. \qquad [ x, y ] = [ p_x, p_y ] = 0
\]
\[
	[ x, p_x ] = sum_i \dpar{x}{q_i} \dpar{p_x}{p_i} - \dpar{p_x}{q_i} \dpar{x}{p_i}
\]
y las derivadas
\[
	\dpar{x}{q_1} = \frac{1}{\sqrt{m\omega}} ( \sqrt{ 2p_1} \cos q_1 ) \qquad \dpar{x}{q_2} = 0
\]
\[
	\dpar{x}{p_1} = \frac{1}{\sqrt{2 m p_1}} ( \sin q_1 ) \qquad \dpar{x}{p_2} = \frac{1}{\sqrt{m\omega}}
\]
\[
	\dpar{p_x}{q_1} = -\frac{\sqrt{2 m p_1}}{2} ( \sin q_1 ) \qquad \dpar{p_x}{q_2} = -\frac{\sqrt{m \omega}}{2} \qquad \dpar{p_x}{p_2} = 0
\] 
\[
	\dpar{p_x}{p_1} = \frac{\sqrt{ }}{\sqrt{2p_1}}
\]
\[
	[ x, p_x ] = \frac{1}{ \sqrt{ m \omega} } \sqrt{ 2 p_1 } \cos^2 q_1 + \left( \frac{\sqrt{2 m p_1}}{2} \sin^2 q_1 \right) + \frac{1}{2}
\]

El hamiltoniano luce
\[
	\Ham = \frac{1}{2m} \left( p_x^2 + p_y^2 + \frac{m^2\omega^2}{4}[ x^2 + y^2 ] \right)
\]
\[
	\Ham = \frac{\omega}{2} p_1 + \frac{\omega}{4} p_2^2 + \frac{\omega}{4} q_2^2,
\]
y se ve que $q_1$ resultó cíclica. Luego $p_1$ es una constante y 
\[
	\dot{p}_2 = \dpar{\Ham}{q_2} = \frac{\omega}{2} q_2.
\]
Resolviendo se llega a 
\[
	\ddot{q}_2 + \omega^2 q_2 = 0,
\]
de manera que $q_2$ tiene comportamiento oscilatorio. Pero
\[
	\dot{q}_1 = -\dpar{\Ham}{p_1} = \frac{\omega}{2} \qquad q_1 = - \frac{\omega}{2} t.
\]
\end{ejemplo}

\begin{ejemplo}{\bf Hamilton Jacobi para fuerza central}

\[
	\Ham = \frac{1}{2m} p_r^2 + \frac{1}{2mr^2}p_\vp^2 + V(r), \qquad S = S' - Et
\]
entonces se puede escribir
\[
	E  - \left[ \frac{1}{2m}\Dpar{S}{r}^2 + V(r) \right] = \frac{1}{2m}\Dpar{S}{\vp}^2.
\]

Para esféricas necesitaré:
\[
	V = V(r) + \frac{b(\theta)}{r^2} + \frac{c(\vp)}{r^2 \sin^2\theta}
\]

Si reemplazo en el hamiltoniano a la forma separable
\[
	W(r, \theta, \vp) = W'(r, \theta) + S_\vp(\vp),
\]
de modo que 
\[
	\Dpar{S}{\vp}^2 + C(\vp) = \alpha \vp
\]
que ya se integra directamente. Luego en forma ídem es
\[
	W'(r, \theta) = S_r( r ) + S_\theta(\theta).
\]
\end{ejemplo}

\begin{ejemplo}{\bf Otro ejemplo de potencial}

\[
	V = \frac{\alpha}{r} - F(z) \qquad \qquad ( r \text{ esféricos}, z \text{ cilíndricas })
\]
donde $(\xi, \eta, \vp)$ coordenadas parabólicas. Verifican
\[
	\frac{ a(\xi) + b(\eta) }{\xi + \eta}
\]
\[
	z = \frac 1 2 ( \xi - \eta ), \quad \rho =  \sqrt{\xi \eta}, \quad \vp \in \{ 0, 2\pi \}, 0 < \xi, \eta < \infty 
\] 
\[
	r = \sqrt{ \rho^2 + z^2 } = \sqrt{ \frac{1}{4} ( \xi - \eta )^2 + \xi \eta },
\]
de modo que $r = 1/2 (\xi + \eta)$. Consideremos
\[
	\Lag  = \frac{1}{2m} ( \dot{\rho}^2 + \rho^2 \dot{\vp}^2 + \dot{z}^2 ) - V(\rho,\vp,z)
\]
que lleva a 
\[
	\dot{\rho}^2 + \dot{z}^2 = \frac{1}{4} ( \xi + \eta )\left( \frac{\dot{\xi}^2}{\xi} + \frac{\dot{\eta}^2}{\eta} \right)
\]

El lagrangiano es
\[
	\Lag = \frac{1}{8} m (\xi + \eta) \left( \frac{\dot{\xi}^2}{\xi} + \frac{\dot{\eta}^2}{\eta} \right) +
	\frac{m}{2}\xi\eta\dot{\vp}^2 - V(\xi,\eta,\vp)
\]
y los momentos 
\[
	p_\xi = \dpar{\Lag}{\dot{\xi}} = \frac{m}{4} (\xi + \eta) \frac{\dot{\xi}}{\xi} \qquad 
	p_\eta = \frac{m}{4} (\xi + \eta) \frac{\dot{\eta}}{\eta}, \qquad
	p_\vp = m \xi \eta \dot{ \vp }
\]
mientras que el hamiltoniano es
\[
	\Ham = \frac{2}{m} \Frac{ \xi p_\xi^2 + \eta p_\eta^2 }{ \xi + \eta } + \frac{ p_\vp^2 }{ 2 m \xi \eta } + 
	\frac{ a(\xi) + b(\eta) }{ \xi + \eta }
\]

Ahora vemos el potencial $V(r,z)$ que es 
\[
	V = \frac{1}{r} ( \alpha - Fr ) = \frac{1}{r} ( \alpha - \frac{1}{4} F ( \xi^2 - \eta^2 ) )
\]
\[
	V = \frac{( \alpha - F \xi^2 / 2 ) + ( \alpha + F \eta^2 / 2 )}{\xi + \eta}
\]
 
Reemplazando en el hamiltoniano se tiene 
\[
	S = W(\xi,\eta,\vp) - Et = W(\xi,\eta) + p_\vp \vp - Et
\]
donde
\[
	E(\xi + \eta) = 2 \xi \Dpar{W}{\xi}^2 + 2 \eta \Dpar{W}{\eta}^2 + \frac{(\xi + \eta)p_\vp}{2 m \xi \eta} + a(\xi) + b(\eta)
\]
o bien 
\[
	E\xi - 2 \xi \Dpar{W}{\xi}^2 + E\xi - \frac{p_\vp}{2 m \xi} - a(\xi) =  2 \eta \Dpar{W}{\eta}^2 + E\xi - \frac{p_\vp}{2 m \eta} + b(\eta) - E\eta  
\]
y como $ W(\xi,\eta) = W_\xi(\xi) + W_eta(\eta) $ se tiene 
\[
	\beta =  2 \xi \Dpar{W_\xi}{\xi}^2 + \frac{p_\vp}{2 m \xi} + a(\xi) -E\xi
\]
\[
	W(\xi) = \int \; \sqrt{ \frac{1}{2\xi} \left( \beta + E\xi - \frac{p_\vp}{2 m \xi} - a(\xi) \right)} d\xi,
\]
o bien
\[
	S = W_\xi(\xi) + W_\eta(\eta) + p_\vp \vp - Et
\]

\end{ejemplo}


\begin{ejemplo}{Problema interesante (reacomodar o desaparecer)}

Este sistema tiene los mismos modos normales que si no estuvieran estos nuevos resortes conectados

\includegraphics[scale=0.35]{images/fig_mc_problema_interesante.jpg}

La matriz con resortes entre vecinos es
\[
	\begin{pmatrix}
	2 k_1 & - k_1 & 0 & 0 & ... & - k_1 \\ 
	-k_1 & 2k_1 & -k_1 & 0 & ... & - k_1 \\
	0 & - k_1 & 2k_1 & ... &  \\
	& 0 & 0 & ...& & -k_1 \\
	& & &... & 2k_1 & 0 \\
	-k_1 &  &  & -k_1 & 0 & 2k_1
	\end{pmatrix}
\]
mientras que la matriz con resortes entre terceros
\[
	\begin{pmatrix}
	2 (k_1 +k_2) & - k_1 & -k_2 & 0 & ... & - k_1 \\ 
	-k_1 & 2(k_1+k_2) & -k_1 & 0 & ... & - k_1 \\
	k_2 & - k_1 & 2(k_1+k_2) & ... &  \\
	& 0 & 0 & ...& 2k_1 &  \\
	-k_2& & &... &  & -k_1 \\
	-k_1 & -k_2 &  &  & -k_1 & 2k_1
	\end{pmatrix}
\]

Siempre tiene los mismos autovectores.
Los modos normales están dados por la geometría y las condiciones de contorno.

\end{ejemplo}

\subsection{Popurrí de cosas para reubicar}

{\bf Sobre el potencial}
Si $\vb{F}$ proviene de un potencial que depende del tiempo entonces no será conservativo el campo de fuerzas.
No vale lo mismo la integral $\int F(x,t) dx$ por diferentes caminos (dependerá del tiempo que tarde en cada
camino).

{\bf sobre generatrices}
\[
	F_2 = F_1 + \sum  Q_i P_i,
\]
con
\[
	\begin{cases}
	Q_i = Q_i( q_i, P_i ) \\
	P_i = P_i( q_i )
	\end{cases}
\]
pero tienen que estar escritas $Q_i, P_i$ en términos de las variables de $F_2$

{\bf Ángulo-acción}

Todas los grados de libertad separables (de lo contrario no puedo definir una de las acciones).

{\bf Observación}

Se puede transformar entre $n$ grados de libertad dependiendo del tiempo a $n+1$ grados de 
libertad conservativos.

{\bf Bundle de Hamilton-Jacobi}

Resolviendo Hamilton-Jacobi llego a
\[
	K = H(q_i,) - E = 0
\]
y si $t$ es separable paso al problema $ H(q_i,) = E $.
\[
	\dpar{W}{q_i} = p_i \qquad \qquad \dpar{W}{P_i} = Q_i
\]
con 
\[
	\theta_i = \omega_i(J_1,...,J_n) t + \theta_{i0}
\]
\[
	\dpar{E}{J_i}(J_1,...,J_n) = \dot{\theta}_i = \omega_i (J_1,...,J_n)
\]

El problema general tiene como solución $J_i$ constante $\theta$ lineal.
El asunto es pasar a la solución que sirve 
\[
	\begin{cases}
	q_1(t), ..., q_n(t) \\
	p_1(t), ..., p_n(t)
	\end{cases}
\]
usando las condiciones iniciales y la generatriz $S$.

La derivada
\[
	\dpar{W}{q_i} = p_i(q_1,... , q_n,J_1,...,J_n)
\]
hay que expresarla en función de los nuevos momentos $J_i$ (no de las constantes de separación $\alpha_i$) pero
en realidad el problema es totalmente separable, entonces
\[
	\dpar{W}{q_i} = p_i(q_i,J_1,..,J_n)
\]
para ángulo-acción los $P_i$ no son las constantes de separación.
\[
	W = W(q_i,P_i) \qquad W = W(q_i,J_i)
\]

Energía en función de los momentos
\[
	W = \int \sqrt{ 2 m ( E[J] - V[q] ) } dq, 
\]
entonces $W = W( J, q )$. Si tenemos varios grados de libertad tendremos varias integrales
\[
	\dpar{W}{J_i} = \theta_i(q_i,J_n) = \omega_i t + \theta_{i0}
\]

Con las condiciones iniciales me meto en:
\[
	\dpar{W}{q_i}(q_1,J_1,...,J_n) = p_i(q_1,... , q_n,J_1,...,J_n)
\]
con los $p_{i0}$ y los $q_{i0}$ obtengo los $J_1, ..., J_n$ constantes $n$ ecuaciones $p_{i0} = p_i (q_1,J_1,...,J_n)$
\[
	\dot{\theta}_i = \omega_i (J_1, ..., J_n )
\]
con los $J_1, ..., J_n$ obtengo $\dot{\theta}_i$.

% \bibliographystyle{CBFT-apa-good}	% (uses file "apa-good.bst")
% \bibliography{CBFT.Referencias} % La base de datos bibliográfica

\end{document}
 
