	\documentclass[10pt,oneside]{CBFT_book}
	% Algunos paquetes
	\usepackage{amssymb}
	\usepackage{amsmath}
	\usepackage{graphicx}
	\usepackage{libertine}
% 	\usepackage[bold-style=TeX]{unicode-math}
	\usepackage{lipsum}

	\usepackage{natbib}
	\setcitestyle{square}

	\usepackage{polyglossia}
	\setdefaultlanguage{spanish}


	\usepackage{CBFT.estilo} % Cargo la hoja de estilo
	
	% Tipografías
	% \setromanfont[Mapping=tex-text]{Linux Libertine O}
	% \setsansfont[Mapping=tex-text]{DejaVu Sans}
	% \setmonofont[Mapping=tex-text]{DejaVu Sans Mono}

	%===================================================================
	%	DOCUMENTO PROPIAMENTE DICHO
	%===================================================================
	
	
\begin{document}

\appendix

\chapter{Rotación en el plano}\label{App.rotacion_plana}

Rotación de un sistema de coordenadas en el plano (pasar de cuaderno).

\notamargen{
Para los apéndices faltaría una estructura más consistente. Tal vez subapéndices con aquellos que resulten demasiado 
pequeños como para justificar una sección con letras grandes. La idea sería agrupar temas comunes y poner en 
apéndice todo aquello de uso suficientemente general. Deberíamos evitar también que un tomo dependa de un apéndice en 
otro tomo. En ese caso tal vez convenga duplicar.
}
 
 
\end{document}
