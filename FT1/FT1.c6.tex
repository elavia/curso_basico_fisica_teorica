	\documentclass[10pt,oneside]{CBFT_book}
	% Algunos paquetes
	\usepackage{amssymb}
	\usepackage{amsmath}
	\usepackage{graphicx}
% 	\usepackage{libertine}
% 	\usepackage[bold-style=TeX]{unicode-math}
	\usepackage{lipsum}

	\usepackage{natbib}
	\setcitestyle{square}

	\usepackage{polyglossia}
	\setdefaultlanguage{spanish}


	\usepackage{CBFT.estilo} % Cargo la hoja de estilo

	% Tipografías
	% \setromanfont[Mapping=tex-text]{Linux Libertine O}
	% \setsansfont[Mapping=tex-text]{DejaVu Sans}
	% \setmonofont[Mapping=tex-text]{DejaVu Sans Mono}

	%===================================================================
	%	DOCUMENTO PROPIAMENTE DICHO
	%===================================================================

\begin{document}

% =================================================================================================
\chapter{Ondas planas}
% =================================================================================================

Lejos de las fuentes de campo las ecuaciones de Maxwell son
\begin{align*}
	\divem{E} = 0 \qquad \qquad &\rotorm{E} = -\frac{1}{c}\dpar{\vb{B}}{t} \\
	\divem{B} = 0 \qquad \qquad &\rotorm{B} = \frac{1}{c}\dpar{\vb{E}}{t}
\end{align*}

Podemos derivar con respecto al tiempo en cada ecuación de rotor y reemplazar con la otra
de manera que
\[
	\rotorm{(\rotorm{B})} = \frac{1}{c}\dpar{}{t}\left(-\frac{1}{c}\dpar{\vb{B}}{t} \right)
	= \Nabla(\divem{B}) - \nabla^2 \vb{B}
\]
\[
	\rotorm{(\rotorm{E})} = -\frac{1}{c}\dpar{}{t}\left(\frac{1}{c}\dpar{\vb{E}}{t} \right)
	= \Nabla(\divem{E}) - \nabla^2 \vb{E}
\]
y esto nos lleva a 
\[
	\nabla^2 \vb{B} - \frac{1}{c^2}\dpar[2]{\vb{B}}{t} = 0 \qquad 
	\nabla^2 \vb{E} - \frac{1}{c^2}\dpar[2]{\vb{E}}{t} = 0
\]
dos sendas ecuaciones de onda para \vb{E} y \vb{B}. Pero es sabido que la solución de
\[
	\nabla^2 \psi - \frac{1}{c^2} \dpar[2]{\psi}{t} = 0
\]
es
\[
	\psi = A\euler^{i(\pe{k}{x}-\omega t)} + B\euler^{i(\pe{k}{x}-\omega t)}
\]
de modo que podemos postular como soluciones para nuestras ecuaciones de onda a
\[
	\vb{E} = \vec{\mathbb{E}}_0 \euler^{i(\pe{k}{x}-\omega t)} \qquad 
	\vb{B} = \vec{\mathbb{B}}_0 \euler^{i(\pe{k}{x}-\omega t)}
\]

Se tiene además que $\vb{k} = k \hat{n}$ da a través de $\hat{n}$ la dirección de
propagación de la onda. El número de onda $k$ podrá ser complejo lo cual refleja
atenuación. Las características del medio entran a través de
\[
	k = \sqrt{\mu\epsilon} \frac{\omega}{c}
\]
\notamargen{Acá hay que hacer las cuentas para demostrar todo esto que acá se
dice sin más. Además hay que comentar sobre la introducción de números complejos en
las amplitudes. Cómo son estos vectores complejos.}
Por su parte $\vec{\mathbb{E}}_0$ y $\vec{\mathbb{B}}_0$ son las amplitudes de los campos,
que serán números complejos constantes (no varían con la posición o el tiempo) y podrán
dar desfasajes.

Al utilizar las ecuaciones de divergencia sobre las soluciones se obtiene que 
\[
	\hat{n}\cdot \vec{\mathbb{E}}_0 = 0 \qquad \hat{n}\cdot \vec{\mathbb{B}}_0 = 0
\]
de manera que las ondas se propagan perpendicularmente a los campos, por ello las
ondas electromagnéticas son transversales.

Utilizando las ecuaciones de rotor se llega a la importante relación 
\[
	\vec{\mathbb{B}}_0 = \sqrt{\mu\epsilon} \hat{n} \times \vec{\mathbb{E}}_0
\]
de modo que los vectores $\vec{\mathbb{E}}_0$ y $\vec{\mathbb{B}}_0$ también son perpendiculares.
Si el vector $\vb{k} \in \mathbb{R}$ entonces $\vec{\mathbb{E}}_0$ y $\vec{\mathbb{B}}_0$
tienen la misma fase.

En el vacío o en un medio LIH los campos \vb{E} y \vb{B} estarán en fase.
Asimismo
\[
	\vb{S} \parallel \hat{n}
\]
pues $\vb{S} \propto \pv{E}{H} $.

En un medio anisótropo $\divem{D}=\divem{}(\epsilon \vb{E}) = 0$ siendo $\epsilon$ un tensor.
Allí $\vec{\mathbb{E}}_0 \cdot \hat{n} \neq 0$ salvo que $\epsilon$ estee diagonalizado y
$\vb{E} \parallel$ al eje principal.

Notemos que $\vb{E},\vb{B}$ y $\hat{n}$ forman una terna derecha.

\subsection{Sobre complejos}
\[
	\mathcal{R}(A) = \frac{1}{2}( A + A^* )	\qquad \text{con} \; A \in \mathbb{C}
\]
Sean 
\[
	\vb{A}(\vb{x},t) = \vb{A}(\vb{x})\euler^{-i\omega t} \qquad \qquad 
			\vb{B}(\vb{x},t) = \vb{B}(\vb{x})\euler^{-i\omega t}
\]
siempre trabajaremos en general con dependencias temporales armónicas y metemos $\euler^{i\pe{k}{x}}$ en el 
módulo $vb{A}_0$ que pasa a depender de $\vb{x}$.

Los campos físicos son siempre la parte real de las expresiones complejas.
\[
	\mathcal{R}(\vb{A} + \vb{B}) = \mathcal{R}( \vb{A} ) + \mathcal{R}( \vb{B} )
\]
con operaciones lineales es lo mismo tomar parte real antes o después.
\[
	\mathcal{R}(\vb{A}.\vb{B}) \neq  \mathfrak{R}( \vb{A} ) + \mathcal{R}( \vb{B} )
\]
con operaciones no lineales no es lo mismo.
Para hacer producto necesito tomar la parte real de cada factor y entonces
\[
	\mathfrak{R}( \vb{A} ).\mathfrak{R}( \vb{B} ) = \frac{1}{2}\mathfrak{R}( \vb{A}.\vb{B}^* + 
			\vb{A}.\vb{B} \euler^{ -i 2 \omega t })
\]

Pero como en las aplicaciones estaré interesado en el promedio sobre un número entero de períodos,
\[
	\langle \vb{A} \vb{B} \rangle = \langle \mathfrak{R}( \vb{A} ). \mathfrak{R}( \vb{B} ) \rangle =
			\frac{1}{2} \mathfrak{R} ( \vb{A}.\vb{B}^* )
\]

\subsection{Poynting promedio y energías promedio}

Los campos \vb{E} y \vb{H} en ondas electromagnéticas toman la forma 
\[
	\vb{E} = \vec{\mathbb{E}}(\vb{x})\euler^{-i\omega t} \qquad 
	\vb{H} = \vec{\mathbb{H}}(\vb{x})\euler^{-i\omega t}
\]
de manera que 
\[
	\vb{S}( \vb{x},t ) = \frac{c}{4\pi} \frac{1}{2} \re( \vec{\mathbb{E}} \times \vec{\mathbb{H}}^* + 
		\vec{\mathbb{E}} \times \vec{\mathbb{H}} \euler^{-i2\omega t})
\]
\[
	\langle \vb{S}( \vb{x},t ) \rangle = \frac{c}{8\pi} \re( \vec{\mathbb{E}} \times \vec{\mathbb{H}}^* ) 
\]

En un MLIH es 
\[
	\vec{\mathbb{B}} = \sqrt{ \mu \epsilon } \hat{n} \times \vec{\mathbb{E}} \qquad\qquad 
	\vec{\mathbb{H}} = \sqrt{ \frac{\epsilon}{\mu } } \hat{n} \times \vec{\mathbb{E}}
\]
donde usamos que $\vb{H} = \vb{B}/\mu$
\[
	\langle \vb{S}( \vb{x},t ) \rangle = \frac{c}{8\pi} \re( \vec{\mathbb{E}} \times 
		\sqrt{\frac{\epsilon}{\mu}}(\hat{n} \times \vec{\mathbb{E}})^* )
\]
\[
	\langle \vb{S}( \vb{x},t ) \rangle = \frac{c}{8\pi} \sqrt{\frac{\epsilon}{\mu}} 
		( \hat{n} (\vec{\mathbb{E}}\cdot\vec{\mathbb{E}}^*) - 
		\vec{\mathbb{E}}^*(\vec{\mathbb{E}}\cdot\hat{n}) )
\]
y finalmente
\[
	\langle \vb{S}( \vb{x},t ) \rangle = \frac{c}{8\pi} 
		\sqrt{\frac{\epsilon}{\mu}}|\vec{\mathbb{E}}|^2 \hat{n}
\]
que es el vector de Poynting para ondas en MLIH.
\[
	U(\vb{x},t) = \frac{1}{8\pi}( \pe{H}{B} + \pe{E}{D} )
\]
\[
	\langle U(\vb{x},t) \rangle = \frac{1}{8\pi} \frac{1}{2} \re ( 
	\vec{\mathbb{H}}\cdot\vec{\mathbb{B}}^* + \vec{\mathbb{E}}\cdot\vec{\mathbb{D}}^* )
\]
\[
	\langle U(\vb{x},t) \rangle = \frac{1}{16\pi}
		\re ( \frac{1}{\mu} |\vec{\mathbb{B}}|^2 + \epsilon |\vec{\mathbb{E}}|^2 ) =
		\frac{1}{8\pi} |\vec{\mathbb{E}}|^2
\]
puesto que 
\[
	|\vec{\mathbb{B}}|^2 = \mu\epsilon |\vec{\mathbb{E}}|^2,	
\]
y entonces la densidad de energía promedio es
\[
		\langle U(\vb{x},t) \rangle = \frac{1}{8\pi} |\vec{\mathbb{E}}|^2.
\]

% =================================================================================================
\section{Polarización de ondas}
% =================================================================================================

Una onda plana bien general en $\hat{n}$ es 
\[
	\vb{E}( \vb{x},t )=(\hat{\epsilon}_1 \vec{\mathbb{E}}_1 + 
			\hat{\epsilon}_2\vec{\mathbb{E}}_2) \euler^{i( \pe{k}{x} -\omega t)}
\]

\begin{figure}[htb]
	\begin{center}
	\includegraphics[width=0.4\textwidth]{images/fig_ft1_polariz.pdf}	 
	\end{center}
	\caption{}
\end{figure} 

Si $\vec{\mathbb{E}}_1,\vec{\mathbb{E}}_2$ están en fase entonces $\vb{E}( \vb{x},t )$ está linealmente
polaridaza con $\theta$ fijo.
Es como que \vb{E} viaja siempre por el mismo andarivel, oscilando. Las amplitudes 
$\vec{\mathbb{E}}_1,\vec{\mathbb{E}}_2$ son complejos para permitir la diferencia de fase entre componentes.

Si $\vec{\mathbb{E}}_1,\vec{\mathbb{E}}_2$ tienen fase arbitraria entonces $\vb{E}( \vb{x},t )$ está elípticamente 
polarizada.

Si $|\vec{\mathbb{E}}_1|=|\vec{\mathbb{E}}_2|$ y la fase es $\pi/2$ entonces $\vb{E}( \vb{x},t )$ está circularmente 
polarizada.

\[
	\vec{\mathbb{E}}_2 = \vec{\mathbb{E}}_1 \euler^{i \pi/2} =\vec{\mathbb{E}}_1 i
\]
entonces
\[
	\vb{E}( \vb{x},t )= \vec{\mathbb{E}}_1 (\hat{\epsilon}_1  \pm \hat{\epsilon}_2 )
				\euler^{i( \pe{k}{x} -\omega t)}	
\]
donde el $+$ corresponde a $\mathcal{C}^+$ antihoraria y el $-$ a horaria. Nos definimos por comodidad,
\[
	\hat{\epsilon}_+ \equiv \frac{\hat{\epsilon}_1 + i \hat{\epsilon}_2 }{\sqrt{2}} \qquad\qquad 
	\hat{\epsilon}_- = \frac{\hat{\epsilon}_1 - i \hat{\epsilon}_2 }{\sqrt{2}}
\]
una base de polarizaciones. Se cumplen
\[
	\hat{\epsilon}_\pm \cdot \hat{\epsilon}_\mp^* = 0 \qquad \qquad 
	\hat{\epsilon}_\pm \cdot \hat{\epsilon}_\pm^* = 1
\]
\[
	\hat{\epsilon}_1 = \sqrt{2}( \hat{\epsilon}_+ + i \hat{\epsilon}_- ) \qquad \qquad 
	\hat{\epsilon}_2 = \sqrt{2}( \hat{\epsilon}_+ - i \hat{\epsilon}_- )
\]
luego cualquier polarización se puede escribir como combinación lineal de $\mathcal{C}^+$ y $\mathcal{C}^-$.
Entonces una onda plana general es
\[
	\vb{E}( \vb{x},t )=(\hat{\epsilon}_+ \vec{\mathbb{E}}_+ + 
			\hat{\epsilon}_-\vec{\mathbb{E}}_-) \euler^{i( \pe{k}{x} -\omega t)}
\]

Una onda que rebota en un espejo transfiere impulso lineal. Una onda $\mathcal{C}$ lleva \vb{L} pero no lo transfiere 
en un rebote perfecto. Por ser \vb{L} un vectorial axial (pseudovector) el reflejo es equivalente a una simetría del 
sistema.

Tenemos dos base entonces $\{ \hat{\epsilon}_1 ,\hat{\epsilon}_2 \}$ y $\{ \hat{\epsilon}_+ ,\hat{\epsilon}_- \}$.
Además,
\[
	\frac{\vec{\mathbb{E}}_-}{\vec{\mathbb{E}}_+} = r\euler^{i\alpha} 
\]
si $r = \pm 1, \alpha=0 $ entonces estamos frente a linealmente polarizada.

% =================================================================================================
\section{Reflexión y refracción de ondas en medios}
% =================================================================================================

Partimos de una onda
\[
	\vb{E}( \vb{x},t )= \vec{\mathbb{E}}_0 \euler^{i( \pe{k}{x} -\omega t)}
\]
donde 
\[
	k = \sqrt{\mu\epsilon} \frac{\omega}{c} = \frac{\omega}{v}
\]
siendo $v$ la velocidad en el medio. Los índices de refracción serán 
\[
	n = \sqrt{\mu \epsilon} \qquad  n' = \sqrt{\mu' \epsilon'}
\]
de tal suerte que los campos son 
\[
	\vb{B} = \frac{\sqrt{\mu\epsilon}}{k} \; \pv{k}{E} \qquad \quad 
		\vb{H} = \sqrt{\frac{\epsilon}{\mu}} \frac{1}{k} \; \pv{k}{E}
\]
\begin{figure}[htb]
	\begin{center}
	\includegraphics[width=0.7\textwidth]{images/fig_ft1_reflex1.pdf}	 
	\end{center}
	\caption{}
\end{figure} 

y tenemos 

\[
	|\vb{k}| = |\vb{k}''| \qquad \text{pues} \quad \mu''= \mu,\; \epsilon''=\epsilon
\]

Utilizando las condiciones de contorno llegamos a
\[
	\omega t = \omega' t = \omega'' t
\]
\[
	\pe{k}{x}\:|_{z=0} = \pe{k'}{x}\:|_{z=0}  = \pe{k''}{x}\:|_{z=0} 
\]
La existencia de condiciones de contorno en $z=0$ que deben ser satisfechas en todo $t$ en todo punto $(x,y)$ lleva a 
todos los factores de fase iguales en $z=0$. Se debe tener \vb{B} normal continuo y \vb{D} normal continuo también, lo 
cual viene de $\divem{B}=0$ y $\divem{D}=0$.

La frecuencia $\omega$ es la misma para el medio 1 y el medio 2 pues $\lambda_1 \neq \lambda_2$.

Los tres vectores $\vb{k}, \vb{k}', \vb{k}''$ están en un mismo plano, entonces 
\[
	k \sin(i) = k' \sin(r) = k'' \sin(i'),
\]
y se deducen las consecuencias
\[
	n \sin (i) = n' \sin(i') \qquad \text{Ley de Snell},
\]
\[
	i = i' \qquad \text{Ley de reflexión}
\]
Luego se plantean los contornos 
\[
	D_{\hat{n}} : \qquad [ \vb{D}_2 - \vb{D}_1 ]\cdot\hat{n} = 0 \qquad \rightarrow  \qquad 
		[ \epsilon'\vb{E}_0^{'} - \epsilon (\vb{E}_0 + \vb{E}_0^{''} )  ]\cdot\hat{n} = 0
\]
\[
	E_{\hat{t}} : \qquad \hat{n} \times [ \vb{E}_2 - \vb{E}_1 ] = 0 \qquad \rightarrow \qquad 
		\hat{n} \times [ \vb{E}_0^{'} - (\vb{E}_0 + \vb{E}_0^{''} )  ]  = 0
\]
\[
	B_{\hat{n}} : \qquad  [ \vb{k}' \times \vb{E}_0^{'} - ( \vb{k} \times \vb{E}_0 + 
			\vb{k}'' \times \vb{E}_0^{''} )  ]\cdot\hat{n} 
\]
\notamargen{Igual a cero esto?}
\[
	H_{\hat{t}} : \qquad  \hat{n} \times \left[ \frac{1}{\mu'}\vb{k}' \times \vb{E}_0^{'} - 
			\frac{1}{\mu}( \vb{k} \times \vb{E}_0 + \vb{k}'' \times \vb{E}_0^{''} ) \right]  = 0
\]
de manera que 
\[
	\vb{B} = \frac{\sqrt{\mu\epsilon}}{k} \pv{k}{E} = \frac{c}{\omega} \pv{k}{E} \qquad \qquad 
	\vb{H} = \frac{c}{\mu \omega } \pv{k}{E}
\]

donde $c/\omega$ es el mismo para ambos medios.

Aplicando diligentemente los contornos se llega a las {\it relaciones de Fresnel} que son los cocientes de las 
amplitudes relativas.

Usando $\mu \sim 1$ (válido para medios transparentes) tenemos

\[
	TE \qquad \qquad TM
\]
\[
	\frac{E_0^{''}}{E_0} = -\frac{\sin(i-r)}{\sin(i+r)} \qquad \qquad 
	\frac{E_0^{''}}{E_0} = \frac{\tan(i-r)}{\tan(i+r)} 
\]
\[
	\frac{E_0^{''}}{E_0} = 1 + \frac{\sin(r-i)}{\sin(i+r)} \qquad \qquad 
	\frac{E_0^{''}}{E_0} = \frac{2 \sin(r)\cos(i)}{\sin(i+r)\cos(i-r)} 
\]
\begin{figure}[htb]
	\begin{center}
	\includegraphics[width=0.7\textwidth]{images/fig_ft1_reflex2.pdf}	 
	\end{center}
	\caption{}
\end{figure} 

\notamargen{frecuencias ópticas $\mu'/\mu = 1$}.

Si $i \sim 0$ entonces TE y TM son similares a menos de un signo.

\subsubsection{Polarization (Brewster angle)}

Es un $i_B$ tal que no hay onda \vb{E} reflejada (en TM),
\[
	E_0^{''} = 0,
\]
puest $\tan( i + r) \to \infty$
\[
	i_b = atan \left( \frac{n'}{n}\right) ,
\]
pues $ i_B + r = \pi/2$ entonces 
\[
	\frac{n}{n'}\sin(i_B) = cos(i_B) \rightarrow i_b = atan \left( \frac{n'}{n}\right),
\]
Sirve para producir luz polarizada linealmente.

\begin{figure}[htb]
	\begin{center}
	\includegraphics[width=0.4\textwidth]{images/fig_ft1_reflex3.pdf}	 
	\end{center}
	\caption{}
\end{figure} 

Atención, pero 
\[
	\vb{S}_i \neq \vb{S}_r + \vb{S}_t ,
\]
pues \vb{S} no está relacionado linealmente con \vb{E}, \vb{B}, y lo que sí vale es
\[
	\vb{S}_i \cdot \hat{n} = \vb{S}_r \cdot \hat{n} + \vb{S}_t \cdot \hat{n} 
\]


\subsubsection{Reflexión interna total}

Sea $ n_{inc} > n_{trans} $. Entonces se da que
\[
	n \sin(i) = n'\sin(r),
\]
\[
	\frac{n}{n'} \sin(i) = \sin(r),
\]
y el LHS es mayor igual a 1 para algunos $i$. Existe un ángulo límite 
\[
	\sin(r) = 1 = \frac{n}{n'} \sin(i) 
\]
\[
	i_0 = asin\left( \frac{n'}{n} \right)
\]
de manera que si $i \geq i_0$ entonces $\sin(r) > 1$ y se debe tener un $r\in \mathbb{C}$.

\begin{figure}[htb]
	\begin{center}
	\includegraphics[width=0.4\textwidth]{images/fig_ft1_reflex4.pdf}	 
	\end{center}
	\caption{}
\end{figure} 

Si $\sin(r)>1$ se tiene $\sin(r)^2 > 1$ y como por teorema de Pitágoras es 
\[
	\cos(r)^2 = 1 - \sin(r)^2 \rightarrow \cos(r) = i \sqrt{ \sin(r)^2 - 1 }
\]
donde notemos espcialmente que hemos sacado fuera un $\sqrt{-1} = i$ para que el argumento de la raíz sea positivo en 
este caso especial. Luego 
\[
	\cos(r) = i \sqrt{ \frac{n}{n'}\sin(i)^2 - 1 } = i a 
\]
y si $\sin(r) = 1$ entonces $r = \pi/2$.
Entonces
\[
	\euler^{i(\vb{k}\cdot\vb{x})} = \euler^{i(k \cos(r)z + k \sin(r)x)} =
		\underbrace{\euler^{-kaz}}_{\text{atenuación}} 
		\underbrace{\euler^{ik\sin(r)x}}_{\text{propagación}}
\]


% =================================================================================================
\section{Corrientes en conductores}
% =================================================================================================

La continuidad de la carga y la divergencia de \vb{D},
\[
	\divem{J} + \dpar{\rho}{t} = 0 \qquad \qquad \divem{D} = 4\pi \rho,
\]
nos llevan a
\[
	\divem{J} + \frac{1}{4\pi}\Nabla\cdot \dpar{\vb{D}}{t} = 0
\]
\[
	\Nabla\cdot \left( \vb{J} + \frac{1}{4\pi} \dpar{\vb{D}}{t} \right) = 0
\]
y esto lo puedo pensar como una densidad de corriente estacionaria,
\be
	\divem{J}_e = 0
	\label{diver_jcond}
\ee
siendo $\vb{J}_e$ proveniente de un \vb{E'} tal que $\rotorm{E'} \neq 0$.
\notamargen{Un campo irrotacional no puede mantener una corriente estacionaria, necesito
una FEM para ella. La FEM es una fuente de \vb{E} no conservativo.}

Recordando la ley de Ohm microscópica, $\vb{J} = \sigma \vb{E}$ ,
\[
	\vb{D} = \epsilon \vb{E} = \frac{\epsilon}{\sigma} \vb{J}
\]
y esto nos conduce a una ecuación diferencial para \vb{J},
\[
	\vb{J}_e = \vb{J} + \frac{\epsilon}{4\pi\sigma} \dpar{\vb{J}}{t} =
	\left( 1 + \frac{\epsilon}{4\pi\sigma} \dpar{}{t} \right)
\]
y entonces 
\[
	\vb{J} = \vb{J}_e + \vb{J}_0 \euler^{-4\pi\sigma/\epsilon t }
\]
siendo el segundo término del RHS la parte no estacionaria de la corriente. Evidentemente, si $t \to \infty$ esta 
tiende a cero.

Dado que se verifica \eqref{diver_jcond} se tiene 
\[
	-\dpar{\rho}{t} = \divem{J}_0 \euler^{-4\pi \sigma/\epsilon t}
\]
y definimos un tiempo de relajación
\[
	\tau = \frac{\epsilon}{4\pi\sigma}
\]
que es un tiempo característico en el cual se alcanzarían condiciones estacionarias.

Podemos distinguir dos comportamientos entonces en términos de este tiempo de relajación $\tau$, si $t<\tau$
\[
	\vb{J} = \vb{J}_e + \vb{J}_0 \euler^{-t/\tau}
\]
y en cambio cuando $t \gg \tau$ se tendrá $\vb{J} \approx \vb{J}_e$ de manera que 
\[
	\divem{J} = \divem{J}_e.
\]

Por otra parte con respecto a los conductores, si se da que  ($\sigma \ll 1$) estamos en presencia de un conductor malo 
y no se alcanza {\it nunca} la condición de $\vb{E}=0$ en el interior. Tienen un $\tau$ grande. Si estamos ante un 
conductor perfecto ($\sigma \to \infty$) la corriente es estacionaria y se tiene un $\vb{E}=0$ en el interior, el 
tiempo $\tau$ es pequeño, tendiendo a cero.

Podemos desarrollar un enfoque similar en términos de la densidad de carga $\rho$.

\[
	\divem{J} = - \dpar{\rho}{t}  \qquad \qquad \vb{J} = \sigma \vb{E} = \frac{\sigma}{\epsilon}\vb{D}
\]
\[
	\dpar{\rho}{t} + \frac{4\pi\sigma}{\epsilon} \rho = 0 \qquad \qquad 
			\divem{J} = \frac{\sigma}{\epsilon} \divem{D} = \frac{4\pi\sigma}{\epsilon} \rho
\]
Entonces 
\[
	\rho = \rho_0 \euler^{-t / \tau } \qquad \qquad \tau \equiv \frac{\epsilon}{4\pi\sigma} ,
\]
y una vez que $t \gg \tau$ y se estabiliza el sistema es $\rho=\rho_0$ entonces 
\[
	\divem{J} = 0 \qquad\qquad \dpar{\rho}{t} = 0
\]


% =================================================================================================
\section{Campo electromagnético en un medio conductor}
% =================================================================================================

Tenemos un campo EM de fuentes lejanas y queremos ver qué sucede en un medio conductor.
Se verifican
\ben
\begin{aligned}
	\divem{B}=0 &\qquad 
	\rotorm{\vb{H}}= \frac{4\pi}{c}\vb{J} + \frac{1}{c}\dpar{\vb{D}}{t} \\
	\divem{D}=0 &\qquad 
	\rotorm{\vb{E}}=-\frac{1}{c}\dpar{\vb{B}}{t}
\end{aligned}
\een
siendo $\rho_L=0$ y $\mu,\epsilon$ homogéneos.
Modelando de acuerdo
\[
	\vb{B} = \mu \vb{E} \qquad \vb{D} = \epsilon \vb{E}
\]
y siendo la ley de Ohm microscópica
\[
	\vb{J} = \sigma \vb{E}, 
\]
y reemplazando en la ecuación del rotor para \vb{H} se tiene 
\[
	\rotorm{H} = \frac{4\pi}{c}\sigma \vb{E} + \frac{\epsilon}{c}\dpar{\vb{E}}{t}
		= \rotorm{\frac{\vb{B}}{\mu}}
\]
\[
	\rotorm{(\rotorm{E})}= -\frac{1}{c}\rotorm{\left(\dpar{\vb{B}}{t}\right)},
\]

\[
	\Nabla{(\divem{E})}-\lapm{\vb{E}} = -\frac{1}{c}\dpar{}{t}{(\rotorm{\vb{B}})}
\]
y ahora podemos introducir la expresión que tenemos para el rotor de \vb{H} y 
usar que la divergencia de \vb{E} es nula de manera que
\[
	-\lapm{\vb{E}}= -\frac{\mu}{c} \dparbis{ 4\pi\vb{J} + \dpar{\vb{D}}{t} }{t}
\]
y entonces
\[
	-\lapm{\vb{E}} + \frac{4\pi\mu\sigma}{c^2} \dpar{\vb{E}}{t} + 
			\frac{\mu\varepsilon}{c^2}\ddpar{\vb{E}}{t} = 0.
\label{onda_gen}
\]
que no es otra cosa que una ecuación de ondas general.
Un par de casos particulares interesantes son el caso $\sigma=0$ que corresponde a
un dieléctrico, para el que se tiene 
\[
	\lapm{\vb{E}} - \frac{\mu\varepsilon}{c^2}\ddpar{\vb{E}}{t} = 0,
\]
una ecuación de ondas usual. Para el caso general $\sigma > 0$ (conductor) podemos
pensar en una solución general del tipo onda plana armónica,
\[
	\vb{E}(\vb{x})=\vb{E_0}\;\euler^{i(\vb{k}\cdot\vb{x}-\omega t)},
\]
cuyas derivadas temporales son fáciles de hallar y resultan proporcionales a $\vb{E}$
de manera que reemplazando este {\it ansatz} en la ecuación arribamos a
\[
	\lapm{\vb{E}} + \frac{4\pi}{c^2}i\mu\sigma\omega\vb{E} +
		\frac{1}{c^2}\mu\varepsilon\omega^2 \vb{E} = 0,
\]
que se puede agrupar de manera más inteligente como 
\[
	\lapm{\vb{E}} + \frac{\mu\varepsilon\omega^2}{c^2}
		\left( 1 + i 4 \pi \frac{\sigma}{\varepsilon\omega}\right)\vb{E} = 0
\]
Podemos definir una especie de número de onda efectivo
\[
	K^2 \equiv k^2 \left( 1 + i 4 \pi \frac{\sigma}{\varepsilon\omega}\right)
\]
y considerar la ecuación de onda homogénea
\[
	\lapm{\vb{E}} + K^2 \vb{E} = 0,
\]
con los diferentes casos particulares ocurriendo dentro de $K^2$. Así para el caso de
un excelente conductor,
\[
	4 \pi \frac{\sigma}{\varepsilon\omega} \gg 1
\]
se tiene 
\[
	\lapm{\vb{E}} + i \frac{4 \pi \sigma \mu\omega}{c^2} \vb{E} = 0
\]
que es una ecuación de difusión para la corriente de conducción (?). Por el contrario en el
caso de un conductor pobre 
\[
	4 \pi \frac{\sigma}{\varepsilon\omega} \ll 1
\]
resulta en 
\[
	\lapm{\vb{E}} + \frac{\mu\varepsilon\omega^2}{c^2} \vb{E} = 0
\]
que es una ecuación de ondas usual dando como resultado una propagación. Tiende a la ecuación
de ondas con $\sigma=0$.

En general podemos escribir
\[
	K^2 = k^2 \left( 1 + \frac{i}{\tau \omega} \right)
\]
donde $\tau$ es la relajación del medio y $\omega$ es la vibración del campo. Se puede poner en
términos del período,
\[
	K^2 = k^2 \left( 1 + \frac{iT}{ 2 \pi \tau } \right)
\]
y si $\tau \gg T$ se tiene propagación.

Para metales $\tau \approx 10^{-14}$ segundos y entonces es válida la ecuación de difusión
hasta la región de radiofrecuencias. Por ejemplo, si 
\[
	\frac{4\pi\sigma}{\varepsilon\omega} \gg 1 \quad \rightarrow \quad \frac{1}{\tau\omega} \gg 1 
			\quad \rightarrow \quad \frac{1}{\tau} \gg \omega
\]
y para metales se cumple que $1.10^{14} \gg 6.10^6$ siendo este último un valor razonable para ondas de radio.
\notamargen{Estos ejemplitos hay que revisarlos y reescribirlos.}


Si consideramos los campos funciones de la distancia $\xi$ de una plano al origen O,
tendremos 

\begin{figure}[htb]
	\begin{center}
	\includegraphics[width=0.3\textwidth]{images/fig_ft1_conduc1.pdf}	 
	\end{center}
	\caption{}
\end{figure} 
los campos son constantes en los planos de normal $\hat{n}$ (ver ilustración).

\[
	\Nabla = \hat{n} \dpar{}{\xi}
\]
y de acuerdo a Maxwell,
\[
	\hat{n} \cdot \dpar{\vb{D}}{\xi} = 0 \qquad \qquad \hat{n} \cdot \dpar{\vb{B}}{\xi} = 0
\]
\[
	\hat{n} \times \dpar{\vb{E}}{\xi} = -\frac{1}{c} \dpar{\vb{B}}{t} \qquad \qquad
	\hat{n} \times \dpar{\vb{H}}{\xi} = \frac{4\pi}{c} \sigma \vb{E} + \frac{\varepsilon}{c} 
\dpar{\vb{E}}{t}
\]
y si tomamos producto escalar de la última ecuación con la normal resulta
\[
	\hat{n} \cdot \left( \hat{n} \times \dpar{\vb{H}}{\xi} \right) = 
		\frac{4\pi}{c} \sigma E_n + \frac{\varepsilon}{c} \dpar{E_n}{t} = 0 
\]
de manera que 
\[
	E_n = E_n^0 \euler^{-\frac{4\pi\sigma}{\varepsilon} t}.
\]

El $E_{\hat{n}}$ (electrostático) se apaga exponencialmente con el tiempo de relajación
del conductor. 
\[
	\hat{n} \cdot \left( \hat{n} \times \dpar{\vb{E}}{\xi} \right) = \frac{\mu}{c} \dpar{H}{t} = 0
\]
$H_n$ solo es constante en el tiempo y uniforme en el espacio.

No se ve influenciado por el conductor. Asimismo la energía está metida casi por
completo en el campo magnético cuando es un muy buen conductor.

\[
	K^2 = \mu \varepsilon \frac{\omega^2}{c^2} \left[ 1 + i\frac{4\pi\sigma}{\varepsilon \omega} \right]
\]
de manera que podemos escribir
\[
	K = \beta + i \frac{\alpha}{2}
\]
siendo $\beta$ el término responsable de la propagación, $\alpha$ el término que se atenua. Esta escritura
es sólo conveniente puesto que resulta que 
\[
	\beta = \sqrt{ \mu \varepsilon }\frac{\omega}{c} 
	\left[ \frac{1 + \sqrt{ 1 + (\omega\tau )^{-2}}}{2}\right]^{1/2} = k
\]
\[
	\frac{\alpha}{2} = \sqrt{ \mu \varepsilon }\frac{\omega}{c} 
	\left[ \frac{1 + \sqrt{ -1 + (\omega\tau )^{-2}}}{2}\right]^{1/2} = k
\]
y ahí vemos la simetría.

Entonces resulta que para el caso de un mal conductor $ \frac{4\pi\sigma}{\omega \varepsilon} \ll 1 $ o bien 
$ \frac{4\pi\sigma}{\varepsilon} \ll \omega $ o bien $ 1/\tau \ll \omega $ se tiene 
\[
	K = \sqrt{ \mu \varepsilon }\frac{\omega}{c} + i \frac{2\pi\sqrt{\mu}\sigma}{c\sqrt{\varepsilon}}
\]
y en cambio por el mismo razonamiento pero para un excelente conductor, $1/\tau \gg \omega$
\[
	K = \frac{\sqrt{2\pi\omega\mu\sigma}}{c}( 1 + i )
\]
y aquí la parte de atenuación $\alpha/2 = \alpha(\omega)/2$ de modo que hay distorsión.
En general tendremos
\[
	\vb{E} = \vec{\mathbb{E}}_0 \euler^{-\alpha/2 \hat{n}\cdot\vb{x}} \euler^{i[\beta \hat{n}\cdot\vb{x} - 
\omega t]}
\]
que nos muestra una primera exponencial real que es el factor de atenuación. Entre más conductor un medio más
rápida es la atenuación.

\[
	\vb{H} = \frac{c}{\mu\omega}\left[ \beta + i \frac{\alpha}{2} \right] \hat{k}\times \vb{E}
\]
de modo que si $\vb{k} \in \mathbb{R}^3$ entonces $\vb{E}_0, \vb{B}_0$ tienen la misma fase. Si en cambio 
\vb{k} pertenece a $\mathbb{C}^3$ hay crecimiento o decaimiento exponencial en algunas direcciones.
Vemos en una onda plana el caso de la atenuación se vería así
\[
	\euler^{i(\vb{k}\cdot\vb{x}-\omega t)} =
	\euler^{i( k\hat{n}_R \cdot\vb{x} + k\hat{n}_I \cdot\vb{x} -\omega t)} =
	\euler^{i( k\hat{n}_R \cdot\vb{x} -\omega t) } + \euler^{ k\hat{n}_I \cdot\vb{x} }
\]

\subsection{Algunos ejemplos y casos}

Recordemos 
\[
	\omega = 2\pi\nu  \qquad k=\frac{2\pi}{\lambda}=\frac{\omega}{c}\sqrt{\mu\epsilon} =\frac{\omega}{v}
\]

\begin{figure}[htb]
	\begin{center}
	\includegraphics[width=0.4\textwidth]{images/fig_ft1_conduc2.pdf}	 
	\end{center}
	\caption{}
\end{figure} 

Noción de algunos numeritos a partir de la tabla

\begin{center}
\begin{tabular}{ c c c }
& $\nu$ & $\lambda$ \\
celular & 1.10$^9$ Hz & 0.3 m \\
radio AM  & 1.10$^6$ Hz  & 300 m \\
radio FM & 1.10$^8$ Hz & 3 m
\end{tabular}
\end{center}



\begin{figure}[htb]
	\begin{center}
	\includegraphics[width=0.4\textwidth]{images/fig_ft1_conduc3.pdf}	 
	\end{center}
	\caption{}
\end{figure} 

Para una malla metálica de interespaciado $\ell$ tenemos los casos
$\lambda \gg \ell$ significa que es macizo para la onda. La penetración no dependerá
de la $\sigma(\omega)$, es un conductor macizo.

$\lambda \ll \ell$ es transparente para la onda. No ve el conductor.

\begin{figure}[htb]
	\begin{center}
	\includegraphics[width=0.4\textwidth]{images/fig_ft1_conduc4.pdf}	 
	\end{center}
	\caption{}
\end{figure} 

En un buen conductor $ \mu_M > \mu_E $ el material se comporta inductivamente.
En un mal conductor $ \mu_E > \mu_M $ el material se comporta capacitivamente.


% \bibliographystyle{CBFT-apa-good}	% (uses file "apa-good.bst")
% \bibliography{CBFT.Referencias} % La base de datos bibliográfica

\end{document}
