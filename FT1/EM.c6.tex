	\documentclass[10pt,oneside]{CBFT_article}
	% Algunos paquetes
	\usepackage{amssymb}
	\usepackage{amsmath}
	\usepackage{graphicx}
	\usepackage{libertine}
	\usepackage[bold-style=TeX]{unicode-math}
	\usepackage{lipsum}

	\usepackage{natbib}
	\setcitestyle{square}

	\usepackage{polyglossia}
	\setdefaultlanguage{spanish}

	\usepackage{CBFT.estilo} % Cargo la hoja de estilo

	% Tipografías
	% \setromanfont[Mapping=tex-text]{Linux Libertine O}
	% \setsansfont[Mapping=tex-text]{DejaVu Sans}
	% \setmonofont[Mapping=tex-text]{DejaVu Sans Mono}

	%===================================================================
	%	DOCUMENTO PROPIAMENTE DICHO
	%===================================================================

\title{CBFT Electromagnetismo}
\author{Ondas planas}
\date{\today}

\begin{document}
\maketitle
\tableofcontents


% =================================================================================================
\section{Reflexión y refracción de ondas electromagnéticas en una interfaz}
% =================================================================================================

 
% =================================================================================================
\section{Coeficientes de Fresnel}
% =================================================================================================


% =================================================================================================
\section{Ondas en medios dieléctricos}
% =================================================================================================

Para el campo eléctrico como onda plana,
\eqn{
\vb{E}(\vb{x},t)=\vb{E_0}\;\euler^{i(\vb{k}\cdot\vb{x}-\omega t)},
\label{onda_plana}
}
podemos obtener el campo magnético \vb{H} a través de la ecuación de Maxwell del rotor
de \vb{E}. Para una expresión como \eqref{onda_plana} se ve que:
\notamargen{Esto es un bosquejo de lo que usamos en la sección siguiente.
Esta deducción debe estar entonces.}
\eqnn{ \rotorm{\vb{E}}=i\vb{k}\times\vb{E}=-\dpar{\vb{B}}{t} }
y es fácil de integrar esta ecuación para obtener
\eqnn{ \vb{H}=\frac{1}{\omega\mu}\vb{k}\times\vb{E} }

% =================================================================================================
\section{Ondas electromagnéticas en conductores}
% =================================================================================================

En el seno de un conductor el campo eléctrico dará origen a corrientes de
modo que ahora las ecuaciones de Maxwell resultan 
\notamargen{Habría que hacer un buen gráfico con la onda entrando en el conductor,
y la geometría simplificada.}

\eqnn{
\begin{aligned}
\divem{B}=0 &\qquad 
\rotorm{\vb{H}}=\vb{J} + \dpar{\vb{D}}{t} \\
\divem{D}=0 &\qquad 
\rotorm{\vb{E}}=-\dpar{\vb{B}}{t}
\end{aligned}
}
de manera que aparece un término más como fuente de campo magnético, que 
son las corrientes \vb{J}. Estaremos suponiendo que las otras fuentes están 
lejanas pero hacen campo en el recinto de interés.

Resolveremos este conjunto de ecuaciones para un caso con muchas simplificaciones.
Asumiremos medios lineales, isótropos y homogéneos (MLIH), de manera que tendremos
\eqnn{
\vb{B}=\mu\vb{H} \qquad \vb{D}=\varepsilon\vb{E} \qquad  \vb{J}=\sigma\vb{E} \qquad, 
}
donde en esta última ecuación $\sigma$ es la conductividad y es una constante, 
al igual que también son constantes $\mu$ y $\varepsilon$.

Tomaremos primeramente la ecuación del rotor de \vb{E}, y le aplicamos el rotor 
nuevamente obteniendo
\eqnn{
\rotorm{(\rotorm{E})}=\rotorm{\left(-\dpar{\vb{B}}{t}\right)},
}
la cual se puede trabajar con una identidad del miembro izquierdo y 
conmutando la operación rotor con la derivada parcial,
\eqnn{
\nabla{(\divem{E})}-\lapm{\vb{E}}=-\dparbis{(\rotorm{\vb{B}})}{t}
}
y ahora podemos introducir la expresión que tenemos para el rotor de \vb{H} y 
usar que la divergencia de \vb{E} es nula de manera que
\eqnn{
-\lapm{\vb{E}}=-\mu \dparbis{ \vb{J} + \dpar{\vb{D}}{t} }{t}
}
y entonces
\eqn{
-\lapm{\vb{E}} + \mu\sigma \dpar{\vb{E}}{t} + \mu\varepsilon\ddpar{\vb{E}}{t}
= 0.
\label{onda_gen}
}

Este proceso podría hacerse también para el campo \vb{B} y llegaríamos a una 
ecuación equivalente. El campo eléctrico verifica una ecuación de onda 
tridimensional con un término disipativo (es el de la derivada $\partial_t$).

Plantearemos como solución de esta ecuación una onda plana general,
	\eqnn{
	\vb{E}(\vb{x})=\vb{E_0}\;\euler^{i(\vb{k}\cdot\vb{x}-\omega t)}
	}
notando que 
\eqnn{
\dpar{\vb{E}}{t} = -i\omega \vb{E} \qquad \text{y} \qquad 
\ddpar{\vb{E}}{t} = -\omega^2 
\vb{E}
}
la ecuación \eqref{onda_gen} se reduce a:
	\eqnn{
	\lapm{\vb{E}} + i\mu\sigma\omega\vb{E} + \mu\varepsilon\omega^2 \vb{E} = 0.
	}
o bien, agrupando de manera conveniente
	\eqnn{
	\lapm{\vb{E}} + \mu\varepsilon\omega^2\left( 
	1+i\frac{\sigma}{\varepsilon\omega}\right)\vb{E} = 0.
% 	\label{onda_helm}
	}
La conveniencia de esta forma de agrupar reside en que así es posible examinar 
varios casos límites interesantes.

Podemos tomar ahora el laplaciano a la onda \vb{E}, el cual se puede ver que 
resulta $k^2$, donde $k$ es el módulo de \vb{k}.

\begin{notas}{Laplaciano en notación indicial:}
Podemos hacer la cuenta $\nabla^2\vb{E}$ de manera rápida usando la notación 
indicial, para el caso de una onda plana. Así escribiríamos
\[ 
\nabla^2\vb{E} = \nabla^2(\vb{E}_0 \: e^{i(\vb{k}\vb{x}-\omega t)}),
\]
que en indicial pasa a ser
\[
\partial_j\partial_j (E_{l}^0 \: e^{i( k_mx_m-\omega t)}),
\]
donde estamos usando convención de suma de Einstein y hemos pasado el subíndice
cero a supraíndice por razones de claridad. El carácter vectorial reside en el
coeficiente $\vb{E}_0$ que es constante. El laplaciano de un vector es un vector
y estamos viendo el componente $l$-ésimo. Introduciéndo la primer derivada en el
argumento,
\[ 
E_l^0 \partial_j(e^{i( k_mx_m-\omega t)}\: ik_m \partial_jx_m)
\]
y usando que $\partial_jx_m=\delta_{jm}$
\[ 
E_l^0 \partial_j(e^{i( k_mx_m-\omega t)}\: ik_j)
=
i E_l^0 k_j \partial_j(e^{i( k_mx_m-\omega t)})
=
i^2 E_l^0 k_j^2 e^{i( k_mx_m-\omega t)}
\]
de manera que volviendo a la notación vectorial tenemos
\[
\nabla^2 \vb{E} = - \vb{E}_0 |\vb{k}|^2 \: e^{i(\vb{k}\vb{x}-\omega t)}) = - k^2 \vb{E}. 
\]
\end{notas}

Es decir que tenemos
\eqnn{
-k^2\vb{E} + 
\mu\varepsilon\omega^2\left(1+i\frac{\sigma}{\varepsilon\omega}\right)\vb{E}=
\left[-k^2 + 
\mu\varepsilon\omega^2\left(1+i\frac{\sigma}{\varepsilon\omega}\right)\right]\vb{
E}=0
}
lo que significa finalmente que
\eqnn{
|\vb{k}| = 
\sqrt{\mu\varepsilon\omega^2}\left(1+i\frac{\sigma}{\varepsilon\omega}\right)^{
\frac{1}{2}}
}
en donde se ha indicado expresamente el módulo de \vb{k} encerrando el vector 
entre las barras. Hay varias observaciones importantes para hacer: el módulo de 
\vb{k} es un complejo. El módulo, que refiere al tamaño de la {\it flecha} que 
representa al vector resulta ser un número complejo. ¿Qué representa esto?



\bibliographystyle{CBFT-apa-good}	% (uses file "apa-good.bst")
\bibliography{CBFT.Referencias} % La base de datos bibliográfica

\end{document}
