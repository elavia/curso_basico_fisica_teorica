	\documentclass[10pt,oneside]{CBFT_book}
	% Algunos paquetes
	\usepackage{amssymb}
	\usepackage{amsmath}
	\usepackage{graphicx}
	\usepackage{bm}
	\usepackage{libertine}
% 	\usepackage[bold-style=TeX]{unicode-math}
	\usepackage{lipsum}

	\usepackage{natbib}
	\setcitestyle{square}

	\usepackage{polyglossia}
	\setdefaultlanguage{spanish}


	\usepackage{CBFT.estilo} % Cargo la hoja de estilo

	% Tipografías
	% \setromanfont[Mapping=tex-text]{Linux Libertine O}
	% \setsansfont[Mapping=tex-text]{DejaVu Sans}
	% \setmonofont[Mapping=tex-text]{DejaVu Sans Mono}

	%===================================================================
	%	DOCUMENTO PROPIAMENTE DICHO
	%===================================================================

\begin{document}

% =================================================================================================
\chapter{Teorema de Green}
% =================================================================================================

% =================================================================================================
\section{Imágenes y método de Green}
% =================================================================================================

El método de las imágenes es un procedimiento gráfico de encontrar problemas equivalentes simulando
con cargas extras (cargas imagen) las condiciones de contorno.

Queremos conseguir un problema equivalente al de la figura, una carga puntual frente a un plano
conductor conectado a tierra.

\includegraphics[width=0.3\textwidth]{images/fig_ft1_imagenes1.jpg}

La solución del problema de arriba será la del problema siguiente 

\includegraphics[width=0.3\textwidth]{images/fig_ft1_imagenes2.jpg}

porque en el recinto de interés son la misma solución. Se puede garantizar esto por unicidad.

\begin{figure}[htb]
	\begin{center}
	\includegraphics[width=0.6\textwidth]{images/fig_ft1_imagegreen1.pdf}	 
	\end{center}
	\caption{}
\end{figure} 

Los problemas que ilustra la figura satisfacen iguales condiciones de contorno en el recinto punteado,
entonces sus soluciones internas son la misma: $\phi_1 = \phi_2$ por unicidad.

\[
	\phi(\vb{x}) = \frac{q}{[ (x-x')^2 + (y-y')^2 + (z-z')^2 ]^{1/2}} - 
	\frac{q}{[ (x+x')^2 + (y-y')^2 + (z-z')^2 ]^{1/2}}
\]

\subsection{El Método de Green}

El concepto tras el método de Green es evaluar el $\phi$ de una carga puntual ante cierta configuración
de contornos conductores; luego desde allí se extraerá información para el caso de una distribución de
carga general. Considerar una carga puntual corresponde a una excitación elemental.

Restando entre sí
\[
	\Nabla\cdot(\phi\Nabla\psi) = \phi\lapm{\psi} + \Nabla\phi\cdot\Nabla\psi
\]
y
\[
	\Nabla\cdot(\psi\Nabla\phi) = \psi\lapm{\phi} + \Nabla\psi\cdot\Nabla\phi
\]
e integrando ambos miembros y utilizando el teorema de la divergencia, se llega a
\[
	\int_V \left[ \phi\lapm{\psi} - \psi\lapm{\phi}\right] dV =
	\int_S \left[ \phi\Nabla\psi - \psi\Nabla\phi \right] \hat{n} dS,
\]
que es la segunda identidad de Green.

Consideremos lo que llamaremos caso A, según vemos en figura, caracterizado según
\[
	\rho_{\text{int}} \qquad \vb{x}'\in R, \vb{x}\in R
\]
\begin{figure}[htb]
	\begin{center}
	\includegraphics[width=0.2\textwidth]{images/fig_ft1_imagegreen2.pdf}	 
	\end{center}
	\caption{}
\end{figure} 
\be
	\psi = \frac{1}{|\vb{x}-\vb{x}'|} \qquad \lapm{\psi} = -4\pi \delta(\vb{x}-\vb{x}')
	\label{laplace_eq}
\ee
\[
	-\phi(\vb{x})4\pi + \int_V 4\pi \frac{\rho(\vb{x}')}{|\vb{x}-\vb{x}'|} \; dV' =
	\int_S \left( \phi\dpar{\psi}{n}-\frac{1}{|\vb{x}-\vb{x}'|}\dpar{\phi}{n}\right)\; dS 
\]
donde estamos usando la abreviatura $\Nabla\phi\cdot\vb{n}=\partial\phi/\partial n$ que es la
derivada normal en la superficie. Despejando
\[
	\phi(\vb{x}) = \int_V \frac{\rho(\vb{x}')}{|\vb{x}-\vb{x}'|} \; dV' +
	\frac{1}{4\pi} \int_S \left( \frac{1}{|\vb{x}-\vb{x}'|}\dpar{\phi}{n} -
	\phi\frac{\partial}{\partial n} \left[\frac{1}{|\vb{x}-\vb{x}'|} \right] \right)\; dS ,
\]
donde la primer integral es debido a las cargas internas y la segunda se puede ver como el efecto
de las cargas fuera del recinto $ R $. 

Recordemos que las condiciones tipo Dirichlet corresponden a $\phi|_S$ (el potencial evaluado sobre
la superficie) y las tipo Neumann a $ \partial\phi/\partial \hat{n}|_S $ (la derivada normal del
potencial evaluada sobre la superficie).
Aquí se presentan condiciones tipo Dirichlet y Newmann simultáneamente en todo punto de la superficie,
lo cual no puede ser.

El caso B, según figura, corresponde a
\[
	\rho_{\text{ext}} \qquad \vb{x}'\notin R, \vb{x}\in R
\]
y, como la delta de Dirac es ahora nula, se tiene  
\[
	\int_V \frac{\rho(\vb{x}')}{|\vb{x}-\vb{x}'|} \; dV' = 
	\frac{1}{4\pi} \int_S \left( \phi\frac{\partial}{\partial n} \left[\frac{1}{|\vb{x}-\vb{x}'|} \right]
	- \frac{1}{|\vb{x}-\vb{x}'|}\dpar{\phi}{n}  \right)\; dS,
\]
es decir que la integral de superficie proviene de las cargas fuera de $R$ que producen campo en el interior
del recinto $R$.
Nótese que hay cambios de signo debidos a la diferente orientación de la superficie.

\includegraphics[width=0.4\textwidth]{images/fig_ft1_green_normales.jpg}

\begin{figure}[htb]
	\begin{center}
	\includegraphics[width=0.2\textwidth]{images/fig_ft1_imagegreen3.pdf}
	\end{center}
	\caption{}
\end{figure}

Entonces el potencial 
\[
	\phi(\vb{x}) = \int_{ V = R_1 } \frac{\rho_{\text{int}}(\vb{x}')}{|\vb{x}-\vb{x}'|} \; dV' +
	\int_{ V \neq R_1 } \frac{\rho_{\text{ext}}(\vb{x}')}{|\vb{x}-\vb{x}'|} \; dV'
\]

Hemos tomado en la fórmula de Green $\psi=1/|\vb{x}-\vb{x}'|$ que es solución particular de \eqref{laplace_eq},
interpretándose $\psi$ como el potencial de una carga puntual unitaria. Verifica:
\[
	\lapm{\left( \frac{1}{|\vb{x}-\vb{x}'|} \right)} = - 4\pi \delta( |\vb{x}-\vb{x}'| )
\]
Pero se puede considerar una solución general $G$ (la función de Green) que cumple 
\[
	G \equiv \frac{1}{|\vb{x}-\vb{x}'|} + f( \vb{x}, \vb{x}'),
\]
siendo $f$ una función arbitraria. Luego,
\[
	\lapm{G} = -4\pi  \delta( \vb{x}, \vb{x}' ) + \lapm{f}
\]
donde $f$ satisface Laplace (si el reciento no incluye a $\vb{x}'$).
Con $\lapm{f( \vb{x}, \vb{x}' )}$.

Entonces $f( \vb{x}, \vb{x}' )$ representan la o las imágenes necesarias para que $G$ cumpla el contorno
necesario $G_D|_S=0$.


% =================================================================================================
\subsection{Funciones de Green}
% =================================================================================================

Utilizando $G$ en la expresión del potencial
\be
	\phi(\vb{x}) = \int_{V'} G(\vb{x},\vb{x}') \rho(\vb{x}')  \; dV' +
	\frac{1}{4\pi} \int_{S'} \left( G(\vb{x},\vb{x}')\dpar{\phi}{n} -\phi\frac{\partial}{\partial 
	n} G(\vb{x},\vb{x}') \right)\; dS' ,
	\label{green1}
\ee
Pero para poder utilizar \eqref{green1} necesito tener un solo tipo de condiciones de contorno.
Si tenemos condiciones tipo Dirichlet (valor de $\phi$ en la frontera $R_1$), podemos tomar $G$ nula 
en $S$ de manera que se anula el término que multiplica a $\partial \phi / \partial n$ (que de todas
maneras no puedo conocerlo).
Entonces, llamando a esta función de Green $G_D$ (por Dirichlet) se ve que debe cumplir 
\[
	\lapm{G_D} = -4 \pi \delta(\vb{x},\vb{x}') \qquad \qquad G_D |_{S_1} = 0,
\]
condiciones que físicamente se asocian a un potencial que cumple una ecuación de Poisson para una carga 
puntual en situada en $ \vb{x} = \vb{x}' $ y que además es nulo sobre la superficie $S_1$, como si 
reemplazáramos esa superficie por un conductor metálico puesto a tierra.
Entonces la expresión 
\[
	\phi(\vb{x}) = \frac{q}{[ (x-x')^2 + (y-y')^2 + (z-z')^2 ]^{1/2}} - 
	\frac{q}{[ (x+x')^2 + (y-y')^2 + (z-z')^2 ]^{1/2}}	
\]
con $q=1$ es la función de Green para este problema.

La solución del problema de potencial será 
\[
	\phi(\vb{x}) = \displaystyle \int_{V'} G_D \rho \; dV' - \frac{1}{4\pi}
				\int_{S} \phi|_S\frac{\partial}{\partial n} G_D \; dS'
\]
junto con el dato $ \phi|_S $.

La condición de contorno de $G$ equivale, en el contexto físico del electromagnetismo, a
reemplazar el contorno por un conductor metálico puesto a tierra.
Entonces $G$ es el potencial de la configuración de conductores con el contorno puesto a tierra
frente a una carga puntual con magnitud unitaria.

La función de Green da la geometría del problema.

\[
	\dpar{\phi_1}{n}|_S - \dpar{\phi_2}{n}|_S = -4\pi\sigma \qquad \qquad \phi_2|_S = \phi_1|_S
\]

\[
	\textrm{Neumann} \quad 	\begin{cases}
				G_N : \lapm{G_N} = -4\pi \delta(\vb{x},\vb{x}') \\
				\Nabla G_N \cdot \hat{n}|_S = -\frac{4\pi}{S}  \\
				\left.\dpar{\phi}{n}\right|_S \\
				\phi(\vb{x}) = \displaystyle <\phi>|_S + \int_{V'} G_N \rho \; dV' + 
				\frac{1}{4\pi} \int_{S} G_N|_S \dpar{G_N}{n} \; dS
			\end{cases}
\]

\subsection{Green para el problema externo de una esfera}

En este problema las condiciones adecuadas son las de Dirichlet, ver Figura
\begin{figure}[htb]
	\begin{center}
	\includegraphics[width=0.4\textwidth]{images/fig_ft1_green1.pdf}	 
	\end{center}
	\caption{}
\end{figure} 
y podemos escribir la función de Green como 
\[
	G = \frac{1}{|\vb{r} - D\hat{r}'|} - \frac{a/D}{|\vb{r} - a^2/D\hat{r}'|} \qquad G|_{r=a}
\]
sujeta a que 
\[
	q' = -q a/D \qquad d = a^2/D
\]
\begin{figure}[htb]
	\begin{center}
	\includegraphics[width=0.6\textwidth]{images/fig_ft1_green2.pdf}	 
	\end{center}
	\caption{$G_D$ es el potencial de la configuración (a) y se evalúa teniendo en cuenta la
	otra (b) que se resuelve casualmente por imágenes. La (c) se resuelve alterando las condiciones.}
\end{figure} 

El caso (c) de la Figura se resuelve con 
\[
	-\frac{V}{4\pi} \int_S \dpar{G}{n} dS = -\frac{V}{4\pi} \int_S \Nabla G\cdot d\vb{S} =
	-\frac{V}{4\pi} \int_V \lapm{G} \: dV	
\]
\[
	= -\frac{V}{4\pi} (-4\pi)\int_V \delta(\vb{x}-\vb{x}') \: dV	= V 
\]

\section{Algunos campos}

En distribuciones infinitas de carga la integral de Poisson diverge pero ello se debe a que en
realidad no existen distribuciones infinitas de carga.
\begin{figure}[thb]
	\begin{center}
	\includegraphics[width=0.8\textwidth]{images/fig_ft1_campohilos.pdf}	 
	\end{center}
	\caption{}
\end{figure} 

\section{Notas método de Green}

Función de Green libre (sin contornos) lleva directo a la integral de Poisson
\[
	G(\vb{x}, \vb{x}') = \frac{1}{|\vb{x} - \vb{x}'|}
\]
entonces 
\[
	\phi(\vb{x}) = \int_V \rho \:G \:dV = \int_{V'} \frac{ \rho(\vb{x}) }{|\vb{x}-\vb{x}'|} dV'
\]
\[
	\nabla^2 \left( \frac{1}{|\vb{x} - \vb{x}'|} \right) = 4\pi \delta(\vb{x}-\vb{x}')
\]
\[
	G(\vb{x}, \vb{x}') =  \frac{1}{|\vb{x} - \vb{x}'|} + f(\vb{x}, \vb{x}') \qquad 
	\textrm{con} \quad \lapm{f}(\vb{x}, \vb{x}') = 0 \quad \textrm{si} \quad \vb{x}\neq\vb{x}'
\]

Para condiciones de Neumann se toma:
\[
	\Nabla G_N|_S = -\frac{4\pi}{S} = \left. \dpar{G}{n} \right|_S
\]
la integral 
\[
	- \frac{1}{4\pi} \int_S \phi|_S \left.\dpar{G}{n}\right|_S  dS
\]
no se puede anular con 
\[
	\left.\dpar{G}{n}\right|_S = 0
\]
salvo que el volumen de integración no contenga a $\vb{x}=\vb{x}'$ en cuyo caso:
se excluye $\vb{x}=\vb{x}'$ de la integración.
\[
	- \frac{1}{4\pi} \int_S \phi|_S \left.\dpar{G}{n}\right|_S  dS =
	\frac{1}{S} \int_S \phi|_S dS = <\phi>|_S
\]
que es el valor promedio de $\phi$ en la superficie $S$.

Se suele tomar la superficie $S \to \infty$ de modo que resulte nulo $<\phi>|_S$.
Se toma el volumen $V$ rodeado por dos superficies una cerrada y finita y la otra
en infinito entonces
\[
	<\phi>|_S = 0 \qquad \qquad \left. \dpar{G}{n}\right|_S = 0
\]
esto es el llamado {\it problema exterior}.

% =================================================================================================
\section{Condiciones de contorno}
% =================================================================================================

La ley de Gauss nos dice
\[
	\int \vb{E} \cdot d\vb{S} = 4 \pi Q_n
\]
para el cilindrito de la figura
\[
	( \vb{E}_2 - \vb{E}_1 )\cdot \hat{n} \Delta S = 4 \pi \sigma \Delta S 
\]
\[
	( \vb{E}_2 - \vb{E}_1 )\cdot \hat{n} = 4 \pi \sigma 
\]
\[
	\rotorm{E} = 0 \Rightarrow \int_\Gamma \vb{E} \cdot d\vb{\ell} = 0 =
	( \vb{E}_2 - \vb{E}_1 )\cdot d\vb{\ell}  = ( \vb{E}_1 + \vb{E}_2 ) \cdot \hat{n}\times\hat{\eta} d\ell
\]
donde esto vale en electrostática (nula la integral de línea del campo \vb{E}) y además
\[
	\hat{n}\times\hat{\eta} = \frac{d\vb{\ell}}{d\ell} 
\]

\begin{figure}[htb]
	\begin{center}
	\includegraphics[width=0.4\textwidth]{images/fig_ft1_contorno1.pdf}	 
	\end{center}
	\caption{}
\end{figure} 
y puesto que vale la permutación
\[
	0 = ( -\vb{E}_2 + \vb{E}_1 )\cdot(\hat{n}\times\hat{\eta}) \longrightarrow 
	0 = \hat{\eta} \cdot ( ( -\vb{E}_2 + \vb{E}_1 ) \times \hat{n} )
\]

de modo que la componente tangencial es continua y entonces
\[
	\hat{n} \times ( \vb{E}_2 - \vb{E}_1 ) = 0
\]
\[
	E_{2\hat{n}} - E_{1\hat{n}} = 4 \pi \sigma \qquad \qquad E_{2\hat{t}} -E_{1\hat{t}} = 0
\]
\[
	-\Nabla\phi_2\cdot\hat{n} + \Nabla\phi_1\cdot\hat{n} = 4 \pi \sigma
\]
\[
	\frac{\Nabla(\phi_2-\phi_1)\cdot \hat{n}}{4 \pi} = \sigma
\]
\[
	\sigma = \frac{1}{4\pi}\dpar{(\phi_1-\phi_2)}{n}
\]
esta es la densidad de carga inducida sobre la frontera entre medios.

Para los medios magnéticos
\[
	\rotorm{H} = \frac{4 \pi}{c} \vb{J}_l
\]
\[
	\int_S (\rotorm{H}) \cdot d\vb{S} = \int_S \frac{4 \pi}{c} \vb{J}_l \cdot d\vb{S} = 
	\frac{4 \pi}{C} \vb{g}_l\cdot \hat{s} d\ell
\]
donde hicimos la transformación
\[
	\int \vb{H}\cdot d\ell = (\vb{H}_2-\vb{H}_1)\cdot d\ell
\]
y donde recordemos que la altura de $\Gamma$ tiene a cero.
\[
	\frac{4 \pi}{c} \vb{g}_l \cdot \vb{s} = ( -\vb{H}_2 + \vb{H}_1 )\cdot( \hat{n} \times \hat{s} ) d\ell
\]
\[
	\frac{4 \pi}{c} \vb{g}_l \cdot \vb{s} \; d\ell = (\vb{H}_1 -\vb{H}_2 \times \hat{n})\cdot \hat{s} 
d\ell
\]

\begin{figure}[htb]
	\begin{center}
	\includegraphics[width=0.4\textwidth]{images/fig_ft1_contorno2.pdf}	 
	\end{center}
	\caption{}
\end{figure} 

de manera que 
\[
	\frac{4 \pi}{c} \vb{g}_l = \hat{n} \times ( \vb{H}_2 - \vb{H}_1 )
\]
\[
	\hat{n}\times\hat{s} = \frac{d\vb{\ell}}{d\ell} 
\]
\[
	B_{2\hat{n}} - B_{1\hat{n}} = 0 \qquad \qquad H_{2\hat{t}} - H_{1\hat{t}} = \frac{4 \pi}{c} g_l
\]
\[
	\int_S \vb{B}\cdot d\vb{S} = 0 \Rightarrow (\vb{B}_2 - \vb{B}_1 )\cdot\hat{n} = 0
\]

% =================================================================================================
\section{Desarrollo multipolar}
% =================================================================================================

\[
	\phi(\vb{x}) = \int_{V'} \frac{\rho(\vb{x})}{|\vb{x}-\vb{x}'|} \; dV'
\]
Cuando la expresión es muy complicada podemos desarrollarla en una serie de potencias
\[
	\phi(\vb{x}) = \frac{Q}{|\vb{x}|} + \frac{\vb{x}\cdot\vb{p}}{|\vb{x}|^3} +
	\sum_{i,j}^3 \frac{1}{2|\vb{x}|^5} x_i Q_{ij} x_j
\]
donde está centrado en el origen de coordenadas. El último término, matricialmente sería
\[
	\frac{1}{2} \frac{\vb{x}^t Q \vb{x}}{|\vb{x}|^5}
\]
y es el término cuadrupolar.

Los momentos son el momento dipolar,
\[
	\vb{p} = \int_V \vb{x} \: \rho(\vb{x}) \: dV
\]
el momento monopolar
\[
	Q = \int_V \rho(\vb{x}) dV
\]
que es la carga total, y el momento cuadrupolar
\[
	Q_{ij} = \int_V  \rho(\vb{x}) \left[ 3 x_i x_j - \delta_{ij} |\vb{x}|^2 \right] \: dV
\]

El momento cuadrupolar refleja apartamiento de la esfera perfecta, los momentos dipolar y cuadrupolar
indican desbalance de carga.
Asimismo $Q_{ij} = Q_{ji}$ es simétrico por ser producto de vectores polares.
Es siempre diagonalizable. Tiene traza nula,
\[
	Q_{xx} + Q_{yy} + Q_{zz}  = 0
\]
se da también que $Q_{ij} (i\neq j)$ mide desbalance lejos de los ejes.
Una esfera con $\rho$ uniforme tiene todos los momentos multipolares nulos salvo el monopolo.

\begin{figure}[htb]
	\begin{center}
	\includegraphics[width=0.8\textwidth]{images/fig_ft1_multipolo2.pdf}	 
	\end{center}
	\caption{}
\end{figure}

Una simetría de reflexión implica que el $\vb{p}_\perp = 0$ donde la notación significa perpendicular
al plano. Esto es así porque no hay desbalance. Para una simetría de revolución $Q_{xx}=Q_{yy}$ entonces
el $Q_{ij}$ puede darse con un sólo número.

Si en una distribución dada, los momentos multipolares hasta el orden $\ell -1$ son nulos entonces
el momento multipolar de orden $\ell$ no depende del origen de coordenadas.

\begin{figure}[htb]
	\begin{center}
	\includegraphics[width=0.6\textwidth]{images/fig_ft1_multipolo3.pdf}	 
	\end{center}
	\caption{}
\end{figure}

En la figura vemos que no ambos no tienen desbalance de carga respecto del origen; el disco uniformemente
cargado tendrá monopolo no nulo y dipolo nulo (siempre respecto del origen), los anillos cargados con carga
opuesta tendrán monopolo y dipolo nulos (respecto del origen y de cualquier otro punto). Pero si muevo las
distribuciones se tendrá desbalance el disco pero no los anillos.

\begin{figure}[htb]
	\begin{center}
	\includegraphics[width=0.3\textwidth]{images/fig_ft1_multipolo4.pdf}	 
	\end{center}
	\caption{}
\end{figure}

Para átomos en general son monopolo, dipolo neutros; el cuadrupolo se da con un solo número. 
En la Figura tenemos un elipsoide con densidad de carga $\rho$ uniforme. Tiene simetría de revolución
de modo que el momento cuadripolar es un número. $Q_{zz} = 0 $ puesto que una esfera no tiene
desbalance, entonces $\overleftrightarrow{Q} = 0 $ 


% =================================================================================================
\section{Dipolo eléctrico}
% =================================================================================================

\[
	\phi(\vb{x}) = \frac{ \vb{p}\cdot\vb{x} }{|\vb{x}|^3} 
\]
si está en el origen, y
\[
	\phi(\vb{x}) = \frac{ \vb{p}\cdot(\vb{x}-\vb{x}_0) }{|\vb{x} - \vb{x}_0|^3} 
\]
si está en un punto $\vb{x}_0$
\[
	\vb{E}(\vb{x}) = \frac{3 \vb{p}\cdot\hat{n} }{|\vb{x} - \vb{x}_0|^3}  \hat{n} - 
		\frac{ \vb{p} }{|\vb{x} - \vb{x}_0|^3}	
\]
donde debemos notar que \vb{p} no depende de \vb{x}.

\begin{figure}[htb]
	\begin{center}
	\includegraphics[width=0.3\textwidth]{images/fig_ft1_dipolar2.pdf}	 
	\end{center}
	\caption{Dipolo centrado en el origen.}
\end{figure}

\[
	\phi(\vb{x}) = \frac{p\hat{z}\cdot r\hat{r}}{r^3} = \frac{p}{r^2} \cos(\theta)
\]
siendo 
\[
	\hat{n} = \frac{\vb{x}-\vb{x}_0}{|\vb{x}-\vb{x}_0|}
\]
\[
	\vb{E}(r,\theta) = \frac{2 p\cos(\theta)}{r^3} \hat{r} + \frac{p\sin(\theta)}{r^3} \hat{\theta}
\]
tiene simetría de revolución, puesto que no depende de $\hat{\phi}$.

Las líneas de campo cuplen que $d\vb{\ell}$ a través de una línea de campo es tal que 
\[
	d\vb{\ell} \parallel \vb{E} \quad \Rightarrow \quad  \vb{E} \times d\vb{\ell}  = 0
\]
la línea de campo sigue la dirección del campo. En el caso del dipolo no tendrán componente en
$\hat{\phi}$ (como es de esperar).

\subsection{Inteacción de un campo externo con una distribución de carga}

Si tenemos un campo \vb{E} con sus fuentes lejos,

\begin{figure}[htb]
	\begin{center}
	\includegraphics[width=0.4\textwidth]{images/fig_ft1_dipolar3.pdf}	 
	\end{center}
	\caption{}
\end{figure}

y que cumple $\divem{E}=0$ y $\rotorm{E}=0$ (irrotacionalidad), se da la siguiente fuerza sobre la distribución
\[
	\vb{F} = \int_V \rho(\vb{x}) \: \vb{E}(\vb{x}) \: dV,
\]
y si \vb{E} no varía demasiado en $V$, entonces podemos representar bien por una serie
\[
	E^\ell(\vb{x}) = E^\ell + x_j \partial_j E^\ell + \frac{1}{2} x_j x_k \partial_j\partial_k E^\ell
\]
entonces 
\[
	F_i = \int_V \rho E_i dV \approx E_i \int_V \rho dV + \int_V \rho  x_j \partial_j E_i dV +
		\frac{1}{2} \int_V \rho x_j x_k \partial_j\partial_k E_i dV 
\]
o bien 
\[
	F_i = \int_V \rho E_i dV \approx E_i q + (\vb{p}\cdot\Nabla) E_i  +
		\vb{x}\cdot\left[ (\vb{x}\cdot\Nabla)\Nabla E_i\right]
\]
de lo cual extraemos que el campo interactúa con la carga, el gradiente del campo interactúa con el dipolo
y la divergencia del campo interactúa con el cuadrupolo.
Un campo uniforme entonces no hace fuerza sobre un dipolo.
Para un campo inhomogéneo, el torque $\vb{\Tau} = \vb{x} \times \vb{F}$ se puede escribir como 
\[
	\vb{\Tau} = q\vb{x} \times \vb{E} = \vb{p} \times \vb{E}
\]
donde $\vb{p}\equiv q\vb{x}$ es el momento dipolar y vemos que el torque tiende a centrar el dipolo
según la dirección del campo \vb{E} aunque no lo logra por la agitación térmica.

La energía de un dipolo será
\[
	U = -\vb{p}\cdot\vb{E}
\]
entonces
\[
	\vb{F} = -\Nabla U = \Nabla(\vb{p}\cdot\vb{E}) = (\vb{p}\cdot\Nabla)\vb{E} + (\vb{E}\cdot\Nabla)\vb{p}
	+ \vb{p}\times(\rotorm{E}) + \vb{E}\times(\Nabla\times\vb{p})
\]
siendo los últimos tres términos nulos según lo que consideramos previamente de manera que
\[
	\vb{F} = (\vb{p}\cdot\Nabla)\vb{E}.
\]

\subsection{Capa dipolar}

El potencial de un dipolo es
\[
	\phi(\vb{x}) = \frac{ \vb{p}\cdot(\vb{x}-\vb{x}_0) }{|\vb{x} - \vb{x}_0|^3} 
\]
y el potencial de una capa dipolar
\[
	\phi(\vb{x}) = \int_S \frac{ \vb{D}(\vb{x}')\cdot(\vb{x}-\vb{x}') }{|\vb{x} - \vb{x}'|^3} \: dS'
\]
siendo $\vb{D}$ el momento dipolar por área que viene de acuerdo a la definición
\[
	D = \lim_{\substack{\sigma\to\infty \\ \epsilon\to 0}} \: \sigma\epsilon
\]
refiérase a la ilustración bajo esta línea
\begin{figure}[htb]
	\begin{center}
	\includegraphics[width=0.2\textwidth]{images/fig_ft1_campo_dipolar3.pdf}	 
	\end{center}
	\caption{}
\end{figure}

Veamos algún detalle más sobre la capa dipolar, que está ilustrado en la Figura siguiente.
\[
	\frac{ \vb{D}\cdot(\vb{x}-\vb{x}')}{|\vb{x} - \vb{x}'|^3} dS = 
	\frac{ D \cdot(\vb{x}-\vb{x}')}{|\vb{x} - \vb{x}'|^3} d\vb{S} = 
	- \frac{ D \cos(\theta)}{|\vb{x} - \vb{x}'|^2} dS = 
	- \frac{ D \cos(\theta)}{r^2} dS  
\]

\begin{figure}[htb]
	\begin{center}
	\includegraphics[width=0.6\textwidth]{images/fig_ft1_campo_dipolar1.pdf}	 
	\end{center}
	\caption{}
\end{figure}

\[
	\frac{ \vb{D}\cdot(\vb{x}-\vb{x}')}{|\vb{x} - \vb{x}'|^3} dS = - D d\Omega
\]
puesto que
\[
	\phi(\vb{x}) = -D \int_S d\Omega \qquad \qquad \frac{\cos(\theta)}{r^2}dS \equiv d\Omega
\]

Para las condiciones de contorno se da lo siguiente
\[
	E_2^{\hat{n}} - E_1^{\hat{n}} = 4\pi\sigma
\]
\[
	- \Nabla (\phi_2 - \phi_1)\cdot \hat{n} = 4\pi\sigma
\]
\[
	\dpar{\phi_1 - \phi_2}{\hat{n}} = 4\pi\sigma
\]
\[
	\phi_1 - \phi_2 = 4\pi\sigma\epsilon
\]

\begin{figure}[htb]
	\begin{center}
	\includegraphics[width=0.4\textwidth]{images/fig_ft1_campo_dipolar2.pdf}	 
	\end{center}
	\caption{}
\end{figure}

desde donde deducimos que el potencial tiene un salto al surcar la capa dado por 
\[
	\phi_2 - \phi_1 = 4\pi D
\]

\subsection{Momento dipolar por unidad de volumen}

El potencial de un dipolo es
\[
	\phi(\vb{x}) = \frac{ \vb{p}\cdot(\vb{x}-\vb{x}') }{|\vb{x} - \vb{x}'|^3} 
\]
y el potencial de muchos de ellos sale de la integración
\[
	\phi(\vb{x}) = \int_V \frac{ \vb{P}(\vb{x}')\cdot(\vb{x}-\vb{x}') }{|\vb{x} - \vb{x}'|^3}  \: dV
\]
\begin{figure}[htb]
	\begin{center}
	\includegraphics[width=0.3\textwidth]{images/fig_ft1_dipolarvol.pdf}	 
	\end{center}
	\caption{}
\end{figure}
donde $\vb{P}$ es la llamada polarización, el momento dipolar por unidad de volumen, siendo $V$ un volumen
que incluye a la zona de polarización (ver Figura).
\[
	\phi(\vb{x}) = \int_V \vb{P}(\vb{x}')\cdot \Nabla' \left(\frac{1}{|\vb{x} - \vb{x}'|} \right)\: dV
\]
y si usamos el teorema de la divergencia para convertir una de las integrales resulta
\[
	\phi(\vb{x}) = \int_S \frac{\vb{P}(\vb{x}')}{ |\vb{x} - \vb{x}'|} \: dS
	- \int_V  \frac{  \Nabla' \cdot \vb{P}(\vb{x}') }{|\vb{x} - \vb{x}'|} \: dV 
\]
lo que habilita a pensar en como que 
\[
	\vb{P}\cdot\hat{n} \equiv \sigma_P
\]
está presente en el borde del cuerpo polarizado, y en su interior existe
\[
	- \Nabla\cdot\vb{P} \equiv \rho_P
\]
siempre que $\divem{P}\neq 0$ es decir que la polarización no sea homogénea.

% =================================================================================================
\section{El potencial vector}
% =================================================================================================

Haremos una especie de desarrollo multipolar del potencial vector \vb{A},
\[
	\vb{A}(\vb{x}) = \frac{1}{c} \int_V \frac{\vb{J}(\vb{x}')}{|\vb{x}-\vb{x}'|} \: dV' 
\]
pero como se puede escribir
\[
	\frac{1}{|\vb{x}-\vb{x}'|} \approx \frac{1}{|\vb{x}|}  + \frac{\vb{x}\cdot\vb{x}'}{|\vb{x}|^3} 
\]
en torno a $\vb{x}'=0$ será
\notamargen{Recordar que Biot \& Savart es para densidad de corriente estacionaria,
i.e. $\divem{J}=0$}
\[
	\vb{A}(\vb{x}) = \frac{1}{c} \int_V \frac{\vb{J}(\vb{x}')}{|\vb{x}|} \: dV' 
	+ \frac{1}{c} \frac{ \vb{x} \: }{|\vb{x}|^3} \cdot \int_V \vb{x}' \vb{J}(\vb{x}') \: dV' 
\]
\[
	\vb{A}(\vb{x}) = \frac{1}{c|\vb{x}|} \int_V \vb{J}(\vb{x}') \: dV' 
	+ \frac{1}{c} \frac{ \vb{x} \: }{|\vb{x}|^3} \cdot \int_V \vb{x}' \vb{J}(\vb{x}') \: dV' 
\]
y el primer término es nulo lo cual puede verse porque sale integrando con alguna identidad (?)
y usando que $\divem{J}=0$. Correspondería al orden monopolar y el hecho de que sea nulo refleja
la no existencia de monopolos.
\[
	\vb{A}(\vb{x}) = \left[ \left(\frac{1}{2c} \int_V \vb{x}'\times \vb{J} \: dV \right) \times \vb{x} \right] 
			\frac{1}{|\vb{x}|^3}
\]
y si definimos el paréntesis como \vb{m} (momento magnético) entonces
\[
	\vb{A}(\vb{x}) = \frac{\vb{m} \times \vb{x} }{|\vb{x}|^3}
\]
en el origen, y 
\[
	\vb{A}(\vb{x}) = \frac{\vb{m} \times (\vb{x}-\vb{x}') }{|\vb{x}-\vb{x}'|^3}
\]
en $\vb{x}'$, las cuales son expresiones a primer orden y que utilizan el gauge de Coulomb, $\divem{A}=0$.

De esta manera tendremos
\[
	\mathcal{M}(\vb{x}') = \frac{1}{2c} \left[ \vb{x}' \times \vb{J}(\vb{x}')\right]
\]
que es la magnetización o densidad de momento magnético, y entonces el momento magnético pasa a ser 
\[
	\vb{m} = \int_v \mathcal{M}(\vb{x}') \:dV'.
\]

Se puede trabajar con el potencial vector así
\[
	\rotorm{A}  = \Nabla \times \left( \frac{\vb{m} \times \vb{x} }{|\vb{x}|^3} \right) =
	\left( \frac{\vb{x}}{|\vb{x}|^3}\cdot\Nabla \right)\vb{m}-(\vb{m}\cdot\Nabla)\frac{\vb{x}}{|\vb{x}|^3}, 
\]
la cual luego de mucho álgebra vectorial se puede llevar a la forma
\[
	\vb{B} = \frac{3 (\vb{m}\cdot\hat{n})\hat{n} - \vb{m}}{|\vb{x}|^3},
\]
que nos dice que bien lejos cualquier distribución de corriente localizada presenta como \vb{B} el 
campo magnético de un dipolo magnético dado por \vb{m}(\vb{x}). Esta aproximación corresponde, por 
supuesto, al primer orden del desarrollo.

\subsection{interpretacion del momento magnético}

Se puede pensar \vb{m} como una espira.
\[
	dA = \frac{ x d\ell \sin(\alpha) }{2}
\]
siendo el área orientada
\[
	\vb{A} = \frac{1}{2} \int_\Gamma \vb{x}\times d\vb{\ell} 
\]
y entonces
\[
	\vb{m} = \frac{I}{c}\vb{A}
\]
\begin{figure}[htb]
	\begin{center}
	\includegraphics[width=0.5\textwidth]{images/fig_ft1_mmag.pdf}	 
	\end{center}
	\caption{}
\end{figure}

Desde volumen a espira hacemos la transformación del modo usual,
\[
	\vb{m} = \frac{1}{2c} \int_V  \vb{x} \times \vb{J}(\vb{x})  \:dV =
		\frac{1}{2c} \int_\Gamma  \vb{x} \times \:d\vb{\ell}
\]
usando que
\[
	\vb{J} \: dV = J d\vb{\ell} dS = \frac{I}{dS} d\vb{\ell} dS = I \: d\vb{\ell} 
\]

A modo de ejemplo, para una espira circular de radio $r$ es
\[
	m = \frac{i}{c} \pi r^2.
\]

\subsection{Interacción del campo magnético con una distribución de corriente}

Hacemos una expansión de Taylor del campo \vb{B} con $|\vb{x}|\ggg|\vb{x}'|$,
\[
	\vb{B} = \vb{B}_0 + (\vb{x}\cdot\Nabla)\vb{B}
\]
y entonces como la fuerza es
\[
	\vb{F} = \frac{1}{c} \int_V \vb{J}(\vb{x}') \times \vb{B}(\vb{x}') \: dV'
\]
\begin{figure}[htb]
	\begin{center}
	\includegraphics[width=0.5\textwidth]{images/fig_ft1_campocorr.pdf}	 
	\end{center}
	\caption{}
\end{figure}
resulta que
\[
	\vb{F} =  \frac{1}{c} \int_V \vb{J} \times \vb{B}_0 \: dV' +
		\frac{1}{c} \int_V \vb{J} \times (\vb{x}'\cdot\Nabla)\vb{B} \: dV'
\]
siendo el primer término nulo.
\[
	\vb{F} = \Nabla\times(\pv{B}{m}) = (\pe{m}{\Nabla})\vb{B} =
		(\pe{m}{\Nabla})\vb{B} = \Nabla(\pe{m}{B})
\]
Si el campo es homogéneo la fuerza es nula, pero como $\vb{F} = -\Nabla U$
\[
	F_m = \Nabla(\pe{m}{B}) \quad \Rightarrow \quad U_M = -\pe{m}{B}
\]
\[
	F_e = \Nabla(\pe{p}{E}) \quad \Rightarrow \quad U_e = -\pe{p}{E}
\]
siendo $U_{m,e}$ la energía de los dipolos en campos externos.


Mediante identidades vectoriales podemos llegar a una expresión
\[
	\vb{F} = - \Nabla \times \frac{1}{c} \int_V \vb{J}(\vb{x}'\cdot\vb{B}) dV' =
	-\Nabla \times \frac{1}{2c} (-\vb{B}) \times \int_V \vb{x} \times \vb{J} dV' =
\]
\[
	\vb{F} = \Nabla \times \vb{B} \times \frac{1}{2c}\int_V \vb{x} \times \vb{J} dV'
\]
\[
	\vb{F} = \Nabla \times (\pv{B}{m}) = \Nabla(\pe{m}{B})
\]
La fuerza de un campo $\vb{B}$ externo sobre una distribución de corrientes es el gradiente de cierta
energía
\[
	\vb{F} = \Nabla(\vb{m}\cdot\vb{B}) = (\vb{m}\cdot\Nabla)\vb{B}
\]
de donde se ve claramente que si \vb{B} es uniforme entonces la fuerza es nula.
\vb{m} es una constante que depende de la distribución de corrientes.

% =================================================================================================
\section{Pertubación por un conductor sobre un campo eléctrico uniforme}
% =================================================================================================

Se tiene un campo uniforme con $Q,R \to \infty$ pero con $2Q/R^2 = cte$, según se ve en la Figura.


El potencial $\phi$ de la esfera es constante por ser conductor.
Puedo definir 
\[
	\phi|_{esf} \equiv 0
\]
pues $\phi(\infty)\neq 0 $ porque hay densidad de carga $\rho$ en el infinito.

\begin{figure}[htb]
	\begin{center}
	\includegraphics[width=0.4\textwidth]{images/fig_ft1_perturbacion1.pdf}	 
	\end{center}
	\caption{}
\end{figure}
Para la carga superior,
\[
	\phi_1 = \frac{-Q}{|\vb{x} - R\hat{z}|} + \frac{a/R Q}{|\vb{x} - a^2/R\hat{z}|}
\]
mientras que para la inferior
\[
	\phi_2 = \frac{Q}{|\vb{x} + R\hat{z}|} + \frac{ a/R Q}{|\vb{x} + a^2/R\hat{z}|}
\]

Recordemos que
\[
	(1+\alpha)^{(-1/2)} \approx 1 - \frac{1}{2}\alpha \qquad \alpha \ll 1
\]
y podemos trabajar el denominador
\[
	|\vb{x} - R\hat{z}| = \sqrt{ x^2 + R^2 - 2Rx\cos(\theta)}
\]
\[
	\frac{1}{|\vb{x} - R\hat{z}|} = \frac{1}{\sqrt{ x^2 + R^2 - 2Rx\cos(\theta)}} =
	\frac{1}{R(1 + x^2/R^2 - 2x/R\cos(\theta))^(1/2)}
\]
\[
	\frac{1}{|\vb{x} - R\hat{z}|} \approx \frac{1}{R}\left(1+ \frac{x}{R} \cos(\theta) \right)
\]
de manera que luego
\begin{multline*}
	\phi(r) \approx Q\left[ \frac{1}{R}\left(1+ \frac{x}{R} \cos(\theta) \right) + 
	\frac{a}{Rx}\left(1+ \frac{a^2}{Rx} \cos(\theta) \right) + \right. \\
	\left. \frac{1}{R}\left(1 - \frac{x}{R} \cos(\theta) \right) -
	\frac{a}{Rx}\left(1 - \frac{a^2}{Rx} \cos(\theta) \right) \right]
\end{multline*}
\[
	\phi(x) \approx -\frac{2Qx}{R^2} \cos(\theta) + \frac{2a^3Q}{R^2x^2}\cos(\theta)
\]
y haciendo $x\equiv r$ y tomando el límite,
\[
	\phi(r) = -E_0 r \cos(\theta) + E_0\frac{a^3}{r^2} \cos(\theta)
\]
y la carga total sobre la esfera es nula puesto que estuvo aislada todo el tiempo.
Respecto de la Figura, si hacemos un Gauss en la zona indicada se obtiene $Q_n=0$,
entonces $\phi(r=a)=0$.

\begin{figure}[htb]
	\begin{center}
	\includegraphics[width=0.2\textwidth]{images/fig_ft1_perturbacion3.pdf}	 
	\end{center}
	\caption{}
\end{figure}

El segundo término es como un dipolo puntual,
\[
	E_0\frac{a^3}{r^2} \cos(\theta) = E_0\frac{a^3 \hat{z}\cdot\vb{r}}{r^3} 
\]
donde
\[
	\vb{p} \equiv E_0 a^3 \hat{z}
\]





% \bibliographystyle{CBFT-apa-good}	% (uses file "apa-good.bst")
% \bibliography{CBFT.Referencias} % La base de datos bibliográfica

\end{document}
