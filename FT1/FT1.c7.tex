	\documentclass[10pt,oneside]{CBFT_book}
	% Algunos paquetes
	\usepackage{amssymb}
	\usepackage{amsmath}
	\usepackage{graphicx}
	\usepackage{libertine}
	\usepackage[bold-style=TeX]{unicode-math}
	\usepackage{lipsum}

	\usepackage{natbib}
	\setcitestyle{square}

	\usepackage{polyglossia}
	\setdefaultlanguage{spanish}


	\usepackage{CBFT.estilo} % Cargo la hoja de estilo

	% Tipografías
	% \setromanfont[Mapping=tex-text]{Linux Libertine O}
	% \setsansfont[Mapping=tex-text]{DejaVu Sans}
	% \setmonofont[Mapping=tex-text]{DejaVu Sans Mono}

	%===================================================================
	%	DOCUMENTO PROPIAMENTE DICHO
	%===================================================================

\begin{document}

% =================================================================================================
\chapter{Relatividad especial}
% =================================================================================================

% =================================================================================================
\section{Transformación de vectores}
% =================================================================================================

Digamos que un vector transforma 
\[
	X'_{i} = a_{ij} X_j
\]
de manera que se verifique que las leyes físicas sean invariantes frente a rotaciones propias.

Einstein postula que:
\begin{itemize}
 \item Todos los sistemas inerciales son equivalentes.
 \item La velocidad de la luz en un sistema inercial es constante. No depende del estado de
 movimiento del observador.
\end{itemize}

Sea un sistema $S'$ que se mueve con velocidad \vb{v} de otro $S$ en forma paralela a un eje (ver figura).
\begin{figure}[htb]
	\begin{center}
	\includegraphics[width=0.4\textwidth]{images/fig_ft1_transfvec.pdf}	 
	\end{center}
	\caption{}
\end{figure} 

Se verifica entonces la transformación de Lorentz
\begin{align*}
	x^{1'} &= x^1  \\
	x^{2'} &= x^2  \\
	x^{3'} &= \gamma \: [ x^3 - \beta x^0]  \\
	x^{0'} &= \gamma \: [ x^0 - \beta x^3] 
\end{align*}
donde son 
\[
	\gamma = \frac{1}{(1 - v^2/c^2)^{1/2}} \qquad \qquad x^0 = ct 
\]

A la transformación [1] se le puede dar forma de rotación en funciones hiperbólicas como sigue
\[
	x^{0'} = x^0 \cosh( \eta ) - x^3 \sinh( \eta )
\]
\[
	x^{3'} = -x^0 \sinh( \eta ) + x^3 \cosh( \eta )
\]
donde seguimos viendo que las leyes son lineales en las coordenadas (el espacio es isótropo)
\notamargen{Debiéramos dar ideas de estas cosas importantes de relatividad especial}
\[
	\begin{pmatrix}
	 x^{0'} \\
	 x^{3'} \\
	\end{pmatrix}
	=
	\begin{pmatrix}
	\cosh( \eta ) & \sinh( \eta ) \\
	-\sinh( \eta ) & \cosh( \eta ) \\
	\end{pmatrix}
	\begin{pmatrix}
	 x^{0} \\
	 x^{3} \\
	\end{pmatrix}
\]
y no es otra cosa  que una rotación en eje $\hat{0}, \hat{3}$ con el ángulo $\eta = atanh( \beta )$. Notemos
que se verifica la invariancia del módulo de la transformación
\[
	(x^{0'})^2 -  ( (x^{1'})^2  + (x^{2'})^2 + (x^{3'})^2 ) =
		(x^{0})^2 -  ( (x^{1})^2  + (x^{2})^2 + (x^{3})^2 ) 
\]
o en una notación más feliz
\[
	(ct')^2 - ( x'^2 + y'^2 + z'^2 ) = (ct)^2 - ( x^2 + y^2 + z^2 )
\]
	
Este espacio 4D es el de Minkowski y no es euclídeo.
\[
	\begin{pmatrix}
	 x^{0'} \\
	 x^{1'} \\
	 x^{2'} \\
	 x^{3'} \\
	\end{pmatrix}
	=
	\begin{pmatrix}
	\gamma & 0 & 0 & -\beta\gamma \\
	0 & 1 & 0 & 0 \\
	0 & 0 & 1 & 0\\
	-\beta\gamma  & 0 & 0 & \gamma \\
	\end{pmatrix}
	\begin{pmatrix}
	 x^{0} \\
	 x^{1} \\
	 x^{2} \\
	 x^{3} \\
	\end{pmatrix}
\]

La transformación inversa se obtiene tomando los reemplazos
\[
	x^{i'} \to x^i \quad ,\quad  x^i \to x^{i'} \quad ,\quad  \beta \to -\beta
\]
El elemento invariante de línea es 
\[
	ds^2 = (dx^0)^2 - (dx^1)^2 - (dx^2)^2 - (dx^3)^2 = ds^{'2}
\]
o bien 
\[
	ds^2 = g_{\alpha\beta}dx^{\alpha}dx^{\beta}
\]
que es el tensor de la métrica. Se verifica
\[
	g_{\alpha\beta} = g^{\alpha\beta} =
	\begin{pmatrix}
	 1 & 0 & 0 & 0 \\
	 0 & -1 & 0 & 0 \\
	 0 & 0 & -1 & 0 \\
	 0 & 0 & 0 & -1 \\
	\end{pmatrix}
\]

\subsubsection{Cuadrivectores en el espacio 4D}

Un cuadrivector contravariante es
\[
	A^{\mu} = ( A^0, \vb{A})
\]
mientras que el covariante es
\[
	A_{\mu} = ( A^0, -\vb{A})
\]
y vemos que las partes temporales son las mismas cambiando el signo de la espacial. Las reglas de 
transformación son
\[
	A'^{\alpha}= \dpar{x'^{\alpha}}{x^{\beta}} A^{\beta} \qquad\qquad 
		A'_{\alpha}= \dpar{x^{\beta}}{x'^{\alpha}} A_{\beta}
\]
luego el producto interno es
\[
	\widetilde{A}\cdot\widetilde{B} \equiv A_\alpha B^\alpha
\]
donde estamos usando convención de suma de Einstein, que significa que 
\[
	\widetilde{A}\cdot\widetilde{B} = A^0 B^0 -\pe{A}{B}
\]
que es invariante por ser un escalar de Lorentz,
\[
	A_\alpha B^\alpha = A'_\alpha B'^\alpha
\]

\subsubsection{Intervalos entre eventos}

Los intervalos deben ser invariantes relativistas y de Lorentz, si el intervalo es temporal se tiene 
\[
	x^0 > x^i x_i \Rightarrow \delta s^2 > 0  
\]
y los eventos pueden estar conectados causalmente
\[
	x^0 < x^i x_i \Rightarrow \delta s^2 < 0 
\]
y los eventos no pueden estar conectados causalmente. Se cumple
\[
	\delta s^2 = (x^0)^2 - [ (x^1)^2 + (x^2)^2 + (x^3)^2 ]
\]

\subsubsection{Operadores diferenciales}

Tenemos la derivada respecto a una coordenada contravariante
\[
	\partial_\alpha \equiv \dpar{}{x^\alpha} = \left( \dpar{}{x^0}, \Nabla \right)
\]
que es la derivada covariante, y también la derivada respecto de una coordenada covariante
\[
	\partial^\alpha \equiv \dpar{}{x_\alpha} = \left( \dpar{}{x^0}, - \Nabla \right)
\]
que es la derivada contravariante. Note la asimetría entre derivo respecto de arriba y es derivada abajo
y viceversa. La notación abreviada puede inducir a confusiones.

La cuadridivergencia de un cuadrivector es un invariante,
\[
	\partial_\alpha A^\alpha = \dpar{A^0}{x^0} + \Nabla\cdot\vb{A}
\]
\[
	\partial^\alpha A_\alpha = \dpar{A^0}{x^0} - \Nabla\cdot(-\vb{A})
\]
y aquí vemos $\partial_\alpha A^\alpha = \partial^\alpha A_\alpha$. Esto nos lleva al D'Alembertiano
\[
	\Box \equiv \partial_\alpha \partial^\alpha = \dpar[2]{}{x^0} - \nabla^2
\]
S es el intervalo entre los eventos 1 y 2, y es un invariante lorentziano
\[
	s^2 = c^2( t_1 - t_2 )^2 - | \vb{x}_1 - \vb{x}_2 |^2
\]
El intervalo es temporal si $s^2 >0$ en cuyo caso se tiene 
\[
	c \delta t >  | \vb{x}_1 - \vb{x}_2 |
\]
lo cual significa que existe frame inercial donde $x_1=x_2$ los eventos ocurren en el mismo sitio de manera
que pueden estar conectados causalmente; puesto que $c\delta t > 0$ y $t_2>t_1$. Por el contrario si 
$c^2 < 0$ se tiene 
\[
	c \delta t <  | \vb{x}_1 - \vb{x}_2 |
\]
y existe entonces frame inercial donde los dos eventos son en el mismo sitio $x_1=x_2$ y entonces $c\delta t 
< 0$ y $t_2 < t_1$ de manera que no pueden estar conectados causalmente.

\begin{figure}[htb]
	\begin{center}
	\includegraphics[width=0.6\textwidth]{images/fig_ft1_intervalos.pdf}	 
	\end{center}
	\caption{}
\end{figure} 

Según se interpreta claramente del gráfico de la figura [ampliar].
\[
	x'^0 = \gamma (x^0 - \beta x^3) \qquad x'^3 = \gamma (x^3 - \beta x^0)
\]
y si ahora es $x'^0 = 0$ entonces para un observador en $S'$ se tiene 
\[
	0 = \gamma (x^0 - \beta x^3) 
\]
o bien $x^0 = \beta x^3$ y aquí es $x'^3 = 0$ de modo que 
\[
	\frac{x^3}{\beta} = x^0
\]
y entonces $a$ de la figura puede ser causado por un suceso en el origen pero $b$ no tiene 
conexión causal con el origen.

\subsection{Transcurso del tiempo en un sistema con V grande}

Sea $v/c$ no despreciable 
\[
	c \Delta t' = \gamma ( c\Delta t - \beta \Delta z) \qquad \qquad \gamma >1
\]
\[
	\Delta t' = \gamma \Delta t \left( 1 - \beta \frac{\Delta z}{c\Delta t} \right)
\]
pero si en $S'$ la partícula está en reposo es $v = dz/dt $ de manera que 
\begin{figure}[htb]
	\begin{center}
	\includegraphics[width=0.3\textwidth]{images/fig_ft1_vgrande.pdf}	 
	\end{center}
	\caption{}
\end{figure} 
\[
	\Delta t' = \gamma \Delta t ( 1 - \beta^2)
\]
\[
	\Delta t' = \Delta t ( 1 - \beta^2)^{1/2}
\]
de modo que $ \Delta t' < \Delta t$, en $S'$ el tiempo transcurre más lentamente.

\subsubsection{Número de onda y conteo}

Un proceso de conteo (discreto) es invariante lorentziano
\[
	x'^3 = \gamma ( x^3 - \beta x^0 )
\]
siendo \vb{v} entre sistemas $SS'$.
El número de crestas es 
\[
	\#_s = \frac{ z_1 - z }{ \lambda } = \frac{ k }{ 2\pi }( z_1 - z ) = \frac{ k }{ 2\pi }( ct - z ) = 
	\frac{ 1 }{ 2\pi }( \omega t - kz )
\]
\[
	\#_s' = \frac{ 1 }{ 2\pi }( \omega' t' - k'z' )
\]
y se puede generalizar
\[
	\pe{k'}{x'} - \omega' t' = \pe{k}{x} - \omega t
\]
\[
	-\left( \pe{k'}{x'} - \frac{\omega' x'^0 }{c} \right) = -\left( \pe{k}{x} - \frac{\omega x^0 }{c}  
\right)
\]
es un invariante lorentziano como
\[
	k_\alpha x^\alpha = k^\alpha x_\alpha
\]
donde el cuadrivector de onda se define
\[
	k^\alpha = \left( \frac{\omega}{c}, \vb{k}\right).
\]

% =================================================================================================
\section{Forma covariante del electromagnetismo}
% =================================================================================================

Partimos de la ecuación de continuidad para la carga,
\[
	\dpar{\rho}{t} + \divem{J} = 0
\]
la cual con la definición del cuadrivector corriente
\[
	J^\mu = ( c\rho , \vb{J} )
\]
se puede escribir como 
\[
	\partial_\mu J^\mu = \dpar{c\rho}{ct} + \divem{J} = 0 .
\]

La formulación covariante empleaba el gauge de Lorentz (así las ecuaciones son validas en cualquier sistema
inercial), el gauge de Lorentz era
\[
	\frac{1}{c} \dpar{\phi}{t} + \divem{A} = 0
\]
siendo el cuadripotencial
\[
	A^\mu = ( \phi , \vb{A} ) 
\]
y entonces 
\[
	\partial_\mu A^\mu = \dpar{\phi}{ct} + \divem{A} = \frac{1}{c} \dpar{\phi}{t} + \divem{A} = 0 .
\]

Se podía ver que resultan ecuaciones de onda inhomogéneas para los potenciales
\[
	\Nabla^2 \vb{A} - \frac{1}{c^2} \dpar[2]{\vb{A}}{t} = -\frac{4\pi}{c} \vb{J}
\]
que viene a ser 
\[
	\partial_\mu\partial^\mu = \Box \vb{A} = \frac{4\pi}{c} \vb{J}
\]
y para el potencial $\phi$
\[
	\Nabla^2 \phi - \frac{1}{c^2} \dpar[2]{\phi}{t} = - 4\pi \phi
\]
que desemboca en 
\[
	\partial_\mu\partial^\mu = \Box \phi = \frac{4\pi}{c} ( c\rho )
\]

Al aplicar el D'Alembertiano a un cuadrivector obtenemos otro cuadrivector 
\[
	\Box A^\mu = \frac{4\pi}{c} J^\mu.
\]

Los campos \vb{E}, \vb{B} forman parte de un tensor de segundo rango antisimétrico llamado tensor
de intesidad de campo 
\[
	F^{\alpha\beta} = \partial^\alpha A^\beta - \partial^\beta A^\alpha
\]
que matricialmente se puede ver como 
\[
	F^{\alpha\beta} =
	\begin{pmatrix}
	 0 & -E_x & -E_y & -E_z \\
	 E_x & 0 & -B_z & B_y \\
	 E_y & B_z & 0 & -B_x \\
	 E_z & -B_y & B_x & 0 \\
	\end{pmatrix}
\]
También se suele definir un tensor de intensidad de campo dual
\[
	\mathcal{F}^{\alpha\beta} =  \frac{1}{2} \varepsilon^{\alpha\beta\gamma\delta} F_{\gamma\delta}
\]
que no es otra cosa que 
\[
	\mathcal{F}^{\alpha\beta}=
	\begin{pmatrix}
	 0 & -B_x & -B_y & -B_z \\
	 B_x & 0 & E_z & -E_y \\
	 B_y & -E_z & 0 & E_x \\
	 B_z & E_y & -E_x & 0 \\
	\end{pmatrix}
\]
y donde $\varepsilon^{\alpha\beta\gamma\delta}$ es el tensor de Levi-Civita de cuatro dimensiones, que es nulo
cuando se repite un índice.
Entonces las ecuaciones de Maxwell en forma covariante explícita resultan 
\[
	\partial_\alpha \mathcal{F}^{\alpha\beta} =  0 \qquad \qquad 
	\partial_\alpha F^{\alpha\beta} =  \frac{4 \pi}{c} J^\alpha.
\]

\subsection{Transformación de los campos}

L transformación de Lorentz era 
\begin{align*}
	ct' &= \gamma \: [ ct - \pe{\beta}{x} ] \\
	\mathbf{x'}_\parallel &= \gamma \: [ \mathbf{x}_\parallel - {\beta}ct ] \\
	\mathbf{x'}_\perp &= \mathbf{x}_\perp
\end{align*}
con $\vb{\beta} = \vb{v}/c$ y donde la transformación de los campos \vb{E}, \vb{B}

\begin{figure}[htb]
	\begin{center}
	\includegraphics[width=0.4\textwidth]{images/fig_ft1_transfCampo1.pdf}	 
	\end{center}
	\caption{}
\end{figure} 

\[
	\vb{E}' = \vb{E}_\parallel + \gamma \: ( \vb{E}_\perp  + \pv{\beta}{B} )
\]
\[
	\vb{B}' = \vb{B}_\parallel + \gamma \: ( \vb{B}_\perp  - \pv{\beta}{E} )
\]
que se pueden poner como 
\[
	\vb{E}' = - \frac{ \gamma^2 }{ \gamma + 1 }\vb{\beta} (\pe{\beta}{E}) +
		\gamma \: ( \vb{E}  + \pv{\beta}{B} )
\]
\[
	\vb{B}' = - \frac{ \gamma^2 }{ \gamma + 1 }\vb{\beta} (\pe{\beta}{B}) + 
		\gamma \: ( \vb{B}  - \pv{\beta}{E} )
\]
y recordemos que la transformación de Galileo era
\[
	\vb{E}' = \vb{E} + \frac{1}{c} \pv{V}{B} \qquad \qquad 
	\vb{B}' = \vb{B} - \frac{1}{c} \pv{V}{E}
\]
siendo el segundo término el que da origen a las corrientes de Foucault al mover un conductor en el seno
de un campo \vb{B}.

\begin{figure}[htb]
	\begin{center}
	\includegraphics[width=0.4\textwidth]{images/fig_ft1_transfCampo2.pdf}	 
	\end{center}
	\caption{}
\end{figure} 

Según la figura superior la transformación de los campos satisface 
\begin{align*}
	E'_x = \gamma ( E_x - \beta B_y ) \qquad B'_x = \gamma ( B_x + \beta E_y ) \\
	E'_y = \gamma ( E_y + \beta B_x ) \qquad B'_y = \gamma ( B_y - \beta E_x ) \\
	E'_z = E_z \qquad B'_z = B_z 
\end{align*}

Las contracciones del producto escalar entre el tensor de intensidad son invariantes. Así, por ejemplo,
\begin{align*}
	F^{\alpha\beta}F_{\alpha\beta} &= 2( B^2 - E^2 ) \\
	\mathcal{F}^{\alpha\beta}\mathcal{F}_{\alpha\beta} &= 2( E^2 - B^2 ) \\
	\mathcal{F}^{\alpha\beta}F_{\alpha\beta} &= -4 \: \pe{B}{E}
\end{align*}

Sea 
\[
	\mathcal{F}^{\alpha\beta}F_{\alpha\beta} = -4 \: \pe{B}{E} = 0,
\]
entonces $\vb{E} \perp \vb{B}$ o alguno de los campos es nulo en todo sistema inercial. Para una carga 
que se mueve con velocidad \vb{v} se tiene $\vb{B}=0$ en un sistema en el que $q$ está en reposo de manera
que 
\[
	\pe{B}{E} = \pe{B'}{E'} = 0
\]
siempre y entonces $\vb{E'} \perp \vb{B'}$ para cualquier sistema inercial S'.

Un sistema electromagnético dependiente del tiempo intercambiará \vb{p} con el campo entonces no vale el
principio de acción y reacción ,
\[
	\dtot{\vb{P}_M}{t} + \dtot{\vb{P}_c}{t} = \int_{S(v)} \overline{T}\cdot d\vb{S}
\]
mientras que 
\[
	\dtot{\vb{P}_c}{t} = \dtot{}{t} \left( \frac{1}{4\pi c} \int \pv{E}{B} dV \right)
\]

\subsection{Covarianza con medios materiales}

En presencia de medios materiales puede definirse
\[
	G^{\alpha\beta} =
	\begin{pmatrix}
	 0 & -D_x & -D_y & -D_z \\
	 D_x & 0 & -H_z & H_y \\
	 D_y & H_z & 0 & -H_x \\
	 D_z & -H_y & H_x & 0 \\
	\end{pmatrix}
\]
y 
\[
	F^{\alpha\beta} \to G^{\alpha\beta}, \quad E_i \to D_i, \quad B_i \to H_i
\]
si las relaciones constitutivas son 
\[
	\vb{D} = \vb{E} + 4\pi\vb{P} \qquad\qquad \vb{H} = \vb{B} - 4\pi\vb{M}
\]
desde 
\[
	G^{\alpha\beta} = F^{\alpha\beta} + R^{\alpha\beta}
\]
y con 
\[
	\partial_\alpha G^{\alpha\beta} = \frac{4\pi}{c} J^\beta
\]
donde la información de $P_i$ y $M_i$ está en el tensor $R^{\alpha\beta}$.
Recordemos que los campos transforman según 
\[
	\vb{P}' = \vb{P}_\parallel + \gamma \: ( \vb{P}_\perp  - \pv{\beta}{M} )
\]
\[
	\vb{M}' = \vb{M}_\parallel + \gamma \: ( \vb{M}_\perp  + \pv{\beta}{P} )
\]

Entonces de un sistema inercial a otro una \vb{P} da origen a una \vb{M} y viceversa.

% =================================================================================================
\section{Principio de Hamilton y relatividad}
% =================================================================================================

Habiéndonos situado en un espacio de Minkowski, tenemos la acción
\[
	S = -\alpha \int_a^b ds,
\]
siendo $\alpha$ una constante a fijar luego, y $ds$ un arco en el espacio minkowskiano. La acción debe ser 
un invariante pues es un extremo.
\[
	ds = \sqrt{ c^2 dt^2 - dx^2 - dy^2 - dz^2 } = c dt \sqrt{ 1 - v^2/c^2 }
\]
de manera que 
\[
	S = -\alpha \int_{t_1}^{t_2} c dt  \sqrt{ 1 - v^2/c^2 } = \int_{t_1}^{t_2} \mathcal{L} dt
\]
y donde $\mathcal{L}$ es el lagrangiano, 
\[
	\mathcal{L} = -\alpha c \left( 1 - v^2/c^2 \right)^{1/2} \approx -\alpha c + \frac{\alpha v^2}{2c}
\]
y luego 
\[
	\mathcal{L} \to T = \frac{m v^2}{2} \; \text{(baja velocidad)}
\]
de manera que fijamos el valor de la constante a partir de este límite de baja velocidades,
\[
	\mathcal{L} = -m c^2 \left( 1 - v^2/c^2 \right)^{1/2}
\]
es el lagrangiano relativista.

A partir de las ecuaciones de Euler-Lagrange es 
\[
	p_i = \dpar{\mathcal{L}}{\dot{q}_i} = \dpar{\mathcal{L}}{v_i}
\]
y haciendo el álgebra,
\[
	p_i = \frac{ m v }{\sqrt{ 1 - v^2/c^2 }}
\]
que es el momento relativista. Entonces
\[
	\dtot{\vb{P}}{t} = m \dtot{}{t}\left( \frac{\vb{v}}{\sqrt{1 - v^2/c^2 }}\right).
\]

Para un movimiento circular, el módulo de la velocidad permanece constante.
\[
	\dtot{|\vb{v}|}{t} = 0 \quad \Rightarrow \quad \dtot{\vb{P}}{t} =
		\left( \frac{ m }{\sqrt{1 - v^2/c^2 }}\right) \dtot{\vb{v}}{t} =
		m \: \gamma \: \dtot{\vb{v}}{t}
\]
si en cambio es $ \dtot{|\vb{v}|}{t} \neq 0 $ se tiene 
\[
	\dtot{\vb{P}}{t} = m \left( \left( \frac{ 1 }{\sqrt{1 - v^2/c^2 }}\right) \dtot{\vb{v}}{t} 
	+ \vb{V} (1 - v^2/c^2)^{-3/2} \frac{v}{c^2} \dtot{v}{t} \right)
\]
\[
	\dtot{\vb{P}}{t} = m \gamma \dtot{\vb{v}}{t} + m \vb{v} \gamma^3 \frac{v}{c^2} \dtot{v}{t}
\]
donde el primer término en el RHS está asociado a la variación en la dirección y el segundo a la variación
en la magnitud (hemos usado con $ \gamma^3 v^2/c^2 > \gamma $ ?). De esto se desprende que la inercia es
mayor para variar la longitud de \vb{v} que su dirección. Es más fácil cambiar dirección que rapidez.

Entonces
\[
	E = \pe{p}{v} - \Lag = m \gamma v^2 + m c^2 \gamma^{-1} = m \gamma c^2
\]
y esta es la energía relativista de una partícula libre. Veamos el límite de bajas velocidades, es decir
que si $v/c \ll 1$ entonces 
\[
	\gamma = \sqrt{ 1 - v^2/c^2 } \approx 1 + \frac{v^2}{2c^2},
\]
y resulta 
\[
	E \approx  m c^2 + \frac{m v^2}{2} = E_0 + \frac{m v^2}{2}
\]
donde $E_0$ es una energía en reposo, que no depende de \vb{v} y podemos expresar la energía cinética como 
\[
	E - m c^2 = \frac{m v^2}{2} = T.
\]
Si es 
\[
	\vb{p} = m \vb{w},
\]
con $\vb{w} = \gamma \vb{v}$ entonces 
\[
	E^2 = m^2 \gamma^2 c^4 \qquad p^2 = m^2 \gamma^2 v^2 
\]
y
\[
	\frac{E^2}{c^2} = m^2 c^2 \gamma^2
\]
\[
	\frac{E^2}{c^2} - p^2 = m^2 \gamma^2 (c^2 -v^2) = m^2 c^2
\]
y esta es la relación fundamental entre energía y momento 
\[
	\frac{E^2}{c^2} = p^2 + m^2c^2.
\]

Para partículas con $m_0 = 0$ y $v=c$ será 
\[
	\frac{E^2}{c^2} = p^2 \qquad \qquad p = \frac{h\nu}{c} = k\hbar.
\]

La formulación hamiltoniana comenzará a partir de 
\[
	\Ham = \sqrt{ p^2 + m^2 c^2} \: c,
\]
sobre el que se puede operar para obtener el límite clásico (de bajas velocidades) como 
\[
	\Ham = \left( 1 + \frac{ p^2 }{ m^2 c^2 } \right)^{1/2} m c^2
\]
y si se cumple $ p/(mc) \ll 1$ entonces 
\[
	\Ham \approx mc^2 + \frac{ p^2 }{2m^2}
\]
donde el último término  es el hamiltoniano de la mecánica clásica para nuestra partícula
libre.

El cuadrimomento se define como 
\[
	p^\mu = ( m\Gamma c , m \Gamma \vb {u}), \qquad \Gamma \equiv \frac{ 1 }{ \sqrt{ 1 - v^2/c^2 } } 
\]
o bien 
\[
	p^\mu = ( E/c , \vb{p} )
\]
siendo 
\[
	p^\mu p_\mu = \frac{E^2}{c^2} - p^2 = m^2 c^2
\]
el invariante asociado a la conservación (del cuadrimomento).

\subsection{Partícula en un campo electromagnético}

Dado que es de la mecánica clásica $\Lag = T - V$ la acción correspondiente la podemos expresar  como 
\[
	S = S_0 + S_inter = \int_{t_1}^{t_2} T dt - \int_{t_1}^{t_2} V dt
\]
es decir la suma de una parte libre y una de interacción. Luego 
\[
	S_{inter}^{NR} = \int_{t_1}^{t_2} -e \phi dt =  -\int_{t_1}^{t_2} \frac{ e \phi }{c} d(ct) = 
		-\int_{x_1}^{x_2} \frac{ e A^0 }{c} dx^0
\]
si usamos los cuadrivectores 
\[
	A^\mu = ( \phi, \vb{A} ) \qquad x^\mu = ( ct, \vb{x} ) 
\]
y generalizamos 
\[
	S_{inter} = - \frac{e}{c} \int_{x_1}^{x_2} A_\mu dx^\mu
\]
tendremos 
\[
	S_inter = \frac{e}{c} \int_{x_1}^{x_2} \left( \vb{A}\cdot d\vb{x} - c \phi dt \right) = 
		\frac{e}{c} \int_{x_1}^{x_2} \left( \vb{A}\cdot\vb{v} - c \phi \right) dt
\]
y finalmente el lagrangiano de una partícula en un campo electromagnético es 
\[
	\Lag = - m c^2 \sqrt{ 1 - v^2/c^2 } + \frac{e}{c} \pe{A}{v} - e \phi
\]
donde el primer término es el lagrangiano de partícula libre y la interacción viene luego. Esta lagrangiano 
no es invariante de medida; sin embargo no perjudica porque en las ecuaciones de movimiento sólo entran las
derivadas del mismo. Recordemos además que $\Lag$ no es invariante relativista pero la acción $S$ sí lo es.

Para construir el hamiltoniano necesitamos el momento conjugado,
\[
	\vb{P} = \dpar{\Lag}{\vb{v}} = \vb{p} + \frac{e}{c}\vb{A} = m\gamma \vb{v} + \frac{e}{c}\vb{A}
\]
y siguiendo la prescripción usual $\Ham = \dpar{\Lag}{\vb{v}}\vb{v} -\Lag $,
\begin{multline*}
	H = ( m\gamma \vb{v} + \frac{e}{c}\vb{A} )\vb{v} + mc^2(1-v^2/c^2)^{1/2} -
		\frac{e}{c}\pe{A}{v} + e\phi = \\
		m\gamma v^2 + e\phi + mc^2(1 - v^2/c^2)^{1/2}
\end{multline*}
y 
\[
	H =  m \gamma v^2 + e\phi + \frac{m c^2}{\gamma}  
\]
de manera que el hamiltoniano en un campo es 
\[
	H = m \gamma c^2 + e \phi
\]
\[
	\vb{P} = m \gamma v + \frac{e}{c} \vb{A} \qquad \qquad H = m \gamma c^2 + e \vb{\phi}
\]
y
\[
	\left( \vb{p} - \frac{e}{c}\vb{A} \right)^2 = m^2 \gamma^2 v^2 \qquad 
	\left( \frac{H}{c} - \frac{e}{c}\phi \right)^2 = m^2 \gamma^2 c^2
\]
\[
	\left( \frac{H}{c} - \frac{e}{c}\phi \right)^2 - \left( \vb{p} - \frac{e}{c}\vb{A} \right)^2 =
	m^2 \gamma^2 ( c^2 - v^2 ) = mc^2,
\]
con ustedes el invariante. Entonces el cuadrimomento de una partícula en un campo electromagnético,
sometida a un potencial electromagnético es 
\[
	p^\mu = \left( \frac{H-e\phi}{c}, \vb{p} - \frac{e}{c}\vb{A} \right)
\]
que es un caso particular del xxxx.

Para el caso de H es 
\[
	H = c \sqrt{ m^2 c^2 + ( \vb{p} - \frac{e}{c}\vb{A} )} + e \phi
\]
y el no relativista
\[
	H^{nr} = m c^2 ( 1 + \frac{1}{m^2 c^2}(\vb{p} - \frac{e}{c}\vb{A} )^2 )^{1/2} + e \phi
\]
usando la aproximación de baja velocidad,
\[
	H^{nr} \approx m c^2 + \frac{1}{2m} ( \vb{p} - \frac{e}{c}\vb{A} )^2 + e \phi
\]
donde tiro el término de reposo $mc^2$ y
\[
	H^{nr} \approx \frac{1}{2m} ( \vb{p} - \frac{e}{c}\vb{A} )^2 + e \phi
\]

Aplicando las ecuaciones de Euler-Lagrange al lagrangiano electromagnético hallado se llega a
\[
	\dtot{\vb{P}}{t} = \dtot{}{t}(m\gamma\vb{v}) = e \left(  \vb{E} + \frac{1}{c}\pv{v}{B} \right)
\]
qu es la fuerza de Lorentz con la corrección relativista. Es la misma expresión hallada otrora pero
sin tener en cuanta la relatividad.

Si $\vb{E}=0$ entonces 
\[
	\dtot{\vb{P}}{t} = m \gamma \dtot{\vb{v}}{t} \quad \text{pues} \quad \dtot{v}{t} = 0 
\]
y el campo \vb{B} sólo variará la dirección de \vb{v}, no su módulo.
El radio de giro de una partícula ciclotrón es mayor con la aproximación relativista que con la newtoniana
porque su inercia es mayor $\gamma > 1$. Planteamos
\[
	|\vb{F}| = e v B
\]
que desde el punto de vista relativista significa
\[
	e v B =  m \gamma \dtot{\vb{v}}{t}
\]
mientras que clásicamente 
\[
	m \frac{v^2}{r} = e v B 
\]
y sale el radio de giro desde acá
\[
	r_B = \frac{m\gamma v}{eB} \qquad\qquad r_B^{nr} = \frac{m v}{eB}
\]
y luego $ r_B > r_B^{nr}$.

\subsection{Cambio de gauge}

El cambio de gauge es una transformación 
\[
	A'^\mu = A^\mu - \partial^\mu f
\]
entonces 
\[
	A'0 = \phi - \partial^0 f \qquad \qquad \vb{A}' = \vb{A} + \Nabla f
\]
El cambio de gauge no es invariante pero $\delta S = 0$ sí es invariante.
La cuadridensidad de fuerza de Lorentz 
\[
	f^\beta = - \partial_\alpha T^{\alpha\beta}.
\]

\subsection{Especie de tiro oblicuo}

La situación física es la depicted en la figura bajo estas líneas
\[
	\dtot{\vb{P}}{t} = e\vb{E} = \frac{}{t}( m \gamma \vb{v} )
\]
que lleva a un sistema hartocomplicado de resolver que es 

\begin{align*}
 \dtot{P_x}{t} &= m \dtot{}{t}\left( \frac{v_x}{ \sqrt{ 1 - (v_x^2 + v_y^2)/c^2 } }\right) = eE \\
 \dtot{P_y}{t} &= m \dtot{}{t}\left( \frac{v_y}{ \sqrt{ 1 - (v_x^2 + v_y^2)/c^2 } }\right) =0 \\
\end{align*}


\begin{figure}[htb]
	\begin{center}
	\includegraphics[width=0.4\textwidth]{images/fig_ft1_tirooblicuo.pdf}	 
	\end{center}
	\caption{}
\end{figure} 

Cualitativamente vemos que $v_x$ crece a medida que ingresa en la zona de campo \vb{E} entonces como $v_y$ es
constante se tiene que $\gamma$ aumenta y aumenta la inercia de modo que disminuye $|\vb{v}|$ y describe
aproximadamente una parábola.

\subsection{cuadrivelocidad}

\vb{u} no transforma como cuadrivector (¿que u?), pero lo que sí transforma así es
\[
	W^\mu = ( \Gamma c , \Gamma \vb{u} ) 
\]
donde $ \Gamma \equiv 1/( 1 - u^2/c^2)^{1/2}$. Luego tenemos la fórmula de Einstein de suma de velocidades,
que tiene como límite a $c$,

\begin{figure}[htb]
	\begin{center}
	\includegraphics[width=0.4\textwidth]{images/fig_ft1_4vel.pdf}	 
	\end{center}
	\caption{}
\end{figure} 

\[
	u_\parallel = \frac{ u'_\parallel + v}{ 1 + \frac{\pe{v}{u'}}{c^2} } \qquad \qquad 
	u_\perp = \frac{ u'_\perp}{\gamma\left(1 + \frac{\pe{v}{u'}}{c^2}\right)}
\]
De esta manera el cuadrimomento es 
\[
	p^\mu = ( m\Gamma c, m \Gamma \vb{u}) \qquad \Rightarrow \qquad m W^\mu = p^\mu.
\]

% \bibliographystyle{CBFT-apa-good}	% (uses file "apa-good.bst")
% \bibliography{CBFT.Referencias} % La base de datos bibliográfica

\end{document}
