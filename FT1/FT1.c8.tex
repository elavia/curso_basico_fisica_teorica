	\documentclass[10pt,oneside]{CBFT_book}
	% Algunos paquetes
	\usepackage{amssymb}
	\usepackage{amsmath}
	\usepackage{graphicx}
% 	\usepackage{libertine}
% 	\usepackage[bold-style=TeX]{unicode-math}
	\usepackage{lipsum}

	\usepackage{natbib}
	\setcitestyle{square}

	\usepackage{polyglossia}
	\setdefaultlanguage{spanish}


	\usepackage{CBFT.estilo} % Cargo la hoja de estilo

	% Tipografías
	% \setromanfont[Mapping=tex-text]{Linux Libertine O}
	% \setsansfont[Mapping=tex-text]{DejaVu Sans}
	% \setmonofont[Mapping=tex-text]{DejaVu Sans Mono}

	%===================================================================
	%	DOCUMENTO PROPIAMENTE DICHO
	%===================================================================

\begin{document}

% =================================================================================================
\chapter{Radiación}
% =================================================================================================


% =================================================================================================
\section{Potenciales retardados}
% =================================================================================================

Veremos el campo debido a cargas en movimiento.
Usando el gauge de Lorentz y las ecuaciones de Maxwell se había llegado al 
set de ecuaciones
\[
	\lapm{\vb{A}} - \frac{1}{c^2} \dpar[2]{\vb{A}}{t} = -\frac{4\pi}{c} \vb{J}
\]
\[
	\lapm{\phi} - \frac{1}{c^2} \dpar[2]{\phi}{t} = - 4 \pi \phi
\]
para las cuales buscaremos ahora soluciones.
Estas soluciones serán los {\it potenciales retardados}, que son potenciales que pertenecen al gauge
de Lorentz.

Las ecuaciones anteriores son ecuaciones diferenciales hiperbólicas que incluyen el tiempo, 
con forma general 
\be
	\lapm{\psi} - \frac{1}{c^2} \dpar[2]{\psi}{t} = - 4 \pi f(\vb{x},t)
	\label{onda_general}
\ee
siendo $f$ la que da la distribución de fuentes.

Resolveremos \eqref{onda_general} con una función de Green. Hacemos Fourier respecto a la frecuencia, de
manera que podamos remover el tiempo (además luego nos interesarán fuentes armónicas y por sobre todo
cualquier perturbación puede descomponerse en Fourier).

Suponemos que podemos escribir
\[
	\psi(\vb{x},t) = \frac{1}{2\pi}\int_{-\infty}^{+\infty} \psi(\vb{x},\omega) \euler^{-i\omega t} 
	d\omega
\]
\[
	f(\vb{x},t) = \frac{1}{2\pi}\int_{-\infty}^{+\infty} f(\vb{x},\omega) \euler^{-i\omega t} d\omega
\]
siendo sus inversas
\[
	\psi(\vb{x},\omega) = \int_{-\infty}^{+\infty} \psi(\vb{x}, t) \euler^{i\omega t} dt
\]
\[
	f(\vb{x},\omega) = \int_{-\infty}^{+\infty} f(\vb{x},t) \euler^{i\omega t} dt
\]
luego la ecuación resulta 
\[
	\int_{-\infty}^{+\infty} \lapm{\psi}(\vb{x},\omega) \euler^{ -i\omega t} d\omega +
	\int_{-\infty}^{+\infty} \frac{\omega^2}{c^2}\psi(\vb{x},\omega) \euler^{ -i\omega t} d\omega = 
		- 4 \pi \int_{-\infty}^{+\infty} f(\vb{x},\omega) \euler^{ -i\omega t} d\omega
\]
de manera que se satisface la ecuación de Helmholtz inhomogénea,
\[
	(\nabla^2 + k^2)\psi(\vb{x},\omega) = - 4\pi f(\vb{x},\omega),
\]
para cada valor de frecuencia $\omega$.
Recordemos que los campos se producen en el vacío y se propagan en un medio no dispersivo.

Una función de Green satisfacerá 
\[
	(\nabla^2 + k^2) G(\vb{x},\vb{x}') = - 4\pi \delta(\vb{x}-\vb{x}'),
\]
donde $\vb{x}-\vb{x}' = \vb{R}$ y la función de Green será simétricamente esférica pues pedimos la
no existencia de contornos, entonces llamando a aquélla $G_k(R)$ se tiene 
\be
	\frac{1}{R}\dtot[2]{}{R}(RG_k) + k^2 G_k = - 4 \pi \delta (\vb{R})
	\label{helmholtz_green}
\ee
donde hemos usado el laplaciano en esféricas. Debemos distinguir dos casos, si $R=0$ entonces la
anterior resulta 
\[
	\lim_{kR \to 0} G_k(R) = \frac{1}{R}
\]
es decir que \eqref{helmholtz_green} tiende a la ecuación de Poisson, 
mientras que de ser cierto $R\neq 0$ en cambio
\[
	\dtot[2]{}{R}(RG_k) + k^2 (RG_k) = 0
\]
y entonces se propone como solución general 
\[
	G_k(R) = \frac{ A }{ R } \euler^{ i k R } + \frac{ B }{ R } \euler^{ -i k R }
\]
donde $A, B$ dependerán de las condiciones de contorno y siendo que el primer término del RHS representa
una onda divergente esférica y el segundo una onda convergente esférica.

Juntado todo se tiene
\[
	\begin{cases}
	G_k(R) = A G_k^+(R) +  B G_k^+(R) \\
	\\
	\displaystyle G_k^{\pm}(R) = \frac{\euler^{ \pm i k R}}{R}  \\
	\\
	A + B = 1
	\end{cases}
\]
Restaurando el tiempo se tiene
\[
	G_k(R)\euler^{- i \omega t } = 
	A \: \frac{\euler^{ i k R - i \omega t }}{R} + 
	B \: \frac{\euler^{ - ( i k R - i \omega t )}}{R}
\]

Para el problema físico habrá que analizar las condiciones de contorno para ver cuál corresponde
a la situación particular.

Se puede interpretar $G_k$ como el potencial de una carga unitaria que aparece en $\vb{x}=\vb{x}'$ en el
instante $t=t'$ y luego desaparece (mmm, qué misterio!).

Ahora necesitamos meter la dependencia temporal,
\[
	\left(\nabla^2_x - \frac{1}{c^2}\dpar[2]{}{t} \right) G^{\pm}(\vb{x},\vb{x}',t,t') = 
	- 4 \pi \delta(\vb{x}-\vb{x}') \delta(t-t')
\]
\[
	- 4 \pi f(\vb{x},\omega) = -4 \pi \int_{-\infty}^{+\infty} f(\vb{x},t) \euler^{i\omega t} dt =
		-4 \pi \int_{-\infty}^{+\infty} \delta(\vb{x}-\vb{x}') \delta(t-t') \euler^{i\omega t} dt
\]
\[
	- 4 \pi f(\vb{x},\omega) = -4 \pi \delta(\vb{x}-\vb{x}') \euler^{i\omega t'} 
\]
de modo que tenemos 
\[
	f(\vb{x},\omega) = \delta(\vb{x}-\vb{x}') \euler^{i\omega t'} ,
\]
usando lo cual se llega a
\[
	G^{\pm}(R,\tau) = \frac{1}{2\pi} \int_{-\infty}^{+\infty} G_k(R,t') \euler^{-i\omega t} d\omega
\]
donde $\tau$ es el tiempo relativo entre los tiempos de observación y fuente $(t')$ y $R$ es la distancia 
relativa entre observación y fuente.
\notamargen{Habría que optar por una notación más consistente, porque la que está es un poco confusa.
Evidentemente entre la carpeta y las notas de final cambié o ahorré notación.}

Entonces
\[
	[\nabla^2 + k^2] G_k(R,t') = - 4 \pi \delta(\vbx - \vbx') \euler^{i \omega t'}
\]
donde la función de Green ahora está en función de un tiempo particular $t'$. No es más que
\[
	G_k(R,t)^{\pm} = \frac{\euler^{\pm i k R} }{ R }\euler^{i \omega t'}
\]
y consecuentemente,
\[
	G^{\pm}(R,\tau) = \frac{1}{2\pi} \int_{-\infty}^{+\infty} 
	 \frac{\euler^{\pm i k R} }{ R }\euler^{i \omega (t'-t)} \: d\omega
\]
donde $\tau \equiv t'-t$ es una manera de acortar la notación y reflejar el hecho de que la relación
entre los puntos primados y sin primar es una relación que no se altera por traslación.
Una forma más natural, agrupando,
\[
	G^{\pm}(R,\tau) = \frac{1}{2\pi} \int_{-\infty}^{+\infty} 
	 \frac{ \euler^{ \pm i \omega/c R - i \omega \tau } }{ R } \: d\omega =
	\frac{1}{2\pi} \int_{-\infty}^{+\infty} 
	 \frac{ \euler^{ - i \omega( \pm R/c - \tau )} }{ R } \: d\omega
\]
que no es otra cosa que una forma de expresar la delta de Dirac (ver Apéndice Cuchuffo)
\[
	G^{\pm}(R,\tau) = \frac{1}{R} \delta\left( \pm \frac{R}{c} - \tau \right) =
	\frac{1}{R} \delta\left( \tau \mp \frac{R}{c} \right)
\]
donde usamos propiedades de la delta de Dirac. Esta expresión es por supuesto, para
un medio no dispersivo.
La forma funcional de la delta indica que la señal tiene un tiempo de propagación finito y
para ir de un lugar a otro no lo hace instantáneamente.

En la notación expandida es
\[
	G^+(\vb{x},\vb{x}',t,t') = 
	\frac{1}{|\vb{x} - \vb{x}'|} \delta\left( t-t' - \frac{1}{c}(\vb{x} - \vb{x}')\right) 
	= \frac{ \delta(t' - [t-(1/c)|\vb{x} - \vb{x}'|]) }{|\vb{x} - \vb{x}'|} ,
\]
la función de Green retardada
\[
	G^-(\vb{x},\vb{x}',t,t') = 
	\frac{1}{|\vb{x} - \vb{x}'|} \delta\left( t-t' + \frac{1}{c}(\vb{x} - \vb{x}')\right) 
	= \frac{ \delta(t' - [t + (1/c)|\vb{x} - \vb{x}'|]) }{|\vb{x} - \vb{x}'|},
\]
la función de Green avanzada.

$G^+$ exhibe el comportamiento causal del efecto observado en \vb{x} a $t$ causado por la acción de la
fuente en el tiempo $(t-R/c)$ donde $R/c$ es la diferencia de tiempo de la señal en propagarse.
Al valor 
\[
	t' = t - \frac{R}{c}
\]
se lo llama el tiempo retardado. Es un poco más práctica la nomenclatura
\[
	G^+(R,t,t') = \frac{ \delta(t' - [ t - (R/c) ]) }{ R } 
	\qquad 
	G^-(R,t,t') = \frac{ \delta(t' - [ t + (R/c) ]) }{ R } ,
\]

Entonces una solución particular de \eqref{onda_general} es 
\[
	\psi^\pm (\vb{x},t) = \int \int G(\vb{x},\vb{x}',t,t') f(\vb{x}',t') d^3x' dt' 
\]
donde $G$ es una combinación de $G^+, G^-$, dado que estas última satisfacen también la
Ec. \eqref{onda_general}.
Necesitaremos condiciones tipo Cauchy.

Consideraremos una onda incidente de fuentes muy lejanas
\[
	\psi_{in}(\vb{x},t) + \int\int G^+ f \: dv' dt
\]
con una fuente que radía a partir de cierto tiempo. Esta solución no incluye radiación
de la fuente luego de encenderse
\[
	\psi_{s}(\vb{x},t) + \int\int G^- f \: dv' dt.
\]

Consideraremos, luego, el problema de una fuene en una cierta región que se enciende en
un instatne y radía. Es decir, $\psi_{in} = 0$,
\[
	\psi(\vb{x},t) = \int\int \frac{\delta(t' - [ t - R/c ] )}{ R } \: dv' dt
\]
\[
	\psi(\vb{x},t) = \int\int \frac{f(\vbx',t - R/c )}{ R } \: dv'
\]
que son los potenciales retardados.

Una diferente explicación, de las notas, es que la solución particular de \eqref{onda_general} es
\[
	\psi^\pm (\vb{x},t) = \int \int G(\vb{x},\vb{x}',t,t') f(\vb{x}',t') d^3x' dt' 
\]
y dos soluciones son 
\[
	\psi_{in}(\vb{x},t) + \int\int G^+ f \: dv' dt \qquad \qquad 
	\psi_{s}(\vb{x},t) + \int\int G^- f \: dv' dt
\]
con $f(\vb{x}',t')$ una fuente que es diferente de cero solo en un intervalo $\sim t'$. Entonces $\psi_{in}$ 
satisface \eqref{onda_general} homogénea en $t \to -\infty$. $\psi_{s}$ es la onda en $t \to +\infty$ solución homogénea.
La situación más común es el caso de $\psi_{in}$ con $\psi_{in}=0$ entonces 
\[
	\psi (\vb{x},t) = \int_{-\infty}^{+\infty} \int_v' \frac{\delta(t'-[t -(R/c) ]) }{ R } 
			f(\vb{x}',t') dv' dt',
\]
e integrando con la delta
\[
	\psi (\vb{x},t) = \int_v' \frac{ f( \vb{x}', t -(R/c) ) }{ R } dv',
\]
que es una fuente en una cierta región que se enciende un instante e irradia.

\subsection{Fuente armónica}

Sea una fuente armónica en el tiempo 
\[
	\vb{J}(\vb{x}', t') = \vb{J}(\vb{x}') \euler^{-i\omega t'}
\]
entonces el potencial vector es 
\[
	\vb{A}(\vb{x}, t) = \left. \frac{4\pi}{c} \int_v' \frac{\vb{J}(\vb{x}')}{|\vb{x} - \vb{x}'|} 
	\euler^{-i\omega t'} \right|_{t_{ret}} dv' = \left. \frac{4\pi}{c} \int_v' 
	\frac{\vb{J}(\vb{x}')}{|\vb{x} - \vb{x}'|} \euler^{-i \omega t} \euler^{i \omega R/c } 
	\right|_{t_{ret}} dv'
\]
\[
	\vb{A}(\vb{x}, t) = \frac{4\pi}{c} \euler^{-i \omega t} \int_v \frac{\vb{J}(\vb{x})}{ R }
		\euler^{i \omega R/c } dv
\]
se puede ver como 
\[
	\vb{A}(\vb{x}) \euler^{-i \omega t} = \frac{4\pi}{c} \int_v' \frac{\vb{J}(\vb{x}')}{|\vb{x}-\vb{x}'|}
		\euler^{i k |\vb{x}-\vb{x}'| } dv' \euler^{-i \omega t}
\]

Si la fuente oscila armónicamente con frecuencia $\omega$ entonces los campos tendrán la misma frecuencia 
$\omega$.
\[
	\vb{A}(\vb{x}) = \frac{1}{c} \int_v' \frac{\vb{J}(\vb{x}')}{|\vb{x}-\vb{x}'|}
		\euler^{i \omega/c |\vb{x}-\vb{x}'| } dv'
\]
y 
\[
	\vb{A}(\vb{x},t) = \vb{A}(\vb{x}) \euler^{-i \omega t} \quad \text{si} \quad
		\vb{J}(\vb{x}',t') = \vb{J}(\vb{x}') \euler^{-i \omega t'}
\]
La aproximación consiste en desarrollar 
\[
	\frac{\euler^{i k |\vb{x}-\vb{x}'| }}{|\vb{x}-\vb{x}'|}
\]
y ver condiciones asintóticas. 

\begin{ejemplo}{\bf Desarrollo}

Esto parece que está en el Jackson (chap. 16 1era,2da eds. y chap.9 3era).
\[
	\frac{\euler^{i k |\vb{x}-\vb{x}'| }}{|\vb{x}-\vb{x}'|} =
	i k \sum_{\ell=0} j_\ell(kr_<) h^{(1)}_\ell(kr_>) 
	\sum_{m=-\ell}^{\ell} Y_{\ell m}^*(\theta',\vp') Y_{\ell m}(\theta,\vp)
\]
donde son funciones de Bessel esféricas y de Hankel las siguientes
\[
	j_\ell(x) =  \Frac{\pi}{2x}^{1/2} J_{\ell +1/2}(x) \qquad \qquad 
	h^{(1)}_\ell(x) =  \Frac{\pi}{2x}^{1/2} \left[ J_{\ell +1/2}(x) + i \: N_{\ell +1/2}(x) \right]
\]
siendo los primeros integrantes de la expansión
\[
	J_0 = \frac{ \sin(x) }{ x }  \qquad 
	J_1 = \frac{\sin(x)}{x} - \frac{\cos(x)}{x}
\]
\[
	h^{(1)}_0(x) = \frac{ \euler^{ix} }{ ix }  \qquad 
	h^{(1)}_1(x) = -\frac{ \euler^{ix} }{ ix } \left( 1 + \frac{i}{x} \right)
\]
La aproximación más baja resulta en ondas EM centradas en la fuente.

\[
	\vb{A}(\vbx) = \frac{4\pi}{c} i k 
	\sum_{\ell=0}^\infty h^{(1)}_\ell(kr) \sum_{m=-\ell}^{\ell} Y_{\ell m}(\theta,\vp)
	\int_V \vb{J}(\vbx') J_\ell(kr') Y_{\ell m}(\theta',\vp') d^3x' 
\]
Consideremos aquí soluciones asintóticas. Definamos $ kr \equiv x$ y $ kr' \equiv x'$,
si $x \ll 1, \ell$ se tiene
\[
	j_ell(x) \approx \frac{x^\ell}{(2\ell+1)!!}\left( 1 - \frac{x^2}{2(2\ell+3)} + ...\right),
\]
en cambio para $ x \gg \ell $
\[
	h^{(1)}_\ell(x) \approx (-i)^{\ell+1} \frac{\euler^{ix}}{x}
\]
con lo cual si hacemos
\[
	\vb{B} = \rotorm{A} \qquad \vb{E} = \frac{i}{k}\rotorm{B}
\]
y trabajaremos en el orden más bajo posible $(\ell=0)$ y $ k r' \ll 1$ (el tamaño de la antena es siempre
mucho mayor a la longitud de onda emitida)
\[
	\vb{A}^{(0)}(\vbx) = \frac{4\pi}{c} i k 
	h^{(1)}_0(kr) \frac{1}{\sqrt{4\pi}} \int_V \frac{ \vb{J}(\vbx') }{\sqrt{4\pi}} d^3x' =
	\frac{ i k }{c} \frac{\euler^{ikr}}{ikr} \int_V \vb{J}(\vbx') \: d^3x'
\]
y si tomamos la divergencia de la corriente, tensorialmente, 
\[
	\partial_i'(x_j'J_i) = \delta_{ij}J_i + ( \partial_i'J_i ) x'_j
\]
y se integra en el volumen,
\[
	\int_V \:  J_j + (\Nabla'\cdot\vb{J}) x_j' \: dV = 0,
\]
que es nula porque se integra en un volumen que englobe a todas las fuentes.
Entonces,
\[
	\vb{A}^{(0)}(\vbx) = \frac{ i k }{c} \frac{\euler^{ikr}}{ikr} \int_V \: -(\Nabla'\cdot\vb{J}) \vbx' \: d^3x'
\]
y será
\[
	\vb{A}(\vb{x})^{(0)} = - i k \vb{p} \frac{\euler^{ikx}}{x},
\]
una onda esférica saliente.
Esto corresponde al $A$ que crea un dipolo magnético que oscila armónicamente con el tiempo,
y faltaría ``pegarle'' la dependencia temporal.
\[
	\vb{B} = - i k \Nabla \times \left( \vb{p} \frac{\euler^{ikr}}{r} \right)
\]
pero si usamos la identidad ID 1b 
\[
	\vb{B}^0 = - i k \left[ 
	- \vb{p} \times \Nabla( \frac{\euler^{ikr}}{r} )
	\right]
\]
y tomándole el gradiente a la epxresión de la onda esférica (cuenta que he hecho mil veces)
\[
	\vb{B}^0 = k^2 (\rver\times\vb{p})  \frac{\euler^{ikr}}{r} \left[ 1 - \frac{1}{ikr} \right].
\]
Para que el campo sea de radiación, el vector $\vb{S}$ debe tener flujo no nulo en el infinito.
Entonces para que el flujo no sea nulo, si los campos van como $1/r$, como el integrando de la integral
de flujo va como $1/r^2$ y el área aumenta como $r^2$ entonces el flujo tenderá a un valor constante.
Cerca de las antenas tendremos un campo que cae como $1/r^2$ y que por ello tiene flujo no nulo
en el infinito.

\end{ejemplo}


Cuando $\ell = 0$ (el primer término de la sumatoria en $\ell$) y
$ kx' \ll 1$ tenemos una antena ineficiente. La longitud de onda $\lambda$ de la radiación es mucho mayor al 
tamaño del emisor, $ 2\pi x' \ll \lambda$ (longitud de onda larga).
En cambio tenemos $ 2\pi x \gg \lambda$ que es la condición de campo lejano (siempre la usaremos).

\notamargen{Recordemos que los potenciales retardados $\vb{A},\phi$ usan la medida ({\it gauge}) de Lorentz.
Además estamos considerando siempre $\lambda$ mucho menores al tamaño de la fuente.}

Por lo tanto,
\[
	\vb{A}(\vb{x})^{(0)} = - i k \vb{p} \frac{\euler^{ikx}}{x}
\]
es una onda esférica saliente. Es el potencial vector \vb{A} de un dipolo magnético oscilante armónicamente.
Recordemos que falta siempre {\it pegarle} un factor $\exp(i\omega t)$. Usando $\vb{E} 0 i/k \rotorm{B}, 
\vb{B} = \rotorm{A}$ tenemos 
\be
	\vb{B}(\vb{x})^{(0)} = k^2 (\hat{r}\times\vb{p})\frac{\euler^{ikx}}{x}\left(1 -\frac{1}{ikx}\right)
	\label{B_rad}
\ee
siendo $\hat{r}$ la dirección de propagación y $x\equiv|\vb{x}|$ que puede ser $|r\hat{r}|$ en esféricas.
El que contribuye a la radiación es el primer término de \eqref{B_rad} (campo lejano) mientras que el segundo
se va a cero rápidamente.

Cerca de la antena es 
\[
	\vb{B}(\vb{x})^{(0)} = i k (\hat{r}\times\vb{p})\frac{1}{r^2}, 
\]
pues $kx\ll 1$ y entonces $\exp(ikx) \sim 1$ (campo cercano) de manera que si $\lambda \to \infty$ entonces
$\vb{B}^{(0)} \sim 0$. 
\notamargen{Creo que en estas expresiones es $r$ en lugar de $x$.}

El campo \vb{E} cerca de la antena surge de la evaluación de
\[
	\vb{E} = \frac{i}{k}\rotorm{B} 
\]
que luego de algunas cuentas donde usamos la identidad del rotor de un rotor (ID 2) y
anular términos
\[
	\divem{p}=0 = 0 = (\rver\cdot\Nabla)\vb{p}
\]
y ver que
\[
	\Nabla\cdot\rver = \rver \left[ \frac{1}{r^2}\dpar{r^2\rver}{r} \right] = \frac{2}{r}
\]
da
\[
	\vb{E}^{(0)} = \frac{1}{r^3}( \: 3\hat{r}(\hat{r}\cdot\vb{p}) - \vb{p} \:)
\]
que es el campo de un dipolo eléctrico. $\vb{E},\vb{B}$ son transversales a $\hat{r}$ y tienen la misma 
longitud (en unidades CGS).
La potencia media (en un número entero de períodos) será 
\[
	\langle dP \rangle = \langle\vb{S}\rangle\cdot d\vb{S} = \langle \vb{S}\rangle\cdot\hat{n}r^2 d\Omega
\]
y entonces
\[
	\vm{ \frac{dP}{d\Omega} } = \langle\vb{S}\rangle\cdot\hat{n}r^2 
\]
\[
	\vm{ \frac{dP}{d\Omega} } = \frac{c}{8\pi} k^4 p^2 \sin(\theta)^2 
\]
y este cálculo podemos ver de dónde sale 
\[
	\langle\vb{S}\rangle = \frac{c}{2 4 \pi} \re\{ \vb{E}\times\vb{B}^* \} =
		\frac{c}{8\pi} \re\{ ( \vb{B}^0 \times \hat{r} )\times k^2(\hat{r} \times \vb{p})/r \}
\]
\[
	\langle\vb{S}\rangle =
	\frac{c}{8\pi} \re\{(-p k^2/r \sin(\theta) \hat{\theta})\times(-p k^2/r\sin(\theta)\hat{\phi} \}
	= \frac{c}{8\pi} p^2 k^4 \sin(\theta)^2 \hat{r}\cdot\hat{r}
\]
\notamargen{Tenemos un cálculo auxiliar de esta cuenta pero no sé si suma meterlo acá.
Aporto además otra expresión para el valor medio del la potencia por solid angle,
$c/(8\pi)[r^2\rver \cdot \pv{E}{B^*}]$.}

Luego, la potencia irradiada es máxima en $\theta=\pi/2$ (ver figura)

\begin{figure}[htb]
	\begin{center}
	\includegraphics[width=0.4\textwidth]{images/fig_ft1_pot_irrad.pdf}	 
	\end{center}
	\caption{Gráfico polar de la potencia radiada.}
\end{figure} 

Entonces, 
\begin{itemize}
 \item Si $\vb{B}=0$ se da que $\vb{S}=0$, es decir que no hay radiación. Una configuración
 que no hace campo magnético, no radía.
 \item Un monopolo no produce campo de radiación por su simetría esférica.
 
 \includegraphics[width=0.4\textwidth]{images/fig_ft1_radiacion_esferica_no.jpg}
 
 Una corriente $J\hat{r}$ no produce \vb{B} y se tienen
 \[
	\vb{B}^0_{rad} = \frac{k^2}{r}( \hat{r} \times \vb{p} ) \euler^{ikr} \qquad \qquad 
	\vb{E}^0_{rad} = \frac{k^2}{r}( \hat{r} \times \vb{p} ) \euler^{ikr} \times \hat{r}
 \]
	\begin{figure}[htb]
		\begin{center}
		\includegraphics[width=0.4\textwidth]{images/fig_ft1_pot_irrad2.pdf}	 
		\end{center}
		\caption{Estado de polarización del campo de radiación.}
	\end{figure} 
 \item Para que un campo sea de radiación debe tener flujo \vb{S} no nulo en el infinito.
 Si los campos van como $1/r$ entonces el Poynting va como $1/r^2$ y $dS$ va como $r^2$
 de modo que $\langle\vb{S}\rangle \cdot d\vb{S}$ tiene valor constante (un flujo que se
 va y no retorna a la fuente). Si el campo va como $1/r^2$ y entonces no produce flujo
 lejos.
 \item Si hacemos la aproximación $\ell=1$ en $\sum_\ell$ resulta que se obtiene un momento
 magnético oscilante más un cuadrupolo eléctrico.
 \item La radiación a orden $\ell=0$ es un dipolo eléctrico oscilante (ver figura)
	\begin{figure}[htb]
		\begin{center}
		\includegraphics[width=0.4\textwidth]{images/fig_ft1_pot_irrad3.pdf}	 
		\end{center}
		\caption{}
	\end{figure} 
 \item La distribución angular de potencia para la parte cuadrupolar que surge con $\ell=1$ es
	\[
		\left\langle\dtot{P}{\Omega}\right\rangle = \frac{ck^6}{128\pi} Q_0^2 \sin(\theta)^2 
\cos(\theta)^2
	\]
	que es para una fuente con simetría de revolución.
	\[
		\vm{ \dtot{P}{\Omega} } = \frac{ck^6}{128\pi} |\hat{r} \times \vb{Q} |^2, 
	\]
	donde $\vb{Q}$ es un vector que vale $\hat{n} \cdot \overline{Q}$, o bien indicialmente $n_iQ_{ij}$.
\end{itemize}

\subsection{Radiación a orden $\ell=1$}

La aproximación siguiente, $\ell=1$, corresponde a
\[
	\vb{A}(\vbx) = \frac{4\pi}{c} i k \sum_{m=-1}^{1}
	h^{(1)}_\ell(kr) Y_{1 m}(\theta,\vp)
	\int_V \vb{J}(\vbx') j_1(kr') Y_{1 m}^*(\theta',\vp') d^3x' 
\]
donde utilizamos
\[
	j_1(kr') = \frac{kr'}{3} \qquad 
	Y_{11} = - \sqrt{ \frac{3}{8\pi} } \: \sin\theta \: \euler^{i \phi} \qquad 
	Y_{10} = \sqrt{ \frac{3}{4\pi} } \: \cos\theta 
\]
\[
	Y_{1,-1} = (-1)^{1} Y_{11}^* = \sqrt{ \frac{3}{8\pi} } \: \sin\theta \: \euler^{-i \phi} 
\]
e introduciéndolo en la cuenta anterior,
\[
	\vb{A}(\vbx) = \frac{ i k^2 }{ 2 c } h^{(1)}_1(kr) \left[ 
	 \int_V \vb{J}(\vbx') r' 2 \left( \sin\theta \: \euler^{-i \phi} \: \sin\theta'\:\euler^{i\phi'} +
	\cos\theta\cos\theta' \right) \right] d^3x'  
\]
pero este término se puede dividir en dos: (1) momento dipolar magnético oscilante y (2) cuadripolo
eléctrico.


\[
	\vb{A} = \frac{1}{cR} \dot{\vb{p}}(t') + \frac{\dot{\vb{m}}(t')}{cR} \times \hat{n} + 
	\frac{1}{6c^2R} {\overline{Q}}(t') \cdot \hat{n}
\]
que es la radiación dipolar eléctrica,  magnética y cuadrupolar eléctrica.

\begin{ejemplo}{\bf Sinopsis de radiación}

De la sinopsis de radiación destaco
\[
	A_\mu(\vbx,t) = \frac{1}{c} \int \: \frac{J_\mu(\vbx',t')}{|\vbx-\vbx'|} 
	\delta\left( t' + \frac{|\vbx-\vbx'|}{c} - t \right) \: d^3x dt'
\]
y para carga puntual
\[
	\phi(\vbx,t) = e \left| \frac{1}{kR} \right|_\text{ret} \qquad \qquad 
	\vb{A}(\vbx,t) = \frac{e}{c} \Frac{\vb{u}}{kR}_\text{ret}
\]
y campo eléctrico de una partícula cargada
\[
	\vb{E}(\vbx,t) = e \left[ \frac{ (\nver -\vb{\b} ) \cdot ( 1 - \beta^2 ) }{ k^2 R^2 } \right]_\text{ret} +
	\frac{ e }{ c } \left[ \frac{ \nver }{ k^3 R } \times \{ ( \nver - \vb{\b} ) \times \vb{\b} \} \right]_\text{ret}
\]
el segundo término es un campo de aceleración, y el corchete $[...]$ es la parte que sobrevive
a grandes distancias.
\[
	\vb{B}(\vbx,t) = \nver \times \vb{E}(\vbx,t).
\]
 
\end{ejemplo}



\begin{ejemplo}{\bf Problema 10}

Estamos haciendo la aproximación de que cualquier punto sobre la flecha roja tiene el mismo
tiempo retardado, ver ilustración debajo.
\[
	z(t') = a \cos( \omega t' )
\]
\includegraphics[width=0.45\textwidth]{images/fig_ft1_antena_probl.jpg}

\[
	R = | \vbx - \vb{r}(t_\text{ret}) | \qquad 
	\nver = \frac{ \vbx - \vb{r}(t_\text{ret}) }{| \vbx - \vb{r}(t_\text{ret}) |}
\]
luego si $|\vbx|\gg a$ es $ R \sim |\vbx| $ y $ \nver \sim \vbx/|\vbx|$
\[
	t'= t_\text{ret} = t - \frac{R(t_\text{ret}))}{c}
\]
\[
	k = 1 - \nver\cdot\vb{\b}(t_\text{ret})
\]
Resultan además
\[
	\vb{x}(t') = a \cos(\omega t') \zver \qquad \qquad 
	\vb{\b}(t') = - \frac{\omega a}{c} \sin(\omega t')\zver
\]
\[
	k = 1 + \frac{a\omega}{c}\sin(\omega t')\sin\theta
\]
\[
	|\nver\times(\nver\times\vb{z})| = \sin\theta
\]
y la energía saliente respecto al tiempo 
\[
	\dtot{\mathcal{E}}{t} = dP = \frac{c}{4\pi} | R \vb{E}_a |^2 \: d\Omega
\]
Poynting da el flujo en tiempo $t$ pero necesitamos respecto a $t'$.
\[
	1 = \frac{t}{t'} - \frac{1}{c} \nabla\vb{R} \left( \dtot{\vbx}{t'}(t') \right)
\]
donde en el producto escalar el primero es $\nver$ y el segundo $\vb u$. Luego,
\[
	\dtot{P}{\Omega}|_{t'} = \frac{ c }{ 4 \pi } \Frac{e}{c}^2 
	\frac{[ \nver \times {\nver - \vb{\b}}\times\vb{\b}]}{k^5}
\]
\[
	\dtot{P}{\Omega}|_{t'} = \frac{ e^2 }{ 4 \pi c}
	\frac{ \sin^2\theta \omega^4 a^2 \cos^2(\omega t')}
	{[ 1 + \omega_0 a/c \sin(\omega t')\cos\theta ]^5}
\]
con $\beta = \omega_0 a / c $ y entonces
\[
	\dtot{P}{\Omega}|_{t'} = \frac{ e^2 c^2 B^4}{4 \pi a^2}
	\frac{\sin^2\theta  \cos^2(\omega t')}	{[ 1 + \omega_0 a/c \sin(\omega t')\cos\theta ]^5}
\]
y 
\[
	\vm{ \dtot{P}{\Omega}|_{t'} }
\]
es la integral de la guía [cualquier cosa que eso signifique]. Caso relativista.

\end{ejemplo}

\subsection{Radiación dipolar eléctrica y magnética y cuadrupolar eléctrica}

Véase la figura para entender la nomenclatura de cosas. Suponemos que $ R \gg a $, donde $a$ es el 
tamaño de la fuente localizada. Consideramos $ t_\text{ret} = t - |\vb{R} - \vbx|/c$ el tiempo que
le toma a la radiación ir de $\vbx$ a $\vb{R}$.

\includegraphics[width=0.3\textwidth]{images/fig_ft1_rad_dipolar_cuadrupolar.jpg}

\[
	\vb{A}(\vb{R},t) = \frac{1}{c} \int \frac{ \vb{J}(\vb{r},t_\text{ret})}{|\vb{R}-\vb{r}|} \: dV
\]
\notamargen{Acá uso $\vb r$ porque estaba así la carpeta pero luego moveré todo a $\vb x$
vector dado que es mi convención usual.}
Las aproximaciones de la distancia llevan a
\[
	|\vb{R} -\vb r | = \sqrt{ R^2 \left( 1 - 2 \frac{1}{r^2}\pe{r}{R} + \Frac{r}{R}^2 \right) } 
	\sim \left( 1 - \frac{\pe{r}{R}}{R^2} \right) R = R - \pe{r}{\nver}
\]

Como el campo eléctrico y magnético son perpendiculares y en módulo iguales sólo me calentaré
por $\vb A$ y dentro de éste solamente por el campo que dará radiación. Es decir,
\[
	\vb{A}(\vb{R},t) = \frac{ 1 }{ c R } \int \vb{J}(\vb r,t') \: dV
\]
En la expansión
\[
	\Nabla\times\frac{\vb a}{R} = \Nabla\Frac{1}{R} \times\vb a + \frac{1}{R} \rotorm{a}
\]
me quedo sólo con el último término ya que el primero da algo que va como $1/R^2$. El que
sobrevive, será
\[
	\rotorm{a}(t') = \Nabla t' \times \dtot{\vb a}{t'}
\]
Y como $ t' = t - R/c + \pe{\nver}{r} / c $ su gradiente  será $ \Nabla t' = - \nver/c + \sigma(1/R)$
pero no los quiero.
\[
	\vb{B}_\text{rad} = \rotorm{A} = - \frac{1}{c} \nver \times \vb A
\]

Estamos en $\lambda$ muy grandes respecto a la dimensión de las fuentes, se hace un Taylor en torno
a $t_r \equiv t - R/c$
\[
	\vb{J}(\vb r,t_\text{ret}) \approx \vb{J}(\vb r,t_r ) + 
	\left. \dpar{\vb J}{t}\right|_{t_r-t-R/c} - \frac{\pe{\nver}{r}}{c}
\]
donde el primer término del rhs es el que generará radiación dipolar eléctrica y como
\[
	\left| \dpar{\vb J}{t} \frac{\pe{\nver}{r}}{c} \right| \ll |\vb J|
\]
y en el caso armónico con $ \omega a / c \ll 1$ se construye
\[
	\vb{A}(\vb{R},t) \approx \frac{ 1 }{ c R } \int \vb{J}(\vb r,t_r) \: dV
	+ \frac{ 1 }{ c R } \dtot{}{t_r} \int \vb{J}(\vb r,t_r) \frac{\pe{\nver}{r}}{c} \: dV
\]
Defino por conveniencia algunas cosas:
\[
	\vb M = \frac{1}{2c} \pv{r}{J} \qquad 
	\pv{\nver}{M} = \frac{1}{2c} \nver\times( \pv{r}{J}) =
	\frac{1}{2c} \left[ \vb{r}(\nver\cdot\vb J) - \vb{J}(\nver\cdot\vb r)\right]
\]
donde el corchete en el rhs es $\alpha$ o bien la reescritura
\[
	\pv{\nver}{M} =  \frac{1}{2c} \left[ \vb{r}(\nver\cdot\vb J) + \vb{J}(\nver\cdot\vb r)\right]
	- \frac{1}{c} \vb{J}(\nver\cdot\vb r)
\]

Ahora trabajaremos con cargas puntuales y defino $t'= t_r $ de manera que
\[
	\int \vb{J}(\vb r,t') \: dV = \dtot{}{t}( \sum e_i \vb{r}_i(t') )
\]
y para $\a$ se tiene
\[
	\a = \sum_i e_i \left[ \vb{r}_i (\nver\cdot\vb{M}_i) + \vb{\mu}_i(\vb{r}_i\cdot\nver) \right]=
	\dtot{}{t}\left[ \sum e_i \vb{r}_i( \vb{r}_i\cdot\nver )\right]
\]
y ahora tiene la pinta de cuadripolo lo que está dentro del corchete en el extremo del rhs.
Le puedo agregar algo en la dirección $\nver$ y no cambio el campo,
\[
	\a' = \dtot{}{t'} \left[ \sum e_i 3 \vb{r}_i( \vb{r}_i\cdot\nver) - \nver r^2_i \right] \frac{1}{3}
\]
y será
\[
	\a'= \frac{1}{3} \dtot{}{t'} Q(t')\cdot\nver
\]
donde $Q_{ij} = \sum_i e_i( 3 x_jx_k - \delta_{jk}r_i^2)$.
Tenemos
\[
	\vb{A} = \frac{1}{cR} \dot{\vb{p}}(t') + \frac{\dot{\vb{m}}(t')}{cR} \times \hat{n} + 
	\frac{1}{6c^2R} {\overline{Q}}(t') \cdot \hat{n}
\]
Luego, los campos de radiación serán
\[
	\vb{B}_\text{rad} = \frac{1}{c^2R}
	\left[ 
	\ddot{ \vb{p} } \times \nver + \frac{1}{cR} (\dot{\vb m}(t')\times\nver) \times \nver + 
	\frac{1}{6c} ( \dddot{Q}(t')\cdot\nver )\times \nver 
	\right]
\]
\[
	\vb{E}_\text{rad} = \frac{1}{c^2R}
	\left[ 
	( \ddot{ \vb{p} } \times \nver ) \times \nver + 
	\nver \times \ddot{\vb m} +
	\frac{1}{6c} \{ ( \dddot{Q}(t')\cdot\nver )\times \nver \}\times \nver
	\right]
\]

Veamos algunas aplicaciones de esta teoría desarrollada.
Suponiendo radiación dipolar eléctrica,
\[
	\vb{P}(t') = \vb{P}_0 \euler^{ - i \omega t }
\]
la aceleración
\[
	\ddot{ \vb{P} }(t') = - \omega^2 \vb{P}_0 \euler^{ - i \omega t }
\]
de modo que para el campo magnético de radiación se tiene 
\[
	\vb{B}_\text{rad} = \frac{\omega^2}{c^2R} \vb{P}_0\times\nver \euler^{i \omega R/c} \euler^{ - i \omega t }
	\sim - \frac{\omega^2}{c^2} \frac{ \vb{P}_0\times\nver }{R}
\]
luego,
\[
	\vm{ \dtot{P}{\Omega} } = \frac{ \omega^4 }{ 8 \pi c^3 } P_0^2 \sin^2\theta
\]
y la potencia emitida en todo ángulo será
\[
	\vm{P} = \frac{ \omega^4 }{ 3 c^3 } P_0^2.
\]

Para la radiación dipolar magnética, será
\[
	\vb{m} = \vb{m}_0 \euler^{ - i \omega t },
\]
y las cuentas son las mismas con el reemplazo obvio,
\[
	\vm{ \dtot{P}{\Omega} } = \frac{ \omega^4 }{ 8 \pi c^3 } m_0^2 \sin^2\theta,
\]
\[
	\vm{P} = \frac{ \omega^4 }{ 3 c^3 } m_0^2.
\]


\subsection{Ejemplo de antena}

Sea una pequeña antena de longitud $d$ (ver figura) tal que 
\[
	\vb{J}(\vb{x}') = I \sin( k[d/2 - |z|] ) \delta(x') \delta(y')  \hat{z}
\]
que tiene nodos de la corriente en los extremos. Luego considerando fuente armónica ($A=A(x)\exp(i\omega t)$)
será
\[
	\vb{A}(\vb{x}) = \frac{1}{c} \int_V' \frac{ \vb{J}(\vb{x}') \euler^{ i k |\vb{x}-\vb{x}'| }}
	{ |\vb{x}-\vb{x}'| } dv'
\]

\begin{figure}[htb]
	\begin{center}
	\includegraphics[width=0.3\textwidth]{images/fig_ft1_antena.pdf}	 
	\end{center}
	\caption{}
\end{figure} 

Hacemos algunas aproximaciones geométricas de distancia amparadas en la figura de más abajo.
	\begin{figure}[htb]
		\begin{center}
		\includegraphics[width=0.4\textwidth]{images/fig_ft1_antena2.pdf}	 
		\end{center}
		\caption{}
	\end{figure} 

Estas aproximaciones son clásicas de los problemas de difracción.
\[
	|\vb{x}-\vb{x}'| = \sqrt{ x^2 + x'^2 - 2xx'\cos(\theta)} =  
	x( 1 - 2x'/x \cos(\theta) + (x'/x)^2)^{1/2} 
\]
y quedándonos a primer orden,
\[
	|\vb{x}-\vb{x}'|\approx x (1 - x'/x \cos(\theta))
\]
de manera que aceptamos una buena aproximación y una bruta,
\[
	|\vb{x}-\vb{x}'| \approx x - x' \cos( \theta ) \qquad \qquad |\vb{x}-\vb{x}'| \approx x
\]
para así escribir
\[
	\approx \frac{1}{|\vb{x}|} \euler^{ikx} \euler^{-ikx'\cos(\theta)}
\]
donde notamos que hemos aproximado de una forma dentro del argumento de la exponencial compleja
y de otra en el denominador de la fracción.

Así, resulta
\[
	\vb{A}(\vb{x}) = \frac{1}{c} \frac{\euler^{ i k x} }{x} \int_V' \vb{J}(\vb{x}') 
		\euler^{ i k x' \cos(\theta) } dv'
\]

Existe condición de contorno que en los extremos la corriente debe ser nula, entonces debe haber nodos
del seno (en $\pm d/2$) y los $d$ posibles son $ n\lambda/2$.
\[
	\vb{A}(\vb{x}) = \hat{z} \frac{2I\euler^{ikx}}{ckx}\left[ \cos( kd/2 \cos(theta) )- 
		\cos( kd/2 ) \right]\frac{1}{\sin(\theta)^2}
\]
entonces 
\[
	\vb{A}(\vb{x}) = A_z \hat{z} \qquad \qquad \vb{A}(\vb{x}) =  A_z\cos(\theta)\hat{\theta} - 
		A_z\sin(\theta) ?
\]
\notamargen{Falta un vegsor}.

Entonces con $ kx' \ll 1$ (longitud de onda larga, $\lambda \gg d $) tenemos
\[
	\left\langle\dtot{P}{\Omega}\right\rangle = \frac{I^2}{2c\pi}\left( \frac{kd}{2} \right)^4 
		\sin(theta)^2
\]
identificando con $|\vb{p}| = Id^2/(2c)$ y este es el primer término multipolar. El paréntesis es muy
chico.
Con media longitud de onda ($kd=\pi$) ($\lambda/2=d$) es
\[
	\left\langle\dtot{P}{\Omega}\right\rangle=\frac{I^2}{2c\pi}\frac{\cos(\pi/2 
	\cos(\theta))^2}{\sin(\theta)^2}
\]
y finalmente para una longitud de onda ($\lambda=2$ y $kd=2\pi$) se tiene 
\[
	\left\langle\dtot{P}{\Omega}\right\rangle=\frac{I^2}{2c\pi} \left[ \frac{ 2\cos(\pi/2 
	\cos(\theta))^2}{\sin(\theta)^2} \right]^2
\]
Las ilustraciones sucesivas de la figura bajo estas líneas dan cuenta de estas diferentes longitudes.

\begin{figure}[htb]
	\begin{center}
	\includegraphics[width=0.1\textwidth]{images/fig_ft1_antena3.pdf}	 
	\end{center}
	\caption{}
\end{figure} 


Como referencia tengamos en cuenta que las expresiones salen de 
\[
	\vb{B}_{rad} = -\frac{1}{c} \hat{n} \times \dot{\vb{A}} = ik \hat{n} \times \vb{A} 
\]
y
\[
	\vb{E}_{rad} = \vb{B}_{rad} \times \hat{n}
\]
Estas equivalencias son para campos de radiación nomás,
\[
	\vb{B}_{rad} = ik \hat{n} \times \vb{A} \qquad \qquad \vb{E}_{rad} = \vb{B}_{rad} \times \hat{n}
\]

% =================================================================================================
\section{Campos de una partícula cargada en movimiento}
% =================================================================================================

Escribimos la densidad de corriente y la densidad de carga según
\[
	\vb{J}(\vb{x}',t') = q\vb{v} \delta[ \vb{x}' - \vb{r}(t')]
\]
\[
	\rho(\vb{x}',t') = q \delta[ \vb{x}' - \vb{r}(t')]
\]
de manera que 
\[
	\vb{A}(\vb{x},t) = \frac{1}{c} \int_{t'}\int_{V'} 
	\frac{ q\vb{v} \delta[ \vb{x}' - \vb{r}(t')] \delta[ t'-t +R/c] }{|\vb{x} -\vb{x}'|} dV' dt'
\]
\[
	\phi(\vb{x},t) = \frac{1}{c} \int_{t'}\int_{V'} 
	\frac{ q \delta[ \vb{x}' - \vb{r}(t')] \delta[ t'-t +R/c] }{|\vb{x} -\vb{x}'|} dV' dt'
\]
donde hemos usado $R\equiv |\vb{x}-\vb{x}'|$ de modo que es $R = R(t')$.
\[
	\vb{A}(\vb{x},t) = \frac{1}{c} \int_v' \frac{ q\vb{v} \delta[ t'-t +R/c] }{|\vb{x} -\vb{r}(t')|} dt' 
	\Rightarrow \vb{A}(\vb{x},t) = \left. \frac{q}{c} \frac{\vb{v}(t')}
	{(1-\hat{n}\cdot{\vb{\beta}})R(t')} \right|_{t'=t-R/c}
\]
\[
	\phi(\vb{x},t) = \frac{1}{c} \int_v' \frac{q \delta[ t'-t +R/c]}{|\vb{x} -\vb{r}(t')|} dt' 
	\Rightarrow \phi(\vb{x},t) = \left. \frac{q}{c} \frac{1}{(1-\hat{n}\cdot{\vb{\beta}})R(t')} 
	\right|_{t'=t-R/c}
\]
cuyas expresiones son los potenciales de Liènard-Wiechert. Hemos usado en las cuentas que 
\[
	\delta[t' - ( t - R(t')/c )] = \frac{1}{\dtot{}{t'}( t' + R(t')/c )} \delta( t-t')
\]
(idea que viene de $\delta f = (1/(df/dx_0)) \delta(x-x_0) $) y que 
\[
	R = |\vb{x}-\vb{x}'| = \sqrt{ x^2 + x'^2 - 2\pe{x}{x'} } \qquad \dtot{R}{t'} = 
	\frac{\dot{\vb{x}'}\cdot(\vb{x}-\vb{x}')}{R} = -\frac{\pe{R}{v}}{R} = -\hat{n}\cdot\vb{v}
\]
\[
	1 + \frac{1}{c} \dtot{R}{t'}  = 1 -\hat{n}\cdot\frac{\vb{v}}{c} = 1 - \hat{n}\cdot\vb{\beta}
\]
según la figura que ilustra bajo estas líneas

\begin{figure}[htb]
	\begin{center}
	\includegraphics[width=0.4\textwidth]{images/fig_ft1_campo_part_car.pdf}	 
	\end{center}
	\caption{}
\end{figure} 

y como los campos serán 
\[
	\vb{B} = \rotorm{A} \qquad \vb{E} = -\frac{1}{c}\dpar{\vb{A}}{t} - \Nabla\phi
\]
se tiene que 
\[
	\vb{E} = \left. q \frac{ (\hat{n}-\vb{\beta})(1-\beta^2)}{K^3R^2} \right|_{ret} + \left. 
	\frac{q}{c} \frac{\hat{n}\times[ (\hat{n}-\vb{\beta})\times \dot{\vb{\beta}} ]}{K^3R} \right|_{ret}
\]
donde se ve que vale $\vb{B} = \hat{n} \times \vb{E}$ que ya sabíamos para $\vb{E}_{rad}$ y $\vb{B}_{rad}$
y donde $K\equiv 1 - \hat{n}\cdot\vb{\beta} $

De acuerdo a la figura xxxx en $t'$ se produce el campo. Cuando la radiación llega a $\vb{x}$ en tiempo $t$ 
la partícula se halla en $\vb{x}'$ (tiempo $t$), de manera que la moraleja es que $t$ y $t'$ son instantes
de tiempo diferentes en un mismo sistema inercial.

\begin{figure}[htb]
	\begin{center}
	\includegraphics[width=0.4\textwidth]{images/fig_ft1_campo_part_car2.pdf}	 
	\end{center}
	\caption{}
\end{figure} 

Podemos sacar un par de frases importantes ya que 
\begin{itemize}
 \item Si una partícula se mueve con $\vb{v}$ constante puedo pasar a un frame inercial $S'$ donde es 
 $\vb{v}=0$ y entonces $\vb{B}'=0$ de manera que como $\pe{B}{E}=\pe{B'}{E'}=0$ se tiene $\vb{B}\perp\vb{E}$
 en todo frame inercial.
 \item El $\vb{E}_{rad}$ estará dado por el $\vb{E}_a$.
 \item Toda partícula que está acelerada en un frame inercial debe irradiar ondas EM, entonces una partícula
 recorre una circunferencia (en un campo $\vb{B}$) si aceptamos que lo que irradia es despreciable.
\end{itemize}

Sea ahora una partícula $e$ con $|\vb{v}|$ constante, entonces 
\[
	\vb{B}_{bs} = e\frac{\vb{\beta}\times\hat{n}}{\gamma^2k^3R^2} \;\; \text{(Lienard-Wiechert)}
	\qquad \qquad 
	\vb{B}_{bs} = e\frac{\vb{v}\times\hat{n}}{cR^2} \;\; \text{(Biot-Savart)}
\]
y 
\[
	\vb{E}_{v} = e\frac{ \hat{n}- \hat{\beta} }{\gamma^2k^3R^2}
\]
donde vemos que difieren en 
\[
	\frac{1-\beta^2}{(1-\hat{n}\cdot{\vb{\beta}})}
\]


% =================================================================================================
\section{Campo de una carga en movimiento}
% =================================================================================================

El campo de velocidad es 
\[
	\vb{E}_v = e \frac{(\hat{n}-\hat{\beta})}{\gamma^2(1-\hat{n}\cdot\vb{\beta})^3R^2} =
		e \left[ \frac{\vb{R}-R\vb{\beta}}{\gamma^2(1-\hat{n}\cdot\vb{\beta})^3R^2} \right]
\]
referidas las magnitudes a la figura XXXX.

\begin{figure}[htb]
	\begin{center}
	\includegraphics[width=0.4\textwidth]{images/fig_ft1_campo_carga_mov2.pdf}	 
	\end{center}
	\caption{}
\end{figure} 

\[
	|\vb{E}_v| = e\frac{\sqrt{ R^2 + \beta^2 R^2 - 2R^2 \beta \cos(\theta)}}
		{\gamma^2( 1 - \vb{R}\cdot\vb{\beta}/R)^3 R^3} =
		e\frac{\sqrt{ 1 + \beta^2 - 2 \beta \cos(\theta)}}
		{\gamma^2( 1 - \beta \cos(\theta))^3 R^2}
\]
entonces como $\cos(\theta) = \beta$
\[
	\dtot{ |\vb{E}_v| }{\theta}= 0
\]
siendo los extremos $\theta=0,\pi$ que representan un movimiento hacia adelante o hacia atrás.
\[
	|\vb{E}_v(\cos(\theta) = \beta)| = \frac{e\gamma}{r^2}
\]
\[
	|\vb{E}_v(\cos(\theta) = 1)| = \frac{e (1+\beta^2-2\beta)^2}{R^2(1-\beta^2)^{-1}(1-\beta)^3}
\]
\[
	|\vb{E}_v^{(\theta = 1)}| = \frac{e }{R^2(1-\beta^2)^2 \gamma^2} = \frac{e}{r^2 \gamma^2}
\]
puesto que es $r=R(1-\beta)$. Vemos que es similar al campo estático pero con un factor corrector.

Campo de aceleración, es
\[
	\vb{E}_a = \frac{e}{c} \frac{ \hat{n} \times [ (\hat{n}-\vb{\beta})\times \dot{\vb{\beta}} ]}{K^3 R} 
		\approx \frac{e}{c} \frac{ \hat{n} \times ( \hat{n} \times \dot{\vb{\beta}})}{K^3 R} 
		= \frac{e}{c} \frac{}{K^3 R}
\]
donde usamos que $v/c \ll 1$ y por ende $ 1 - \hat{n}\cdot\vb{\beta}\approx 1$ entonces es 
\[
	\hat{n} \cdot \hat{\beta} \approx \hat{n}
\]

\begin{figure}[htb]
	\begin{center}
	\includegraphics[width=0.4\textwidth]{images/fig_ft1_campo_carga_mov.pdf}	 
	\end{center}
	\caption{}
\end{figure}

% =================================================================================================
\section{Cálculo de potencia irradiada}
% =================================================================================================

Se realiza calculando el vector de Poynting,
\[
	\vb{S} = \frac{c}{4\pi} \pv{E}{B} = \frac{c}{4\pi} |\vb{E}_a|^2 \hat{n} =  \frac{e^2}{4\pi c}
		\hat{n} \left| \frac{\hat{n} \times \dot{\vb{\beta}} }{R} \right|^2
\]
si es 
\[
	dP = \vb{S}\cdot\hat{n} R^2 d\Omega =
		\frac{e^2}{4\pi c} \left| \hat{n} \times \dot{\vb{\beta}} \right|^2 d\Omega
\]
y entonces 
\[
	\dtot{P}{\Omega} = \frac{e^2}{4\pi c} \frac{\dot{v}^2}{c^2} \sin(\theta)^2 = 
		\frac{e^2 \dot{v}^2 }{4\pi c^3} \sin(\theta)^2.
\]

Luego, si integramos,
\[
	P = \frac{e^2 \dot{v}^2 }{4\pi c^3} \int\int \sin(\theta)^3 d\theta d\phi
\]
y se llega a que 
\[
	P = \frac{2 e^2 \dot{v}^2 }{3 c^3}
\]
que es la fórmula de Larmor con $v/c \ll 1$. Ahora podemos prescindir de la restricción no relativista
usando que la $P$ es invariante lorentziano.
\[
	P = \frac{2 e^2}{3 c^3 m^2} \left( \dtot{\vb{p}}{t} \dtot{\vb{p}}{t} \right)
\]
y como $p_{\mu}( E/c , -\vb{p} )$ y $p^{\mu}( E/c , \vb{p} )$. 
\[
	-\dtot{p_\mu}{\tau} \dtot{p^\mu}{\tau} = \dtot{\vb{p}}{\tau} \dtot{\vb{p}}{\tau} - 
		\dtot{}{\tau}(E^2/c^2)
\]
\[
	\tau = \gamma (t-\beta x_\parallel) \qquad \qquad \dtot{\tau}{t} = \gamma \qquad \Rightarrow \qquad 
	\dtot{\vb{p}}{\tau} \dtot{\tau}{t}= \gamma \dtot{\vb{p}}{\tau} =  \dtot{\vb{p}}{t}
\]
y luego para una trayectoria rectilínea
\[
	\dtot{P}{\Omega} = \frac{e^2}{4\pi c^3} \frac{ a^2 \sin(\theta)^2}{(1-\beta \cos(\theta))^5} 
\]
\[
	P = \frac{2 e^2 a^2 \gamma^6}{3 c^3} \qquad \qquad a = Z_0 \omega^2 \euler^{-i \omega t}
\]
Según vemos en la figura la distribución angular de potencia es una especie de {\it as de pique} en el cual
a mayor velocidad los lóbulos se pegan al eje de simetría. Compárese con el caso no relativista.
\begin{figure}[htb]
	\begin{center}
	\includegraphics[width=0.4\textwidth]{images/fig_ft1_frenado3.pdf}	 
	\end{center}
	\caption{}
\end{figure} 

Se tiene además 
\[
	\theta_{max} \approx \frac{1}{2\gamma}
\]

\begin{figure}[htb]
	\begin{center}
	\includegraphics[width=0.3\textwidth]{images/fig_ft1_frenado2.pdf}	 
	\end{center}
	\caption{}
\end{figure} 

% =================================================================================================
\section{Frenado magnético}
% =================================================================================================

Sea la Figura. Hacemos
\[
	\vb{E}' = \frac{1}{c}\pv{v}{B} = \omega \frac{r}{c} \hat{\phi} \times (-B\hat{z}) = 
	-\frac{\omega r B}{c} \hat{r}
\]
y la densidad de potencia disipada por corrientes de Foucault será
\[
	\mathfrak{P} = \pe{P}{E'} = \sigma E'^2 = \frac{\sigma \omega^2 r^2 B^2}{c^2}
\]
donde 
\[
	P = \int \int \frac{\sigma \omega^2 r^2 B^2}{c^2} r d\theta dr =
		\frac{\sigma \omega^2 a^4 B^2 2\pi }{4 c^2}.
\]
Son corrientes de {\it Foucault} las que frenan el disco.

\begin{figure}[htb]
	\begin{center}
	\includegraphics[width=0.4\textwidth]{images/fig_ft1_frenado.pdf}	 
	\end{center}
	\caption{}
\end{figure} 

En un disco fijo con $\vb{B} = B_0 \euler^{i\omega t}$  habrá $\vb{E} = E(r) \hat{\phi}$ de manera que 
$\pe{J}{E} = \sigma E(r)^2$ entonces se disipará energía por efecto Joule. Se calientan los 
transformadores en un ejemplo usual de la vida real.


\subsection{Esponja electromagnética}

En $t=0$ se distribuye una $\sigma$ en la cara interna. Se genera una \vb{J} y un campo \vb{E} radial que 
no produce \vb{B} entonces $\vb{S}=0$ no hay radiación.
La carga se mueve 

\begin{figure}[htb]
	\begin{center}
	\includegraphics[width=0.25\textwidth]{images/fig_ft1_esponja.pdf}	 
	\end{center}
	\caption{}
\end{figure} 

por el interior hasta llegar a la superficie y alcanzar situación estática $\vb{E}_{II} = 0$ .
La energía disipada lo hace en forma de calor pero no se radía.

% \bibliographystyle{CBFT-apa-good}	% (uses file "apa-good.bst")
% \bibliography{CBFT.Referencias} % La base de datos bibliográfica

\end{document}
