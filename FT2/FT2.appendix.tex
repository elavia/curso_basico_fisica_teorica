	\documentclass[10pt,oneside]{CBFT_book}
	% Algunos paquetes
	\usepackage{amssymb}
	\usepackage{amsmath}
	\usepackage{graphicx}
% % 	\usepackage{bm}
% 	\usepackage{libertine}
% 	\usepackage[bold-style=TeX]{unicode-math}
	\usepackage{lipsum}

	\usepackage{natbib}
	\setcitestyle{square}

	\usepackage{polyglossia}
	\setdefaultlanguage{spanish}


	\usepackage{CBFT.estilo} % Cargo la hoja de estilo

	% Tipografías
	% \setromanfont[Mapping=tex-text]{Linux Libertine O}
	% \setsansfont[Mapping=tex-text]{DejaVu Sans}
	% \setmonofont[Mapping=tex-text]{DejaVu Sans Mono}

	%===================================================================
	%	DOCUMENTO PROPIAMENTE DICHO
	%===================================================================

\begin{document}
 
 
\appendix

% ~~~~~~~~~~~~~~~~~~~~~~~~~~~~~~~~~~~~~~~~~~~~~~~~~~~~~~~~~~~~~~~~~~~~~~~~~~~~~~~~~~~~~~~
\chapter{Tensor de Levi-civita}\label{App.levi_civita}
% ~~~~~~~~~~~~~~~~~~~~~~~~~~~~~~~~~~~~~~~~~~~~~~~~~~~~~~~~~~~~~~~~~~~~~~~~~~~~~~~~~~~~~~~

\[
	\epsilon_{ijk} = \begin{cases}
			\pm 1 &  i,j,k \text{ diferentes } \\
	                0     &  \text{ Si algún par es igual }
	                 \end{cases}
\]
y el signo es de acuerdo a si la permutación es par o impar.


% % ~~~~~~~~~~~~~~~~~~~~~~~~~~~~~~~~~~~~~~~~~~~~~~~~~~~~~~~~~~~~~~~~~~~~~~~~~~~~~~~~~~~~~~~
% \chapter{Delta de Dirac}\label{App.delta_dirac}
% % ~~~~~~~~~~~~~~~~~~~~~~~~~~~~~~~~~~~~~~~~~~~~~~~~~~~~~~~~~~~~~~~~~~~~~~~~~~~~~~~~~~~~~~~
% 
% La delta de Dirac tiene representaciones numéricas en términos de límites. En sí, debe entenderse como un proceso límite.
% Las dos más utilizadas son las representaciones lorentziana,
% \[
% 	\delta (x) = \lim_{\epsilon \to 0} \: \frac{\epsilon }{\pi (x^2 + \epsilon^2)}
% \]
% gaussiana
% \[
% 	\delta (x) = \lim_{\epsilon \to 0} \: \frac{\euler^{-x^2/(4\epsilon)} }{ 2 \sqrt{ \pi } \sqrt{ \epsilon } }
% \]
% y la que utiliza la función sinc, [CHECK]
% \[
% 	\delta (x) = \lim_{\epsilon \to 0} \: \frac{\sin( x / \epsilon ) }{\pi x}
% \]
% 
% \notamargen{Hacer grafiquitos de estas funciones.}
% 
% La variable que tiene a cero $\epsilon$ cuantifica el ancho mientras que $1/\epsilon$ cuantifica la altura.
% 
% \[
% 	\int g(x) \delta'(x-a) \: dx = - g'(a)
% \]
% 
% % ~~~~~~~~~~~~~~~~~~~~~~~~~~~~~~~~~~~~~~~~~~~~~~~~~~~~~~~~~~~~~~~~~~~~~~~~~~~~~~~~~~~~~~~
% \chapter{Coordenadas esféricas y cilíndricas}\label{App.coord_esf_cil}
% % ~~~~~~~~~~~~~~~~~~~~~~~~~~~~~~~~~~~~~~~~~~~~~~~~~~~~~~~~~~~~~~~~~~~~~~~~~~~~~~~~~~~~~~~
% 
% Acá se condensan algunas expresiones asociadas al aspecto de los operadores diferenciales en los diferentes
% sistemas de coordenadas curvilíneos.
% 
% El prototipo de sistema curvilíneo es el esférico.
% Teníamos
% \[
% 	\dX = dr \rver + r d\theta \thetaver + r\sin\theta \phiver \qquad 
% 	dV = r^2 \sin\theta dr d\theta d\phi
% \]
% donde $r^2 \sin\theta$ es el jacobiano de la transformación.
% 
% La idea  es que en cualquier sistema curvilíneo de coordenadas $\{ q_i \}$ se tiene para el diferencial total
% de una función $f$ 
% \[
% 	df = \Nabla f \cdot \dX
% \]
% y como en coordenadas cartesianas es
% \[
% 	df = \Nabla f \cdot \dX = \sum_i \dpar{f}{q_i} dq_i,
% \]
% la idea es que tiene que valer lo mismo en todo sistema de coordenadas.
% Luego, en un sistema donde las coordenadas no son las cartesianas se tiene
% \[
% 	\dX = h_1 dq_1 \hat{e}_1 + h_2 dq_2 \hat{e}_2 + h_3 dq_3 \hat{e}_3
% \]
% donde los $ h_i $ dan la métrica del espacio coordenado. El gradiente será 
% \[
% 	\Nabla f = g_1 \hat{e}_1 + g_2 \hat{e}_2 + g_3 \hat{e}_3
% \]
% donde $g_i$ se ajusta pidiendo que el escalar $df$ sea un invariante
% \[
% 	\Nabla f \cdot \dX = \sum_i h_i g_i dq_i,
% \]
% de lo cual surge que 
% \[
% 	\Nabla \equiv \sum_i \frac{1}{h_i} \dpar{}{q_i} \hat{e}_i.
% \]
% Este es el operador gradiente en un sistema curvilíneo.
% 
% La divergencia en cartesianas es 
% \[
% 	\divem{F} = \dpar{F_x}{x} + \dpar{F_y}{y} + \dpar{F_z}{z} 
% \]
% y utilizando el teorema de la divergencia de Green,
% \[
% 	\int_\Omega \divem{F} d\Omega = \int_{\partial \Omega} \vb{F}\cdot d\vb{S}
% \]
% se arriba a 
% \[
% 	\divem{F} = \frac{1}{h_1h_2h_3} \left( 
% 	\dpar{}{q_1}\left[ h_2 h_3 F_1 \right] + \dpar{}{q_2}\left[ h_1 h_3 F_2 \right] + \dpar{}{q_3}\left[ h_1 h_2 F_3 \right]
% 	\right)
% \]
% 
% Luego, el laplaciano (que será el operador más usado en este curso) resulta de 
% \[
% 	\lapm{\vp} = \divem{\Nabla \vp},
% \]
% es decir la divergencia del gradiente de la función.
% 
% En un sistema curvilíneo la delta será algo como
% \[
% 	\delta (\vb{x} - \vb{x}') = \frac{1}{h_1 h_2 h_3 } \delta(q_1 - q_1')\delta(q_2 - q_2') \delta(q_3 - q_3').
% \]
%  
% \chapter{Rejunte} 
% 
% {\bf Simplificaciones}
% 
% Recordemos que
% \[
% 	\sqrt{ x^2 }  = |x|,
% \]
% la simplificación de las raíces cuadradas implican el módulo para tener en cuenta las dos posibilidades del signo.
% 
% {\bf Ángulo entre dos vectores}
% 
% En esféricas el ángulo entre dos vectores $\vb{v}_1$ y $\vb{v}_2$, escribiendo los versores en cartesianas resulta 
% \[
% 	\cos \gamma \equiv \frac{ \vb{v}_1 \cdot \vb{v}_2 }{ |\vb{v}_1 \cdot \vb{v}_2 |} =
% 	\cos( \theta_1 ) \cos( \theta_2 ) + \sin( \theta_1 ) \sin( \theta_2 ) \cos (\vp_1 - \vp_2 )
% \]
% 
% {\bf Cálculo diferencial e integral vectorial sobre superficies}
% 
% Hablar sobre la normal $\hat{n}$ de una superficie y versores tangenciales $\hat{t}$.
% 
% {\bf Identidades vectoriales}
% Esta,
% \[
% 	A \cdot {B \times C } = C \cdot {A \times B },
% \]
% y otras.
% 
% 
% {\bf ID 1}
% Siendo $\vb{A}$ un vector generico y $\phi$ un campo escalar,
% \[
% 	\divem{(\phi \vb{A})} = \phi \: \divem{A} + \vb{A}\cdot{\Nabla \phi}
% \]
% {\bf ID 1b}
% \[
% 	\Nabla\times(\phi \vb{A}) = \phi \rotorm{A} - \vb{A}\times\Nabla\phi
% \]
% 
% 
% {\bf ID 2}
% \[
% 	\Nabla\times[\vb{M}\times\vb{N}] = 
% 	\vb{M} \: ( \divem{N} ) - \vb{N} \: ( \divem{M} ) +
% 	( \vb{N} \cdot \Nabla ) \: \vb{M} - ( \vb{M} \cdot \Nabla ) \: \vb{N}
% \]
% 
% {\bf ID 2b}
% \[
% 	\vb{A} \times ( \pv{B}{C}) = \vb{B}(\pe{A}{C}) - \vb{C}(\pe{A}{B})
% \]
%  
% {\bf ID 3} 
% \[
% 	\int_V \: \rotorm{R}\: dV = \int_S \: \vb{R}\times\hat{n}\: dS
% \]
% 
% {\bf ID 4} 
% \[
% 	\Nabla\cdot\pe{A}{B} = \vb{A} \times (\rotorm{B}) +
% 	\vb{B} \times (\rotorm{A}) +
% 	(\vb{B}\cdot\nabla)\vb{A}+
% 	(\vb{A}\cdot\nabla)\vb{B}
% \]
% 
% {\bf ID 5}
% \[
% 	\Nabla\cdot(\pv{A}{B}) = \vb B (\rotorm{A}) - \vb A (\rotorm{B})
% \]
% 
% 
% \subsubsection{Bessel Functions}
% \label{apendice_bessel}
% 
% \[
% 	J_0( x ) = \sum_{i=0}^{\infty} \frac{(-1)^i}{i! \Gamma(i+1)} \Frac{x}{2}^{2i}
% \]
% 
% Algunas integrales útiles
% \[
% 	\int x^n J_{n-1} dx = x^n J_n(x)
% \]
% \[
% 	\int_0^x x J_{n}^2(x) \: dx = \frac{x^2}{2} \left[ J_n(x)^2 - J_{n-1}(x) J_{n+1}(x) \right] 
% \]
%  
\end{document}
