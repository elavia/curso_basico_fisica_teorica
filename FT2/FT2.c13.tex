	\documentclass[10pt,oneside]{CBFT_book}
	% Algunos paquetes
	\usepackage{amssymb}
	\usepackage{amsmath}
	\usepackage{graphicx}
% 	\usepackage{libertine}
% 	\usepackage[bold-style=TeX]{unicode-math}
	\usepackage{lipsum}

	\usepackage{natbib}
	\setcitestyle{square}

	\usepackage{polyglossia}
	\setdefaultlanguage{spanish}
	



	\usepackage{CBFT.estilo} % Cargo la hoja de estilo

	% Tipografías
	% \setromanfont[Mapping=tex-text]{Linux Libertine O}
	% \setsansfont[Mapping=tex-text]{DejaVu Sans}
	% \setmonofont[Mapping=tex-text]{DejaVu Sans Mono}

	%===================================================================
	%	DOCUMENTO PROPIAMENTE DICHO
	%===================================================================

\begin{document}

% =================================================================================================
\chapter{Partículas idénticas}
% =================================================================================================

Más apropiado sería partículas indistinguibles. Si en algún punto del espacio se solapan las funciones de 
onda (interfieren) de dos partículas del mismo tipo cosa de que tengan misma masa, carga, etc. (dos 
electrones por ejemplo) no podemos distinguir cual es cual. 

Una tal situación se ilustra en la figura debajo; las dos situaciones son indistinguibles

\includegraphics[width=0.65\textwidth]{images/fig_ft2_identical_particles.jpg}

Sean dos estados $\Ket{k'},\Ket{k''}$ con $k^{(i)}$ índice colectivo. 
Si las tengo en una zona común tendré como estado total a $\Ket{k'} \otimes \Ket{k''} $
pero si las partículas son del mismo tipo, 
\[
	\Ket{k'} \otimes \Ket{k''} \qquad \Ket{k''} \otimes \Ket{k'},
\]
representan el mismo sistema cuántico y son ortogonales. No se pueden distinguir estos estados.
\notamargen{No sé si reescribí esta sección diferente a la carpeta porque lo mejoré o si es
equivalente. Se verá en su momento.}

En la zona de interferencia es 
\[
	\Ket{k'}_1 \otimes \Ket{k'}_2 \quad \text{o} \quad \Ket{k''}_1 \otimes \Ket{k''}_2
\]
donde ambos estados son ortogonales y los subíndices numéricos identifican a la partícula. 

\begin{figure}[htb]
	\begin{center}
	\includegraphics[width=0.9\textwidth]{images/teo2_29.pdf}
	\end{center}
	\caption{}
\end{figure} 

	

Entonces un estado general será
\[
	\Ket{K} = c_1 \Ket{k'}_1 \otimes \Ket{k''}_2 + c_2 \Ket{k''}_1 \otimes \Ket{k'}_2
\]
con $|c_1|^2 +|c_2|^2= 1$. 
Esta es la ``degeneración de intercambio'', y la reventaremos con un postulado extra que le
añadiremos a la mecánica cuántica.

Definiremos un operador permutación $P$ que intercambia kets en un producto tensorial.
Es decir que opera según 
\[
	P_{12}( \Ket{k'}_1\otimes \Ket{k''}_2 ) = \Ket{k''}_1\otimes \Ket{k'}_2
\]
y además satisface las siguientes propiedades
\[
	P_{12} = P_{21} \qquad P_{12}^2 = \mathbb{1} \qquad P_{12}^\dagger = P_{12} \qquad 
	P_{12}P_{12}^\dagger = 1 \qquad \text{autovalores:} \; \pm 1
\]

Su función es la de intercambiar etiquetas, no el orden de las partículas.
Sean operadores $\hat{A}_1,\hat{A}_2$ que actúan sobre las partículas 1,2; es decir 
\[
	\hat{A}_1 \equiv \hat{A}_1\otimes\mathbb{1}_2, \qquad 
	\hat{A}_2 \equiv \mathbb{1}_1\otimes\hat{A}_2
\]
Veamos qué sucede sobre autoestados y operadores
\[
	\hat{A}_1 \Ket{a'} \Ket{a''} = a' \Ket{a'} \Ket{a''} \qquad 
	\hat{A}_2 \Ket{a'} \Ket{a''} = a'' \Ket{a'} \Ket{a''} 
\]
\[
	P_{12}A_1P_{12}^{-1}P_{12}\Ket{a'} \Ket{a''} = P_{12} a' \Ket{a'}_1 \Ket{a''}_2 =
	a' \Ket{a''}_1 \Ket{a'}_2
\]
\[
	= P_{12}A_1P_{12}^{-1} \Ket{a''}_1 \Ket{a'}_2 = a' \Ket{a''}_1 \Ket{a'}_2
\]
\[
	= A_2 \Ket{a''}_1 \Ket{a'}_2 = a' \Ket{a''}_1 \Ket{a'}_2
\]
y
\[
	P_{12}\hat{A}_1P_{12}^{-1} = \hat{A}_2, \qquad P_{21} A_1 - A_2 P_{12} = 0
\]
\notamargen{La idea que tenía en la carpeta era la siguiente: si el operador es
simétrico, entonces conmuta con el operador de permutación $P$, no sé si quise
decir eso en las notas de final.}


Luego $\hat{A}$ es simétrico si $[\hat{P}_{12},\hat{A}_{12}]=0$. Sea $[\hat{P}_{12},\hat{H}]=0$ entonces es 
$P_{12}$ constante de movimiento y 
\[
	P_{12} \Ket{\alpha} = \pm \Ket{\alpha}
\]

Un operador que cumple lo de arriba es el hamiltoniano.
Para dos partículas será 
\[
	H = \frac{p_1^2}{2m_1} + \frac{p_2^2}{2m_2} + v(|x_1 - x_2|) + V_e(\vb{x}_1)+ V_e(\vb{x}_2)
\]
donde si las partículas son idénticas se puede hacer $m_1=m_2\equiv m$ y veo que se cumple que
$ [ H, P_{12} ] = 0 $ y $ P_{12} \Ket{\a} = \pm \Ket{\a} $
Defino ahora dos estados, simétrico y antisimétrico
\[
	\Ket{k' k''}_s = \frac{1}{\sqrt{2}}\left( \Ket{k'}_1\Ket{k''}_2 + \Ket{k''}_1\Ket{k'}_2 \right) \qquad 
	\Ket{k' k''}_a = \frac{1}{\sqrt{2}}\left( \Ket{k'}_1\Ket{k''}_2 - \Ket{k''}_1\Ket{k'}_2 \right)
\]
con 
\[
	P_{12}\Ket{\phantom{k}}_s = + \Ket{\phantom{k}}_s \qquad \qquad 
	P_{12}\Ket{\phantom{k}}_a = - \Ket{\phantom{k}}_a
\]
Puedo introducir operadores de simetrización y antisimetrización 
\[
	\hat{S}_{12} \equiv \frac{1}{\sqrt{2}} \left( \mathbb{1} + \hat{P}_{12} \right)
\]
\[
	\hat{A}_{12} \equiv \frac{1}{\sqrt{2}} \left( \mathbb{1} - \hat{P}_{12} \right)
\]
que verifican 
\[
	S^2 = S, \quad  A^2= A, \quad SA=AS=0, \quad [S,A] = 0,
\]
es decir que no son otra cosa que proyectores, 
\[
	\hat{S}_{12} (c_1\Ket{k'}\Ket{k''} + c_2\Ket{k''}\Ket{k'} ) = \frac{1}{\sqrt{2}}(c_1+c_2)
	(\Ket{k'}\Ket{k''} + \Ket{k''}\Ket{k'} )
\]
es simétrico y 
\[
	\hat{A}_{12} (c_1\Ket{k'}\Ket{k''} + c_2\Ket{k''}\Ket{k'} ) = \frac{1}{\sqrt{2}}(c_1-c_2)
	(\Ket{k'}\Ket{k''} - \Ket{k''}\Ket{k'} )
\]
es antisimétrico.
En general se complica bastante con más de dos partículas 
\[
	P_{ij}(\Ket{k'}_1\Ket{k''}_2...\Ket{k^i}_i...\Ket{k^j}_j...) =
	(\Ket{k'}_1\Ket{k''}_2...\Ket{k^j}_i...\Ket{k^i}_j...)
\]
pués tenemos 
\[
	[P_{ij},P_{k\ell}] \neq 0 \quad \text{en general}
\]
Las permutaciones para tres partículas pueden descomponerse en permutaciones de a dos,
como por ejemplo
\[
	P_{123} = P_{12}P_{13} 
\]
\[
	P_{123}\Ket{k'}\Ket{k''}\Ket{k'''} = P_{12}\Ket{k'''}\Ket{k''}\Ket{k'} = \Ket{k''}\Ket{k'''}\Ket{k'}
\]

Con tres partículas hay $3!$ estados; uno totalmente simétrico $\Ket{\phantom{k}}_s$, uno totalmente antisimétrico 
$\Ket{\phantom{k}}_a$ y cuatro sin simetría definida.
Los estados con simetría definida serán 
\begin{align*}
	\Ket{k'k''k'''}_{s/a} = \frac{1}{\sqrt{6}}&\left( \Ket{k'k''k'''} +  \Ket{k''k'''k'} + \Ket{k'''k'k''}\right. \\
	& \left. \pm \Ket{k''k'k'''} \pm \Ket{k'k'''k''} \pm \Ket{k'''k''k'} \right)
\end{align*}
donde el $\Ket{}_a$ tiene el signo $(-)$ en las permutaciones anticíclicas y el $(+)$ en las cíclicas.
Existe un determinante de Slater como método mnemotécnico de obtener los estados $\Ket{}_a$.
\[
	\Ket{ \Psi }_a = \frac{1}{3!}\begin{vmatrix} \; \Ket{k'} & \Ket{k''} & \Ket{k'''} \\  
	\; \Ket{k'} & \Ket{k''} & \Ket{k'''} \\ \; \Ket{k'} & \Ket{k''} & \Ket{k'''} \end{vmatrix}
\]
La obtención de estos estados corresponde a aplicar 
\[
	A_{123} = \frac{1}{\sqrt{3!}}\left( \mathbb{1} + P_{231} + P_{312} - P_{212} - P_{132} - P_{321} \right)
\]
\[
	( \mathbb{1} + P_{23}P_{21} + P_{31}P_{32} - P_{21}P_{23} - P_{13}P_{12} - P_{32}P_{31} )
\]
Si dos $k^{(i)}$ coinciden ya no hay estado antisimétrico posible.

\section{Postulado de simetrización}

Permitirá romper la degeneración de intercambio. 
Postulamos que toda partícula es de uno de dos tipos de acuerdo a su simetría 

\begin{center}
\begin{tabular}{|c|l|l|l|l|}
\multirow{2}{*}{Sistemas de N part. idénticas} & N & simetrica & BE & entero\\
& N & antisimetrica & FD & semientero
\end{tabular}
\end{center}

\begin{center}
\begin{tabular}{lll}
Función de onda & Estadística & Spin \\
\hline \\
Simétrica & Bosones $P_{ij}\Ket{ N \text{ bosones}} = 
+ \Ket{ N \text{ bosones}} $ & Entero\\
Antisimétrica & Fermiones $P_{ij}\Ket{ N \text{ fermiones}} = 
- \Ket{ N \text{ fermiones}} $ & Semi-entero
\end{tabular}
\end{center}

En la naturaleza no ocurren simetrías mixtas.

% \subsection{Principio de exclusión de Pauli}

Para fermiones, suponiendo un sistema de dos partículas idénticas, es
\[
	\Ket{ \Psi }_a = \frac{1}{\sqrt{2}}( \Ket{k'}_1\Ket{k''}_2 - \Ket{k''}_1\Ket{k'}_2)
\]
y entonces si $k'=k''$ se tiene que 
\[
	\Ket{ \Psi }_a = 0.
\]
de manera que no es posible tener dos fermiones con iguales números cuánticos.
Esto es el principio de exclusión de Pauli. 
Por el contrario los bosones sí pueden tener iguales números cuánticos.

\notamargen{En la carpeta hay un ejemplo que no entiendo en la p81 donde la moraleja
es que el hamiltoniano no tiene modo de vincular estados de bosones con estados de
fermiones. Habría que ver de entenderlo y juzgar luego si aporta introducirlo aquí.}


\subsection{Sistema de dos electrones de spin $1/2$}

Sistema de dos electrones de spin $1/2$, que son fermiones. Sea que el hamiltoniano
no depende del spin total y por ello $[H,S]=0$  con $S = S_1 + S_2$. Se tendrá 
\[
	\Ket{\Psi}^{sist} = \Ket{\Psi}^{spa} \otimes \Ket{\Psi}^{spin},
\]
pero como $\Ket{\Psi}^{sist}$ tiene que ser antisimétrica tendremos 
\[
	P_{12} \Ket{\Psi}^{sist} = - \Ket{\Psi}^{sist} 
\]
que, según la separación anterior, implica que 
\[
	P_{12} \Ket{\Psi}^{sist} = P_{12} \Ket{\Psi}^{spa} \otimes P_{12}\Ket{\Psi}^{spin} 
\]

No obstante, aún antes de saber lo de la parte espacial ya tengo información de la
parte de spin.
Para dos electrones con spin $1/2$ se tiene $j_1+j_2$ entonces $ 0 \leq j \leq 1 $ de modo que $|m_1|\leq j_1$ y
$|m_2|\leq j_2$ entonces $0 \leq S \leq 1$ y $|m_{s_1}|\leq s_1$ y $|m_{s_2}|\leq s_1$.
\[
	\left.\begin{aligned}
	& \Ket{\uparrow \uparrow} \\
	& \Ket{\downarrow \downarrow } \\
	& \frac{1}{\sqrt{2}}( \Ket{ \uparrow\downarrow }+\Ket{ \downarrow\uparrow} )
	\end{aligned}
	\right\} \; \text{triplete} \; s=1 \qquad \text{Estados simétricos}
\]
\[
	\left.\begin{aligned}
	\frac{1}{\sqrt{2}}( \Ket{\uparrow\downarrow} - \Ket{\downarrow\uparrow} )
	\end{aligned}
	\right\} \; \text{singlete} \; s=0 \qquad \text{Estados antisimétricos}
\]

Además se tiene $ [P_{12}, S_\pm] = 0 $ y obtengo los del triplete con la bajada y subida $S_\pm$,
de modo que todos están relacionados.
Luego, $ P_{12} \ket{\Psi}^{spin} $ será simétrico $S=1$ o antisimétrico $S=0$ y esto es 
independiente del hamiltoniano $H$ y viene de que pudimos separar parte espacial y spin.

Ahora bien, como simétrico por antisimétrico en $S\otimes A$ es antisimétrico
\[
	P_{12} \Ket{\Psi}_{sist} =
	P_{12}^{spa} \Ket{\Psi}_{spa} \otimes P_{12}^{spi} \Ket{\Psi}_{spin}
\]
y se tienen en cada caso $A \otimes S$ en $S=1$ o bien $S \otimes A$ en $S=0$ será
en el primer caso $\Ket{\Psi}_{spa}$ antisimétrica con $(-1)^\ell$ y $\ell$ impar 
y en el segundo caso $\Ket{\Psi}_{spa}$ simétrica con $(-1)^\ell$ y $\ell$ par. 


Entonces
\begin{gather*}
	\begin{aligned}
	s=0 \qquad \Rightarrow \; &\Ket{\Psi}^{spa} \text{es simétrica} \\
	s=1 \qquad \Rightarrow \; &\Ket{\Psi}^{spa} \text{es antisimétrica}
	\end{aligned}	
\end{gather*}	

Vistos desde el centro de masa dos electrones verifican que $ P_{12} = \Pi $ 
(el operador $P_{12}$ es paridad) y entonces
\notamargen{Dos electrones solo pueden acoplarse a impulso total par $\vb J = \vb L + \vb S$.
Acá hay un mix de explicaciones, evidentemente carpeta no coincidía con las notas.}
\[
	P_{12}\Ket{n\ell m} = (-1)^\ell \Ket{n\ell m}
\]
\[
	\ell \;\text{par} \; \rightarrow \Ket{\Psi}^{spa} = P_{12} \Ket{\Psi}^{spa} \qquad 
	\ell \;\text{impar} \; \rightarrow -\Ket{\Psi}^{spa} = P_{12} \Ket{\Psi}^{spa}
\]
Necesitaré $\ell$ par con $s=0$ entonces $\ell+s=j$ par. En cambio, si $\ell$ impar con $s=1$ entonces 
$\ell+s=j$ par. Dos electrones sólo se acoplan a momento total $j$ par.

Sean los siguientes estados 
\[
	\Ket{ \Psi }_{sa} = \frac{1}{\sqrt{2}}( \Ket{k'}\Ket{k''} \pm \Ket{k''}\Ket{k'})
\]
\[
	\Ket{ \Psi_F }_{sa} = \frac{1}{\sqrt{2}}( \Ket{a'}\Ket{a''} \pm \Ket{a''}\Ket{a'})
\]
y la probabilidad será
\[
	\text{Prob}\; = |_{sa}\Braket{\Psi|\Psi}_{sa}|^2 = 
	\left|\frac{1}{2}(_1\Bra{a'}_2\Bra{a''} \pm _1\Bra{a''}_2\Bra{a'})(\Ket{k'}_1\Ket{k''}_2 \pm 
	\Ket{k''}_1\Ket{k'}_2)\right|^2 =
\]
\[
	= \frac{1}{4}\left| \Bra{a'}\Bra{a''}\Ket{k'}\Ket{k''} \pm \Bra{a''}\Bra{a'}\Ket{k'}\Ket{k''} 
	\pm \Bra{a'}\Bra{a''}\Ket{k''}\Ket{k'} \pm \Bra{a''}\Bra{a'}\Ket{k''}_1\Ket{k'} \right|^2
\]
\[
	= \frac{1}{4}\left| 2\Braket{a'|k'}\Braket{a''|k''} \pm 2\Braket{a''|k'}\Braket{a'|k''}\right|^2
\]
\[
	= \left| \underbrace{\Braket{a'|k'}\Braket{a''|k''}}_{\text{término directo}} \pm 
	\underbrace{\Braket{a''|k'}\Braket{a'|k''}}_\text{término de intercambio}\right|^2
\]
\begin{multline*}
	\text{Prob}\; = |_{sa}\Braket{\Psi|\Psi}_{sa}|^2 = \left| \Braket{a'|k'}\Braket{a''|k''} \right|^2 + 
	\left|\Braket{a''|k'}\Braket{a'|k''}\right|^2 \\
	\pm 2\mathcal{Re}\left(\underbrace{ \Braket{a'|k'}\Braket{a'|k''}^*\Braket{a''|k''}\Braket{a''|k'}^* 
	}_\text{Interferencia}\right)
\end{multline*}

Vemos que aparece una interferencia que será importante solamente si hay solapamiento. En el caso de no 
solaparse o con partículas clásicas solo el primer término es de importancia.

\includegraphics[width=0.4\textwidth]{images/fig_ft2_extra_identical.jpg}


\section{El átomo de helio}


\begin{figure}[htb]
	\begin{center}
	\includegraphics[width=0.4\textwidth]{images/teo2_30.pdf}
	\end{center}
	\caption{}
\end{figure} 
\[
	H = \frac{p_1}{2m} + \frac{p_2}{2m} - \frac{2e^2}{r_1} -  \frac{2e^2}{r_2} + \frac{e^2}{r_{12}}
\]
y si el último término es $\sim 0$ decimos que en ese caso $H$ está desacoplado 
\[
	\Psi = \Psi_1 \otimes \Psi_2
\]
\[
	[\vb{H},\vb{S}] = 0 \qquad \vb{S} = \vb{S}_1 + \vb{S}_2 = \begin{cases} 0 \\ 1 \end{cases}
\]
S es constante de movimiento y para la $\Ket{\psi_{spin}}$ se tiene 
\[
	S=0 \qquad \frac{1}{\sqrt{2}}( \Ket{ \uparrow\downarrow }-\Ket{ \downarrow\uparrow}) \quad \text{singlete}
\]
\[
	S=1 \qquad \begin{aligned}
	& \Ket{\uparrow \uparrow} \\
	& \Ket{\downarrow \downarrow } \\
	& \frac{1}{\sqrt{2}}( \Ket{ \uparrow\downarrow }+\Ket{ \downarrow\uparrow} )
	\end{aligned} \quad \text{triplete}
\]
Sea $ e^-_1 \Ket{100}$ y $e^-_2 \Ket{n\ell m}$
\[
	\Ket{\Psi}_{He} = \frac{1}{\sqrt{2}}\left(  \Ket{100}\Ket{n\ell m} \pm  \Ket{n\ell m}\Ket{100}
	\right) \Ket{\Psi_{spin}}
\]
de modo que con $S=0$ será
\[
	\Ket{\Psi}_{He} = \frac{1}{\sqrt{2}}\left( \Ket{100}\Ket{n\ell m} + \Ket{n\ell m}\Ket{100}\right) 
	\frac{1}{\sqrt{2}}( \Ket{ \uparrow\downarrow }-\Ket{ \downarrow\uparrow})
\]
y en cambio con $S=1$
\[
	\Ket{\Psi}_{He} = \frac{1}{\sqrt{2}}\left(  \Ket{100}\Ket{n\ell m} - \Ket{n\ell m}\Ket{100}\right) 
			\begin{cases}
	                   \Ket{\uparrow \uparrow} \\
			   \Ket{\downarrow \downarrow } \\
			   \frac{1}{\sqrt{2}}( \Ket{ \uparrow\downarrow }+\Ket{ \downarrow\uparrow} )
	                  \end{cases}
\]

Podemos pensar en teoria de perturbaciones ahora y calcular 
\[
	E_{He} = E_{100} + E_{n\ell m} + \Delta E
\]
donde 
\[
	\Delta E = \Braket{\Psi|\frac{e}{r_{12}}|\Psi}
\]
y el término en el {\it sandwich} lo considero una perturbación.
\[
	\Delta E = \Bra{\Psi^{spin}}^\dagger \frac{1}{2}\left(\Bra{100}\Bra{n\ell m} \pm \Bra{n\ell m}\Bra{100}
	\right) \frac{e}{r_{12}} \left(\Ket{100}\Ket{n\ell m} \pm \Ket{n\ell m}\Ket{100}\right)\Ket{\Psi^{spin}}
\]
\[
	\Delta E = \Bra{100}\Bra{n\ell m}\frac{e}{r_{12}}\Ket{100}\Ket{n\ell m} \pm  
	\Bra{n\ell m}\Bra{100}\frac{e}{r_{12}}\Ket{100}\Ket{n\ell m}
\]
que se puede escribir más resumidamente como
\[
	\Delta E = I \pm J
\]

\begin{figure}[htb]
	\begin{center}
	\includegraphics[width=1.0\textwidth]{images/teo2_15.pdf}
	\end{center}
	\caption{}
\end{figure} 

Esta separación de los niveles en $\pm J$ se debe al carácter de fermión de las partículas.

\includegraphics[width=0.4\textwidth]{images/fig_ft2_extra_identical2.jpg}

% \bibliographystyle{CBFT-apa-good}	% (uses file "apa-good.bst")
% \bibliography{CBFT.Referencias} % La base de datos bibliográfica

\end{document}
