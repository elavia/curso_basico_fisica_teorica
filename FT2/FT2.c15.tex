	\documentclass[10pt,oneside]{CBFT_book}
	% Algunos paquetes
	\usepackage{amssymb}
	\usepackage{amsmath}
	\usepackage{graphicx}
% 	\usepackage{libertine}
% 	\usepackage[bold-style=TeX]{unicode-math}
	\usepackage{lipsum}

	\usepackage{natbib}
	\setcitestyle{square}

	\usepackage{polyglossia}
	\setdefaultlanguage{spanish}
	



	\usepackage{CBFT.estilo} % Cargo la hoja de estilo

	% Tipografías
	% \setromanfont[Mapping=tex-text]{Linux Libertine O}
	% \setsansfont[Mapping=tex-text]{DejaVu Sans}
	% \setmonofont[Mapping=tex-text]{DejaVu Sans Mono}

	%===================================================================
	%	DOCUMENTO PROPIAMENTE DICHO
	%===================================================================

\begin{document}

% =================================================================================================
\chapter{Introducción a la mecánica cuántica relativista}
% =================================================================================================

Serán útiles los siguientes cuadrivectores de la relatividad especial
\[
	p^\mu = (E/c, \vb{p}) \qquad p_\mu = (E/c, -\vb{p}) 
	\qquad x_\mu = (ct, -\vbx ) \qquad x^\mu = (ct, \vb{x})
\]
\[
	\dpar{}{x^\mu} = \left( \frac{1}{c}\dpar{}{t} , \Nabla \right) \equiv \partial_\mu
	\qquad \dpar{}{x_\mu} = \left( \frac{1}{c}\dpar{}{t} , -\Nabla \right) \equiv \partial^\mu
\]
y las contracciones
\[
	p^\mu p_\mu  = \frac{E^2}{c^2} - p^2 = m^2 c^2 
	\qquad 
	x^\mu x_\mu  = c^2 t^2 - x^2 
\]
donde la última igualdad en la primera contracción es para una única partícula libre, que es lo que
consideramos por el momento.

De lo que sabemos en mecánica cuántica tenemos 
\be
	H = \frac{p^2}{2m} 
	\label{ham_libre_clasico}
\ee
\be
	E =  i\hbar \dpar{}{t} \quad \vb{p} = - i\hbar \Nabla
	\label{em_relativ}
\ee
La cuántica es de por sí covariante (susceptible de escribirse en términos de cuadrivectores)
y vemos que la siguiente ecuación
\[
	P_\mu = i \hbar \partial_\mu = i \hbar \dpar{}{x^\mu}
\]
condensa las dos en \eqref{em_relativ}.
Veremos luego que en relatividad la expresión \eqref{ham_libre_clasico} no sirve y tendremos que
usar \eqref{em_relativ}.

Paritmos de la ecuación de Schrödinger para la partícula libre, que es
\be 
	i \hbar \dpar{\psi}{t} = - \frac{\hbar^2}{2m} \nabla^2 \psi
	\label{schro_freepart}
\ee
y entonces podemos hacer la cuenta 
\[
	\psi^* \times \eqref{schro_freepart} \rightarrow  
	i \hbar \psi^*\dpar{\psi}{t} = - \frac{\hbar^2}{2m} \psi^* \nabla^2 \psi
\]
y conjugando la ecuación,
\[
	\psi \times \eqref{schro_freepart}^* \rightarrow  
	-i \hbar \psi\dpar{\psi^*}{t} = - \frac{\hbar^2}{2m} \psi \nabla^2 \psi^*
\]
y restando ambas expresiones se obtiene 
\[
	i \hbar \left( \psi^*\dpar{\psi}{t} + \psi\dpar{\psi^*}{t} \right) =
	\frac{\hbar^2}{2m} \left( \psi \nabla^2 \psi^* -\psi^* \nabla^2 \psi \right)
\]
\[
	i \hbar  \dpar{ (\psi^* \psi) }{t} + \frac{\hbar^2}{2m} 
	\Nabla \cdot \left( \psi^* \Nabla \psi - \psi \Nabla \psi^* \right) = 0
\]
la cual se puede reescribir como
\[
	\dpar{ (\psi^* \psi) }{t} +  
	\Nabla \cdot \left( \frac{\hbar}{2mi} [\psi^* \Nabla \psi - \psi \Nabla \psi^*] \right) = 0
\]
que es una analogía de la conservación de la carga en electrodinámica. 
Recordemos que la conservación de la carga era
$ \partial_t \rho + \nabla \cdot \vb{J} = 0$.
Entonces, la ecuación de Schrödinger se puede interpretar como una especie de conservación de la probabilidad
en el tiempo. Notemos que $\psi^* \psi = |\psi|^2 \geq 0$.
Todo esto parece bastante razonable y es parte del fundamento de las hipótesis básicas de la
mecáncia cuántica.

Tratando de llevar la \eqref{ham_libre_clasico} a algo relativista podemos pensar en las expresiones
correspondientes, que son
\[
	E^2 = c^2 p^2 + m^2 c^4
\]
\[
	E = \sqrt{ c^2 p^2 + m^2 c^4 } = H \qquad \text{con} \; H\psi = E\psi
\]

Pero esto se pone muy complicado debido a la raíz. Para evitarla se puede considerar directamente
el cuadrado.
Entonces,
\[
	H^2 = E^2 = c^2p^2 + m^2c^4
\]
lo cual nos lleva a la ecuación
\be 
	-\hbar \dpar[2]{\psi}{t} = - \hbar^2c^2 \Nabla^2\psi + m^2 c^4 \psi
	\label{ec_kg}
\ee
que es la llamada ecuación de Klein-Gordon.
En términos de las contracciones de cuadrivectores 
\[
	p^\mu p_\mu = m^2c^2 \qquad -\partial_\mu\partial^\mu \psi = \frac{m^2c^2}{\hbar^2}\psi,
\]
donde el operador $\Box^2 \equiv \partial_\mu\partial^\mu$ es el dalembertiano, se puede poner
como
\[
	\left( \Box^2 + \frac{m^2c^2}{\hbar^2} \right) \psi = 0.
\]
Ahora, procediendo de modo ídem al caso anterior, para construirme una corriente coservada
se tienen
\[
	\psi^* \cdot \eqref{ec_kg} = - \hbar^2 \psi^* \partial_t^2 \psi =
	- \hbar^2 c^2 \psi^* \nabla^2 \psi + m^2 c^4 \psi^*\psi
\]
\[
	\psi \cdot \eqref{ec_kg}^* = - \hbar^2 \psi \partial_t^2 \psi^* =
	- \hbar^2 c^2 \psi \nabla^2 \psi^* + m^2 c^4 \psi\psi^*
\]
y restando ambas ecuaciones tenemos
\[
	\hbar^2 \dpar{}{t}\left(  \psi^* \partial_t \psi -  \psi \partial_t \psi^* \right) =
	\hbar^2 c^2 \Nabla\cdot( \psi^* \Nabla \psi - \psi \Nabla \psi^* )
\]
\[
	\dpar{}{t}\left( \frac{i}{c^2}[ \psi^* \partial_t \psi -  \psi \partial_t \psi^* ]\right) +
	i\Nabla\cdot( \psi \Nabla \psi^* - \psi^* \Nabla \psi ) = 0
\]

El problema es que no puede asegurarse que esta $\rho \equiv i/c^2[ \psi^* \partial_t \psi -  \psi \partial_t 
\psi^* ]$ sea definida positiva, lo cual sería necesario para seguir una coherencia.

Soluciones tipo onda plana de la anterior ecuación serían
\[
	\psi = N \euler^{ i/\hbar(\pe{p}{x} - Et)},
\]
cuya derivada temporal es
\[
	\partial_t \psi = -N \frac{iE}{\hbar} \euler^{ i/\hbar(\pe{p}{x} - Et)},
\]
lo cual conduce a la densidad
\begin{multline*}
	\rho = \frac{i}{c^2}\left( N^*\euler^{ -i/\hbar(\pe{p}{x} - Et)}(-N)
	\frac{iE}{\hbar} \euler^{ i/\hbar(\pe{p}{x} - Et)} - \right. \\ 
	\left. N \euler^{ i/\hbar(\pe{p}{x} - Et)}N^*\frac{E}{\hbar} 
	\euler^{ i/\hbar(\pe{p}{x} - Et)}\euler^{i\pi/2}
	\right) 
\end{multline*}
o bien
\[
	\rho = -\frac{i}{c^2}\left( 2|N|^2 \frac{iE}{\hbar} \right) < 0 
	\qquad \text{si} \quad E > 0
\]
para una onda plana.
Necesito considerar $E<0$ pues $E=\pm\sqrt{c^2p^2+m^2c^4}$ y la base debe ser completa.
Las energías negativas no son un problema, pero lo que sí es un problema es
\[
	\rho = \frac{2E}{\hbar c^2} \Psi \Psi^*
\]
donde $ \Psi \Psi^* \geq 0 $ lo cual lleva a densidad de probabilidad negativa y esto es
difícil de aceptar proque matemáticamente es incoherente.
Además si hay energías negativas, nunca puede alcanzarse un estado fundamental y la materia
sería inestable.

La densidad $\rho$ es positiva si tuviese $E<0$ pero esto causa el problema de tener materia
inestable, pues nunca se alcanza el fundamental. 
Acá muere, en este atolladero, la ecuación de Klein-Gordon.

\subsection{La ecuación de Dirac}

Dirac retoma la densidad negativa y por medio de un trick matemático partiendo de 
\[
	H \Psi = i \hbar \dpar{\Psi}{t}
\]
con $H^2 = E^2 = c^2p^2 + m^2c^4$ pide una ecuación lineal en el impulso \vb{p} con lo cual
\[
	H = c \: \vb{\alpha} \cdot \vb{p} + \beta m c^2
\]
usando $H\psi = E\psi$ y con $\beta,\vb{\alpha},\vb{p}$ operadores (de modo que será cuidadoso
con la no conmutatividad). Así
\[
	H^2 = (c \vb{\alpha} \cdot \vb{p} + \beta m c^2)(c \vb{\alpha} \cdot \vb{p} + \beta m c^2)
\]
\[
	H^2 = c^2 \alpha_i p_i \alpha_\ell p_\ell + c^3 \alpha_i p_i \beta m +
	\beta m c^3 \alpha_i p_i + \beta^2 m^2 c^4
\]
\[
	H^2 = c^2 \alpha_i \alpha_\ell p_i  p_\ell + c^3 m p_i 
	\underbrace{( \alpha_i \beta + \beta \alpha_i )}_{=0} + \beta^2 m^2 c^4
\]
\[
	H^2 = 
	c^2 \underbrace{\left( \frac{\alpha_i \alpha_\ell +
	\alpha_\ell\alpha_i}{2} \right)}_{\delta_{i\ell}}
	p_i p_\ell + m^2 c^4 \underbrace{\beta^2 }_{ = 1 }
\]
\[
	\alpha_i \alpha_\ell + \alpha_\ell\alpha_i = 2 \delta_{i\ell} \qquad 
	\alpha_i \beta + \beta \alpha_i= 0 \qquad
	\beta^2 = 1
\]
Como se ve, estos no pueden ser simples escalares. 
Intentamos buscar los requisitos para que la ecuación converja al resultado relativista de
la energía $E^2$. 
Para ello, Dirac pide 
\begin{itemize}
 \item $\vb{\alpha},\beta$ hermíticos
 \item $\beta^2=1 \; \alpha^2=1$ autovalores $\pm 1$
 \item traza nula
 \[
	\alpha_i \beta = -\beta \alpha_i  \quad \rightarrow  \quad 
	\beta \alpha_i \beta = -\beta^2 \alpha_i = -\alpha_i
 \]
 \[
	Tr(\alpha_i) = -Tr(\beta \alpha_i \beta) = -Tr(\beta\beta\alpha_i)
 \]
 \item dimensión par 
 \[
	\vb{\alpha} = \begin{pmatrix} 0 & \vec{\sigma} \\ \vec{\sigma} & 0 \\ \end{pmatrix} \qquad 
	\beta = \begin{pmatrix} \mathbb{1} & 0 \\ 0 & -\mathbb{1} \\ \end{pmatrix}
 \]
 donde cada elemento de la matriz es de $2\times2$ porque $\sigma$ son las matrices de Pauli.
\end{itemize}

Por alguna oscura razón no alcanza solamente con las matrices de Pauli, entonces se prueba con las
matrices anteriores $\alpha$ y $\beta$, que son de $4\times 4$. La función de onda será vectorial
ahora.
Entonces,
\[
	H \vec{\psi} = i \hbar \dpar{\vec{\psi}}{t}, 
	\qquad H \in 4\times 4, \vec{\psi} \in 4\times 1, \qquad
	\vec{\psi} = \begin{pmatrix} 
			\psi_1 \\
			\psi_2 \\
			\psi_3 \\
			\psi_4
		\end{pmatrix}
\]
y así podemos llegar a una ecuación válida relativísticamente hablando
\be \label{ec_dirac}
	i \hbar \dpar{\psi}{t} = -i \hbar c \sum_k \alpha_k \dpar{\psi}{x_k} + m c^2 \beta \psi
\ee
Tomándole daga y haciendo la evaluación de la densidad,
\[
	-i \hbar \dpar{\psi^\dagger}{t} = i \hbar c \sum_k \dpar{\psi^\dagger}{x_k}\alpha_k + 
	m c^2\psi\alpha_k\beta
\]
\[
	\psi^\dagger\cdot\eqref{ec_dirac} - \eqref{ec_dirac}^\dagger\cdot \psi \rightarrow 
	i\hbar \dpar{}{t}(\psi^\dagger psi) = -i\hbar c \sum_k \dpar{}{x_k}(\psi^\dagger \alpha_k \psi)
\]
\[
	\dpar{}{t}(\psi^\dagger \psi) + c \sum_k \dpar{}{x_k}(\psi^\dagger \alpha_k \psi) = 0
\]
Y si $\rho \equiv \psi^\dagger\psi$ ahora tenemos una densidad de probabilidad como requiere la naturaleza.
Todo esto es para la partícula libre.

\subsection{Ejemplo: partícula libre quieta}

Sea una partícula libre en reposo,
\[
	\vb{p} = 0 \qquad H=\beta m c^2
\]
\[
	i \hbar \dpar{\psi}{t} = \beta m c^2 \psi
\]
\[
	i \hbar \dpar{}{t} \begin{pmatrix} \psi_1 \\ \psi_2 \\ \psi_3 \\ \psi_4  \end{pmatrix} =
	\begin{pmatrix}
	mc^2 & 0 & 0 & 0 \\ 
	0 & mc^2 & 0 & 0 \\ 
	0 & 0 & -mc^2 & 0 \\ 
	0 & 0 & 0 & -mc^2 
	\end{pmatrix}
	\begin{pmatrix}
	\psi_1 \\ 
	\psi_2 \\ 
	\psi_3 \\ 
	\psi_4  
	\end{pmatrix}
\]
Tenemos cuatro ecuaciones, dos con energía positiva y dos con energía negativa
\[
	i \hbar \dpar{\psi_i}{t} = mc^2 \psi_i \qquad i \hbar \dpar{\psi_i}{t} = -mc^2 \psi_i
\]
donde hemos de remarcar que no nos hemos desecho de las energías negativas,
\[
	\psi_1 = \euler^{-imc^2t/\hbar}\begin{pmatrix} 1 \\ 0 \\ 0 \\ 0  \end{pmatrix} \qquad 
	\psi_3 = \euler^{imc^2t/\hbar}\begin{pmatrix} 0 \\ 0 \\ 1 \\ 0  \end{pmatrix}
\]
\[
	\psi_2 = \euler^{-imc^2t/\hbar}\begin{pmatrix} 0 \\ 1 \\ 0 \\ 0  \end{pmatrix} \qquad 
	\psi_4 = \euler^{imc^2t/\hbar}\begin{pmatrix} 0 \\ 0 \\ 0 \\ 1  \end{pmatrix}
\]
Como aún tenemos degeneración de orden dos, necesitaremos un operador que conmute con el 
hamiltoniano $H$ para discriminar entre estados. Se define
\[
	\vec{\Sigma} = \begin{pmatrix} \vec{\sigma} & 0 \\ 0 & \vec{\sigma} \\ \end{pmatrix} \qquad 
	[H,\vec{\Sigma}] = 0
\]
\[
	\Sigma_3 =  \begin{pmatrix} \sigma_3 & 0 \\ 0 & \sigma_3 \\ \end{pmatrix} = 
	\begin{pmatrix} 1 & 0 & 0 & 0 \\ 0 & -1 & 0 & 0 \\ 0 & 0 & 1 & 0 \\ 0 & 0 & 0 & -1 \end{pmatrix}
\]
\begin{align*}
	\psi_1, E=mc^2, \Sigma_3=1 \qquad  &\psi_2, E=mc^2, \Sigma_3=-1 \\
	\psi_3, -E=mc^2, \Sigma_3=1 \qquad  &\psi_4, -E=mc^2, \Sigma_3=-1
\end{align*}
Podemos identificar 
\[
	\vec{S} = \frac{\hbar}{2} \vec{\Sigma}
\]
de modo que tenemos dos estados para energía positiva $ E > 0 $ que son $ S_z = \pm \hbar/2 $ y
dos estados con energía negativa $ E < 0 $ con $S_z = \pm \hbar / 2 $; es decir un total de cuatro
combinaciones diferentes.

Luego, si $p \neq 0$ entonces
\[
	[H, \vb{\Sigma}] = 2 i c \: \vb{\alpha} \times \vb{p}
\]
y como $\vb L = \vbx \times \vb p $ se tiene
\[
	[ H, \vb L] = - i \hbar c \: \vb{\alpha} \times \vb{p}
\]
pero, utilizando el $\vb S$ previamente definido,
\[
	[ H, \vb L + \frac{\hbar}{2}\vb{\Sigma} ] =  0.
\]
Esta ecuación da el spin correcto, el factor giromagnético de Landé, etc. Todo ello metido en la misma
desde el vamos.

\subsection{Energías negativas}

Veamos ahora el asunto de las energías negativas.

Como 
\[
	E = \pm  \sqrt{ c^2 p^2 + m^2 c^4 }
\]
hay $E<0$ y además un {\it gap} de ancho $2mc^2$ entre ellas. Veamos la ilustración aqui abajo.

\includegraphics[width=0.5\textwidth]{images/fig_ft2_energias_negativas_1.jpg}

El problema es que dado que la materia es estable, en este caso buscaría su grado de estabilidad mayor,
decaer a energía menor, y no tendría límite puesto que siempre puede ir a un nivel de energía menor; pero
nunca quedaría estable.
Las $E<0$ harían que la materia jamás alcance un estado fundamental y por ende jamás se estabilice.
Dirac piensa que los estados de $E<0$ están todos llenos. No decaen más electrones allí dentro. 
Este vacío lleno se llamó {\it mar de Dirac}. Iluminando ese vacío se lo puede excitar.

\includegraphics[width=0.4\textwidth]{images/fig_ft2_energias_negativas_2.jpg}
\includegraphics[width=0.6\textwidth]{images/teo2_14.pdf}

\begin{figure}[htb]
	\begin{center}
	\includegraphics[width=0.6\textwidth]{images/teo2_13.pdf}
	\end{center}
	\caption{}
\end{figure} 

Podemos hacer saltar a la zona positiva una carga $(-e)$ dejando un huevo positivo (equivalente a una carga 
$+e$). Es una creación de pares $\gamma \to e^-e^+$, sin embargo el proceso inverso $ e^-e^+ \to \gamma$ de 
aniquilación de pares ocurre prontamente.

El agujero se llena rápidamente con un electrón de $ E > 0 $ en un proceso de aniquilación de pares.

\includegraphics[width=0.6\textwidth]{images/fig_ft2_energias_negativas_3.jpg}


Se observó experimentalmente.
Sin embargo, esto es para fermiones únicamente.
Los bosones no obedecen principio de exclusión entonces no esperaríamos formar materia estable con bosones.
Feynman interpreta energías negativas como partículas yendo hacia atrás en el tiempo.


\begin{ejemplo}{\bf Ejercicio surtido 1}
 
Consideramos tres fermiones en un pozo de potencial 1D de longitud L. Son fermiones imaginarios sin spin.
\[
	V = \begin{cases}
	     0      & 0 \geq x \geq L \\
	     \infty & \text{Otro lugar }
	    \end{cases}
\]
Cada partícula se describe con un número $n$ de energía
\[
	\Ket{\vp_n(x)} = \sqrt{ \frac{2}{L} } \sin\Frac{n\pi x}{L} \qquad 
	E_n = \frac{\pi^2\hbar^2}{2 m L^2} n^2 
\]
y como los fermiones no tienen que estar en un mismo nivel de energía, tendremos para el nivel
fundamental posibilidades como
\[
	E_0 = \frac{\pi^2\hbar^2}{2 m L^2} ( 1^2 + 2^2 + 3^2 )
\]
de manera que 
\[
	\Ket{\Psi} = \frac{1}{\sqrt{N!}} \sum_{\a} \mathcal{E}_\a P_\a \Ket{\Psi} =
	\frac{1}{\sqrt{N!}}
	\begin{pmatrix}
	\Ket{1;\vp_1} & \Ket{1;\vp_2} & \Ket{1;\vp_3} \\
	\Ket{2;\vp_1} & \Ket{2;\vp_2} & \Ket{2;\vp_3} \\
	\Ket{3;\vp_1} & \Ket{3;\vp_2} & \Ket{3;\vp_3} 
	\end{pmatrix}
\]
que es el determinante de Slater (una fila por partícula) y donde 
\[
	\mathcal{E}_\a = \begin{cases}
	 1  & \a \; \text{par} \\
	 -1 & \a \; \text{impar} 
	\end{cases}
\]
Entonces,
\begin{multline*}
	\Ket{\Psi} = \frac{1}{\sqrt{ 3! }} 
	\left[ 
	\Ket{ 1, \vp_1 ; 2 \vp_2, 3 \vp_3} + \Ket{ 1, \vp_3 ; 2 \vp_1, 3 \vp_2} + 
	\Ket{ 1, \vp_2 ; 2 \vp_3, 3 \vp_1}  \right. \\
	\left. - \Ket{ 1, \vp_3 ; 2 \vp_2, 3 \vp_1} - \Ket{ 1, \vp_1 ; 2 \vp_3, 3 \vp_2} -
	\Ket{ 1, \vp_2 ; 2 \vp_1, 3 \vp_3}
	\right] 
\end{multline*}
donde el signo menos es por número de permutaciones impar (anticíclico) y el signo más por un número
par de permutaciones (cíclico).

Ahora supongamos que las partículas son electrones, en lugar de fermiones. Sea $s=1/2$,
\[
	\Ket{\vp + } = \sqrt{\frac{2}{L}} \: \sin\Frac{n \pi x}{L} \begin{pmatrix}
	                                                            1 \\
	                                                            0
	                                                           \end{pmatrix}
\]
\[
	\Ket{\vp - } = \sqrt{\frac{2}{L}} \: \sin\Frac{n \pi x}{L} \begin{pmatrix}
	                                                            0 \\
	                                                            1
	                                                           \end{pmatrix}
\] 

\includegraphics[width=0.6\textwidth]{images/fig_ft2_ubicacion_niveles.jpg}

Luego, para la izquierda
\[
	\Ket{\Psi_A} =
	\frac{1}{\sqrt{N!}}
	\begin{pmatrix}
	\Ket{1;\vp_1 +} & \Ket{1;\vp_1 -} & \Ket{1;\vp_2 +} \\
	\Ket{2;\vp_1 +} & \Ket{2;\vp_1 -} & \Ket{2;\vp_2 +} \\
	\Ket{3;\vp_1 +} & \Ket{3;\vp_1 -} & \Ket{3;\vp_2 +} 
	\end{pmatrix}
\]
y para la derecha
\[
	\Ket{\Psi_A} =
	\frac{1}{\sqrt{N!}}
	\begin{pmatrix}
	\Ket{1;\vp_1 +} & \Ket{1;\vp_1 -} & \Ket{1;\vp_2 -} \\
	\Ket{2;\vp_1 +} & \Ket{2;\vp_1 -} & \Ket{2;\vp_2 -} \\
	\Ket{3;\vp_1 +} & \Ket{3;\vp_1 -} & \Ket{3;\vp_2 -} 
	\end{pmatrix}
\]
donde hay que aclarar que aquí no hemos factorizado.

\end{ejemplo}

\begin{ejemplo}{\bf Ejercicio surtido 2}

Sean dos orbitales espaciales ortonormales y los estados de spin siguientes
$\Ket{\vp}, \Ket{\xi}$ ortonormales y $\Ket{+}, \Ket{-}$.
En la parte a) tenemos dos electrones $\Ket{\vp +}, \Ket{\xi -}$ y hay que calcular la probabilidad de
hallar $\rho_{11}(\vbx,\vbx') d^3x d^3x'$ pero como son dos electrones el sistema se
hallará en un autoestado antisimétrico.
\[
	\Ket{\Psi_A} = \frac{1}{\sqrt{2}}
	\begin{pmatrix}
		\Ket{1 \vp +} & \Ket{1 \xi -} \\
		\Ket{2 \vp +} & \Ket{2 \xi -}
	\end{pmatrix}
\]
\[
	\Ket{\Psi_A} = \frac{1}{\sqrt{2}}
	\left(  \Ket{1 \vp +} \Ket{2 \xi -} - \Ket{1 \xi -}\Ket{2 \vp +}
	\right)
\]
y la densidad de probabilidad es
\[
	\rho = | \Braket{ \vbx \vbx' | \Psi_A } |^2 = \Psi^*_A(\vbx,\vbx') \Psi_A(\vbx,\vbx')
\]
que descomponiendo en términos es algo como
\begin{multline*}
 \frac{1}{2} \left[ \Bra{1 \vp^*(\vbx) + } \Braket{ 2 \xi^*(\vbx') - | 1 \vp(\vbx) + } 
 \Ket{ 2 \xi(\vbx') - }  \right. + \\
 \left. \Bra{1 \xi^*(\vbx)} \Bra{-} \Bra{2 \vp^*(\vbx')} \Braket{ + | 1 \xi(\vbx) } \Ket{-}
	\Ket{2 \vp(\vbx') + } + 0 \right]
\end{multline*}
que resulta
\[
	\frac{1}{2} \left[
	| \vp(\vbx) |^2 | \xi(\vbx') |^2 + | \vp(\vbx') |^2 | \xi(\vbx) |^2 + 0 
	\right] 
\]
donde la ortogonalidad de los spines hace nulo el término de interferencia.

En la parte b) hay que calcular
\[
	\rho_I(\vbx) d^3x = \int_{-\infty}^\infty d^3x \rho_{II}(\vbx,\vbx') =
	\frac{1}{2}( |\vp(\vbx)|^2 + |\xi(\vbx)|^2).
\]
obteniéndose el último término por ortonormalización.

Para la parte c) si no son ortogonales las funciones siguen valiendo, porque los términos que
hemos matado lo hemos hecho merced a ortogonalidad de spin.

La parte d) ahora suponemos que los electrones con estado de spin
\[
	\Ket{\Psi_A} = \frac{1}{\sqrt{2}}
	\begin{pmatrix}
		\Ket{1 \vp +} & \Ket{1 \xi +} \\
		\Ket{2 \vp +} & \Ket{2 \xi +}
	\end{pmatrix}
\]
\[
	\Ket{\Psi_A} = \frac{1}{\sqrt{2}}
	\left(  \Ket{1 \vp +} \Ket{2 \xi +} - \Ket{1 \xi +}\Ket{2 \vp +}
	\right)
\]
y consecuentemente
\[
	\rho_{II}(x,x')d^3xd^3x' = \frac{1}{2}
	\left[ 
	| \vp(\vbx) |^2 | \xi(\vbx') |^2 + | \vp(\vbx') |^2 | \xi(\vbx) |^2
	- 2 \mathfrak{Re}( \vp(\vbx)\vp(\vbx')^*\xi(\vbx)^*\xi(\vbx')  )
	\right]
\]
tendremos los dos términos anteriores más el de interferencia, que no se anula ahora.
Luego, al integrar por ortogonalidad se nula la interferencia y $\rho_I(\vbx)$ da lo
mismo que antes.

\end{ejemplo}





% \bibliographystyle{CBFT-apa-good}	% (uses file "apa-good.bst")
% \bibliography{CBFT.Referencias} % La base de datos bibliográfica

\end{document}
