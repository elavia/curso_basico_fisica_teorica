	\documentclass[10pt,oneside]{CBFT_book}
	% Algunos paquetes
	\usepackage{amssymb}
	\usepackage{amsmath}
	\usepackage{graphicx}
	\usepackage{libertine}
	\usepackage[bold-style=TeX]{unicode-math}
	\usepackage{lipsum}

	\usepackage{natbib}
	\setcitestyle{square}

	\usepackage{polyglossia}
	\setdefaultlanguage{spanish}
	



	\usepackage{CBFT.estilo} % Cargo la hoja de estilo

	% Tipografías
	% \setromanfont[Mapping=tex-text]{Linux Libertine O}
	% \setsansfont[Mapping=tex-text]{DejaVu Sans}
	% \setmonofont[Mapping=tex-text]{DejaVu Sans Mono}

	%===================================================================
	%	DOCUMENTO PROPIAMENTE DICHO
	%===================================================================

\begin{document}

% =================================================================================================
\chapter{Dinámica cuántica}
% =================================================================================================

Queremos ver la evolución temporal de los kets 
\[
	\Ket{\alpha,t_0,t},
\]
notación que refiere al estado $\alpha$ que partió en $t_0$ al tiempo $t$. Pictóricamente
\[
	\Ket{\alpha,t_0} \underbrace{\longrightarrow}_{\text{evoluciona}} \Ket{\alpha,t_0,t}
\]

Emplearemos para ello un operador de evolución temporal $U_{(t,t_0)}$ al cual le pediremos
\[
	\Ket{\alpha,t_0,t} = U \Ket{\alpha,t_0}
\]
con las propiedades

\begin{itemize}
 \item Unitariedad
 \[
	\Braket{ \alpha,t_0,t| \alpha,t_0,t} = 1 \forall t
 \]
 \[
	\Braket{ \alpha,t_0| U^\dagger U| \alpha,t_0} = 1 \quad \Rightarrow \quad 
	U^\dagger U = U U^\dagger = \mathbb{1}
 \]
 para conservación de la probabilidad.
 \item Linealidad
 \[
	U(t_2,t_0) = U(t_2,t_1) U(t_1,t_0) \qquad t_2>t_1>t_0
 \]
 \item Límite a $\mathbb{1}$
 \[
	U_{(t,t_0)} \to \mathbb{1} \quad \text{si} \quad t\to t_0
 \]
 o bien 
 \[
	U_{(t_0+dt,t_0)} \to \mathbb{1} \quad \text{si} \quad dt\to 0
 \]
\end{itemize}

Se propone entonces un 
\[
	U_{(t+dt,t)} = \mathbb{1} - i\Omega dt 
\]
con $\Omega$ hermítico. Comparando con clásica vemos que $H$ origina la evolución temporal, entonces
identificamos $\Omega$ con $H$, del modo $\Omega = H/\hbar$ así que 
\[
	U_{(t+dt,t)} = \mathbb{1} - \frac{i}{\hbar} H dt .
\]

De esta forma 
\[
	U_{(t+dt,t_0)} =  U_{(t+dt,t)} U_{(t,t_0)}  = \left( \mathbb{1} - \frac{i}{\hbar} H dt \right) U_{(t,t_0)}
\]
\[
	\dpar{U}{t} = \frac{ U_{(t+dt,t_0)} - U_{(t,t_0)} }{dt} = - \frac{i}{\hbar}H U_{(t,t_0)}
\]
y entonces 
\[
	i\hbar\dpar{U}{t} = HU
\]
que es la ecuación para $U_{(t,t_0)}$.
\[
	i\hbar\dpar{}{t} U_{(t,t_0)} \Ket{\alpha,t_0} = H U_{(t,t_0)} \Ket{\alpha,t_0}
\]
y arribamo a la ecuación de Schrödinger para kets
\[
	i\hbar\dpar{}{t} \Ket{\alpha,t_0,t} = H \Ket{\alpha,t_0,t}
\]
donde el inconveniente es que $H=H(t)$.

El concepto se ilustra en la figura siguiente
\begin{figure}[htb]
	\begin{center}
	\includegraphics[width=0.3\textwidth]{images/teo2_5.pdf}	 
	\end{center}
	\caption{}
\end{figure} 



% =================================================================================================
\section{Dinámica cuántica}
% =================================================================================================

\subsection{Casos de solución de $U(t,t_o)$}

\begin{itemize}
 \item Supongamos $ H \neq H(t)$, entonces
 \[
	U( t, t_0) = \euler^{-i/\hbar H (t-t_0)} 
 \]
 \item Sea $ H = H(t)$, entonces
 \[
	U( t, t_0) = \euler^{-i/\hbar \int_{t_0}^t H(t')dt'} 
 \]
 y la integral puede hacerse una vez conocida la expresión de $H(t)$.
 \item Sea $ H = H(t)$ con $[H(t_1),H(t_2)] \neq 0$ entonces
 \begin{multline*}
	U( t, t_0) =  1 + \sum_{n=1}^{\infty} \left( \frac{-i}{\hbar}\right)^n 
		\int_{t_0}^t dt_1 \int_{t_0}^{t_1} dt_2 \int_{t_0}^{t_2} dt_3 ... \times \\
			\int_{t_0}^{t_{n-1}} dt_n H(t_1) H(t_2) ... H(t_n)    
 \end{multline*}
%  \[
% 	U( t, t_0) =  1 + \sum_{n=1}^{\infty} \left( \frac{-i}{\hbar}\right)^n 
% 		\int_{t_0}^t dt_1 \int_{t_0}^{t_1} dt_2 \int_{t_0}^{t_2} dt_3 ... \int_{t_0}^{t_{n-1}} dt_n 
% 			H(t_1) H(t_2) ... H(t_n)  
%  \]
 y esta es la serie de Dyson (del físico Freeman Dyson().)
\end{itemize}

El problema que suscita es debido a que si $H$ a diferentes tiempos no conmuta no podemos poner la exponencial en serie 
de potencias. En realidad $\exp({\square})$ tiene sentido sólo si la serie 
\[
	\sum_{n=0}^{\infty}  \frac{1}{n!}\square^n
\]
tiene sentido; es decir, si no surgen ambigüedades al tomar la potencia $n$-ésima del operador $\square$.
\notamargen{El operador $\square$ no se deja poner sombreros, quiere andar con la cabeza descubierta}

Para el caso 1 es simplemente 
 \[
	a
 \]
pero para el caso 3 es 
 \[
	a
 \]
puesto que al operar es 
\[
	a
\]
pues $[H(t'),H(t'')]\neq 0$. En el caso 2 $(\int_{t_0}^t H(t')dt' )^n$ no tiene problemas puesto que está provista la 
conmutatividad.

\subsection{Soluciones útiles}

Primeramente conseguimos un $\hat{A}$ tal que $[ A, H ]=0$ y entonces (estoy considerando $ H \neq H(t)$ )
\[
	a,
\]
luego 
\[
	a
\]
con $\hat{H}$ y $\hat{A}$ conmutan se tiene
\[
	a
\]

Entonces operamos con el $H$ para 
\[
	a
\]
y así 
\[
	a
\]
de manera que comparando con 
\[
	a
\]
El coeficiente es el mismo pero le hemos sumado una fase $\exp(-iE_{a'}(t-t_0)/\hbar)$ que no es global.

\subsection{Evolución de valores de expectación}

Recordemos primeramente que los autoestados no evolucionan. Luego 
\[
	a
\]

La fase es global es considerar una autoestado. La podemos descartar (setear igual a uno)
\[
	a
\]

El valor de expectacion de un operador respecto a un autoestado no varía.
\[
	a
\]
\[
	a
\]
\[
	a
\]

El valor de expectación de un operador respecto a un estado general tiene una fase no global que produce términos de 
interferencia.

\subsection{Relaciones de conmutación}

\[
	[ A + B, C] = [A, C] + [B,C] 
\]
\[
	[A, B] = - [B,A]
\]
\[
	[A, B\cdot C] = B[A,C] +  [A,B]C
\]
\notamargen{Acá no es baca + caballo puesto que no conmutan.}
\[
	i\hbar[ A, B]_{\text{classic}} = [A, B]
\]
donde $[ , ]_{\text{classic}}$ es el corchete de Poisson.
Las relaciones de conmutación fundamentales son 
\[
	[x_i, x_j] = 0 \qquad [p_i, p_j]=0 \qquad [x_i,p_j] =i\hbar\delta_{ij}
\]
a las que podemos sumar
\[
	[x,f(p)] = i\hbar\dpar{f}{p} \qquad [p,G(x)] = i\hbar\dpar{G}{x} 
\]
\[
	[S_i,S_j] = i\hbar \varepsilon_{ijk}S_k
\]

\subsection{La ecuación de Schrödinger}

\[
	a \text{con} \qquad \hat{H} = \frac{\hat{p}^2}{2m} + V(\hat{x}) 
\]
Puedo meter un bra $\Bra{x'}$ que no depende del tiempo y entonces 
\[
	a
\]
\[
	a
\]
de manera que resulta la ecuación de Schrödinger
\[
	a .
\]

\subsection{Representación de Heisenberg}

Los kets y los operadores no tienen sentido físico, pero sí los valores de expectación : toda física podrá modificar 
los primeros pero debe conservar los valores de expectación. Así tenemos dos representaciones posibles:

\begin{center}
\begin{tabular}{|l|l|}
\hline
Schrödinger & Heisenberg \\
\hline
& \\
$\Ket{\alpha} \to U\Ket{\alpha} \quad $ & $\Ket{\alpha} \to \Ket{\alpha} \quad $ \\
& \\
$A \to A \quad $ & $A \to U^\dagger AU \quad$ \\
& \\
$\Ket{a'} \to \Ket{a'} \quad $ & $\Ket{a'} \to U^\dagger \Ket{a'} \quad $ \\
& \\
\hline
\end{tabular}
\end{center}
Así vemos que en Schrödinger los kets evolucionan y los operadores permanecen fijos; al igual que los autoestados.
En cambio en Heisenberg los kets no evolucionan pero sí lo hacen los operadores y los autoestados.

Deben notars que:
\begin{enumerate}
 \item Los productos internos no cambian con el tiempo
 \[
	a
 \]
 \item Los valores de expectacion son los mismos en ambos esquemas
 \[
	a
 \]
 \[
	\Braket{A}^{(S} = \Braket{A}^{(H} \qquad A(t)^H = U(t)^\dagger A^S U(t)
 \]
\end{enumerate}

El operador $\hat{A}$ en Schrödinger no depende explícitamente del tiempo. La idea es que le ``pegamos'' a los 
operadores la evolución temporal de los kets.
\[
	a
\]
pero a $t=t_0$ las representaciones coinciden,
\[
	a
\]

\subsubsection{La ecuación de Heisenberg}

\[
	a
\]
\[
	\Rightarrow 
\]
\[
	a
\]
\[
	a
\]
\[
	a
\]
y llegamos a la ecuación de Heisenberg
\[
	\dpar{A^{(H)}}{t} = \frac{1}{i\hbar} [ A^{(H)}, H^{(H)}]
\]
si $A^{(H)}$ conmuta con el $H^{(H)}$, entonces $A^{(H)}$ es una cantidad conservada (una constante de movimiento).
En ese caso el operador no depende del tiempo y entonces $A^{(H)} = A^{(S)}$.

\subsubsection{Evolución de autoestados}

\[
	a,
\]
aplico un $U^\dagger$ a ambos lados y entonces 
\[
	a
\]
los $a'$ no dependen de la representación porque tienen significado físico. Entonces los $\Ket{a'}$ evolucionan
\[
	a
\]
\[
	a
\]
\[
	a
\]
puesto que recordemos, nota importante,
\[
	a
\]
entonces $H$ es el mismo en ambas puesto que $\hat{U} =\hat{U}(\hat{H}) $ y $[U,H]=0$.

De esta forma los autoestados evolucionan al revés 
\[
	a
\]

Podemos ver de otro modo la equivalencia
\[
	a
\]
pero 
\[
	a
\]
\[
	a
\]

\subsubsection{Coeficientes}

Los coeficientes en Schrödinger y en Heisenberg son 
\[
	a
\]
Entonces en Schrödinger es 
\[
	a
\]
mientras que en Heisenberg es 
\[
	a
\]

Los coeficientes en las expresiones son iguales como corresponde a todo magnitud que tiene sentido físico, pues 
$|c_a(t)|^2$ es la probabilidad.

\subsection{Teorema de Ehrenfest}

Para una partícula libre, donde $p(t)=p(0)$ es constante de movimiento,
\[
	x^{(H)} = x(0) + \frac{p(0)}{m}t
\]
y se tiene 
\[
	[x(t),x(0)] = -\frac{i\hbar}{m}t
\]
\[
	H = \frac{p^2}{2m} + V(x)
\]
\[
	\dtot{P}{t} = \frac{1}{i\hbar}[p,H] = \frac{1}{i\hbar}[p,V(x)] = 
	\frac{1}{i\hbar}\left( -i\hbar\dpar{V}{x}\right),
\]
de modo que 
\[
	\dtot{P}{t} = -\dpar{V}{x} \qquad \longrightarrow \quad m \dtot[2]{x}{t} = -\dpar{V}{x} 
\]
\[
	p = m \dtot{x}{t} \qquad \dtot{p}{t} = m \dtot[2]{x}{t} 
\]
donde estamos usando 
\[
	\dpar{A^H}{t} = \frac{1}{i\hbar}[A^H,H]
\]

Es necesario remarcar que relaciones como $[x,p]=i\hbar$ son para operadores en la picture de Schrödinger, donde los 
operadores no cambian en el tiempo. Estamos en efecto haciendo $[x(0),p(0)]=i\hbar$
\[
	\Braket{\alpha,t_0|m \dtot[2]{x}{t}|\alpha,t_0} = - \Braket{\alpha,t_0|\dpar{V}{x}|\alpha,t_0}
\]
\[
	m\dpar[2]{}{t}\Braket{\alpha,t_0| x^H |\alpha,t_0} = -\Braket{\alpha,t_0|\dpar{V}{x}|\alpha,t_0}
\]
y entonces el teorema de Ehrenfest es 
\[
	m \dpar[2]{}{t} \Braket{x^{(s)}} = - \Braket{ \dpar{V^{(s)}}{x}}
\]
los valores de expectación son iguales en ambas representaciones.



% \begin{figure}[htb]
% 	\begin{center}
% 	\includegraphics[width=0.35\textwidth]{images/fig_ft1_gauss.pdf}	 
% 	\end{center}
% 	\caption{}
% \end{figure} 




% \bibliographystyle{CBFT-apa-good}	% (uses file "apa-good.bst")
% \bibliography{CBFT.Referencias} % La base de datos bibliográfica

\end{document}
