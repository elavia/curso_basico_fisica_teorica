	\documentclass[10pt,oneside]{CBFT_book}
	% Algunos paquetes
	\usepackage{amssymb}
	\usepackage{amsmath}
	\usepackage{graphicx}
	\usepackage{libertine}
	\usepackage[bold-style=TeX]{unicode-math}
	\usepackage{lipsum}

	\usepackage{natbib}
	\setcitestyle{square}

	\usepackage{polyglossia}
	\setdefaultlanguage{spanish}
	



	\usepackage{CBFT.estilo} % Cargo la hoja de estilo

	% Tipografías
	% \setromanfont[Mapping=tex-text]{Linux Libertine O}
	% \setsansfont[Mapping=tex-text]{DejaVu Sans}
	% \setmonofont[Mapping=tex-text]{DejaVu Sans Mono}

	%===================================================================
	%	DOCUMENTO PROPIAMENTE DICHO
	%===================================================================

\begin{document}

% =================================================================================================
\chapter{Dinámica cuántica}
% =================================================================================================

Queremos ver la evolución temporal de los kets 
\[
	\Ket{\alpha,t_0,t},
\]
notación que refiere al estado $\alpha$ que partió en $t_0$ al tiempo $t$. Pictóricamente
\[
	\Ket{\alpha,t_0} \underbrace{\longrightarrow}_{\text{evoluciona}} \Ket{\alpha,t_0,t}
\]

Emplearemos para ello un operador de evolución temporal $U_{(t,t_0)}$ al cual le pediremos
\[
	\Ket{\alpha,t_0,t} = U \Ket{\alpha,t_0}
\]
con las propiedades

\begin{itemize}
 \item Unitariedad
 \[
	\Braket{ \alpha,t_0,t| \alpha,t_0,t} = 1 \forall t
 \]
 \[
	\Braket{ \alpha,t_0| U^\dagger U| \alpha,t_0} = 1 \quad \Rightarrow \quad 
	U^\dagger U = U U^\dagger = \mathbb{1}
 \]
 para conservación de la probabilidad.
 \item Linealidad
 \[
	U(t_2,t_0) = U(t_2,t_1) U(t_1,t_0) \qquad t_2>t_1>t_0
 \]
 \item Límite a $\mathbb{1}$
 \[
	U_{(t,t_0)} \to \mathbb{1} \quad \text{si} \quad t\to t_0
 \]
 o bien 
 \[
	U_{(t_0+dt,t_0)} \to \mathbb{1} \quad \text{si} \quad dt\to 0
 \]
\end{itemize}

Se propone entonces un 
\[
	U_{(t+dt,t)} = \mathbb{1} - i\Omega dt 
\]
con $\Omega$ hermítico. Comparando con clásica vemos que $H$ origina la evolución temporal, entonces
identificamos $\Omega$ con $H$, del modo $\Omega = H/\hbar$ así que 
\[
	U_{(t+dt,t)} = \mathbb{1} - \frac{i}{\hbar} H dt .
\]

De esta forma 
\[
	U_{(t+dt,t_0)} =  U_{(t+dt,t)} U_{(t,t_0)}  = \left( \mathbb{1} - \frac{i}{\hbar} H dt \right) U_{(t,t_0)}
\]
\[
	\dpar{U}{t} = \frac{ U_{(t+dt,t_0)} - U_{(t,t_0)} }{dt} = - \frac{i}{\hbar}H U_{(t,t_0)}
\]
y entonces 
\[
	i\hbar\dpar{U}{t} = HU
\]
que es la ecuación para $U_{(t,t_0)}$.
\[
	i\hbar\dpar{}{t} U_{(t,t_0)} \Ket{\alpha,t_0} = H U_{(t,t_0)} \Ket{\alpha,t_0}
\]
y arribamo a la ecuación de Schrödinger para kets
\[
	i\hbar\dpar{}{t} \Ket{\alpha,t_0,t} = H \Ket{\alpha,t_0,t}
\]
donde el inconveniente es que $H=H(t)$.

El concepto se ilustra en la figura siguiente
\begin{figure}[htb]
	\begin{center}
	\includegraphics[width=0.3\textwidth]{images/teo2_5.pdf}	 
	\end{center}
	\caption{}
\end{figure} 



% =================================================================================================
\section{Dinámica cuántica}
% =================================================================================================

% Son ecuaciones lineales de modo que vale la superposición (con \vb{E}, \vb{B} y 
% cualquier vector relacionado linealmente con ellos).
% \[
% 	\nabla \cdot \vb{D} = 4 \pi \rho_\ell \qquad \nabla \cdot \vb{B} = 0
% \]
% \[
% 	\nabla \times \vb{E} = - \frac{1}{c} \dpar{B}{t} \qquad \nabla \times \vb{H} =
% 	\frac{4\pi}{c} \vb{J}_\ell + \frac{1}{c}\dpar{D}{t}
% \]
% \[
% 	\vb{F} = q \left( \vb{E} + \frac{1}{c} \vb{v} \times \vb{B} \right)
% \]
% 
% Los vectores pueden ser polares (tienen físicamente bien definido el sentido) o
% axiales (se les atribuye un sentido por convención).
% 
% Las ecuaciones son invariantes ante transformaciones del tipo: rotación
% y reflexión espacial y temporal.

\subsection{Casos de solución de $U(t,t_o)$}

\subsection{Soluciones útiles}

\subsection{Evolución de valores de expectación}

\subsection{Relaciones de conmutación}

\subsection{La ecuación de Schrödinger}

\subsection{Representación de Heisenberg}

\subsubsection{La ecuación de Heisenberg}

\subsubsection{Evolución de autoestados}

\subsubsection{Coeficientes}

\subsection{Teorema de Ehrenfest}


% 
% =================================================================================================
\section{El oscilador armónico}
% =================================================================================================

\subsection{El estado fundamental $\Braket{0}$}

\subsection{Función de onda}

\subsection{Interferencia en experimento de Young}

\subsection{Cambio de cero del potencial}

% 
% La ley de Coulomb reza que
% \[
% 	\vb{F}_{12} = q_1 q_2 \frac{(\vb{x}_1 - \vb{x}_2)}{|\vb{x}_1 - \vb{x}_2 |^3}
% \]
% que es la fuerza sobre 1 debido a 2. De la ley de Coulomb se puede definir 
% \[
% 	\vb{E}_{12}(\vb{x}_1) \equiv \vb{F}_{12}/q_1
% \]
% y tomando $\vb{x}_1 \equiv \vb{x}$ y haciendo el límite $q_1 \to 0$ se tiene
% \[
% 	\vb{E}(\vb{x}) = \sum_{i=1}^N \; q_i \frac{(\vb{x} - \vb{x}_i)}{|\vb{x} - \vb{x}_i |^3}
% \]
% que es el campo eléctrico y que en el paso al continuo resulta
% \[
% 	\vb{E}(\vb{x}) = \int_{V'} \rho(\vb{x}) \frac{(\vb{x} - \vb{x}_i)}{|\vb{x} - \vb{x}_i |^3} dV' 
% \]
% siendo \vb{x} punto campo y $\vb{x}_i$ punto fuente.
% 
% \begin{figure}[htb]
% 	\begin{center}
% 	\includegraphics[width=0.6\textwidth]{images/fig_ft1_ejescargas.pdf}	 
% 	\end{center}
% 	\caption{}
% \end{figure} 
% 
% \subsection{Conservación de la carga}
% 
% La carga total sale de una integral 
% \[
% 	Q = \int_{V'}  \rho(\vb{x}') dV'
% \]
% como muestra la imagen
% \begin{figure}[htb]
% 	\begin{center}
% 	\includegraphics[width=0.25\textwidth]{images/fig_ft1_conserv.pdf}	 
% 	\end{center}
% 	\caption{}
% \end{figure} 
% y si el volumen es fijo podemos tomar la derivada con respecto al tiempo que pasa el interior como
% derivada parcial,
% \[
% 	\dtot{Q}{t} = \int_{V'} \dpar{\rho}{t} (\vb{x}') dV' = - \int_{S\equiv\partial V'} \vb{J} \cdot d\vb{S}
% \]
% y el miembro extremo derecho  se debe a que si la carga varía es a consecuencia de que se va en
% forma de flujo. Aplicando el teorema de la divergencia en el miembro derecho,
% \[
% 	\int_{V'} \dpar{\rho}{t} (\vb{x}') dV' = - \int_{V'} \nabla \cdot \vb{J} \; dV'
% \]
% lo cual vale para todo volumen y entonces esto significa que
% \[
% 	\dpar{\rho}{t} + \nabla \cdot \vb{J} = 0
% \]
% que es la ecuación de continuidad de la carga. Si fuera $\nabla \cdot \vb{J}=0$ esto significa que las líneas
% de \vb{J} no tienen principio ni fin.
% 
% % =================================================================================================
% \section{Interacción magnética}
% % =================================================================================================
% 
% Cuando se da $\nabla \cdot \vb{J}=0$ hablamos de una corriente estacionaria (no hay acumulación de carga en
% ninguna parte). Las corrientes estacionarias producen efectos magnéticos dados por la ley de Biot-Savart
% \[
% 	\vb{B}(\vb{x}) = \frac{1}{c} \int_\Gamma \frac{I d\vb{\ell}' \times (\vb{x} - \vb{x}')}{|\vb{x} - \vb{x}'|^3} 
% \]
% que es válida para un circuito $\Gamma$, que es una curva que se recorre en sentido CCW.
% En el caso de un volumen la expresión es 
% \[
% 	\vb{B}(\vb{x}) = \frac{1}{c} \int_{V'} \frac{ \vb{J}(\vb{x}') \times (\vb{x} - \vb{x}')}{|\vb{x} - \vb{x}'|^3} 
% dV'
% \]
% mientras que la fuerza sobre un circuito $\Gamma$ es
% \[
% 	F = \frac{1}{c} \int_\Gamma I d\vb{\ell} \times \vb{B}
% \]
% y sobre un volumen 
% \[
% 	F = \frac{1}{c} \int_V \vb{J} \times \vb{B} dV
% \]
% 
% La transformación entre estas integrales puede hacerse merced al siguiente razonamiento,
% % \begin{align*}
% %  	I d\vb{\ell} \times \vb{B} = \vb{J}  \cdot d\vb{S} d\vb{\ell}  \times \vb{B} =
% %   	\cos(\theta) dS \vb{J} d\ell \times \vb{B} = \\
% % 	\vb{J} \times \vb{B}  \cos(\theta) dS d\ell  = \vb{J} \times \vb{B}  d\vb{S} \cdot d\vb{\ell}  = 
% % 	\vb{J} \times \vb{B}  dV 
% % \end{align*}
% \[
%   	I d\vb{\ell} \times \vb{B} = \vb{J}  \cdot d\vb{S} d\vb{\ell}  \times \vb{B} =
%   	\cos(\theta) dS \vb{J} d\ell \times \vb{B} = 
% \]
% \[
% 	\vb{J} \times \vb{B}  \cos(\theta) dS d\ell  = \vb{J} \times \vb{B}  d\vb{S} \cdot d\vb{\ell}  = 
% 	\vb{J} \times \vb{B}  dV 
% \]
% 
% \subsection{Fuerza de un circuito sobre otro}
% 
% La fuerza de un circuito 2 sobre otro circuito 1 puede calcularse con un poco de paciencia como sigue
% \[
% 	F_{12} = \frac{1}{c} \int_{\Gamma_1} I_1 d\vb{\ell}_1 \times \left\{
% 	\frac{1}{c} \int_{\Gamma_2} \frac{I_2 d\vb{\ell}_2 \times (\vb{x}_1 - \vb{x}_2)}{|\vb{x}_1 - \vb{x}_2|^3} 
% 	\right\}
% \]
% \[
% 	F_{12} = \frac{I_1 I_2}{c^2} \int_{\Gamma_1} \int_{\Gamma_2} d\vb{\ell}_1 \times \left\{
% 	\frac{d\vb{\ell}_2 \times (\vb{x}_1 - \vb{x}_2)}{|\vb{x}_1 - \vb{x}_2|^3} 
% 	\right\}
% \]
% \[
% 	F_{12} = \frac{I_1 I_2}{c^2} \int_{\Gamma_1} \int_{\Gamma_2} d\vb{\ell}_2  \left\{
% 	\frac{d\vb{\ell}_1 \cdot (\vb{x}_1 - \vb{x}_2)}{|\vb{x}_1 - \vb{x}_2|^3} 
% 	\right\} - \int_{\Gamma_1} \int_{\Gamma_2} \frac{ (\vb{x}_1 - \vb{x}_2)}{|\vb{x}_1 - \vb{x}_2|^3} 
% 	\left\{ d\vb{\ell}_1 \cdot d\vb{\ell}_2 \right\}
% \]
% donde el primer término se comprueba nulo si se reescribe utilizando que
% \[
% 	\frac{ (\vb{x}_1 - \vb{x}_2)}{|\vb{x}_1 - \vb{x}_2|^3} = 
% 	\nabla_{\vb{x}_2} \frac{ 1 }{|\vb{x}_1 - \vb{x}_2|} =
% 	- \nabla_{\vb{x}_1} \frac{ 1 }{|\vb{x}_1 - \vb{x}_2|} 
% \]
% de manera que entonces
% \[
% 	- \int_{\Gamma_2} d\vb{\ell}_2  \int_{\Gamma_1} d\vb{\ell}_1 \cdot \nabla_{\vb{x}_1} \frac{ 1 }{|\vb{x}_1 - \vb{x}_2|} 
% \]
% donde se ve que es nula la última integral dado que 
% \[
% 	\int_{\Gamma_1} d\vb{\ell}_1 \cdot \nabla_{\vb{x}_1} = 0.
% \]
% 
% Entonces, se tiene 
% \[
% 	F_{12} = - \frac{I_1 I_2}{c^2} \int_{\Gamma_1} \int_{\Gamma_2} \frac{ (\vb{x}_1 - \vb{x}_2)}{|\vb{x}_1 - \vb{x}_2|^3} 
% 	\left( d\vb{\ell}_1 \cdot d\vb{\ell}_2 \right)
% \]
% que vale lo mismo si intercambiamos $\Gamma_1$ con $\Gamma_2$ en la integración. Podemos decir que con corrientes estacionarias
% vale el principio de acción y reacción: las fuerzas son iguales y de sentido opuesto.
% 
% 
% % =================================================================================================
% \section{Teorema de Helmholtz}
% % =================================================================================================
% 
% Nos dice que un campo vectorial está completamente determinado por su divergencia y su rotor.
% Por ejemplo, para un campo eléctrico 
% \[
% 	\vb{E} = \int_{V'} \rho \frac{\vb{x} - \vb{x}'}{|\vb{x} - \vb{x}'|^3} dV' = 
% 		- \int_{V'} \rho \nabla_{\vb{x}} \frac{ 1 }{|\vb{x} - \vb{x}'|} dV' = 
% 		- \nabla_{\vb{x}} \int_{V'}   \frac{ \rho }{|\vb{x} - \vb{x}'|} dV' = 
% \]
% y esta última es la integral de Poisson
% \[
% 	\vb{E} = - \nabla_{\vb{x}} \phi (\vb{x}).
% \]
% Entonces $\vb{E}$ es un gradiente y por ello 
% \[
% 	\nabla  \times \vb{E} = 0
% \]
% de manera que $\vb{E}$ es conservativo, cumple $\int \vb{E}\cdot d\vb{\ell} = 0$ o lo que
% es lo mismo, $\vb{E}$ es irrotacional.
% Hemos hecho la construcción de un potencial electrostático.
% 
% % =================================================================================================
% \section{Ley de Gauss}
% % =================================================================================================
% 
% 
% 
% \begin{figure}[htb]
% 	\begin{center}
% 	\includegraphics[width=0.35\textwidth]{images/fig_ft1_gauss.pdf}	 
% 	\end{center}
% 	\caption{}
% \end{figure} 
% \[
% 	\vb{E} \cdot \hat{n} = q \frac{\cos(\theta)}{r^2}
% \]
% y el ángulo sólido es
% \[
% 	\vb{E} \cdot \hat{n} dS = q \frac{\cos(\theta)}{r^2} dS
% \]
% \[
% 	\vb{E} \cdot \hat{n} dS = q d\Omega \qquad \longrightarrow \qquad 
% 	\int_{S\equiv\partial V} \vb{E} \cdot \hat{n} \; dS = q \int_S d\Omega =
% 	\begin{cases}
% 	 0 \quad \textrm{carga exterior}\\
% 	 4\pi \quad \textrm{carga interior}
% 	\end{cases}
% \]
% \[
% 	\int_S \vb{E} \cdot \hat{n} \; dS = 4\pi \sum_i q_i
% \]
% La ley de Gauss es
% \[
% 	\int_S \vb{E} \cdot \hat{n} \; dS = 4\pi Q_n
% \]
% donde $Q_n$ es la carga neta dentro de la superficie $S$. Al continuo pasa como 
% \[
% 	\int_S \vb{E} \cdot \hat{n} \; dS = 4\pi \int_V \rho \: dV
% \]
% de manera que 
% \[
% 	\int_V \divem{E} \; dV = \int_V 4\pi \rho \: dV
% \]
% y entonces
% \[
% 	\divem{E} = 4\pi \rho.
% \]
% 
% Por otro lado si \vb{E} es el gradiente de un potencial $\phi$ se tiene
% \[
% 	\divem{E} = \Nabla\cdot{(-\Nabla\phi)} = - \lapm\phi = 4\pi \rho
% \]
% y se desprenden las ecuaciones de Poisson,
% \[
% 	\lapm\phi = -4\pi \rho
% \]
% y de Laplace
% \[
% 	\lapm\phi = 0
% \]
% que es el caso particular de la anterior con cargas nulas.
% 
% La solución de la ecuación no homogénea es suma de una solución del homogéneo más una solución
% particular. La carga está relacionada a la solución particular.
% 
% \subsection{Gauges}
% 
% Dado que $\divem{B}=0$ entonces existe un \vb{A} tal que 
% \[
% 	\rotorm{A} = \vb{B}
% \]
% pero para caracterizar totalmente el \vb{A} tengo la libertad de definir a conveniencia
% \[
% 	\divem{A} \equiv \; \textrm{``el gauge''}.
% \]
% Casos particulares importantes son el gauge de Coulomb,
% \[
% 	\divem{A} = 0
% \]
% de manera que como 
% \[
% 	\Nabla \times (\rotorm{A}) = \Nabla(\divem{A}) - \lapm{\vb{A}} = \frac{4\pi}{c}\vb{J}
% \]
% se llega para el potencial electromagnético, bajo el gauge de Coulomb, a que 
% \[
% 	\lapm{\vb{A}} = - \frac{4\pi}{c}\vb{J} 
% \]
% 
% 	\begin{table}[hbt]
% 	\centering
%         \begin{tabular}{|c|c|}
% 		\hline
% 		& \\
% 		$\displaystyle{\vb{E} = \int_{V'} \frac{\rho(\vb{x}')(\vb{x}-\vb{x}')}{|\vb{x}-\vb{x}'|^3} dV' 
% 		}$ & $\displaystyle{\vb{B} = \frac{1}{c} \int_{V'} \frac{\vb{J}(\vb{x}') \times 
% 		(\vb{x}-\vb{x}')}{|\vb{x}-\vb{x}'|^3} dV'}$ \\
% 		& \\
% 		\hline
% 		Ley de Gauss & Ley de Ampere \\
% 		& \\
% 		$\displaystyle{\int_S \vb{E}\cdot d\vb{S} = 4\pi Q_n}$ &
% 		$\displaystyle{\int_\Gamma \vb{B}\cdot d\vb{\ell} = \frac{4\pi}{c} I_c}$ \\
% 		& \\
% 		\hline
% 		&\\
% 		$\divem{E} = 4\pi\rho$ & $\divem{B} = 0$ \\
% 		$\rotorm{E} = 0$ & $\rotorm{B} = \frac{4\pi}{c}\vb{J}$ \\
% 		& \\
% 		\hline
% 		& \\
% 		$\vb{E} = - \Nabla\phi$ & $\vb{B} = \rotorm{A}$ \\
% 		& \\
% 		\hline
%         \end{tabular} 
% 	\caption{}
% 	\end{table} 
% 
% La operación de tomar rotor y el producto vecrtorial cambian el carácter de los vectores: de
% polares pasan a axiales y viceversa.
% 
% La fuerza general sobre una distribución de carga es
% \[
% 	\vb{F} = \int_{V'} \rho \vb{E} dV' + \frac{1}{c} \int_{V'} \vb{J} \times \vb{B} dV'. 
% \]
% 
% \subsection{Delta de Dirac}
% 
% Una densidad de carga puntual se puede escribir mediante una delta de Dirac de acuerdo a
% \[
% 	\rho(\vb{x}') = q\: \delta (\vb{x} - \vb{x}') = \begin{cases}
% 	                                               0 \qquad \vb{x} \neq \vb{x}' \\
% 	                                               \infty \qquad \vb{x} = \vb{x}'\\
% 	                                              \end{cases}
% \]
% siendo las dimensiones de la delta las de $1/L^3$ y cumpliéndose 
% \[
% 	\int_{V'} \delta (\vb{x} - \vb{x}') dV' = 1
% \]
% \[
% 	\delta (\vb{x} - \vb{x}') = \frac{1}{h_1h_2h_3} \delta(q_1-q_1') \delta(q_2-q_2') \delta(q_3-q_3')
% \]
% donde $q_1, q_2$ y $q_3$ son coordenadas curvilíneas generales y $h_1h_2h_3$ es el jacobiano
% de la transformación.
% Luego
% \[
% 	\int f(\vb{x}) \delta' (\vb{x} - \vb{x}_0) dx = -f'(\vb{x}_0)
% \]
% 
% 
% 
% \subsection{reflexión}
% 
% Un vector polar sufre reflexión especular mientras que un vector axial ({\it pseudovector})
% sufre una antireflexión especular. Ver la figura.
% 
% \begin{figure}[htb]
% 	\begin{center}
% 	\includegraphics[width=0.6\textwidth]{images/fig_ft1_reflexvect.pdf}	 
% 	\end{center}
% 	\caption{}
% \end{figure} 
% 
% Una reflexión más una rotación permite eliminar componentes de campo.
% Una simetría más una rotación-traslación permite eliminar dependencias.
% 
% Lo primero que debe hacerse es escribir bien la \vb{J} a partir del dato de la corriente
% (que es el que se suele tener) mediante
% \[
% 	i = \int_S \vb{J} \cdot d \vb{S}
% \]
% En cambio, para \vb{A} es más fácil usar
% \[
% 	\vb{B} = \rotorm{A}
% \]
% y despejar de aquí la ecuación diferencial que emplear
% \[
% 	\vb{A} = \frac{1}{c} \int_V \frac{\vb{J}}{|\vb{x}-\vb{x}'|} dV
% \]
% 
% 
% \section{El potencial vector}
% 
% Por la ley de Biot y Savart,
% \[
% 	\vb{B} = \frac{1}{c} \int_{V'} \frac{\vb{J}(\vb{x}') \times (\vb{x}-\vb{x}')}{|\vb{x}-\vb{x}'|^3} 
% 	dV' = \Nabla_x \times \frac{1}{c} \int_{V'} \frac{\vb{J}(\vb{x}')}{|\vb{x}-\vb{x}'|} dV'
% \]
% de modo que
% \be
% 	\vb{A} = \frac{1}{c} \int_{V'} \frac{\vb{J}(\vb{x}')}{|\vb{x}-\vb{x}'|} dV'
% 	\label{potvec}
% \ee
% pero 
% \[
% 	\vb{A}' \equiv \vb{A} + \Nabla \Psi
% \]
% es tan buen potencial vector como \vb{A} puesto que los rotores verifican $\rotorm{A}=\rotorm{A}'=\vb{B}$,
% de lo cual extraemos en conclusión que el potencial vector está definido a menos del gradiente de una
% función escalar.
% 
% Tomándole el rotor a \eqref{potvec} y considerando $\Nabla'\cdot\vb{J}(\vb{x}')=0$ lo cual se verifica si
% la corriente es estacionaria se tiene 
% \[
% 	\rotorm{B} = \frac{4\pi}{c} \vb{J}(\vb{x})
% \]
% y entonces
% \[
% 	\int_S \rotorm{B} \cdot d\vb{S} = \frac{4\pi}{c} \int_S \vb{J}(\vb{x}) \cdot d\vb{S}
% \]
% y por el teorema de Stokes arribamos a
% \[
% 	\int_{\Gamma\equiv\partial S} \vb{B}\cdot d\vb{\ell} = \frac{4\pi}{c} I_\Gamma
% \]
% que es la ley de Ampere. Notemos que $I_\Gamma$ es la corriente concatenada por el lazo $\Gamma$.
% Además
% \[
% 	\rotorm{B} = \Nabla\times(\rotorm{A}) = \Nabla(\divem{A}) - \nabla^2 \vb{A} = \frac{4\pi}{c}\vb{J}
% \]
% pero utilizando el gauge de Coulomb es $\divem{A}=0$ y entonces
% \[
% 	\nabla^2 \vb{A} = -\frac{4\pi}{c}\vb{J}
% \]
% que es una ecuación de Poisson vectorial.
% 
% Magnetostática y electrostáctica son gobernadas por ecuaciones de Poisson para potenciales $\vb{A},\phi$ y
% el problema entonces se reduce a resolverlas para luego hallar los campos por derivación.
% 
% \section{Unicidad de problemas de potencial}
% 
% Si dos problemas satisfacen iguales condiciones de contorno entonces en el recinto encerrado por
% ese contorno tienen igual solución.
% 
% Si en un recinto $R$
% \be
% 	\phi_1|_{cont} = \phi_2|_{cont}
% 	\label{potnounico}
% \ee
% pero se da para el interior de $R$ que $\phi_1\neq\phi_2$ entonces se tiene sucesivamente
% \[
% 	U \equiv \phi_1 - \phi_2 \qquad \qquad \Nabla U = \Nabla \phi_1 - \Nabla \phi_2
% \]
% \[
% 	\lapm{U} = \lapm{\phi_1} - \lapm{\phi_2} = -4\pi \rho + 4\pi\rho = 0
% \]
% \[
% 	\Nabla\cdot\left( U\Nabla U \right) = U\left( \Nabla\cdot\Nabla U \right) + \Nabla U \cdot \Nabla U
% \]
% \[
% 	\int_V \Nabla\cdot\left( U\Nabla U \right) dV = \int_V U \lapm{U}  + (\lapm{U})^2 dV =  \int_V (\lapm{U})^2 dV
% \]
% llegando al último miembro porque el potencial $U$ cumple la ecuación de Laplace. Luego,
% \[
% 	\int_V (\lapm{U})^2 dV = \int_S U\Nabla{U} \cdot d\vb{S} = 0
% \]
% habiéndose pasado a la integral de superficie por el teorema de la divergencia y anulando el valor global porque 
% $U$ en el contorno es nula (recuérdese \eqref{potnounico}). Además, 
% \[
% 	\Nabla{U} \cdot d\vb{S}  \longrightarrow \left.\dpar{U}{\hat{n}}\right|_{cont}
% \]
% luego,
% \[
% 	\Nabla U = 0 \qquad \Nabla\phi_1 = \Nabla\phi_2 
% \]
% y entonces
% \[
% 	\phi_1 = \phi_2 .
% \]
% a menos, por supuesto, de una constante.



% \bibliographystyle{CBFT-apa-good}	% (uses file "apa-good.bst")
% \bibliography{CBFT.Referencias} % La base de datos bibliográfica

\end{document}
