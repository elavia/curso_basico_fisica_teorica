	\documentclass[10pt,oneside]{CBFT_book}
	% Algunos paquetes
	\usepackage{amssymb}
	\usepackage{amsmath}
	\usepackage{graphicx}
	\usepackage{libertine}
	\usepackage[bold-style=TeX]{unicode-math}
	\usepackage{lipsum}

	\usepackage{natbib}
	\setcitestyle{square}

	\usepackage{polyglossia}
	\setdefaultlanguage{spanish}
	



	\usepackage{CBFT.estilo} % Cargo la hoja de estilo

	% Tipografías
	% \setromanfont[Mapping=tex-text]{Linux Libertine O}
	% \setsansfont[Mapping=tex-text]{DejaVu Sans}
	% \setmonofont[Mapping=tex-text]{DejaVu Sans Mono}

	%===================================================================
	%	DOCUMENTO PROPIAMENTE DICHO
	%===================================================================

\begin{document}

% =================================================================================================
\chapter{Dinámica cuántica}
% =================================================================================================

Queremos ver la evolución temporal de los kets 
\[
	\Ket{\alpha,t_0,t},
\]
notación que refiere al estado $\alpha$ que partió en $t_0$ al tiempo $t$. Pictóricamente
\[
	\Ket{\alpha,t_0} \underbrace{\longrightarrow}_{\text{evoluciona}} \Ket{\alpha,t_0,t}
\]

Emplearemos para ello un operador de evolución temporal $U_{(t,t_0)}$ al cual le pediremos
\[
	\Ket{\alpha,t_0,t} = U \Ket{\alpha,t_0}
\]
con las propiedades

\begin{itemize}
 \item Unitariedad
 \[
	\Braket{ \alpha,t_0,t| \alpha,t_0,t} = 1 \forall t
 \]
 \[
	\Braket{ \alpha,t_0| U^\dagger U| \alpha,t_0} = 1 \quad \Rightarrow \quad 
	U^\dagger U = U U^\dagger = \mathbb{1}
 \]
 para conservación de la probabilidad.
 \item Linealidad
 \[
	U(t_2,t_0) = U(t_2,t_1) U(t_1,t_0) \qquad t_2>t_1>t_0
 \]
 \item Límite a $\mathbb{1}$
 \[
	U_{(t,t_0)} \to \mathbb{1} \quad \text{si} \quad t\to t_0
 \]
 o bien 
 \[
	U_{(t_0+dt,t_0)} \to \mathbb{1} \quad \text{si} \quad dt\to 0
 \]
\end{itemize}

Se propone entonces un 
\[
	U_{(t+dt,t)} = \mathbb{1} - i\Omega dt 
\]
con $\Omega$ hermítico. Comparando con clásica vemos que $H$ origina la evolución temporal, entonces
identificamos $\Omega$ con $H$, del modo $\Omega = H/\hbar$ así que 
\[
	U_{(t+dt,t)} = \mathbb{1} - \frac{i}{\hbar} H dt .
\]

De esta forma 
\[
	U_{(t+dt,t_0)} =  U_{(t+dt,t)} U_{(t,t_0)}  = \left( \mathbb{1} - \frac{i}{\hbar} H dt \right) U_{(t,t_0)}
\]
\[
	\dpar{U}{t} = \frac{ U_{(t+dt,t_0)} - U_{(t,t_0)} }{dt} = - \frac{i}{\hbar}H U_{(t,t_0)}
\]
y entonces 
\[
	i\hbar\dpar{U}{t} = HU
\]
que es la ecuación para $U_{(t,t_0)}$.
\[
	i\hbar\dpar{}{t} U_{(t,t_0)} \Ket{\alpha,t_0} = H U_{(t,t_0)} \Ket{\alpha,t_0}
\]
y arribamo a la ecuación de Schrödinger para kets
\[
	i\hbar\dpar{}{t} \Ket{\alpha,t_0,t} = H \Ket{\alpha,t_0,t}
\]
donde el inconveniente es que $H=H(t)$.

El concepto se ilustra en la figura siguiente
\begin{figure}[htb]
	\begin{center}
	\includegraphics[width=0.3\textwidth]{images/teo2_5.pdf}	 
	\end{center}
	\caption{}
\end{figure} 



% =================================================================================================
\section{Dinámica cuántica}
% =================================================================================================

\subsection{Casos de solución de $U(t,t_o)$}

\begin{itemize}
 \item Supongamos $ H \neq H(t)$, entonces
 \[
	U( t, t_0) = \euler^{-i/\hbar H (t-t_0)} 
 \]
 \item Sea $ H = H(t)$, entonces
 \[
	U( t, t_0) = \euler^{-i/\hbar \int_{t_0}^t H(t')dt'} 
 \]
 y la integral puede hacerse una vez conocida la expresión de $H(t)$.
 \item Sea $ H = H(t)$ con $[H(t_1),H(t_2)] \neq 0$ entonces
 \begin{multline*}
	U( t, t_0) =  1 + \sum_{n=1}^{\infty} \left( \frac{-i}{\hbar}\right)^n 
		\int_{t_0}^t dt_1 \int_{t_0}^{t_1} dt_2 \int_{t_0}^{t_2} dt_3 ... \times \\
			\int_{t_0}^{t_{n-1}} dt_n H(t_1) H(t_2) ... H(t_n)    
 \end{multline*}
%  \[
% 	U( t, t_0) =  1 + \sum_{n=1}^{\infty} \left( \frac{-i}{\hbar}\right)^n 
% 		\int_{t_0}^t dt_1 \int_{t_0}^{t_1} dt_2 \int_{t_0}^{t_2} dt_3 ... \int_{t_0}^{t_{n-1}} dt_n 
% 			H(t_1) H(t_2) ... H(t_n)  
%  \]
 y esta es la serie de Dyson (del físico Freeman Dyson().)
\end{itemize}

El problema que suscita es debido a que si $H$ a diferentes tiempos no conmuta no podemos poner la exponencial en serie 
de potencias. En realidad $\exp({\square})$ tiene sentido sólo si la serie 
\[
	\sum_{n=0}^{\infty}  \frac{1}{n!}\square^n
\]
tiene sentido; es decir, si no surgen ambigüedades al tomar la potencia $n$-ésima del operador $\square$.
\notamargen{El operador $\square$ no se deja poner sombreros, quiere andar con la cabeza descubierta}

Para el caso 1 es simplemente 
 \[
	a
 \]
pero para el caso 3 es 
 \[
	a
 \]
puesto que al operar es 
\[
	a
\]
pues $[H(t'),H(t'')]\neq 0$. En el caso 2 $(\int_{t_0}^t H(t')dt' )^n$ no tiene problemas puesto que está provista la 
conmutatividad.

\subsection{Soluciones útiles}

Primeramente conseguimos un $\hat{A}$ tal que $[ A, H ]=0$ y entonces (estoy considerando $ H \neq H(t)$ )
\[
	a,
\]
luego 
\[
	a
\]
con $\hat{H}$ y $\hat{A}$ conmutan se tiene
\[
	a
\]

Entonces operamos con el $H$ para 
\[
	a
\]
y así 
\[
	a
\]
de manera que comparando con 
\[
	a
\]
El coeficiente es el mismo pero le hemos sumado una fase $\exp(-iE_{a'}(t-t_0)/\hbar)$ que no es global.

\subsection{Evolución de valores de expectación}

Recordemos primeramente que los autoestados no evolucionan. Luego 
\[
	a
\]

La fase es global es considerar una autoestado. La podemos descartar (setear igual a uno)
\[
	a
\]

El valor de expectacion de un operador respecto a un autoestado no varía.
\[
	a
\]
\[
	a
\]
\[
	a
\]

El valor de expectación de un operador respecto a un estado general tiene una fase no global que produce términos de 
interferencia.

\subsection{Relaciones de conmutación}

\[
	[ A + B, C] = [A, C] + [B,C] 
\]
\[
	[A, B] = - [B,A]
\]
\[
	[A, B\cdot C] = B[A,C] +  [A,B]C
\]
\notamargen{Acá no es baca + caballo puesto que no conmutan.}
\[
	i\hbar[ A, B]_{\text{classic}} = [A, B]
\]
donde $[ , ]_{\text{classic}}$ es el corchete de Poisson.
Las relaciones de conmutación fundamentales son 
\[
	[x_i, x_j] = 0 \qquad [p_i, p_j]=0 \qquad [x_i,p_j] =i\hbar\delta_{ij}
\]
a las que podemos sumar
\[
	[x,f(p)] = i\hbar\dpar{f}{p} \qquad [p,G(x)] = i\hbar\dpar{G}{x} 
\]
\[
	[S_i,S_j] = i\hbar \varepsilon_{ijk}S_k
\]

\subsection{La ecuación de Schrödinger}

\[
	a \text{con} \qquad \hat{H} = \frac{\hat{p}^2}{2m} + V(\hat{x}) 
\]
Puedo meter un bra $\Bra{x'}$ que no depende del tiempo y entonces 
\[
	a
\]
\[
	a
\]
de manera que resulta la ecuación de Schrödinger
\[
	a .
\]

\subsection{Representación de Heisenberg}

Los kets y los operadores no tienen sentido físico, pero sí los valores de expectación : toda física podrá modificar 
los primeros pero debe conservar los valores de expectación. Así tenemos dos representaciones posibles:

\begin{center}
\begin{tabular}{|l|l|}
\hline
Schrödinger & Heisenberg \\
\hline
& \\
$\Ket{\alpha} \to U\Ket{\alpha} \quad $ & $\Ket{\alpha} \to \Ket{\alpha} \quad $ \\
& \\
$A \to A \quad $ & $A \to U^\dagger AU \quad$ \\
& \\
$\Ket{a'} \to \Ket{a'} \quad $ & $\Ket{a'} \to U^\dagger \Ket{a'} \quad $ \\
& \\
\hline
\end{tabular}
\end{center}
Así vemos que en Schrödinger los kets evolucionan y los operadores permanecen fijos; al igual que los autoestados.
En cambio en Heisenberg los kets no evolucionan pero sí lo hacen los operadores y los autoestados.

Deben notars que:
\begin{enumerate}
 \item Los productos internos no cambian con el tiempo
 \[
	a
 \]
 \item Los valores de expectacion son los mismos en ambos esquemas
 \[
	a
 \]
 \[
	\Braket{A}^{(S} = \Braket{A}^{(H} \qquad A(t)^H = U(t)^\dagger A^S U(t)
 \]
\end{enumerate}

El operador $\hat{A}$ en Schrödinger no depende explícitamente del tiempo. La idea es que le ``pegamos'' a los 
operadores la evolución temporal de los kets.
\[
	a
\]
pero a $t=t_0$ las representaciones coinciden,
\[
	a
\]

\subsubsection{La ecuación de Heisenberg}

\[
	a
\]
\[
	\Rightarrow 
\]
\[
	a
\]
\[
	a
\]
\[
	a
\]
y llegamos a la ecuación de Heisenberg
\[
	\dpar{A^{(H)}}{t} = \frac{1}{i\hbar} [ A^{(H)}, H^{(H)}]
\]
si $A^{(H)}$ conmuta con el $H^{(H)}$, entonces $A^{(H)}$ es una cantidad conservada (una constante de movimiento).
En ese caso el operador no depende del tiempo y entonces $A^{(H)} = A^{(S)}$.

\subsubsection{Evolución de autoestados}

\[
	a,
\]
aplico un $U^\dagger$ a ambos lados y entonces 
\[
	a
\]
los $a'$ no dependen de la representación porque tienen significado físico. Entonces los $\Ket{a'}$ evolucionan
\[
	a
\]
\[
	a
\]
\[
	a
\]
puesto que recordemos, nota importante,
\[
	a
\]
entonces $H$ es el mismo en ambas puesto que $\hat{U} =\hat{U}(\hat{H}) $ y $[U,H]=0$.

De esta forma los autoestados evolucionan al revés 
\[
	a
\]

Podemos ver de otro modo la equivalencia
\[
	a
\]
pero 
\[
	a
\]
\[
	a
\]

\subsubsection{Coeficientes}

Los coeficientes en Schrödinger y en Heisenberg son 
\[
	a
\]
Entonces en Schrödinger es 
\[
	a
\]
mientras que en Heisenberg es 
\[
	a
\]

Los coeficientes en las expresiones son iguales como corresponde a todo magnitud que tiene sentido físico, pues 
$|c_a(t)|^2$ es la probabilidad.

\subsection{Teorema de Ehrenfest}

Para una partícula libre, donde $p(t)=p(0)$ es constante de movimiento,
\[
	x^{(H)} = x(0) + \frac{p(0)}{m}t
\]
y se tiene 
\[
	[x(t),x(0)] = -\frac{i\hbar}{m}t
\]
\[
	H = \frac{p^2}{2m} + V(x)
\]
\[
	\dtot{P}{t} = \frac{1}{i\hbar}[p,H] = \frac{1}{i\hbar}[p,V(x)] = 
	\frac{1}{i\hbar}\left( -i\hbar\dpar{V}{x}\right),
\]
de modo que 
\[
	\dtot{P}{t} = -\dpar{V}{x} \qquad \longrightarrow \quad m \dtot[2]{x}{t} = -\dpar{V}{x} 
\]
\[
	p = m \dtot{x}{t} \qquad \dtot{p}{t} = m \dtot[2]{x}{t} 
\]
donde estamos usando 
\[
	\dpar{A^H}{t} = \frac{1}{i\hbar}[A^H,H]
\]

Es necesario remarcar que relaciones como $[x,p]=i\hbar$ son para operadores en la picture de Schrödinger, donde los 
operadores no cambian en el tiempo. Estamos en efecto haciendo $[x(0),p(0)]=i\hbar$
\[
	\Braket{\alpha,t_0|m \dtot[2]{x}{t}|\alpha,t_0} = - \Braket{\alpha,t_0|\dpar{V}{x}|\alpha,t_0}
\]
\[
	m\dpar[2]{}{t}\Braket{\alpha,t_0| x^H |\alpha,t_0} = -\Braket{\alpha,t_0|\dpar{V}{x}|\alpha,t_0}
\]
y entonces el teorema de Ehrenfest es 
\[
	m \dpar[2]{}{t} \Braket{x^{(s)}} = - \Braket{ \dpar{V^{(s)}}{x}}
\]
los valores de expectación son iguales en ambas representaciones.
% 
% =================================================================================================
\section{El oscilador armónico}
% =================================================================================================

Para el oscilador armónico 1D  el hamiltoniano y energía eran
\[
	H = \frac{p^2}{2m} + \frac{m\omega^2 x^2}{2} \qquad E = \hbar \omega \left( n + \frac{1}{2} \right)
\]
pero este problema puede resolverse usando un nuevo operador $\hat{a}$
\[
	\hat{a} = \sqrt{\frac{m\omega}{2\hbar}}\left( x + i\frac{p}{m\omega} \right) \qquad \text{con} \quad 
	\hat{a}^\dagger = \sqrt{\frac{m\omega}{2\hbar}}\left( x - i\frac{p}{m\omega} \right)
\]
que es suma de $\hat{x}, \hat{p}$ pero que no es hermítico. Cumple que 
\[
	[a , a^\dagger ] = 1 \qquad a a^\dagger =  \frac{H}{\hbar\omega} -1 \qquad 
	H = \hbar\omega \left( a a^\dagger + \frac{1}{2} \right),
\]
donde se define el operador número $\hat{N}\equiv a^\dagger a$ que al verificar $[\hat{N},\hat{H}]=0$ tienen base de 
autoestados en común $\{ \Ket{n} \}$. En efecto 
\[
	\hat{N} \Ket{n} = n\Ket{n} \qquad
	\hat{H} \Ket{n} = \hbar\omega \left( n + \frac{1}{2} \right) \Ket{n}
\]
siendo $n$ el número de cuantos de energía.
Se cumplen además 
\[
	[N,a] = [a^\dagger a,a] = - [ a, a^\dagger a ] = - \left( a^\dagger [a,a] + [a,a^\dagger]a \right) =
	-a
\]
\[
	[N,a^\dagger] = [a^\dagger a, a^\dagger ] = - [a^\dagger , a^\dagger a ] =
	- \left( a^\dagger [a^\dagger,a] + [a^\dagger,a]a^\dagger \right) = a^\dagger
\]

Queremos ver que le  hace $a^\dagger$  a un autoestado $\Ket{n}$ y luego $a$ sobre el mismo.
\[
	N a^\dagger \Ket{n} = ([N, a^\dagger] + a^\dagger N) \Ket{n} =
	a^\dagger \Ket{n} + a^\dagger n \Ket{n} 
\]
\[
	\Hat{N} (a^\dagger\Ket{n}) = (n+1)(a^\dagger\Ket{n})
\]
Entonces, como no hay degeneración y tenemos $N\Ket{n'} = n'\Ket{n'}$ entonces 
\[
	a^\dagger \Ket{n} = c_1 \Ket{n+1},
\]
y procediendo de modo idem para $a\ket{n}$ será
\[
	a \Ket{n} = c_2 \Ket{n-1}
\]
Luego,
\[
	a^\dagger \Ket{n} = c_1 \Ket{n+1} \overbrace{\longrightarrow}^{DC} 
	\Bra{n+1} c_1^* = \Bra{n} a 
\]
\[
	a \Ket{n} = c_2 \Ket{n-1} \overbrace{\longrightarrow}^{DC} \Bra{n-1} c_2^* = \Bra{n} a^\dagger
\]
y entonces 
\[
	\Braket{n|N|n} = n \Braket{n|n} = n =  \Braket{n| a^\dagger a |n} =  \Braket{n-1|c_2^* c_2|n-1} =
	|c_2|^2 \Braket{n-1|n-1}
\]
\[
	n = \Braket{n|aa^\dagger-1|n} = -1 + \Braket{n|aa^\dagger|n} = - 1 + \Braket{n+1|c_1^*c_1|n+1} =
	-1 + |c_1|^2 \Braket{n+1|n+1}
\]
siendo
\[
	|c_2| = \sqrt{n} \qquad |c_1| = \sqrt{n+1} 
\]
\[
	\hat{a}^\dagger \Ket{n} = \sqrt{n+1} \Ket{n+1} \qquad  \hat{a}\Ket{n} = \sqrt{n} \Ket{n-1} 
\]
y entonces de esta forma $\hat{a}^\dagger$ es el operador de creación de cuantos y $\hat{a}$ el de aniquilación.

\subsection{El estado fundamental $\Braket{0}$}

\[
	a \Ket{n}  \overbrace{\longrightarrow}^{DC} \Bra{n} a^\dagger
\]
y desde el postulado para productos internos,
\[
	(\Bra{n}a^\dagger)(a\Ket{n}) \geq 0 \quad n \Braket{n|n} \geq 0 \Rightarrow n \geq 0 
\]
entonces $n$ cabalga por los naturales.
Si hacemos 
\[
	a\Ket{n} = \sqrt{n} \Ket{n-1}, \qquad  a^2 \Ket{n} = \sqrt{n}\sqrt{n-1} \Ket{n-2} \; ...
\]
en algún momento se llega a $\Ket{n=0}$, entonces $E_0 = \hbar\omega/2$ y 
\[
	\Ket{0} \equiv \text{El fundamental}
\]
y no se puede bajar más,
\[
	\hat{a}\Ket{0} = 0.
\]

Por otra parte, con el $\hat{a}^\dagger$ se puede llegar a cualquier estado
\[
	a^\dagger \Ket{0} = \sqrt{1} \Ket{1}, \qquad  a^{\dagger 2} \Ket{0} = \sqrt{1}\sqrt{2} \Ket{2} = 
	\sqrt{1}\sqrt{2}\sqrt{3} \Ket{3}
\]
\[
	\frac{{(a^{\dagger})}^n}{\sqrt{n}!} \Ket{0} = \Ket{n}
\]

Las matrices de $\hat{a},\hat{a}^\dagger$ sólo tienen una diagonal corrida de elementoss 
\[
	\Braket{n'|a|n} = \sqrt{n} \Braket{n'|n-1} = \sqrt{n} \delta_{n',n-1}
\]
\[
	\Braket{n'|a^\dagger|n} =  \sqrt{n-1} \Braket{n'|n+1} = \sqrt{n-1} \delta_{n',n+1}
\]

También puede verse que 
\[
	\Braket{n|x|n}= 0 \qquad \Braket{n|p|n}= 0
\]
y por ello 
\[
	\Braket{(\Delta x)^2}_{\Ket{0}} \Braket{(\Delta p)^2}_{\Ket{0}} = \frac{\hbar^2}{4} 
\]
el estado fundamental es el de incerteza mínima.

\subsection{Función de onda}

Siendo $\Psi_n(x') = \Braket{x'|n}$ quiero evaluar $\Psi_0(x') = \Braket{x'|0}$ y ver que como 
\[
	\Braket{x'|a|0}= 0 
\]
tengo 
\[
	0 = \sqrt{ \frac{m\omega}{2\hbar} } \Braket{x'|x+\frac{ip}{m\omega}|0} =
	\sqrt{ \frac{m\omega}{2\hbar} } \left[ x'\Braket{x'|0} + \frac{i}{m\omega}\Braket{x'|p|0} \right]
\]
\[
	x' \Braket{x'|0} + \frac{i}{m\omega} (-i\hbar) \dpar{}{x} \Braket{x'|0} = 0
\]
entonces 
\[
	x' \Braket{x'|0} = - \frac{\hbar}{m\omega} \dpar{}{x'}\Braket{x'|0} 
\]
\[
	- \int \frac{m\omega}{\hbar} x' dx' = \int \frac{d \Braket{x'|0}}{\Braket{x'|0}} \Rightarrow 
	\Braket{x'|0} = \kappa \euler^{-m\omega x^{'2}/(2\hbar)}
\]
y entonces 
\[
	1 = \int_{-\infty}^{\infty} \Braket{0|x'}\Braket{x'|0} dx' = 
	\int_{-\infty}^{\infty} |\kappa|^2 \euler^{-m\omega x^{'2}/\hbar} dx' =
	|\kappa|^2 \sqrt{\frac{\pi\hbar}{m\omega}} 
\]
\[
	|\kappa| = \left( \frac{m\omega}{\pi\hbar} \right)^{1/2} = \frac{1}{(\pi x_0^2)^{1/4}}
\]
donde usamos el conocido resultado $\int_{-\infty}^\infty \exp( - a x^2) dx = \sqrt{\pi/a}$, llegamos al llamado pack 
gaussiano.
\[
	\Braket{x'|0} = \frac{1}{(\pi x_0^2)^{1/4}} \euler^{-\frac{1}{2}\left( x'/x_0 \right)^2}
\]
El estado fundamental tiene incerteza mínima y debe corresponder a un paquete gaussiano.

Notemos que $\hat{a}^\dagger$ crea sobre ket y aniquila sobre bra, mientras que $\hat{a}$ aniquila sobre ket y crea 
sobre bra,
\[
	a^\dagger \Ket{n} = \sqrt{n+1} \Ket{n+1} \Rightarrow \Bra{n} a = \Bra{n+1} \sqrt{n+1}
\]
\[
	a \Ket{n} = \sqrt{n} \Ket{n-1} \Rightarrow \Bra{n} a^\dagger = \Bra{n-1} \sqrt{n}
\]

\subsection{Interferencia en experimento de Young}

Consideremos la situación depicted en la figura bajo estas líneas

\begin{figure}[htb]
	\begin{center}
	\includegraphics[width=0.6\textwidth]{images/teo2_6.pdf}	 
	\end{center}
	\caption{}
\end{figure} 

Uso $\hat{H}$ de partículas libres.
\[
	\frac{1}{2} \Ket{\alpha} = \Ket{\alpha_1} = \Ket{\alpha_2}
\]
para $t>0$ se tiene 
\[
	\Ket{\tilde{\alpha_1}} = \euler^{ -i H t /\hbar } \Ket{\alpha_1} =
		\euler^{ -i E_\alpha t /\hbar } \Ket{\alpha_1}	
\]
\[
	\Ket{\tilde{\alpha_2}} = \euler^{ -i E_\alpha t /\hbar } \Ket{\alpha_2}	
\]

En la pantalla debe verse la interferencia de los dos estados solapados.
\[
	\Ket{\tilde{\alpha}} = \Ket{\tilde{\alpha_1}} + \Ket{\tilde{\alpha_2}} =
		\euler^{ -i E_\alpha \frac{d_1}{v} /\hbar } \Ket{\alpha_1} +
		\euler^{ -i E_\alpha \frac{d_2}{v} /\hbar } \Ket{\alpha_2}	
\]
\[
	\Ket{\tilde{\alpha}} = \frac{1}{2} \euler^{ -i E_\alpha \frac{d_1}{v} /\hbar } 
		| 1 + \euler^{ -i E_\alpha \frac{d_2-d_1}{v} /\hbar } | \Ket{\alpha_1}
\]
y si definimos
\[
	\beta=E_\alpha \frac{d_2-d_1}{v} /\hbar,
\]
resulta entonces
\[
	\Braket{\tilde{\alpha}|\tilde{\alpha}} = \frac{1}{4}| 1 +  \euler^{ -i E_\alpha \frac{d_2-d_1}{v} /\hbar } |^2 =
		\frac{1}{4}( (1+\cos\beta)^2 + \sin^2\beta ) =
			\frac{1}{2} + \frac{1}{2}\cos\left( \beta \right).
\]


Al partir el estado $\Ket{\alpha_1} $ y volver a unirlo en $\Ket{\alpha_1} + \Ket{\alpha_2}$ vemos una intensidad que 
dependa de la diferencia de camino.

\subsection{Cambio de cero del potencial}

En mecánica clásica la física de un problema no se ve afectada por un cambio de gauge.
Si movemos el cero de potencial, la situación física es la misma.
Veamos qué sucede en mecánica cuántica.
\[
	\Ket{\alpha,t,t_0} = \euler^{ -i (p^2/2m + V(x))(t-t_0)/\hbar} \Ket{\alpha,t_0}
\]
\[
	\Ket{\tilde{\alpha},t,t_0} = \euler^{ -i (p^2/2m + V(x) + V_0)(t-t_0)/\hbar} \Ket{\alpha,t_0}
\]
\[
	\Ket{\tilde{\alpha},t,t_0} = \euler^{ -i V_0(t-t_0)/2 }\Ket{\alpha,t,t_0}
\]
y entonces vemos que $\Ket{\tilde{\alpha},t}$ y $\Ket{\alpha,t}$ difieren en una fase, de manera que los valores de 
expectación no cambian (con $V_0$ constante).

\begin{figure}[htb]
	\begin{center}
	\includegraphics[width=0.6\textwidth]{images/teo2_7.pdf}	 
	\end{center}
	\caption{}
\end{figure} 

Este es un experimento ideal (pensado). Dentro de los cilindros hay campo nulo. Se varia el $V$ abriendo y cerrando la 
llave a la entrada y a la salida.
Se cambia la fase de las partículas inferiores respecto de las superiores, entonces habrá interferencia en $O$.

Clásicamente no hay variación,
\[
	\Delta \text{fase} = -\frac{i}{\hbar}\euler \int _{t_1}^{t_2} V_1(t) - V_2(t) dt = 
	-\frac{i}{\hbar}\euler \Delta V
\]

Lo que realmente cuenta es la diferencia de potencial $\Delta V$, la cual sí tiene sentido físico porque es 
independiente de la medida y porque pueden escribirse los campos en función de aquella.
\[
	E = - \Nabla\phi - \frac{1}{c}\dpar{\vb{A}}{t}
\]
\[
	H = \frac{1}{2m} \left( \vb{p} - \frac{\euler\vb{A}}{c}\right)^2 + \euler\phi 
\]
\[
	\dtot{H}{t} = \frac{1}{i\hbar}[x_i,H] = \frac{p_i  \euler A_i}{m}
\]

% =================================================================================================
\section{El propagador}
% =================================================================================================

Físicamente representa la proababilidad de transición entre autoestados por el paso del tiempo,
$ \Ket{x'}_{t_0} \longrightarrow \Ket{x''}_t$
\[
	a
\]
\[
	b
\]
\[
	c
\]

Podemos pensar que el propagador lleva la función de onda desde $t_0$ a $t$. Se puede escribir:
\[
	a
\]
y metemos un observable $\hat{A}$ donde $[A,H]=0$ y $A\Ket{a'}=a\Ket{a'}$.

El propagador depende del potencial, pero no de la función de onda inicial. Se debe cumplir que:
\[
	b
\]
\[
	c
\]
\[
	d
\]
y entonces el propagador es una función de Green que satisface 
\[
	d
\]
con $K(x'',t;x',t_0)=0 $ si $t<0$ que es la condición de contorno.

\subsection{El propagador de la partícula libre}

\[
	a
\]
\[
	a
\]
\[
	b
\]

También se puede escribir el propagador en la representación de Heisenberg,
\[
	a
\]
\[
	K(x'',t;x',t_0) = \Braket{x'',t | x',t_0}.
\]

El propagador cumple con la propiedad de composición (como el $U(t,t_0)$), es decir:
\[
	a
\]

% =================================================================================================
\section{Integrales de camino de Feynmann}
% =================================================================================================

Consideramos una partícula yendo de $(x_1,t_1)$ a $(x_N,t_N)$. Dividimos el tiempo 
\[
	a
\]
y queremos ver la amplitud de transición desde el estado 1 al $N$.

\[
	a
\]

Se puede pensar como que estamos sumando sobre todos los posibles caminos entre $(x_1,t_1)$ y $(x_N,t_N)$ 
fijos. En mecánica clásica teníamos un solo camino, el que minimizaba la acción $S$
\[
	\delta \int_{t_1}^{t_2} \Lag dt = \delta S = 0
\]
pero en cambio en mecánica cuántica todos los caminos aportan. En un libro de Dirac, Feymann lee 
\[
	a
\]
Definiremos
\[
	\equiv 
\]
Luego para considerar la suma sobre todos los segmentillos a lo largo de un camino tendremos
\[
	\prod_{n=2}^N \euler^{i/\hbar S(n,n-1)} =
\]
y hay que considerar TODOS los posibles caminos 
\[
	\propto \sum_{caminos} \euler^{i/\hbar S(N,1)} 
\]
cuando $\hbar \to 0$ las trayectorias contribuyen con una cantidad que oscila loca y violentamente. Tienden a 
la cancelación para caminos aledaños. Por el $\hbar \sim 0$ la fase es grande y entonces se cancelan.
Esto no ocurre cerca del camino (real) que cumple 
\[
	\delta S(N,1) = 0
\]
Para trayectorias cercanas la $\Delta fase$ no es grande y hay interferencia constructiva.
Para un $\delta t$ infinitesimal es 
\[
	a
\]
\[
	b
\]

Consideremos, por ejemplo, una partícula libre, entonces $V=0$ de modo que resolviendo 
\[
	a
\]
Esto no es otra cosa que el propagador de una partícula libre. Para un $\Delta t$ finito será 
\[
	+
\]
\[
	=
\]
siendo esta última la integral de camino de Feynmann.

En base a éstas Feynamn desarrolla una formulación equivalente de la mecánica cuántica que utiliza los 
conceptos de:
\begin{enumerate}
 \item Superposición
 \item Composición de la transición
 \item Límite clásico con $\hbar \to 0$
\end{enumerate}

Estas integrales contienen toda la información del sistema cuántico, aunque no sea sencillo extraerla.

Consideremos un propagador de $(x',0) \to (x',t)$
\[
	G(t) =
\]
\[
	G(t) =
\]
y tomando Laplace-Fourier 
\[
	\tilde{G}(t)
\]

La expresión 
\[
	\equiv Integral de camino de Feynmann
\]
satisface la ecuación de Schrödinger y es una alternativa a la formulación de la cuántica usual.

% =================================================================================================
\section{Introducción al momento angular (rotaciones)}
% =================================================================================================

El operador $\hat{L}$ será el encargado de realizar las rotaciones. Por el álgebra visto en la mecánica 
clásica sabemos que, dado un vector \vb{v} y una matriz ortogonal $R$ se tiene
\[
	\vb{v}' = R \vb{v} \qquad \text{con} \quad |\vb{v}'|=|\vb{v}|
\]
y 
\[
	|\vb{v}|^2 = V^t V = (V^t R^t) (R V) \qquad \text{pues} \quad R^tR=RR^t = \mathbb{1}
\]
puesto que es una matriz ortogonal. Luego se cumplen 
\[
	clausura	
\]
el producto de dos matrices ortogonales es otra matriz ortogonal
\[
	asociatividad
\]
\[
	E identidad
\]
\[
	E inversa
\]

\subsection{No conmutatividad de las rotaciones clásicas}

Las rotaciones finitas no conmutan. Luego, el grupo de las rotaciones será un grupo abeliano
\[
	R_z(\varphi) = \begin{pmatrix}
	 \\
	\end{pmatrix}
\]
\[
	R_x(\varphi) = \begin{pmatrix}
	 \\
	\end{pmatrix}
\]
\[
	R_y(\varphi) = \begin{pmatrix}
	 \\
	\end{pmatrix}
\]

Si reemplazamos $\cos(\epsilon) \approx 1 - \epsilon^2/2$ y $\sin(\epsilon) \approx \epsilon$ hasta orden dos.
Se puede ver que las rotaciones, en torno a ejes diferentes, sólo conmutan a orden uno $(\epsilon)$ de manera 
que una rotación infinitesimal $d\varphi$ conmuta pero una rotación finita $\varphi$ no lo hace.

% =================================================================================================
\section{Rotaciones cuánticas}
% =================================================================================================

Para las rotaciones cuánticas se pedirá
\[
	D,
\]
rotación infinitesimal o bien
\[
	D,
\]
para rotación finita. Donde $\hat{D}$ es el operador de las rotaciones y $\hat{J}$ es un momento angular 
general. Se postula de esta forma para que $\hat{D}$ cumpla las mismas propiedades que $R$ y la relación de 
conmutación
\[
	R_x R_y - R_y R_x = R_z (\epsilon^2) - \mathbb{1}
\]
\[
	D
\]
de modo que la cuenta lleva a  
\[
	J_x
\]
la cual generalizando se llega a 
\[
	[J_i,J_j] = i \hbar \epsilon_{ijk} J_k
\]
que son las relaciones de conmutación generales para momento angular $\hat{J}$.

Para sistemas de spín $1/2$ es 
\[
	D(\hat{n},\phi) \equiv \euler^{-i/\hbar \vb{S}\cdot\hat{n} }
\]
Se puede ver que ante rotaciones cuánticas $D(\hat{n},\phi)$ los valores de expectación transforman como 
vectores
\[
	=
\]

En general $\vb{J} = (J_x, J_y, J_z)$ se transforma como vector y entonces $\hat{J}$ es un operador vectorial.
Para spín $1/2$ es
\[
	\Ket{alpha} =
\]
\[
	D
\]
\[
	D
\]
Si $\phi=2\pi$ (cosa que debiera dejar al ket incólume) se tiene 
\[
	D
\]

Luego, esto es una muestra del carácter no-clásico del spin; una vuelta completa le cambia el signo al ket 
pero notemos cuidadosamente que el valor de expectación -- que es algo físico -- no varía. Esto muestra que 
el ket no puede tener sentido físico.

\subsection{Angulos de Euler}

Se define una serie de rotaciones 
\[
	1 2 3
\]
lo cual equivale a
\[
	R() = 
\]
\[
	\euler
\]

Pero desconozco cómo operar en los ejes móviles $z',y'$
\[
	R_{y'}(\beta) =
\]
\[
	R_{z'}(\gamma) =
\]
\[
	R() =
\]
Rotación equivalente a [1] pero para ejes fijos, puesto que en mecánica cuántica sabemos rotar en torno a 
ejes fijos.

Los ángulos de Euler son la caracterización de una rotación general en 3D.

Entonces nuestra rotación en 3D cuántica será:
\[
	D() =
\]

\subsection{Autoestados y autovalores de J}

Partimos de 
\[
	[] = 
\]
y
\[
	J^2 = , [J^2,J] = 0
\]
siendo la última muy importante y probándose por evaluación directa. Lleva a 
\[
	[J^2,J_i^n] = 0 \qquad \text{con} \; i=x,y,z \; n\in\mathbb{N}
\]

Se eligen $J^2, J_z$ como observables que conmutan 
\[
	J^2
\]

Definiremos los operadores de subida y de bajada
\[
	J_{\pm} \equiv J_x \pm J_y
\]
que verifican 
\[
	[]
\]
Entonces se tiene 
\[
	J^2() \longrightarrow 
\]
\[
	(J_z) \longrightarrow
\]

\[
	J_{\pm} \Ket{a,b} = C_{\pm} \Ket{a,b\pm\hbar}
\]
\[
	J_+
\]
sube el $J_z$ en una unidad de $\hbar$ o bien baja el $J_z$ en una unidad de $\hbar$.
\[
	J_+J_- =  ,
\]
\[
	J^2 = ,  
\]
\[
	\Braket{a,b|J^2 - J^2_z|a,b} = 
\]
\[
	(a-b^2)\Braket{a,b|a,b} = , a \geq b^2
\]
hay cota para $b$.
Como no puede seguir subiendo debe dar el ket nulo 
\[
	= 0
\]
\[
	= 0
\]
pero 
\[
	J_-J_+
\]
\[
	= 0	\qquad a = b_m(b_m-\hbar)
\]
tiene solución 
\[
	b_M - B_m = - \hbar
\]
pero esto es absurdo.

Luego,
\[
	\Ket{a,b_m} \longrightarrow \Ket{a,b_M}
\]
y como $J_+$ sube de a un $\hbar$ será
\[
	b_M = b_m + n\hbar
\]
y entonces
\[
	b_M = \frac{n\hbar}{2} = \frac{n}{2} \hbar = j \hbar
\]
y se da que $j$ es entero o semientero.

Definiremos 
\[
	b_M \equiv j \hbar \qquad a \equiv j (j+1) \hbar^2 \qquad -j\hbar \leq b \leq j\hbar
\]
pero como $b/\hbar = m$
\[
	b_M \equiv j \hbar \qquad a \equiv j (j+1) \hbar^2 \qquad -j \leq m \leq j
\]
\[
	m = (-j,-j+1,-j+2,...,j-1,j) \qquad 2j+1 \text{valores de} \; m
\]
\[
	J^2 \Ket{j,m} = j( j+1 )\hbar^2\Ket{j,m} \qquad J_z \Ket{j,m} = m \hbar \Ket{j,m}
\]

\subsection{La normalización de $J_\pm$}

\[
	J_+
\]
\[
	\Braket{j,m|J_-J_+|j,m} = 
\]
\[
	c_+ = 
\]
\[
	\Braket{j,m|J_+J_-|j,m} = 
\]
\[
	c_- =
\]
\[
	J_+
\]

\subsection{Elementos de matriz de $J^2, J_z, J_+$}

Asumiendo normalización de $\Ket{j,m}$ se tiene 
\[
	\Braket{} = 
\]
\[
	=
\]

\subsection{Elementos de matriz de $\mathcal{D}(R)$}

Ahora queremos ver cual es la forma de los elementos de matriz de $\mathcal{D}(R)$
\[
	\mathcal{D}(R) =
\]
siendo que $\mathcal{D}(R)$ tiene por efecto rotar el sistema físico.
Lo primero que hay que notar es que 
\[
	\propto \delta_{jj'}
\]
porque $[J^2,J_i]=0$ y entonces $[J^2,J_i^n]=0$ y 
\[
	D
\]
y 
\[
	D
\]
es una matriz para cada $j$ fijo con $\{ (2j+1)\times(2j+1)=\text{dimensión}\}$
\[
	D
\]
pero las rotaciones no cambian el $j$, $\mathcal{D}(R)$ conecta estados con la misma $j$ y $\mathcal{D}(R) 
\in (2j+1)\times(2j+1)$ 
\[
	D
\]

La matriz de $\mathcal{D}(R)$ (no caracterizada por un único $j$) puede ponerse en forma diagonal por bloques:


con cada bloque de $(2j+1)\times(2j+1)$ , pero siendo cada bloque irreducible. Las matrices de rotación con 
$j$ fijo forman un grupo. $\mathcal{D}_{m'm}^{(j)}(R)$ son los elementillos de la matriz.
\[
	\Ket{j,m} \longrightarrow
\]

\subsection{Forma explícita del operador $\mathcal{D}(R)$}

Los ángulos de Euler permitieron caracterizar la rotación más general. Entonces 
\[
	D
\]
\[
	D
\]

En los $d_{m'm}^{(j)}$ está la dificultad de la cuenta.

% =================================================================================================
\section{Formalismo de spinores de Pauli}
% =================================================================================================

Apropiado para trabajar con sistemas de spín $1/2$. Estos sistemas son casos particulares de momento angular,
\[
	j = 1/2 \qquad m=-\frac{1}{2},+\frac{1}{2}
\]
y se definen los spinores $\chi_\pm$ como
\[
	\Ket{+} \equiv  \begin{pmatrix}
	                 1 \\ 0 
	                \end{pmatrix} \equiv  \chi_+
\]

Para spín $1/2$ podemos tomar $\vb{J} = \vb{S}$ por la analogía de las relaciones de conmutación.
A su vez 
\[
	\vb{S} = \frac{\hbar}{2} \vec{\sigma} \; \text{con} \;
\]
que es una especie de vector 
\[
	\sigma
\]
Luego esta equivalencia provee expresión de los operadores $S_i$ en términos de matrices de $2\times 2$, así:
\[
	\frac{i}{2}[ J_- - J_+] = J_y = S_y = \frac{\hbar}{2} \sigma_y
\]

Las matrices de Pauli cumplen las propiedades básicas siguientes 
\[
	\sigma^2_i = \mathbb{1}
\]
\[
	\Ket{+} \qquad (\sigma\dot\vb{a})
\]

\subsection{Aplicación a las rotaciones}

\[
	D
\]
pero 
\[
	()^n= 
\]
\[
	\euler
\]
\[
	d
\]
es el operador de rotación para sistemas de spin $1/2$. Con esta expresión podemos evaluar 
$d^{j=1/2}_{m'm}(\beta)$
\[
	d
\]
donde hemos usado los resultados 
\[
	\cos \sin 
\]

En el caso general el operador de rotación para sistemas de spin $1/2$ lucirá:
\[
	D = 
\]

\subsection{Ejemplo}

\[
	d
\]
Este resultado es intuitivamente lógico.


\subsection{Rotaciones en sistemas con $j=1$}

Ahora tenemos 
\[
	j=1 \qquad m = -1,0,1
\]
recordando $J_y$ en términos de escaleras
\[
	J_y
\]
\[
	\euler
\]
\[
	\left( \frac{J_y}{\hbar} \right)^n =
\]
\[
	\euler =
\]
acá lo vemos como operador (es notación), $d_{m'm}^{j=1}(\beta)$ simboliza la matriz
\[
	d
\]

% =================================================================================================
\section{Momento angular orbital}
% =================================================================================================

\[
	\vb{L} =
\]
verifica el álgebra de $\vb{J}$,
\[
	[]
\]
Consideremos ahora una rotación en torno a $z$, en un $\delta\phi$,
\[
	() =
\]
\[
	() = 
\]
esto es una traslación en $\hat{x},\hat{y}$,
\[
	(1-i\frac{L_z}{\hbar}\delta\phi) \Ket{x',y',z'} = \Ket{}
\]

Esta traslación es debida a una rotación infinitesimal en $\delta\phi$ torno a $z$ entonces genera las 
rotaciones clásicas en torno a $z$.
\[
	\Psi 
\]
\[
	\Psi
\]

Podemos hallar una expresión para $L_z$ en esféricas:
\[
	\Braket{r,\theta,\varphi||\alpha}
\]
identificamos 
\[
	= 
\]
operador $L_z$ en esféricas

Usando 
\[
	L^2 = 
\]
\[
	\Braket{L^2}
\]
\[
	L^2 = -\hbar^2 r^2 \nabla^2_{\theta,\varphi}
\]
donde $\nabla^2_{\theta,\varphi}$ es la parte angular del laplaciano en coordenadas esféricas.
Esto puede obtenerse también partiendo de 
\[
	L^2 = \vb{x}^2\vb{p}^2 - (\vb{x}\cdot\vb{p})^2 + i\hbar \vb{x}\cdot\vb{p}
\]

Sea un $H$ de partícula, sin spín, sujeta a potencial simétricamente esférico. Sabemos que la función de onda 
$\Psi_\alpha(\vb{r}')$ es separable en coordenadas esféricas, entonces:
\[
	\Braket{|} = 
\]
\[
	\Braket{|} = 
\]

Cuando el H es esféricamente simétrico (como en un potencial central) se tiene 
\[
	[] = [] = 0
\]

Trabajaremos solamente en la parte angular  $\Ket{\theta,\varphi} \equiv \Ket{\hat{n}}$
\[
	\Braket{\hat{n}|\ell,m} =
\]
que es la amplitud de hallar $\Ket{\ell,m}$ en la dirección $\hat{n}$.

Podemos vincular ahora los armónicos esféricos con los autoestados de $L_z,L^2$
\[
	L_z
\]
\[
	L^2
\]
\[
	=
\]
Entonces, con la ortogonalidad
\[
	\longrightarrow
\]
y con la completitud 
\[
	\longrightarrow
\]
de manera que llegamos a 
\[
	\int \int 
\]

Podemos hallar una expresión para 
\[
	= 0
\]
\[
	\Rightarrow
\]

Luego usamos $L_-$ para hallar sucesivamente los demás $Y^m_\ell$
\[
	=
\]
y por este camino se llega a 
\[
	Y
\]
con 
\[
	\qquad 
\]

En el caso de momento angular orbital $\ell$ no puede ser semientero porque entonces $m$ sería semientero y 
en una vuelta de $2\pi$
\[
	\euler^{im2\pi} = -1
\]	

Además,
\[
	\text{(no hay signo menos)}
\]

\subsection{Armónicos esféricos como matrices de rotación}

Se pueden hallar autoestados de dirección $\Ket{\hat{n}}$ rotando el $\Ket{\hat{z}}$,
\[
	\hat{n} = 
\]
Necesitamos aplicar 
\[
	D
\]
\[
	n
\]
\[
	l
\]
\[
	l
\]
\[
	Y
\]
pero como $\theta=0$ , $Y_\ell^m = 0$  con $m\neq 0$ se tiene 
\[
	lm
\]
\[
	Y^*,
\]
la matriz de rotación en este caso es un armónico esférico.

La $\Psi$ tiene la misma simetría que el potencial.

% =================================================================================================
\section{Suma de momentos angulares}
% =================================================================================================

\subsection{Dos momentos de spín $1/2$}

Sean dos estados de spín $1/2$
\[
	a
\]
en cada espacio valen las relaciones usuales de conmutación 
\[
	b, [S_{1i},S_{2j}] = 0
\]
donde el último indica que operadores de espacios diferentes conmutan.

Un estado general es 
\[
	a
\]
Hay cuatro estados
\[
	b
\]
que corresponden a los operadores $S_ 1^2, S_2^2, S_{1z}, S_{2z}$ que conmutan (son un CCOC).

Podemos elegir otras base de operadores que comutan que será: $S_ 1^2, S_2^2, S, S_{z}$, de modo que el estado general 
será
\[
	c
\]
Así tendremos
\[
	d
\]
\[
	S 2 =  \qquad 
\]

Dada la repetición de $S_1,D_2$ se suelen identificar a las bases solamente 
\[
	d
\]
Además la base $\{ \Braket{m_1,m_2}\}$ se puede poner como 
\[
	+ \equiv + 1/2 \qquad\qquad - \equiv - 1/2
\]

\subsection{Cambio entre bases}

Podemos hallar a ojo que 
\[
	\cdot \Ket{++} = \Ket{1,1} \qquad \cdot \Ket{--} = \Ket{1,-1} 
\]
de manera que la única forma de tener $m=1$ es con los dos spines up y la única forma de tener $m=-1$ es con los dos 
spines down.

Se hallan los otros con el operador de bajada
\[
	S_- \equiv S_{1-} + S_{2-}
\]
y si descompongo $S_-$ en $S_{1-}$ y $S_{2-}$ para operar en $\Braket{s,m}$ se tiene 
\[
	S
\]
y ahora si opero con $S_-$,
\[
	S
\]

Luego
\[
	\Braket{00}
\]
y puedo usar ortonormalidad 
\[
	= 0 = \qquad \text{con} \; |a|^2 + |b|^2 = 1
\]
\[
	\cdot 
\]


\section{Teoría formal de suma de momentos angulares}

Sea de sumar dos momentos angulares $J_1, J_2$. Las relaciones de conmutación son
\[
	[] = i\hbar \varepsilon_{ijk}J_{1k} \qquad [] = i\hbar \varepsilon_{ijk}J_{2k} \qquad
	[J_{1k},J_{2k}] = 0
\]
\[
	\vb{J} = \otimes + \otimes \equiv \vb{J}_1
\]
\[
	[]
\]

El momento total \vb{J} cumple que 
\[
	J^2 = 
\]
donde vemos que 
\[
	[] =
\]
pero 
\[
	[ J^2 , J_{1z}] \neq 0  \qquad \qquad [ J^2 , J_{2z}] \neq 0
\]

Esto deja dos opciones para elegir un CCOC

\begin{center}
\begin{tabular}{|c|c|}
\hline 
$J_1^2, J_2^2, J_{1z}, J_{2z}$ & $J_1^2, J_2^2, J^2, J_{z}$ \\
$\Ket{j_1,j_2;m_1,m_2}$ & $\Ket{j_1,j_2;j,m}$ \\
base desacoplada & base acoplada \\
\hline
\end{tabular}
\end{center}



Se puede pasar de uan base a otra con una identidad $\mathbb{1}$ apropiada
\[
	\Ket{j_1,j_2;j,m} =
\]
\[
	1.
\]
\[
	a
\]
\[
	2.
\]
\[
	a
\]

Donde los $C_{m_1 m_2}^j$ son los coeficientes de Clebsh-Gordan. En 2 la $\sum$ sería en $j\to\infty$, pero veamos la 
relacion que hace algunos $C_{m_1 m_2}^j=0$. Ante todo abreviaremos suprimiendo los índices $j_1,j_2$ con lo cual 
\[
	C
\]

\subsection{Restricciones para la no nulidad de los coeficientes}

\[
	a
\]
\[
	b
\]
\[
	\Braket{m_1,m_2| j,m} \neq 0 \Rightarrow m = m_1 + m_2
\]

A su vez, en la suma de $J_1$ y $J_2$ resultan los $j$ acotados por una desigualdad triangular 
\[
	|| \leq j \leq j_1 + j_2
\]

Asimismo los $C_{m_1 m_2}^j$ se toman reales, entonces 
\[
	C
\]
y juntando todo se tiene 
\[
	\Rightarrow
\]

Ambas bases tienen la misma dimensión
\[
	\sum = (2j_1 + 1)(2j_2 + 1)
\]

Recordemos que cada $j$ tiene $2j+1$ estados posibles (los $m$ correspondientes a cada $j$) ($|m|\leq j$). Si sumamos 
$j_1=1, j_2=3/2$ tendremos 
\[
	dim
\]
\[
	j = 1/2, 3/2, 5/2
\]

Podemos ver a ojo que 
\[
	\Ket{j}
\]
luego con el $J_=, J_-$ podemos construirnos los siguientes (utilizando ortonormalidad)
\[
	= 
\]
\[
	= \qquad \sum = 1
\]

\subsection{Relación de recurrencia}

\[
	J_\pm = \ ket{j,m} = 
\]
\[
	b
\]
y metiendo un bra $\Bra{m_1,m_2}$ se llega a la relación de recurrencia
\[
	\sqrt{()}
\]

\subsection{Suma de \vb{L} y \vb{S}}

Sea suma \vb{L} y \vb{S}, entonces 
\[
	a
\]
habrá sólo cuatro $C_{m_1 m_2}^j$ no nulos, que serán 
\[
	\Braket{|}
\]
donde vemos que los coeficientes linkean sólo los estados con $j=\ell-1/2$ y $j=\ell+1/2$ y podemos construir una 
matriz de $2\times 2$ para este caso.

Esto tórnase práctico para acoplamiento spin-órbita 
\[
	LS
\]
\[
	LS
\]
\[
	LS
\]

\section{Operadores vectoriales}

Queremos analizar como transforma un operador vectorial $\hat{v}$ bajo rotaciones en mecánica cuántica.
En mecánica clásica,
\[
	V_i = R_{ij} V_j \qquad \text{con} \; R \; \text{matriz diagonal}
\]
En mecánica cuántica tenemos que al rotar
\[
	=
\]
Pediremos entonces que $\Braket{V}$ transforme como un vector y eso lleva a que 
\[
	=
\]
\[
	\mathcal{D}(R)^+
\]
y calculando la expresión anterior 1 llegamos a que debe valer
\[
	[V_i,J_j] =  i\hbar \varepsilon_{ijR}V_R
\]
que es la manera de transformar de un operador vectorial. Podemos probar un caso simple de una rotación infinitesimal 
en $\hat{z}$ y ver que vale.


\section{Operadores tensoriales}

En mecánica clásica 
\[
	T_{ij} 
\]
que es un tensor de rango dos. Esto es un tensor cartesiano. Su problema es que no es irreducible, entonces puede 
descomponerse en objetos que transforman diferente ante rotaciones. Sea la díada $U_iV_j$, tensor de rango dos, que 
puede escribirse como 
\[
	UV =
\]















% \begin{figure}[htb]
% 	\begin{center}
% 	\includegraphics[width=0.35\textwidth]{images/fig_ft1_gauss.pdf}	 
% 	\end{center}
% 	\caption{}
% \end{figure} 




% \bibliographystyle{CBFT-apa-good}	% (uses file "apa-good.bst")
% \bibliography{CBFT.Referencias} % La base de datos bibliográfica

\end{document}
